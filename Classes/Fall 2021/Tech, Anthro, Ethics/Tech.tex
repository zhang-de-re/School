\documentclass[letterpaper,notitlepage,12pt]{article}
\usepackage[
  letterpaper,tmargin=1in,bmargin=1in,lmargin=1in,rmargin=1in]
{geometry}
\usepackage[backend=biber]{biblatex-chicago}
\usepackage{csquotes}
\addbibresource{tech.bib}
\usepackage{setspace}
\usepackage[super]{nth}

\title{Nonideal Theories of Technology}
\author{Alexander Zhang}
\date{}

\doublespacing

\begin{document}

\maketitle

\section{Philosophy for whose sake?}

The following is in part inspired by Shannon Vallor's project in
  \textit{Technology and the Virtues: A Philosophical Guide to a Future Worth
  Wanting}.
Drawing from ancient virtue traditions of Aristotle, Confucius, and Buddha, she
  attempts to motivate a new virtue tradition founded in the technological
  particularities of the modern context.
In doing so she strives to address a \textquote{need for a common framework in
  which these narratives can be situated if humans are going to be able to
  address these emerging problems of a \textit{collective} technosocial action
  wisely and well.}\footcite[p. 9]{vallor_technology_2018}
For the current discussion, I am not so much interested in evaluating the common
  framework that Vallor proposes but in trying to motivate the need which she
  acutely observes.
To put it another way, I want to attend to the subtitle of her project: Why do
  we need a philosophical guide to a future worth wanting?

At the heart of this question is another: Who is our philosophy for?
Clearly the answer is smuggled in here by the framing.
Such a thing is only required to make intelligible actual dialogue across
  disciplines and across the thresholds of ivory towers.

But then, why should philosophers think such a project is worthwhile?
Insofar as there are anxieties about a future that negatively affects us all as
  humanity collectively, then the answer is clear.
The kinds of problems that we might want ethics to address are not purely
  academic.
They are instantiated in the real world affecting and affected by real people.
As Vallor notes, \textquote{ethical discourse that speaks only to the concerns
  of a particular moral or philosophical `tribe' will be helpless to confront
  the ethical impact of the technosocial realities that increasingly address
  humans \textit{collectively}---realities which demand the effective
  cultivation and application of some measure of cooperative human
  wisdom.}\footcite[p. 23]{vallor_technology_2018}
The problems themselves mold, or ought to, philosophy's response to them.
Insofar as we are interested in applying philosophy towards understanding
  problems and finding solutions, then we are ultimately interested in applied
  philosophy.

But as I will argue, we often fall short of this seemingly simple task.
This hinges on a fundamental issue of philosophical methodology.
If we are to ever see philosophy make substantive, constructive contributions,
  then there must be a fidelity to the concrete realities of the problem rather
  than, as Vallor puts it, \textquote{the perils of essentialism,
  overgeneralization, and abstraction.}\footcite[p. 33]{vallor_technology_2018}
It might be objected that this way lies pragmatism, relativism, nihilism,
or---even more disrespectfully---mere sophistry.
Such claims are fundamentally misguided.
It is true that there is interesting and valuable work in teasing out the
metaphysics pertaining to an applied issue like technology.
The problem is the territorial impulse to decree that the task begins and ends
with theorizing the metaphysics, ontology, anthropology, or whatever.
There is, of course, the practical objection that if we cannot proceed until the
first philosophy is completed, then we will never get to the ethics.
Consider J.S. Mill's frustration.

However, this is not the argument that I will make here.
More fundamentally, the issue is that an ideal theory, rather than an nonideal
theory, will be flawed in virtue of not accounting for the nonideal.
Any attempt to proceed from first principles on the assumption that we will
arrive at a theory that makes the applied question trivial will fail if it ever
gets there.
While metaphysics will inform the applied ethical question, the concrete
particularities of the applied issue also informs the metaphysics.
To this end, I will draw out the necessity for a nonideal approach to the ethics
of technology.
An approach founded in idealization---toy models of metaphysics and
anthropology---will be edifying bu ultimately impotent, at least, if we are
genuinely concerned with the question of ethics: of life as it is lived and how
it ought to be.
To illustrate this point, I will take Bernard Stiegler's ethics of technology as
paradigmatically ideal and contrast it with a nonideal reflection on the
contraceptive pill.
This will ground a discussion of what disruption and reorientation might look
like in light of that history.
I will conclude on a remark of the impossibility of an ideal theory of care.

\section{Exegesis on ideal and nonideal theory}

Marx did not invent socialism.
Yet, for all the early Fourierist, Owenites, and Saint-Simonians, socialism has
become bound up with Marxism.
This is not a trivial distinction, although it is one often forgotten by
Marxists since.
Marx's break from the prior theories of socialism is his turn, as he describes
it, from `Utopian' socialism to `Scientific'.
He rejects that our present condition in capitalism is simply because
philosophers have yet to stumble across the \textquote{absolute truth, reason,
  and justice [which] has only to be discovered to conquer all the world by
virtue of its own power.}\footcite{engels_socialism:_1978}
Rather, \textquote{In direct contrast to German philosophy which descends from
  heaven to earth, here we ascend from earth to heaven. That is to say, we do
  not set out from what men day, imagine, conceive, nor from men as narrated,
thought of, imagined, conceived, in order to arrive at men in the
flesh.}\footcite{marx_german_200}
It is probably for this reason why, despite his metamorphosing inspiration by 
the Silesian weavers who smashed the factories autolooms, he calls for workers 
to seize the means of production instead.

This distinction was preserved through feminists like Onora O'Neill and further
developed in philosophy of race by Charles W. Mills.
He further develops the point by arguing \textquote{that the so-called ideal
  theory more dominant in mainstream ethics is in crucial respects obfuscatory,
  and can indeed be thought of as in part \textit{ideological} in the pejorative
  sense of a set of group ideas that reflect, and contribute to perpetuating,
illicit group privilege.}\footcite[p. 166]{mills_ideal_2005}
To make sense of this, however, we have to disambiguate the senses of
\textquote{ideal}.
Out of the gate, the intended referent is not the trivial sense of
ideal-as-normative in which any normative ethics that asserts an ought is
committed to.

Rather, Mills is interested in how we create ideals of, or, rather, idealize,
the world as models.
He distinguishes between two different ways in which we---in ethics, philosophy
in general, or natural and social sciences---create idealized models.

First is the ideal-as-descriptive-model wherein we take an ideal to be
representative of some phenomenon P.
However, \textquote{Since a model is not coincident with what it is modeling, of
  course, an ideal-as-descriptive-model necessarily has to abstract away from
certain features of P}\footcite[p. 166]{mills_ideal_2005}
As a result, we have to make decisions, consciously or otherwise, about which 
features of P are important and which will be omitted.

This, then, is an opportunity to define rather than merely describe.
For certain P we can produce an exemplar, or schematized picture of the actual
nature and mechanism, of what an ideal P is like.
This is the ideal-as-idealized-model.\footcite[p. 167]{mills_ideal_2005}
There are all sorts of occasions when we refer to an ideal-as-idealized-model.
When we try to evaluate the quality of some functioning, like a body's health,
we refer to an idealized anatomy of what \textquote{true} functioning looks
like.
Alternatively, to minimize the cognitive work involved, we might make
simplifying assumptions that some idealized condition pertains like a perfect
vacuum, a frictionless plane, or a resistance-free conductor.

However, while we frequently make use of ideal-as-idealized models, we ought to
ask how useful will it be too use them as a starting point.
Mills gives the example of the idealized frictionless plane.
This might be unproblematic if the actual plane in question is coated in Teflon.
In that case, the actual P approximates the ideal P with only minor deviation.
But of course, if the actual P is covered in Velcro, this reasoning is absurd.
Mill notes, \textquote{Ideal-as-descriptive-model, the model of the actual
  workings of the plane, will be quite different from the
  ideal-as-idealized-model, and one will need to start with an actual
  investigation of the plane's properties. One cannot just conceptualize them
in terms of a minor deviation from the ideal, ideal-as-idealized
model.}\footcite[p. 167]{mills_ideal_2005}
The point is that the degree to which actual P resembles the ideal P is not
something that can be rationalized \textit{a priori}.

Of course, the answer to how useful an ideal-as-idealized-model is as a starting
point depends on the fact of how closely it resembles the actual.
Since we cannot know a priori the degree to which this is the case, it is
methodologically unsound to proceed on an ideal-as-idealized-model based on a
suppressed premise that it is sufficiently approximate to the
ideal-as-descriptive-model.
In the case that it is not, then our conclusions will be invalid and fail to
capture the actual phenomenon we are attempting to model.

The significance of this to ethical theory should be obvious.
Again, as Mills notes, \textquote{if one wants to change the actual P so it
  conforms more closely in its behavior to the ideal P, one will need to work
  and theorize not merely with ideal, ideal-as-idealized-model, but with the
  nonideal, ideal-as-descriptive-model, so as to identify and understand the
  peculiar features that explain P's dynamics and present it from attaining
ideality.}\footcite[p. 167]{mills_ideal_2005}
To assert any kind of `ought', a methodologically sound approach must work with
both.

Moreover, we are morally obligated to do so in ethics.
Proceeding on idealization of ideal-as-ideal-models is deeply problematic.
As O'Neill contends, this may involve attributing abilities to agents from
outside the human norm.\footcite[p. 56]{oneill_abstraction_1987}
Mills extends this argument by asserting that moral ideal theories will also
involve any of the following: an idealized social ontology, silence on
oppression, ideal social institutions, an idealized cognitive sphere, or strict
compliance.\footcite[p. 168-9]{mills_ideal_2005}
I can only let Mills' words to speak for themselves: \blockquote{Now look at
  this list, and try to see it with the eyes of somebody coming to formal
  academic ethical theory and political philosophy for the first time. Forget,
  in other words, all the articles and monographs and introductory texts you
  have read over the years that may have socialized you into thinking that this
  is how normative theory should be done. Perform an operation of Brechtian
  defamiliarization, estrangement, on your cognition. Wouldn't your spontaneous
  reaction be:\textit{How in God's name could anybody think that this is the
appropriate way to do ethics?}}\footcite[p. 169]{mills_ideal_2005}

This reaction is not philosophically na\"{i}ve.
It correctly intuits that \textquote{In modeling humans, human capacities, human
  interaction, human institutions, and human society as an
  ideal-as-idealized-model, in never exploring how deeply different this is from
  an ideal-as-descriptive-model, we are are abstracting away from realities
  crucial to our comprehension of the actual workings of injustice in human
  interactions and social institutions, and thereby guaranteeing that the
ideal-as-idealized-model will never be achieved.}\footcite[p.
170]{mills_ideal_2005}
Mills takes as his main target for critique to be Rawls.
However, this criticism can be applied widely across the field.
As he exclaims, \textquote{Why should anyone think that abstaining from
  theorizing about oppression and its consequences is the best way to bring
  about an end to oppression? Isn't this, on the face of it, just completely
implausible?}\footcite[p. 171]{mills_ideal_2005}
We can see in this that Mills primary concern is the way idealizing values and
norms are taken to be an appropriate methodology for ethical theorizing.
In this way, Mills accusation of ideal theory as ideology is clarified.
Given the actual realities of academic philosophy, demographics both past and
present, this will result in the nonrepresentative interests and perspective of
the over-represented.
To be clear, this should not be construed as a conscious conspiratorial
manipulation.
Rather, it is simply an unfortunate result of the confluence of social
privilege and the difficulty humans seem to have seeing outside of their own
standpoint prompted or unprompted.
The moral fault only comes about if we mistake that limited perspective for the
world as it is despite ample opportunities to recognize the evidence to the
contrary.

Before finishing this exegesis on Mills' ideal and nonideal framework, let's
consider the following objection: \textquote{Suppose it is claimed that the
  foregoing accusations are unfair because, in the end, nonideal theory and its
  various prescriptions are somehow already \textquote{contained} within ideal
  theory. So there is no need for a separate enterprise of this kind---or if
  there is, it is just a matter of \textit{applying} principles, not of
\textit{theory} (applied ethics rather than ethical theory)---since the
appropriate recommendations can, with the suitable assumptions, all be derived
from ideal theory.}\footcite[p. 177]{mills_ideal_2005}
This objection rests on the misapprehension of the obstacles involved.
Again, this requires stepping outside your standpoint, the perspective molded
over the course of a whole life, and into another's.
This is only exacerbated if those dissenting, who would give the nonideal
necessary to inform, are excluded, as has historically been the cases.

Ultimately, those who try to make this case will have to satisfactorily account
for the history of philosophy as it is.
Like Mills points out, \textquote{if it were as easy as all that, just a matter
  of \textit{modus ponens} or some other simple logical rule, then why was it so
  hard to do?\ldots The actual working of human cognitive processes, as
  manifested in the sexism and sometimes racism of such leading figures in the
  canon as Plato, Aristotle, Aquinas, Hobbes, Hume, Locke, Rousseau, Kant,
  Hegel, and the rest, itself constitutes the simplest illustration of the
mistakenness of such an analysis.}\footcite[p. 178]{mills_ideal_2005}
Ideal theory has failed to obviate the need for a nonideal theory for all these
millennia.
More will be said on this later.
For now, Mills point for the need for nonideal theory is hopefully sufficiently
motivated: \textquote{Insofar as concepts crystallize in part from experience,
  rather than being a priori, and insofar as capturing the perspective of
  subordination requires advertence to its reality, an ideal theory that ignores
these realities will necessarily be handicapped in principle.}\footcite[p.
177]{mills_ideal_2005}

\section{Broken Clocks}

In contrast to this, consider the following passage from Stiegler on the pace of
technological innovation as it compares to the pace of cultural evolution:
\blockquote{It is as if time has leapt outside itself: not only because the
  process of decision making and anticipation (in the domain of what Heidegger
  refers to as \textquote{concern}) has irresistibly moved over to the side of
  the \textquote{machine} or technical complex, but because, in a certain sense,
  and as Blanchot wrote recalling a title of Ernst J\"{u}nger, our age is in the
  process of breaking the \textquote{time barrier}. Following the analogy with
  the breaking of the sound barrier, to break the time barrier would be to go
  faster than time. A supersonic device, quicker than its own sound, provokes at
  the breaking of the barrier a violent sonic boom, a sound shock. What would be
  the breaking of a time barrier if this meant going faster than time? What
  \textit{shock} would be provoked by a device going quicker than its
  \textquote{own time}? Such a shock would in fact mean that speed is older than
  time. For either time, with space, determines speed, and there could be no
  question of breaking the time barrier in this case, or else time, like space,
is only thinkable in terms of speed (which remains
unthought).}\footcite{stiegler_technics_1998}

Mills criticism of ideal theory focuses on its idealizing as a kind of ideology.
By smuggling in the values and norms of a privileged minority and presenting it
as ideal-as-idealized-model of reality from which a universal ethics is
theorized, it obscures the nonideal realities of those outside the standpoint of
that minority.
It is not evident that that is what is happening here, or, at least, not
entirely so.
However, the ideal/nonideal framework is a methodological one---it applies
whenever we idealize a model and have need to ask how sound a starting point for
our inquiry, how closely it approximates the real phenomenon, at all.
In this light, we should consider the significance of ideal theories that
proceed on ideal-as-idealized-models of the world, or features of the world, as
the grounds for ethical theorizing in addition to the Mills' critique of
idealizing values and norms.
However, to do this, we have to illustrate the gap between the
ideal-as-idealized-model and the ideal-as-descriptive model.
To that end, let's take Stiegler to be an example of the former.

Stiegler's theory of technology, roughly put, rests on the question of time
following on from Heidegger.
On a basic level he argues the following.
Culture represents a means by which the knowledge relevant to Being is
transmitted through time.
However, technological innovation disrupts culture and thereby disrupts the
transmission of that knowledge.
This requires that culture goes through a period of reorientation to adjust to
the new technologies.
However, since the logic of technology aims towards permanent innovation at an
ever increasing rate and cultural development is capped at an evolutionary or
biological rate, we can infer that at some point the rate of technology will
outpace the rate of cultural evolution, breaking the time barrier. In such a
case, we are doomed to some kind of cataclysm or another---at best, an age of
constant disruption with no stabilizing reorientation possible.
It is unclear whether this crossing is a future possibility or a past already
realized---he refers to it in both ways---but he supposes that, in response, we
have to build structures of care that utilize ritual and transitional objects
that obviate the threat of the techno-logic.

There are a few points we can note to bring out the later arguments.
First is what technology does to knowledge.
Stiegler claims \textquote{when a change of technical system occurs---in
  Bertrand Gilles' sense---the epoch from which it originated comes to an end: a
  new epoch emerges, generally at the cost of military, religious, social and
  political conflict of all kinds. But the \textit{new epoch} emerges only
  when---on the occasion of these conflicts, and due to loss of the scheme of
  the proceeding epoch's knowledge and powers of living, doing and
  conceiving---new ways of thinking, new ways of doing, and new ways of living
  take shape, which are new `new forms of life' in Georges Conguilhem's sense,
  on the basis of precursors \textit{reconfiguring the retentions inherited from
the earlier epoch into so many kinds of protention.}}\footcite[ch.
2.8]{stiegler_age_2019}
More will be said on Stiegler's account of disruption and reorientation in the
next section.
For now, what's worth noting is that it is characteristic of his style---as often
is in ideal-as-idealizing-models---that he simply presents these assertions as
descriptive declarations.
It is declared that \textquote{It is impossible to live in a society without
  positive collective protentions, but the latter are the outcome of
intergenerational and transgenerational transmission.}\footcite[ch.
2.7]{stiegler_age_2019}
Certainly it is the case that for, at least some, this shared history is a
significant ground for how the conceptualize their own selves.
As such, it seems to be reasoned that this kind of culture must be a necessary
condition for authentic, or at least intelligible, being as an
individual.\footnote{To perhaps pull back the curtain, this seems to be another
  case in point for the overarching structure of critique I'm making. I'm happy
  to grant that this is the experience and, therefore, values of many people who
  are born into a culture in which they already belong. This, however, has not
  been my experience. As a first-generation American-born Chinese, I have always
  experienced being the Other: culturally domestic but categorized as foreign.
  To Americans, I am Chinese and, to Chinese, I am American. I am cultural
  flotsam. This is my nonideal reality. What, then, should I make of Stiegler's
  declarations of his ideal reality? Am I the obscured jetsam of the
  ideal-as-idealized model of values and perspectives that don't recognize me or
  my experiences? Or is it simply the case that it is impossible for me to live
in a society?}

Similarly, Stiegler declares that it is an intrinsic feature of technology that
it interrupts the transmission of knowledge via culture. \textquote{A change of
  the technical system always inevitably entails a disadjustment between this
  technical system and what Bertrand Gille called the social system, which had
  hitherto been `adjusted' to the preceding technical system, and which had
thrown forward \textit{along with it} an `epoch'---but where the technical system
  as such fades into the background, forgotten as it disappears into
  everydayness, just as for a fish, what disappears from view, as its `element'
is water.}\footcite[ch. 2.8]{stiegler_age_2019}
This assertion seems to hinge on the intuition that technological innovation can
be generalized  as change since technology, in part, shapes the social world.
As such, if there is an introduction of change in the technology, it follows
there must be change in the social.
The disruption results in a disorientation and a reorientation.
A new culture is formed, presumably at the loss of the old.
Again, operating on the level of abstraction and metaphysical essentialism, this
appears intuitive.

Finally, we can note how Stiegler characterizes care in his account.
\textquote{It is as a \textit{rational form of care} maintaining reason through
  the formation and training of deep attention of a specific kind, aiming at the
  formation of apodictic statements, that is, submitted to anamnesic,
  cumulative, non-contradictory and demonstrative rules of transindividuation,
  that theory finds itself short-circuited. But with it, and behind it, this
  regression affects \textit{every long} transindividuation circuit, not all of
  which are theoretical. The theoretical is a modality of the experience of
  consistencies, that is, of motives that form into thinking for oneself,
  configuring, as anamnesic necessity, a `true self', and as such constructing
thought as creativity}.\footcite[p. 32]{stiegler_what_2013}
Again, more will be said on this point in the final section.
For now, it is enough to observe how care is idealized.
Elsewhere, Stiegler calls for a `politics of care' in explicit contrast and
exclusion of a mere `ethics'.\footcite[p. 97]{stiegler_what_2013}
Here, care is abstracted away from individuals and their interdependencies.
It is instead presented as some kind of socio-political structure instantiated
by policy and committed to protecting the bases of cultural long-circuits like
some kind of conservation project.

I have, by and large, attempted to let Stiegler speak for himself in the brief
space that I can afford.
This is intentional in an effort to illustrate the difference with the nonideal
approach to conceptualizing the disruption and reorientation dynamic between
technologies and cultures in the next section.
Permit me, however, to end this section on one last passage.
In starting from the ideal-as-idealized-model of technology and culture,
Stiegler finds the force of his argument leading him again and again to this
conclusion: \textquote{In doing so, the principle of disruption consists in
  \textit{hastening the Anthropocene's approach towards its limits}. By
  violently accelerating this event, disruption risks triggering in each of us
  the feeling that we are rushing headlong into an abyss, which would be to
  produce a purely negative collective protention---bringing with it the most
barbarous behavior imaginable.}\footcite[ch. 5.24]{stiegler_age_2019}

\section{The power of the pill}

It may not be obvious what a nonideal theory approach to the ethics of
technology might look like.
As such, I will construct an example of how looking at a piece of technology can
help clarify an ideal-as-descriptive-model.
However, before that, it may be helpful to do some ground clearing on what a
nonideal theory is not.

Vallor gives us the bigger picture: \blockquote{While the most influential
  \nth{20]} century philosophers of technology each had a unique way of
  conceptualizing the ethical issues concerning technology, they \textit{all}
  tended to describe an ethics of technology as a response to a singular problem
  or phenomenon\ldots Notice also that each presents humans as somewhat passive
  subjects of the singular technological `problem', who must somehow reclaim
  their freedom with respect to technology. A new generation of thinkers has
  consciously moved away from essentialist conceptions of technology
  \textit{versus} humanity and toward the analysis of individual
  technolog\textit{ies} as features of specific human contexts; away from
  globalizing characterizations of the problem with technology to more neutral
  and localized descriptions of a diverse multitude of ethical issues raised by
  particular technologies and their uses; and finally, away from metaphysical
  concerns about human freedom and essence and toward more empirically-grounded
aspects of human-technology relations.}\footcite[p. 31]{vallor_technology_2018}
Unfortunately for this context, Vallor is interested in developing a common
moral framework from which philosophers and non-philosophers alike can have a
mutually intelligible discourse about technology rather than building a nonideal
theory about technology itself per se.
However, the empirical turn which she describes aligns with Mills criticism of
\textquote{reliance on idealization to the exclusion, or at least
marginalization, of the actual.}\footcite[p. 168]{mills_ideal_2005}
What is not described here is an approach where we treat technology as inert
objects upon which we perform some a utilitarian calculus to determine whether
it is good or bad.
Nor is it one where we do away with the ideal-as-normative and commit to a kind
of relativism or nihilism.
It is simply the methodological choice to recognize the nonideal and to allow to
inform our models so they are closer to the descriptive and, therefore, firmer
ground to stand upon.
Let us look at technologies as they are so that we can recognize how they
actually interact with human beings as they are.
If we are interested in, for example, then shouldn't we look at automation and
the diversity of experiences we have of it from VisiCalc to CNC machining to the
Jennifer unit?

To this end, I think it will be useful to look at the dynamics of the
contraceptive pill and the impacts of social access to it.
The idea that technology disrupts and changes the social is no surprise: it is
often, but not always, liberating.
Suffragist Susan B. Anthony once remarked on the bicycle, \textquote{I think it
  has done more to emancipate women than any one thing in the world. It gives
  her a feeling of self-reliance and independence the moment she takes her seat;
and away she goes, the picture of untrammelled womanhood.}
Similarly, it would not be unreasonable to claim that the pill was a
contributing factor to the second- and third-wave feminist movements in the
latter half of the \nth{20} century. The pharmacological effects are well known:
\textquote{In comparison with the pill, therefore, the diaphragm is 60 times
  more risky in perfect use and seven times typically, whereas the condom is 30
times more risky when perfectly used and four times in typical use}\footcite[p.
731]{goldin_power_2002} However, the significance of the pill extend beyond that:
it was safe, discrete, and---most importantly---put the decision and use in the 
hands of the woman instead of her bumbling partner.

Moreover, the social effects of the contraceptive pill can be observed
empirically.
While the pill was approved by the Food and Drug Administration in 1960, it was
only legally accessible by married women.
It was only in the late 60s and early 70s when state law changes regarding the
age of majority and `mature minor' decisions, primarily in response to the
Vietnam War, created legal access for unmarried woman.
Claudia Goldin and Lawrence F. Katz present an econometric analysis of
cross-state and cross-cohort variation in pill availability to young, unmarried
women and conclude as follows:
\blockquote{The pill directly lowered the costs of engaging in long-term career
  investments by giving women far greater certainty regarding the pregnancy
consequences of sex. In the absence of an almost infallible contraceptive
method, young women embarking on a lengthy professional education would have pay
the penalty of abstinence or cope with considerable uncertainty regarding
pregnancy. The pill had an indirect effect, as well, by reducing the marriage
market cost to women who delayed marriage to pursue a career. With the advent of
the pill, \textit{all} individuals could delay marriage and not pay as large a
penalty. The pill, encouraging the delay of marriage, created a
\textquote{thicker} marriage market for career women. Thus the pill may have
enabled more women to opt for careers by indirectly lowering the cost of career
investment.}\footcite[p. 731]{goldin_power_2002}

The magnitude of these effects is remarkable.
There was a sharp break in the number of applications to major degree programs
around 1970.
As Goldin and Katz observe, \textquote{Throughout the 1960s the ratio of women
  to all students was around 0.1 in medicine, 0.04 in law, 0.01 in dentistry,
  and 0.03 in business administration. By 1980 it was 0.3 in medicine, 0.36 in
law, 0.19 in dentistry, and 0.28 in business.}\footcite[p.
749]{goldin_power_2002}
Unsurprisingly, the sharp increase of women in professional degrees lead to a
sharp increase of women in those professions: \textquote{The percentage of all
  lawyers and judges who are women more than doubled in 1970s (from 5.1 percent
  in 1970 to 13.6 percent in 1980) and was 29.7 percent in 2000. The share of
  female physicians increased from 9.1 percent in 1970 to 14.1 percent in 1980
  and was 27.9 percent in 2000. Similar patterns are apparent for occupations
  such as dentists, architects, veterinarians, economists, and most of the
engineering fields.}\footcite[p. 749]{goldin_power_2002}
It is also worth mentioning that the trend holds true for philosophy as well
(although, philosophy was the second worse in the humanities
overall only doing better that religion) with \textquote{three times as many
women receiving philosophy Ph.Ds in 1985 as in 1965.}\footcite[p.
683]{rorty_special_1987}

Of course, there were many factors driving social change at the time.
As Goldin and Katz are careful to point out: \textquote{No great social movement
is caused by a single factor.}\footcite[p. 767]{goldin_power_2002}
However, alternative explanations don't happen at quite the right time for the
effects they observe.
Of particular note is the resurgence of feminism in the in America.
While this is undoubtedly also a contributing factor, Goldin and Katz note,
\textquote{Although the Civil Right Act of 1964 covered discrimination by
  \textquote{sex}, the Equal Employment Opportunity Commission (EEOG) set up to
  investigate charges did little about sex discrimination until the early 1970s.
  Affirmative action in the form of Executive Order 11246 was amended in 1967 by
  Executive Order 11375 to cover women, but it took many years to put into
  effect. Title IX, guaranteeing women equal access to federally funded colleges
  and universities, was passed in 1972, but its guidelines were not formulated
  until 1975. Antidiscrimination legislation had effects complementary to that
  of the pill, but its timing appears to have been a bit too late for it to have
been the spark.}\footcite[p. 765-6]{goldin_power_2002}

While the expansion of functions which at least some women were reasonably
capable of pursuing as a result, in part, of the pill, the sea change in
demographics presents us with an interesting case of the dynamics of disruption
and reorientation.\footnote{If an intuition pump is preferred over the empirical
  data, consider Japan where the pill was only legalized in 1999 and has seen
  comparatively little increase in the economic status of women since the
1970s.}
The introduction of women's voices in these professions---as peers---left an
indelible mark on the social world.
If culture is understood as a vehicle for the transmission of intergenerational
knowledge, then the introduction of knowledge of the nonideal realities of
women's experiences---knowledge that had always been there but was
obfuscated---was certainly disorientating for members of the prior existing
culture.
We can take the history of the term \textquote{sexual harassment} as a poignant
example because of the disconnect to the history of sexual harassment itself.
While the act of sexual harassment has always been around, it was only given
language 1973 in a report \textquote{Saturn's Rings} by Mary Rowe who was then
working at MIT.
Afterwards, momentum behind the movement to redress systematic blindness of the
problem built up rapidly---primarily by female activists in universities like
Cornell.
Legal activist Catherine MacKinnon's work \textit{Sexual Harassment of Working
Women: A case of Sex Discrimination}---inspired in part from the
awareness-raising in Cornell---is credited for creating the legal claim for
sexual harassment as a form of sex discrimination under Title VII of the Civil
Rights Act of 1964.
The term, however, remained largely unknown outside academia and legal circles
until Anita Hill's testimony in the Clarence Thomas Supreme Court Nomination.
The EEOC has registered the number of sexual harassment cases growing from 10
cases per year prior to 1986 to 624 the following year and 2,217 in 1990 and
then 4,626 by 1995.\footcite{cochran_sexual_2004}

That this rapid social change happened contemporaneously with the surge of women
in long-term degrees and professions is not mere coincidence.
We can see this by attending to the testimonies of these women.
The power of having the term \textquote{sexual harassment} cannot be overstated.
By giving language to it, it became a concept that could be communicated where
there was none before.
Accounts like Susan Brownmiller's describe the experience as revelation:
\blockquote{\textquote{Lin's students had been talking in her seminar about the unwanted
  sexual advances they'd encountered on their summer jobs.} Sauvigne relates.
  \textquote{And then Carmita Wood comes in and tells Lin \textit{her} story. We
    realized that to a person, everyone of us---the women on staff, Carmita, the
    students---had had an experience like this at some point, you know? And none
    of us had ever told anyone before. It was one of those \textit{click, aha!}
  moments, a profound revelation.}\footcite[p. 281]{brownmiller_our_1999}

It is significant that this takes place in a university classroom. Miranda
Fricker argues that what has occurred is a hermeneutical injustice.
There had been a lacuna in the shared hermeneutical resources---a gap in the
knowledge of the epistemic community---that had prevented these women from
being able to render their own experiences intelligible to themselves much less
others.
That this was an injustice arises from the fact that this hermeneutical lacuna
arose from the exclusion of women in participating in the epistemic community as
peers.
As she argues, \textquote{Women's position at the time of second wave feminism
  was still one of marked social powerlessness in relation to men; and
  specifically, the unequal relations of power prevented women from
  participating on equal terms with men in these practices by which collective
  social meanings are generated. Most obvious among such practices are those
  sustained by professions such as journalism, politics, academia, and law---it
  is no accident that Brownmiller's memoirs recounts so much pioneering feminist
  activity in and around these professional spheres and their institutions.
  Women's powerlessness meant that their social position was one of unequal
  hermeneutical participation, and something like this sort of inequality
provides the crucial background condition for hermeneutical
injustice.}\footcite[p. 132]{fricker_epistemic_2011}
The injustice here should not be taken lightly.
A fundamental part of human being is the capacity for knowledge---both in
possessing and professing.
In this case both had been undermined thereby undermining these women as humans
\testit{qua} knowers.
The supporting role the pill had in instantiating the conditions for this to
finally take place---despite centuries of intelligent minds working on
ideal-as-idealized-models which presumably should have just been easily
\textquote{applied}---demonstrates the richness of the kind of disruption
technology can bring that was obscured in Stiegler.

The issue might be reframed in terms of ideology---in the descriptive sense
Sally Haslanger gives as \textquote{an element in a social system that
  contributes to its survival and yes is susceptible to change through some form
of cognitive critique.}\footcite[p. 75]{haslanger_but_2007}
She includes practical knowledge and rituals like habitual gestures in this
conception of ideology.
Moreover, this ideology, with a small \textquote{i}, maps onto different social
groups with their different experiences, beliefs, and frameworks for
understanding what actions and events mean.
Taking this together, it seems Haslanger's sense of \textquote{ideology} has the 
same, or sufficiently proximate, referent as Stiegler's sense of
\textquote{culture}.
If that is the case, then Haslanger gives us an alternative theory of disruption
and reorientation.

Haslanger begins from an examination of a mother and daughter disagreement about
whether crop-tops are cute.
If we are not rudely dismissive, then we might notice that there is some
interesting features of social knowledge at play here.
It seems that the daughter is right when she asserts crop-tops are cute and
girls who wear track suits are dorks because the truth or falsity of the
assertion indexes to the social structures she is embedded in---or social
milieus.
The practices of these social structures as a \textquote{product of schema (a set
  of dispositions to respond in certain ways) and resources (a set of tools and
material goods)} is not subjective but real constraints of the social 
world.\footcite[p. 79]{haslanger_but_2007}
In this way they instantiate social meaning.
The daughter says something true insofar as she accurately describes the
practices of the social milieu she lives, in part, in.
\textquote{Plausibly, cuteness and dorkiness are features that must be judged
from the social milieus because they are partly constituted by those
milieus.}\footcite[p. 81]{haslanger_but_2007}
Conversely, the mother says something true when she asserts the opposite because
it is true indexed to her social milieu.
Both assert the truth and yet both contradict each other.
In light of this, Stiegler's intuition that cultural change disrupts knowledge
is better clarified.

However, if we take this as a clarification of Stiegler's disruption, then
Haslanger gives us insight into reorientation as well.
As she notes, \textquote{There is something tempting about the idea that we live
  in different social worlds (or milieus); that what's true is one social world
  is obscure form another; that some social worlds are better for its
  inhabitants than others; and that some social worlds are based on illusion and
distortion.}\footcite[p. 81]{haslanger_but_2007}
If we take this position then we have two ways to address disagreement between
social milieus: understanding and critique. These dimensions map onto how other
social structures may be more or less accessible or more or less in harmony to
our own respectively. As such, the genuine disagreement between the mother and
the daughter is a function of the accessibility and harmony of the milieus.

This suggests in turn the picture of reorientation through adjudicating social
truths across milieus. Haslanger rejects, tacitly, the kind of conception which
Stiegler seems gesture at: \textquote{A simplistic hypothesis might be that once
  one is exposed to a different social reality by engaging with assessors from
  another milieu, one will come to see the weaknesses of one's own milieu. On
  this view, the very exposure to another milieu, even in one's current
  (inadequate) milieu and provide opportunities for improvement. Critique,
  strictly speaking, is not necessary; one need only broaden the horizons of
  those in the grip of an unjust structure and they will gain
\textquote{consciousness} and gravitate to liberation.}\footcite[p.
81]{haslanger_but_2007}
She rejects this, however, on the grounds that it tends to smuggle in an
objectivist view about moral and epistemic value that is itself relative to
milieus.
If the claim is to privilege privileged milieus in determining whose horizons
need to be broadened and who already have \textquote{consciousness}, then there
needs to be an objective standard that is not milieu-relative.
This, however, demands a kind of ideal theory that our ideal-as-idealized-models
have never lived up to.

Instead, Haslanger proposes a notion of critique wherein \textquote{the
  assessor's claim is a genuine critique of a speaker's only if there is some
  common ground (factual, epistemic, or social) between the speaker's milieu and
  the assessor's milieus, and the assessor's claim is truth relative to the
  common ground. To say that a critique is genuine, in this sense, is not to say
  that it is the final word; rather, it is to say that a response is called
for.}\footcite[p. 87]{haslanger_but_2007}
This illuminates the frustration between the mother and daughter.
To simply give a flat denial to the other is, at least, incomplete or, at worst,
pointless.
Rather, they ought to engage in the dialogue necessary to find the common ground
they both accept (is accessible and harmonious to both starting points) from
which they can assess the truth-value of the claim indexed to the common ground.
As Haslanger asserts, \textquote{An advantage of this notion of critique is that
  it would help make sense of the idea that ideology critique is transformative.
  If critique isn't just a matter of reasoned disagreement, but is a matter of
  forming or finding a common milieu, then because a milieu is partly
  constituted by dispositions to experience and respond in keeping with the
  milieu, then possibilities for agency other than those scripted by the old
milieu become socially available.}\footcite[p. 87]{haslanger_but_2007}
It is precisely this element of transformation that is missing in Stiegler.
It is for this reason his ideal theory is drawn to a pessimism about technology.
It abstracts away the role of human interaction across social milieus in
creating an ideal-as-idealized-model of cultural evolution.

What should we make of the time barrier in light of this clarification of
disruption and reorientation?
It is true there are real concerns of the extreme reckless accelerationism
embraced by technological fetishists.
However, this is a concern distinct from that of the pace of technology
outrunning the pace of cultural evolution.
It is true that the kind of ideology critique described by Haslanger and the
redress of hermeneutical injustice described by Fricker teaches effort and time.
The task is monumental but necessary.
But why should we think that this means that there is a cultural evolutionary
limit to the rate of change?
Given the account above, there seems to be no necessary condition for such a
limit.\footnote{In addition listed here, the account implies that a breaking of
  the time barrier is impossible although the argument is outside the scope of
  this paper. The argument, roughly, is that there is good reason to think that
mere calculation, which is how Stiegler characterizes technology in the 
abstract, would not be able to cross that threshold since the issue is not a
question of the volume of new information but the subsumption of new knowledge
or understandings. Since this is always embodied in some human being---even if
it is not in the privileged social milieu---it will always be cultural. For it
to go beyond that, there must be generation of knowledge or understanding
outside of humanity. This suggests AI. However, Searle's Chinese Room and
Block's blockhead machine forcibly argue that programming---that is to say,
calculation---can never instantiate the features of strong AI---that is to say,
actual intelligence. As such, there seems to be no feasible technology on the
horizon that could break the time barrier as Stiegler describes.}
The only alternative is to suppose that there is a maximum limit on the human
capacity to extend moral and epistemic regard.
However, this seems to be a complete abdication of morality.
To give into the inclination to conserve and slow-walk in the face of sire moral
urgency can only stand on an ideal-as-idealized-model that chooses not to
recognize the nonideal realities finally given voice by technology.
It seems wrong to blame technology for unveiling when technology also represents
a vehicle by which the dialogue towards common ground can be had.
This recalls the call for unity  from those Alabama clergymen that 
\textquote{We recognize the natural impatience of people who feel their hopes 
  are slow in being realized. But we are convinced that these demonstrations are 
unwise and untimely.}\footcite{noauthor_white_1963}
We need only look to Martin Luther King Jr. about the myth of time:
\textquote{For years now I have heard the word \textquote{wait}. It rings in the
  ear of every Negro with a piercing familiarity. The \textquote{wait} has almost
always meant \textquote{never}.}\footcite{king_letter_2018}

\section{Care for the individual I see in front of me or for the individual I
idealize in my mind?}

We can finally turn to the question of care.
What should we make of Stiegler's ideal theorization of care where we discard
the mere \textquote{ethics} for a politics, idealized but undefined? 
I won't belabor the point: the structure of the critiques should be evident 
at this point.
It is significant that the ethos of care has emerged from a paradigmatic nonideal
theory.
For millennia philosophers constructed ideal-as-idealized-models around the value
of autonomy.
It was on this theoretical background that Lawrence Kohlberg constructed his
model of moral development.
It was in this research that Carol Gilligan observed the moral life of women was
being excluded and discounted by the ideal-as-idealized-models operationalized
in Kohlberg's theory.
From this history we have an ethics of care that recognizes the reality of
interdependence experienced by every human being.
It is unclear how we can have a conception of care that abstracts away the
relations between individuals which instantiate it, the particularities of
individuals that contextualize it, or the epistemic humility demanded by it.
Is it the ideal-as-idealized-model of human individuals, the essentialism and
abstraction that obscures the complex richness of an individual's life, that
allows Stiegler to exclaim:
\blockquote{I myself, no doubt, and without ever having thought of it prior to
  writing it here and now. I myself said to myself, in the early 1970s, in a bar
  where I made the night owls of Toulouse listen to and discover black music:

  I'm Malcolm X.

  I said this to myself by idealizing Malcolm X and identifying myself with him,
and, of course by deluding myself.}\footcite[ch. 6.35]{stiegler_age_2019

\nocite{searle_minds_1980}
\nocite{block_psychologism_1981}

\printbibliography

\end{document}
