\documentclass[letterpaper,notitlepage,12pt]{article}
\usepackage[
  letterpaper,tmargin=1in,bmargin=1in,lmargin=1in,rmargin=1in]
{geometry}
\usepackage[backend=biber]{biblatex-chicago}
\usepackage{csquotes}
\addbibresource{tech.bib}
\usepackage{setspace}
\usepackage[super]{nth}

\title{Nonideal Theories of Technology}
\author{Alexander Zhang}
\date{}

\begin{document}

\maketitle

\section{Philosophy for whose sake?}

The following is in part inspired by Shannon Vallor's project in
  \textit{Technology and the Virtues: A Philosophical Guide to a Future Worth
  Wanting}.
Drawing from ancient virtue traditions of Aristotle, Confucius, and Buddha, she
  attempts to motivate a new virtue tradition founded in the technological
  particularities of the modern context.
In doing so she strives to address a \textquote{need for a common framework in
  which these narratives can be situated if humans are going to be able to
  address these emerging problems of a \textit{collective} technosocial action
  wisely and well.}\footcite[p. 9]{vallor_technology_2018}
For the current discussion, however,  I am not engaged in evaluating the common
  framework that Vallor proposes.
Rather, the project here is to motivate the need which she acutely observes.
To put it another way, I want to attend to the subtitle of her project: What
does a philosophical guide to a future worth wanting in regards to technology
look like?

This represents a methodological issue about how we approach the ethics of
technology.
First, however, to make sense of what the methodology ought to be, we need to
be clear on why we need a philosophy of technology.
Insofar as there are anxieties about a future that negatively affects us all as
humanity collectively, then the answer appears to be clear.
But if we think that this calls for philosophy's contribution, then it implies
that we think philosophy can be useful.
However, if we believe that philosophy can make substantive, constructive
contributions, then there must be a fidelity to the concrete realities of the
problem rather than, as Vallor puts it, \textquote{the perils of essentialism,
overgeneralization, and abstraction.}\footcite[p. 33]{vallor_technology_2018}

The kinds of problems that we want an ethics of technology to address are not
purely academic.
They are instantiated in the real world affecting and affected by real people.
Insofar as we are interested in applying philosophy towards understanding
problems and finding solutions, then we are ultimately interested in applied
philosophy.
As such, a guide, like Vallor's, that attempts to make intelligible actual
dialogue across disciplines and the thresholds of ivory towers is necessary.

To be clear, this should not be construed as stipulating philosophy's role as an
apologist for technology.
It might be objected that this way lies pragmatism, relativism, nihilism,
or---even more disrespectfully---mere sophistry.
It might be claimed that in being useful, we never actually question the project
of technology but only tweak it, thereby allowing the producers of technology to
continue their work without questioning their fundamental assumptions.
Such claims are fundamentally misguided.
The mistake is in thinking that we can philosophize about
\textquote{Technology}, with a capital T, without examining the real world
instantiations and experiences of actual technologies.

This should not be trivialized.
This cross-disciplinary framework and dialogue is necessary in part because, as 
Vallor notes, any action would need to be collective cross-disciplinary action
in order to respond to the tangled web of globalized interdependencies.
More pertinent, however, it is necessary because we ought to recognize the scale
of the task set in front of us: \textquote{ethical discourse that speaks only to the concerns
  of a particular moral or philosophical `tribe' will be helpless to confront
  the ethical impact of the technosocial realities that increasingly address
  humans \textit{collectively}---realities which demand the effective
  cultivation and application of some measure of cooperative human
  wisdom.}\footcite[p. 23]{vallor_technology_2018}
Making sense of technology is a hard problem.
It represents an inquiry that spans the course of human history from the plow to
cryptocurrency and beyond.
It is bound up with the richness and diversity of all those human experiences.
It is fundamentally complex.

The claim here is not that we ought to be merely pragmatists and give up
questioning whether the foundational metaphysics of such projects are useful.
The claim here is that if we correctly appreciate the object of study for what
it is, then we would appreciate that the task demands literacies across subjects
including and beyond philosophy.
To make sense of technology, for any project, we need to understand
technologies.

This, hopefully, seems to be a trivial tautology.
But, as I will argue, we often fall short of this.
The issue arises out of a methodological choice by some philosophers to pursue
an ideal theory of technology resting on abstract, essentialist, or
overgeneralized conceptions of Technology rather than a nonideal approach that
begins with examining technologies and humans as they actually are: embodied in
the world.
To illustrate this point, I will take Stiegler's criticism of technology as
paradigmatic and contrast it with an example of a nonideal approach examination
of the contraceptive pill. To be explicit, the nonideal model is not presented
as contradiction to Stiegler's general critique but a demonstration of the
methodological points: ideal-as-descriptive-models of technology are
possible, ideal-as-descriptive-models of technology have better explanatory
and predictive power, and ideal-as-descriptive models of technology make more
meaningful the understanding of technology by understanding technologies.
Additionally, it should be clear that this account is not a defense of
technological optimism or accelerationism nor should it be taken to be
exhaustive as both are outside the scope of this paper.
Indeed, it may appear that there is substantive agreement between Stiegler's and
the nonideal approach's account.
The question here is not whether the conclusions arrive at the right answer but
whether the underlying methodology is unsound such that we fail to fully
understand the right reasons behind the conclusions.
That being said, I conclude by drawing out the downstream ramifications to
Stiegler's account if we concede, for they are concessions, that Stiegler can or
ought to accept and incorporate the nonideal approach's account; namely, the
time argument and the care proposal are fundamentally flawed.

\section{Exegesis on ideal and nonideal theory}

Central to this argument is the distinction of ideal and nonideal theory.
As such, we start with an exegesis on the framework as provided by Charles W.
Mills that traces back through feminists like Onora O'Neill and to Karl Marx.
It is a distinction on the methodological choice we make when we try to
construct theories about the world: in what direction do we proceed from, the 
theory to the world or the world to the theory?
Marx, in distinguishing his \textquote{Scientific} socialism from the
\textquote{Utopian} socialism of Fourier, Owen, and Saint-Simon, rejected the
notion that the \textquote{absolute truth, reason, and justice has only
to be discovered to conquer all the world by virtue of its own
power}\footcite{engels_socialism:_1978}
Rather, \textquote{In direct contrast to German philosophy which descends from
  heaven to earth, here we ascend from earth to heaven. That is to say, we do
  not set out from what men say, imagine, conceive, nor from men as narrated,
thought of, imagined, conceived, in order to arrive at men in the
flesh.}\footcite{marx_german_200}
In making sense of Mills more developed framework of this intuition, we will see
why we should be concerned with philosophies of technology that begin from the
idealized rather than the descriptive.

First, to motivate the concern regarding this methodological distinction, it's
worth noting the stakes.
For Mills, working in philosophy of race, the problem of ideal theory is
not merely academic.
As he argues, \textquote{that the so-called ideal
  theory more dominant in mainstream ethics is in crucial respects obfuscatory,
  and can indeed be thought of as in part \textit{ideological} in the pejorative
  sense of a set of group ideas that reflect, and contribute to perpetuating,
illicit group privilege.}\footcite[p. 166]{mills_ideal_2005}
While Mills conclusion is not central to the main thrust of the argument here,
it is salient.
Insofar as ethical theories, including ethics of technology, serve to justify
and permit real action, it matters that we get it right for the right reasons.
The unintended consequences may be significant.

To make sense of this, however, we have to disambiguate the senses of
\textquote{ideal}.
It may seem to be the case that any ethical theory entails some kind of ideal.
Mills target, however, is a particular sense.
Out of the gate, the intended referent is not the trivial sense of
ideal-as-normative in which any normative ethics that asserts an ought is
committed to.
Rather, Mills is interested in a particular way in how we create ideals of, or, 
rather, idealize, the world as models.
Again, it seems that any attempt to render the world more intelligible would
entail idealization in this sense.
However, he distinguishes between two different ways in which we---in ethics, 
philosophy in general, or natural and social sciences---create idealized models:
ideal-as-descriptive-model and ideal-as-idealized-model.
What Mills takes issue with specifically is ideal in the sense of the
ideal-as-idealized model rather than idealization per se.

First is the ideal-as-descriptive-model wherein we take a model to be
representative of some phenomenon P.
In this, we attempt to describe the phenomenon as it is observed in a 
way that is more readily intelligible.
However, \textquote{Since a model is not coincident with what it is modeling, of
  course, an ideal-as-descriptive-model necessarily has to abstract away from
certain features of P}\footcite[p. 166]{mills_ideal_2005}
The phenomenon in question may be too complex to be immediately grasped or we
may need arrive at some generalizable conclusions from specific
particularities.
For example, the results of a vaccine trial in of themselves tell us the 
effectiveness of the candidate specific to the sample population.
Moreover, the data collected may not be perfect due to external factors, e.g.
non-random attrition of participants, crossover, or compliance.
However, we might infer efficacy for the general population from the results 
on the assumption that the specific sample population is sufficiently
representative and the trial sufficiently powered.
We have to make decisions, consciously or otherwise, about which 
features of P are important and which will be omitted. 

This, then, is an opportunity to define rather than merely describe.
For certain P we can construct an exemplar, or schematized picture of what we
believe the actual nature and mechanism, of what an ideal P is like.
This idealized model is the ideal-as-idealized-model.\footcite[p. 167
]{mills_ideal_2005}
In this, we attempt to to model how P should work rather than how attempting to
accurately describe how P actually works.
There are all sorts of occasions when we refer to an ideal-as-idealized-model.
When we try to evaluate the quality of some functioning, like a body's health,
we refer to an idealized anatomy of what \textquote{true} functioning looks
like.
Alternatively, to minimize the cognitive work involved, we might make
simplifying assumptions that some idealized condition pertains like a perfect
vacuum, a frictionless plane, or a resistance-free conductor.

However, while we frequently make use of ideal-as-idealized models, we ought to
ask how useful will it be to use them as a starting point.
Mills gives the example of the frictionless plane.
Proceeding from the ideal-as-idealized-model of the frictionless plane might be 
unproblematic if the actual plane in question is coated in Teflon.
In that case, the actual P approximates the ideal P with only minor deviation.
But of course, if the actual P is covered in Velcro, this reasoning is absurd.
Mill notes, \textquote{Ideal-as-descriptive-model, the model of the actual
  workings of the plane, will be quite different from the
  ideal-as-idealized-model, and one will need to start with an actual
  investigation of the plane's properties. One cannot just conceptualize them
in terms of a minor deviation from the ideal, ideal-as-idealized
model.}\footcite[p. 167]{mills_ideal_2005}
The point is that the degree to which actual P resembles the ideal P is not
something that can be rationalized \textit{a priori}.

Of course, the answer to how useful an ideal-as-idealized-model is as a starting
point depends on the fact of how closely it resembles the actual.
Since we cannot know a priori the degree to which this is the case, it is
methodologically unsound to proceed on an ideal-as-idealized-model based on a
suppressed premise that it is sufficiently approximate to the
ideal-as-descriptive-model.
In the case that it is not, then our conclusions will be invalid and fail to
capture the actual phenomenon we are attempting to model.

To be clear, the way I will use the distinction is to pick out the nonideal,
ideal-as-descriptive-model, from the ideal theory, ideal-as-idealized-model.
It's plausible that there we can understand the models to be on a continuum
between idealized and descriptive.
However, the useful distinction to be made here is whether or not we proceed
from a model that is sufficiently close to the actual phenomena in question.
To go back to the vaccine example, we might try to infer human efficacy at
different stages of research: after conceptualizing the strategy to develop the
candidate, after testing the candidate in vitro, or after testing the candidate
in vivo on increasingly human analogous animals.
In each case, it seems that the model is formed from an increasingly better
empirical position.
However, the distinction of ideal and nonideal doesn't hinge on whether just any
empirical observation is used to inform the model.
Attempting to conclude human efficacy on the basis of any of these stages up
until randomized clinical trials in humans  would
still be operating from ideal-as-idealized-models because it is known, if not 
fully understood why, that success in a test tube or a chimp is not a sufficiently 
reliable indicator of success in actual human beings.
As such, a model idealized from these grounds, though empirical, do not suffice
as ideal-as-descriptive models of human efficacy because they do not account for
the quite possibly significant deviation of the actual phenomenon in 
question---efficacy in human beings---from the idealized model---theoretical
efficacy, efficacy in a test tube, or efficacy in an animal.

The significance of this to ethical theory should be obvious.
Proceeding on ideal-as-idealized-models is deeply problematic.
As O'Neill contends, this may involve attributing abilities to agents from
outside the human norm.\footcite[p. 56]{oneill_abstraction_1987}
Mills extends this argument by asserting that moral ideal theories will also
involve any of the following: an idealized social ontology, silence on
oppression, ideal social institutions, an idealized cognitive sphere, or strict
compliance.\footcite[p. 168-9]{mills_ideal_2005}
As Mills asks of us: \blockquote{Now look at
  this list, and try to see it with the eyes of somebody coming to formal
  academic ethical theory and political philosophy for the first time. Forget,
  in other words, all the articles and monographs and introductory texts you
  have read over the years that may have socialized you into thinking that this
  is how normative theory should be done. Perform an operation of Brechtian
  defamiliarization, estrangement, on your cognition. Wouldn't your spontaneous
  reaction be:\textit{How in God's name could anybody think that this is the
appropriate way to do ethics?}}\footcite[p. 169]{mills_ideal_2005}

We should be careful to note the methodological point being made here.
Of course, any outside critique of a theory will be similarly skeptical.
The underlying issue Mills points to, instead, is that \textquote{if one wants 
  to change the actual P so it
  conforms more closely in its behavior to the ideal P, one will need to work
  and theorize not merely with ideal, ideal-as-idealized-model, but with the
  nonideal, ideal-as-descriptive-model, so as to identify and understand the
  peculiar features that explain P's dynamics and present it from attaining
ideality.}\footcite[p. 167]{mills_ideal_2005}
To assert any kind of \textquote{ought}, a methodologically sound approach must 
work with both.
To put a finer point on it, the kind of critical response to an ideal theory
cannot be yet another ideal theory.

This reaction to Mills's question is not philosophically na\"{i}ve.
It correctly intuits that \textquote{In modeling humans, human capacities, human
  interaction, human institutions, and human society as an
  ideal-as-idealized-model, in never exploring how deeply different this is from
  an ideal-as-descriptive-model, we are are abstracting away from realities
  crucial to our comprehension of the actual workings of injustice in human
  interactions and social institutions, and thereby guaranteeing that the
ideal-as-idealized-model will never be achieved.}\footcite[p.
170]{mills_ideal_2005}
As he exclaims, \textquote{Why should anyone think that abstaining from
  theorizing about oppression and its consequences is the best way to bring
  about an end to oppression? Isn't this, on the face of it, just completely
implausible?}\footcite[p. 171]{mills_ideal_2005}
We can see in this that Mills primary concern is the way idealizing values and
norms are taken to be an appropriate methodology for ethical theorizing.
In this way, Mills accusation of ideal theory as ideology is clarified.
Given the actual realities of academic philosophy, demographics both past and
present, this will result in the nonrepresentative interests and perspective of
the over-represented.
To be clear, this should not be construed as a conscious conspiratorial
manipulation.
Rather, it is simply an unfortunate result of the confluence of social
privilege and the difficulty humans seem to have seeing outside of their own
standpoint prompted or unprompted.
The moral fault only comes about if we mistake that limited perspective for the
world as it is despite ample opportunities to recognize the evidence to the
contrary.

Some examples may help illustrate what Mills is asking of us and how we have
fallen short.
Mills takes as his main target for critique to be Rawls who justifies his
methodology with the claim: \textquote{The reason for beginning with ideal
  theory is that it provides, I believe, for the systematic grasp of these more
pressing problems\ldots At least, I shall assume that a deeper understanding can
be gained in no other way, and that the nature and aims of a perfectly just
society is the fundamental part of the theory of justice.}\footcite[p. 8]{
rawls_theory_1999}
We should note that Rawls does admit that \textquote{Obviously the problems of
  partial compliance theory are the pressing and urgent matters. These are the
things that we are faced with in everyday life.}\footcite[p.
8]{rawls_theory_1999}
In criticism of Rawls, Mills reiterates the feminist critique of Rawls's move to
postulate an ideal family and to not include gender in the original position.
However, I want to highlight that Mills overall argument for the necessity of
nonideal theory has two elements: the methodological argument that ideal theory
obfuscates relevant evidence and the historical argument that ideal theory has,
by and large, simply failed to incorporate the relevant evidence afterwards.
To do this, I will turn to two other examples that illuminate each of these
points.

The first example is the case of the idealization of autonomy.
The concept of autonomy, and it's more colloquial cousin, independence, has been
central to ethical theory for centuries.
Notably, autonomy lies at the heart of Kant's moral theory.
As Christine Korsgaard notes \textquote{The Kingdom of Ends provides us with a
way of representing the sense in which moral laws are laws of
autonomy.}\footcite[p. 23]{korsgaard_creating_1996}
But the supposition that human beings, as a result of their rationality, can be
defined ideally as autonomous is an ideal-as-idealized-model.

The claim here is not that ideal theories of autonomy are fundamentally mistaken
about the value of autonomy.
The claim is simply that while autonomy may be a significant moral value,
constructing an ideal-as-idealized-model from it such that it forms the very
basis of the conceptualization of moral life obscures other relevant features of
moral life.
As recent work in feminist theory as noted, every human being for some
significant part of their life are not autonomous: they are dependent on others.
We are not independent but interdependent for a significant period up until
adulthood and, if we survive, are likely to be as well in periods of
vulnerability such as sickness or old age.

The critique of nonideal theory is not (necessarily) that autonomy is a myth, 
but that in constructing an ideal-as-idealized-model on the basis of autonomy,
ideal theories exclude the moral life of care from consideration and,
consequently, fail to give us a holistic picture.
As Mills notes, \textquote{the point is that idealization here obfuscates the
  reality of care giving that makes any achievement of autonomy possible in the
  first place, and only through nonideal theory are we sensitized to the need to
balance this value against other values, and rethink it.}\footcite[p.
177]{mills_ideal_2005}
Insofar as these theories inform our understanding of moral life, which they do,
this leads us to similarly dismiss others and their testimony of their
informative experience.
Consider Carol Gilligan's recounting of Amy's testimony of moral life in
Lawrence Kohlberg's experiments in developing his theory of moral development:
\textquote{But as the interviewer conveys through the repetition of questions
  that the answers she gave were not heard or not right, Amy's confidence begins
to diminish, and her replies become more constrained and unsure.}\footcite[p.
29]{gilligan_different_1993}
Ask whether this experience is unique to Amy or if it's not heartbreakingly
common.

The second example responds to the obvious objection to Mills framework.
\blockquote{Suppose it is claimed that the
  foregoing accusations are unfair because, in the end, nonideal theory and its
  various prescriptions are somehow already \textquote{contained} within ideal
  theory. So there is no need for a separate enterprise of this kind---or if
  there is, it is just a matter of \textit{applying} principles, not of
\textit{theory} (applied ethics rather than ethical theory)---since the
appropriate recommendations can, with the suitable assumptions, all be derived
from ideal theory.}\footcite[p. 177]{mills_ideal_2005}
To put a finer point on it, we might question to what degree have we really
committed to just doing ideal theory or that ideal theory alone is sufficient.
It might be supposed that philosophers who do postulate ideal theories recognize
and concede that in applying the ideal, practical considerations will need to be
accounted for.
After all, Rawls, in stipulating his starting assumption of ideal theory
concedes that going from theory to applied will have to take into account the
nonideal.

We can grant that this is a position we might charitably ascribe to ideal
theorists.
The argument here is not that they are not well-intentioned or overtly ascribe
to the kind of oppressive ideologies Mills is concerned about.
However, the issue is whether or not this is sufficient to get them off the
moral hook.
But Mills response to Rawls's concession asks, \textquote{But then why, in the
  thirty-plus years up to his death, was he still at the beginning? Why was this
  promised shift of theoretical attention endlessly deferred, not just in his
own writing but in the vast majority of his followers?}\footcite[p.
179]{mills_ideal_2005}
Herein lies Mills historical rejoinder.

It might be supposed that this underestimates the difficulty constructing ideal
theory.
In fact, the reverse is true: ideal theorists underestimate the difficulty in
applying their constructions to the nonideal.
Consider Robert Nozick's libertarian account on the principle of
rectification: \blockquote{Some people steal from others, or defraud them, or
  enslave them, seizing their product and preventing them from living as they
  choose, or forcibly exclude others from competing in exchanges. None of these
  are permissible modes of transition from one situation to another. And some
  persons acquire holdings by means not sanctioned by the principle of justice
  on acquisition. The existence of past injustice (previous violations of the
  first two principles of justice in holdings) raises the third major topic
  under justice in holdings: the rectification of injustice in holdings\ldots
  I do not know of a thorough or theoretically sophisticated treatment of such
  issues. Idealizing greatly, let us suppose theoretical investigation will
  produce a principle of rectification. This principle uses historical
  information about previous situations and injustices done in them (as defined
  by the first two principles of justice and rights against interference), and
  information about the actual course of events that flowed from these
  injustices, until the present, and it yields a description (or descriptions)
  of holdings in the society. The principle of rectification presumably will
  make use of its best estimate of subjunctive information about what would have
  occurred (or at probability distribution over what might have occurred, using
  the expected value) if the injustice had not taken place. If the actual
  description of holdings turns out not to be one of the descriptions yielded by
the principle, then one of the descriptions yielded must be
realized.}\footcite[p. 152-53]{nozick_anarchy_2012}
The significance of this third principle in Nozick's libertarianism cannot be
underplayed: it is one the few, if only, cases where Nozick permits coercive
redistribution.
Moreover, like much of his theory, the implications of this principle are not
conservative but radical.

As Mills notes, \textquote{There could hardly be a greater and more clear-cut
  violation of property rights in U.S. history than Native American
expropriation and African slavery.}\footcite[p. 180]{mills_ideal_2005}
Given the controversy regarding the question of reparations today to either
group for past injustices, Nozick's theory seems particularly pertinent.
It's hard to imagine a more straight-forward case for the application of ideal 
theory to the nonideal.
Yes, Mills observes: \blockquote{But in the large literature of Nozick---not as
  large as Rawls, but substantive nonetheless---the matter of reparations for
  Native Americans and blacks has hardly ever been discussed. Whence this
  silence, considering that not even the mental effort of doing a Rawlsian
  race-behind-the-veil job is is required? Doesn't discussion of this issue
  \textquote{logically} follow from Nozick's own premises? And the answer is, of
  course, that logic radically underdetermines what actually gets thought about,
  researched, and written up in philosophy and books. White philosophers are not
  the population affected by these issues, so for the most part white
  philosophers have not been concerned about them. \textquote{Ideally} one would
  have expected that the pages of libertarian journals, and also mainstream
  journals, would have been ringing with debates on this matter. But of course
  they are not. Only recently, as a result of black activism, has the issue of
reparations becomes less than completely marginal
nationally.}\footcite[p.180]{mills_ideal_2005}

What we should appreciate is that the simple matter of applying ideal principles
to the nonideal is not so simple: \textquote{If it were as easy as all that, 
  just a matter of \textit{modus ponens} or some other simple logical rule, then 
  why was it so
  hard to do?\ldots The actual working of human cognitive processes, as
  manifested in the sexism and sometimes racism of such leading figures in the
  canon as Plato, Aristotle, Aquinas, Hobbes, Hume, Locke, Rousseau, Kant,
  Hegel, and the rest, itself constitutes the simplest illustration of the
mistakenness of such an analysis.}\footcite[p. 178]{mills_ideal_2005}
The objection to Mills's critique of ideal theory rests on a misapprehension of
the obstacles involved.
This requires stepping outside your standpoint, the perspective molded over the
course of a whole life, and into another's.
The difficulty is only exacerbated if those dissenting, who would give the
nonideal necessary to inform, are excluded, as has historically been the case.

In looking at the history of philosophy, ideal theory has failed to obviate the
need for a nonideal theory for all these millennia.
We can grant that philosophers are well-intentioned and that the recognize that
more will have to be said in the application of their idealized principles.
But the question remains: when will the work of accounting for the nonideal be
done?
Historically, it seems rarely if ever.
If concessions and promises are never followed up by action, they hardly rise
above the level of lip service.
Moreover, this implies the concession that ideal theories are evidentially
impoverished.
This presses the need for ideal theorists to satisfactorily account for the
history all the more.

Given the two clarifications made above the necessity for a nonideal approach is
sufficiently motivated.
Indeed, for the purposes of this of this paper, all that is required is the
recognition that we ought to pursue a nonideal theory given the issues raised
above.
It may be asked whether Mills critique requires us to reject an ideal theory out
of hand, to throw the baby out with the bathwater, or if it suffices to
introduce the nonideal into ideal theory and make the relevant adjustments.
However, this is outside the scope of this paper as the argument made below does
not hinge on the answer.
Plausibly, and my intuition, the response will depend on the theory in question
and how much deviation the ideal-as-idealized-model exhibits to the
ideal-as-descriptive-model.
In either case, for the purposes of this paper, it is enough to accept that,
\textquote{Insofar as concepts crystallize in part from experience,
  rather than being a priori, and insofar as capturing the perspective of
  subordination requires advertence to its reality, an ideal theory that ignores
these realities will necessarily be handicapped in principle.}\footcite[p.
177]{mills_ideal_2005}
While Mills is concerned primarily with justice and ideology, the framework
presented here is pertinent to the philosophy of technology.
Theorizing on \textquote{Technology} with a capital T, without due consideration
of actual experiences of technologies as they are embodied in the world and in
interaction with human beings, is an ideal-as-idealized-model.
A nonideal theory of technology examines technologies, in part, to ascertain the
deviation of the actual to the idealized model.

\section{Broken Clocks}

Given Mills's account is primarily concerned with how  ideal theories in 
political and moral philosophy can instantiate or reinforce injustice, it will
help to elaborate how Mills's methodological point can be applied to the
philosophy of technology.
In some ways, Vallor anticipates the critique made here: \blockquote{While the most influential
  \nth{20]} century philosophers of technology each had a unique way of
  conceptualizing the ethical issues concerning technology, they \textit{all}
  tended to describe an ethics of technology as a response to a singular problem
  or phenomenon, whether conceived as the human `enframing' of reality as a
  resource to be manipulated (Martin Heidegger); the oppressive mandate of
  technological efficiency (Jacques Ellul); the spread of `one-dimensional
  thought' enslaved to technopolitical interests (Herbert Marcuse); the
  relentless technological cycle of manufactured needs and desires (Jonas); or
  the suffocating rule of the `device paradigm' (Borgmann).
  Notice also that each presents humans as somewhat passive
  subjects of the singular technological `problem', who must somehow reclaim
  their freedom with respect to technology. A new generation of thinkers has
  consciously moved away from essentialist conceptions of technology
  \textit{versus} humanity and toward the analysis of individual
  technolog\textit{ies} as features of specific human contexts; away from
  globalizing characterizations of the problem with technology to more neutral
  and localized descriptions of a diverse multitude of ethical issues raised by
  particular technologies and their uses; and finally, away from metaphysical
  concerns about human freedom and essence and toward more empirically-grounded
aspects of human-technology relations.}\footcite[p. 31]{vallor_technology_2018}
Unfortunately for this context, vallor is interested in developing a common
moral framework from which philosophers and non-philosophers can have mutually
intelligible discourse about technology rather than building a nonideal theory
of technology itself per se.
However, the empirical turn which she describes aligns with Mills's criticism of
\textquote{reliance on idealization in the exclusion, or at least
marginalization, of the actual}\footcite[p. 168]{mills_ideal_2005}
As such, the following attempts to elaborate on why we ought to make the
methodological choice to recognized the nonideal and allow it to inform our
models so they are closer to the descriptive and, therefore, firmer ground to
stand upon.
That is to say, that to understand technology, we need to look at technologies.

To this end, the following section will take Stiegler's critique of modern
example as an example of ideal theory to be contrasted with an example of
nonideal approaches to understanding technology in later sections.
It may be objected that Stiegler's account is not vulnerable to the kind of
criticisms Vallor levies against \nth{20} century theorists like Heidegger's.
We might assert that Stiegler critiques the kind of ideal theories used by
technological optimists and accelerationists to justify unbridled development.
We might also assert that Stiegler does give reason room for human action given
he asserts that technology and humanity mutually constitute each other.
I raise these points to make clear what the present project is not.
We can grant these objections because they are beside the point made here.e
What is at issue is whether or not Stiegler's model itself is idealized with
appropriate regard to how the actual may deviate substantively from the model.
An ideal theory critique of an ideal theory is still vulnerable to the same
methodological concerns.

Summarizing Stiegler with brevity is difficult given his rhetorical flourishes.
He draws on a wide breadth of philosophical resources.
Moreover, he make connections from his theory to a wide ranging set of real
world events and phenomenon.
It seems, at a cursory glance, plausible that the kind of idealizations deployed
by Stiegler are not problematic in the ways suggested by Mills.
However, I will argue against this by breaking down a few salient points as
representative of Stiegler's methodology as a whole.

One of Stiegler's main concern is that the pace of technological innovation is
accelerating compared to the pace of cultural evolution such that it will break
the \textquote{time barrier}: \blockquote{It is as if time has leapt outside 
  itself: not only because the process of decision making and anticipation (in 
  the domain of what Heidegger refers to as "concern") has irresistibly moved 
  over to the side of the "machine" or technical complex, but because, in a 
  certain sense, and as Blanchot wrote recalling a title of Ernst J\"{u}nger, our 
  age is in the process of breaking the "time barrier." Following the analogy 
  with the breaking of the sound barrier, to break the time barrier would be to 
  go faster than time. A supersonic device, quicker than its own sound, provokes
  at the breaking of the barrier a violent sonic boom, a sound shock. What would 
  be the breaking of a time barrier if this meant going faster than time? What 
  shock would be provoked by a device going quicker than its "own time"? Such a 
  shock would in fact mean that speed is older than time. For either time, with 
  space, determines speed, and there could be no question of breaking the time 
  barrier in this sense, or else time, like space, is only thinkable in terms of
speed (which remains unthought).}\footcite[p. 15]{stiegler_technics_1998}
It might be claimed that the operative word here is \textquote{analogy}.
But it is clear that Stiegler takes this concern seriously as it is a recurring
issue throughout his work: \blockquote{Those who speculate on radical 
  innovation, its chaotic effects and the gains they expect to make thereby bet 
  against the social systems, just as in other times they might have bet on 
  being able to make a profit from famine or poverty. In this case, the bet 
  against the social systems consists in hastening the players (the speculators 
  themselves) as well as the pawns (us) headlong into the wall that is the 
  blocked horizon of the Anthropocene, that is, sending it to its limit, which 
  is approaching at high speed. 

  This blocked horizon is what I once referred to as the ‘wall of time’: this is
  the moment when a ‘shift’ holds sway, that is, a potential for highly chaotic 
bifurcations\ldots}\footcite[ch. 5.26]{stiegler_age_2019}
This is the conclusion that commentators have taken away from
Stiegler: \textquote{Moreover, Stiegler argues, because of the drive of
  permanent innovation in contemporary technology, we do not have time to build
our cultures around a stable from of technology. Technics outruns
ethnics.}\footcite[p. 148]{bishop_building_2020}
Evidently, this is not Stiegler speaking merely metaphorically.
We should take seriously the claim that the speed of technological innovation
will break the speed of time as set by the supposed limitations of human
cultural evolution.

To make sense of this assertion, we can first turn to \textit{Technics and Time}
where he provides the theoretical framework for his conclusion that
\textquote{There is today a conjunction between the question of technics and the
  question of time, one made evident by the speed of technological evolution, by
  the ruptures in temporalization (event-ization) that this evolution provokes,
and by the processes of deterritorialization accompanying it.}\footcite[p.
17]{stiegler_technics_1998}
This conjunction between technology and time is that the techno-logical is
constitutive of temporality.\footcite[p. 18]{stiegler_technics_1998}
To arrive at this, Stiegler's account is presented in two parts.
In the first part, he brings to doubt the conception of history such that
humanity is the subject and technology is merely the objects at the disposal of
humanity.
Rather, the two are mutually constitutive of each other.
In the second part, Stiegler argues, contra Heidegger, human experiences of
temporality is mediated through objects, artifacts, and, in general,
technics---in short, technology.\footnote{I will throughout use technology as
  the generic that can mean any of the above. It odes not appear that anything
  in particular hinges on the distinction besides the fact that the form of
  technology may be multiply realizable such that it also includes technical
practices.}

Towards this first end, Stiegler surveys a number of theories of technological
evolution---that of Bertrand Gille, Andr\'{e} Leroi-Gourhan, and Gilbert
Simondon---an thus connects them with the philosophical anthropology of
Jean-Jacques Rousseau and Leroi-Gourhan.
From Gille, Stiegler draws out the way in which technical systems relate to
social systems.
As he notes, technical systems are a \textquote{\textit{temporal unity}}, a
  stabilized point of technological evolution \textquote{\textit{around a point
  of equilibrium concretized by a particular technology}}\footcite[p.
  31]{stiegler_technics_1998}
Moreover, the technical system had clear connection connection to the social
system via the economic system.\footcite[p. 31--2]{stiegler_technics_1998}
Gille introduces an element of technological determinism by arguing that the
changes from one system to the next occur when the technical system reaches and
breaks the limit of the social system, thereby putting technical innovation as
the engine of both technological and cultural evolution.\footcite[p.
33--5]{stiegler_technics_1998}
Significantly, both Gille and Stiegler characterizes technical innovation and
its relation to the social system as essentially consumerist.\footcite[p.
31--2]{stiegler_technics_1998}
As a result of this, both hold that the conjunction of the technical and social
systems today has the essential quality of being aimed towards perpetual
innovation and, hence, structural instability in the social system.\footcite[p.
37--43]{stiegler_technics_1998}
This is, as Stiegler likes to put it, the techno-logic.

Stiegler draws out two key assertions from Leroi-Gourhan.
First, he notes Leroi-Gourhan's hypothesis of \textquote{\textit{universal,
  technical tendencies}, independent of the cultural localities that
  \textit{ethnic} groupings compose, in which they become concrete as
\textit{technical facts}.}\footcite[p. 43]{stiegler_technics_1998}
Stiegler takes this as evidence of a techno-logic inherent to technology and
extrinsic to humanity claiming that we can distinguish the
\textquote{\textit{technical}, whose essence lies in the universal tendency, and
  the \textit{ethnic}, whose manifestation as a particular concretization
envelops its universality.}\footcite[p. 44]{stiegler_technics_1998}
Notably, for the purposes of this paper, Stiegler notes that \textquote{This set
  of hypotheses on technical evolution depends on an analogy with biology and
zoology}\footcite[p. 44]{stiegler_technics_1998} and that \textquote{The 
investigation will proceed by analogy with the methods of zoology---the whole 
question being \textit{up to what point} the analogy holds.}\footcite[p.
46]{stiegler_technics_1998}
Despite this, Stiegler goes on to affirm: \blockquote{\textit{the tendency does 
    not simply derive from an organizing force---the human---it does not belong 
    to a forming intention that would precede the frequentation of matter, and 
    it does not come under the sway of some willful mastery: the tendency 
    operates, down through time, by selecting forms in a relation of the human 
    living being to the matter it organizes and by which it organizes itself, 
  where none of the terms of the relation hold the secret of the other.} This 
    technical phenomenon is the relation of the human to its milieu, and it is 
    in this sense that it must be apprehended zoo-logically, without its 
    elucidation being possible, for all that, in terms of the common laws of
  zoology.}\footcite[p. 49]{stiegler_technics_1998}

Secondly, Stiegler draws upon Leroi-Gourhan's suggestion that, originally,
geography determined the ethnic group where unity is formed around a shared
becoming rather than a shared past.\footcite[p. 55--8]{stiegler_technics_1998}
As a result, Stiegler characterizes the ethnic, or the cultural, as being
essentially related to time: \textquote{the relation to a collective future 
sketching in its effects the reality of a common becoming\ldots}\footcite[p.
55]{stiegler_technics_1998}
This is a relevant point in his later account in \textit{The Age of Disruption}
when he declares that \textquote{It is impossible to live in a society without
  positive collective protentions, but the latter are the outcome of
intergenerational and transgenerational transmission.}\footcite[ch.
2.7]{stiegler_age_2019}
Certainly, it is the case that for at least some this shared tradition and
future is a significant ground for how they conceptualize their own selves.
As such, it seems to be reasoned that this kind of culture must be a necessary
condition for authentic, or at least intelligible, being as an individual.
Notably, however, there is little to no consideration regarding whether this
experiences of ethnicity is broadly shared, particularly by disadvantaged
minorities.

A brief aside to foreshadow the later argument, it is plausible to me that this
is not the case and is worth a deeper evaluation. 
Notably, part of the minority
experience of their ethnicity---anecdotally---it is something imposed externally
on the basis of a perceived past that defines them as an Other and different.
This has been my experience as a first-generation American-born Chinese:
culturally domestic but categorized as foreign.
To Americans, I am Chinese and, to Chinese, I am American.
I am happy to grant that the kind of experience described by Stiegler is held,
and therefore, valued by people who are born into belonging to a culture or
ethnicity.
I am also happy to grant that Stiegler is concerned with how technology disrupts
experiences of belonging centered around a shared future, traditions of
American, or Chinese, or American and Chinese, or neither American or Chinese.
The concern here is not that Stiegler is creating an idealized tradition but
that he idealized from tradition such that it, wittingly or unwittingly, excludes or
minimizes, inappropriately, experiences and, therefore, values other than
tradition or other mechanisms besides technology by which this value is
disrupted.
If the claim is that we must have society that we are part of in order for there
to be a life worth living and that technological innovation disrupts it, then
this raises the kind of concerns of ideology in the pejorative sense which Mills
raised.
First, it elevates one particular kind of experience, potentially privileged,
as the exemplar by which a life worth living is defined with no regard to
alternative experiences.
The claim is not that belonging, in the sense of sharing a collective future, is
not valuable.
The claim is that we ought to be worried that, in saying it is a necessity, it
results in false negatives by discounting other experiences and values which can
instantiate a life worth living.
Second, it elevates one particular mechanism, technological innovation, as
exemplar for the phenomenon in question with no regard to alternative
mechanisms.
Note, the claim here is not that Stiegler holds his account of technological
disruption as the one and only way in which the phenomenon of disorientation can
be realized.
We can grant that it is logically possible in Stiegler's account that
nontechnological mechanisms can also instantiate Stiegler's concerns or that
Technological disruptions is necessary to disrupt unjust social structures.
But as Mills's notes, what is written in philosophy, and has been thought, is by
far and away a fraction of what is logically possible or ought to be considered.
Hence, the claim here is that, given the historical failure of ideal theory to
consider the nonideal, it is problematic for a theory like Stiegler's to choose
to focus on only certain aspects in particular without due regard to the
holistic picture.

The third theory of technological evolution which Stiegler draws from in
Simondon's account of \textquote{mechanology} or how the technical dynamic
imposes itself on the social dynamic.\footcite[p. 67]{stiegler_technics_1998}
This rests on the supposition that technical innovation aims at
\textquote{automatism} as perfection when it in fact represents the perfection
of an industrial object.\footcite[p. 66]{stiegler_technics_1998}
As a result of giving over intentionality to the automatism of the machine,
Stiegler concludes that this leads tho the techno-logic such that inventiveness
doesn't come from humans but instead \textquote{\textit{comes from the technical
  object itself}. It is \textit{in this sense}, resulting in the indetermination
  of the machine's functioning, and not under the category of autonomization,
  that one may refer to the \textit{autonomy} of the machine---the autonomy of
its \textit{genesis}.}\footcite[p. 68]{stiegler_technics_1998}
As a result, he concludes that this is the beginning of the breaking of the time
barrier: \textquote{With the machine, a discrepancy between technics and culture 
  begins because the human is no longer a \textquote{tool bearer.} For culture 
  and technics to be reconciled, the meaning of \textquote{the machine bearer of
  tools} must be thought, what it means for itself, and what that means for the 
place of the human.}\footcite[p. 69]{stiegler_technics_1998}

With these conceptual pieces, Stiegler is able to essentialize the techno-logic
as the characterization of a Technology with a capital T.
As he declares, \blockquote{There is here an actual techno-logical
  \textit{maieutic}. Certainly, what is invented, exhumed, brought to light, 
  brought into the world by the object exists in the laws of physics. But in
  physics they exist only as possibilities. When they are freed, they are no 
  longer possibilities but realities, irreversibly---pure possibilities that 
  have become effects which must from that point on be taken into account. They 
  become reality only through the technical object's potential of inventiveness, 
  in the process of concretization characterized by the fact that the human has 
  no longer the inventive role but that of an operator. If he or she keeps the 
  inventor's role, it is \textit{qua} an actor listening to cues from the object 
itself, reading from the text of matter.}\footcite[p.
75]{stiegler_technics_1998}
Here, Technology is described like a Platonic ideal.
It is knowledge that is already latent in the \textquote{genetics}, existing
outside of human conciousness until it is simply \textit{exhumed} into the
world.
As a result, the nonideal is trivialized: \blockquote{The maieutic proper to the 
  empiricism of what we are calling the \textit{experience} of the technical 
  object, which is its \textit{functioning}, corresponds here as well to a 
  \textit{selection of combinations}. Operating on a backdrop of chance,
the selection follows phyletic lines whose necessity is their horizon, dotted 
with mutations whose accidental effects become the new functional
principles}\footcite[p.76]{stiegler_technics_1998}
And so, knowingly or not, Stiegler decrees the irrelevance of a nonideal
methodology to his account.
Rather, he has epistemic confidence that he can access Technology with a capital
T and proceed primarily on an analysis and synthesis of abstract concepts.
It is hard to see how such a methodology can satisfy the condition of sufficient
attention to how the actual deviates from the ideal necessary for an
ideal-as-descriptive-model.

Stiegler goes on to examine two philosophical anthropologies to draw out the
claim that technics is time.\footcite[p. 83--5]{stiegler_technics_1998}
To do this, he contrasts the anthropological accounts of Rousseau and his
account of the nature of humanity prior to artifice to Leroi-Gourhan's
paleo-anthropological account that asserts the development of tool use has a
causal connection to the evolution of human cognitive capacities.
The relevant take away fro the rest of this exegesis is that technology relates
to time as externalized memory and, therefore, our experience to
temporality.\footcite[p. 172--4]{stiegler_technics_1998}
This externalization is the technical memory supports which transmit
epiphylogenetic history.
Without these artificial memory supports, there is no already-there, no
experience of the past, and, therefore, no relation to time.\footcite[p.
159]{stiegler_technics_1998}
For the sake of brevity, I will not go into detailed exegesis.
I will note that this may appear to be the kind of empirical investigation that
Mills's critique calls for.
However, Stiegler is more or less explicit about the assumptions at work in this
methodology here: that in making sense of a thing's origins, we directly gain
knowledge of the thing itself as a whole.
To this end, Stiegler references Plato's \textit{Meno} and \textit{Phaedrus}
along with, of course, Kant's transcendental knowledge all of which articulate a
methodology by which knowledge exists, possibly latently, prior to experience
and only needs to be remembered.\footcite[p. 95--100]{stiegler_technics_1998}
As such, Stiegler's anthropological account is not an inquiry in how the actual
deviates from the ideal but an act of remembrance---of exhumation of what is
already presumed to be there to be found.
The direction of this methodology constitutes an ideal-as-idealized-model.

We can now turn briefly to the second part of Stiegler's account it
\textit{Technics and Time}.
This is largely a consideration of Heidegger's philosophy in terms of the prior
part's conclusions.
Stiegler explains his methodology in this account: \blockquote{\textit{If} there 
  is a temporal arche-structure---constituted in the \textquote{already
  there} nature of the horizon of prostheses as anticipation of the end, in
the movement of exteriorization, on \textquote{collapsed zones of genetic
sequencing}---then on the basis of this collapse:

1. Nothing can be said of temporalization that does not relate to the
epiphylogenetic structure put in place each time, and each time in an
original way, by the already-there, in other words by the memory supports that 
organize successive epochs of humanity: that is, technics---the
supplement is elementary, or rather elementary supplementarity \textit{is} (the
relation to) time (diff\'{e}rance).

2. This kind of analysis presupposes an elucidation of the possibility
of anticipation (of the possibility of possibility). Such an elucidation is
the very object of existential analytic, which should accordingly be 
\textit{interpreted in terms of the question of prostheticity}.

The above step could not be taken by jumping abruptly from anthropology to 
phenomenology; the initial questions are quite different from this sort of 
approach. We will first read various occurrences of the myth
of Prometheus and Epimetheus, where an originary bond is presented
that is formed between prostheticity (Prometheus, god of technics); anticipation
(Prometheus, god of foresight); mortality (Prometheus, giver
to mortals of \textit{elpis}---both worry concerning the end and ignorance of the
end); forgetting (the fault of Epimetheus); and reflexivity, or the
\textquote{comprehension of being,} as delay and deferred reaction
(\textit{\={e}pim\={e}theia}, or knowledge that arises from the accumulation of 
experience through the mediation of past faults). From the perspective of this
myth, \textquote{exteriorization} will immediately call forth socialization, 
considered as the relation to death or as anticipation.}\footcite[p.
183--4]{stiegler_technics_1998}
Again, in Stiegler's own words, the significance of this methodological move:
\blockquote{The following reading rests on a confrontation between the 
  Heideggerian existential analytic and the myths of Prometheus and of
Epimetheus in their most known versions (Hesiod, Aeschylus, Plato). In
classical Greek culture a mythology of the origin of technics is to be
found which is also a mythology of the origin of mortality, a thanatology, the 
absence of analysis of which in Heidegger's work strikes us as
highly revealing.}\footcite[p. 16]{stiegler_technics_1998}
While later on Stiegler considers particular examples of technology such as
clocks, live broadcast television, and writing, it is largely as reinforcement
of his critique of his critique of Heidegger that emerges out of his ruminations
on myth.
After this, he returns to speaking about technology, more specifically tools and
tool use, in the abstract divorced from any explanatory process of the actual.
This, again, represents a methodological choice to deploy an
ideal-as-idealized-model.

The issue here isn't that he turns to myth or that there can be no edification
to be found there.
We must admit that philosophy, as an argument, is partly rhetorical and such a
turn can be a successful rhetorical choice.
The question at issue is not whether or not it is persuasive or edifying, but
whether it is a valid methodological choice because it is an inference that
rests on certain premises.
One of those suppressed premises is that myth is sufficiently like the actual
world that we can deduce, in a sufficiently epistemically robust way,
nontrivial knowledge about the characteristics of actual things outside of myth.
However, this premise is unmotivated: Stiegler does not attempt demonstration or
persuasion that the premise should be accepted.
And yet, he uses it to derive significant features of his account.
Stiegler concludes that tradition is intrinsically linked through an
etymological methodology: Epimetheus is related to \textit{mathesis}, a kind of
presuppostional knowledge of things in Heidegger's account, through
\textit{metheia} to \textit{methano} while \textit{epi} means accidental or
artificial facticity thereby leading to \textit{epimethia} as the accumulation
of knowledge marked by accident such as heritage or tradition.\footcite[p.
206--7]{stiegler_technics_1998}
This link between technology and transgenerational knowledge retention is
crucial to Stiegler's account.
It is unclear how such a methodology can be considered sufficiently attentive to
the possible deviation of the actual to the ideal unless we suppose an extreme
kind of nominative determinism or we are biased because of our priors.

To return to the breaking of the time barrier argument, Stiegler draws the
connection between technology and time in \textit{Technics and Time} but to draw
out the further claim that technological innovation is accelerating such that it
will outrun the cultural evolutionary limits, we have to turn to his later
works.
In \textit{The Age of Disruption}, he gives a clearer picture on this argument.
The first is a kind of intuition pump: that we can see evidence that
technological innovation is disrupting the formation of culture because we live
in an epoch without \textit{epokh\={e}}. Stiegler argues that the sped of
digital information transfer is such that it becomes \textquote{\textit{the 
global cause of a colossal social disintegration}} such that \textquote{this
\textit{automatic nihilism} sterilizes and destroys local culture and social
life like a neutron bomb\ldots}\footcite[ch. 1.4]{stiegler_age_2019}
As a result, \textquote{everything that for individuals forms the horizon of
  \textit{their} future, constituted by \textit{their} protentions, is
  outstripped, overtaken and progressively replaced by \textit{automatic}
  protentions that are produced by intensive computing systems operating between
  one and four million times quicker than the nervous systems of psychic
individuals.}\footcite[ch. 1.5]{stiegler_age_2019}
This, evidently, is the time shock that occurs when the speed of technological
innovation breaks the limit of cultural evolution.
Stiegler describes this state as an absence of epoch or an epoch without
epok\={e}.\footcite[ch. 2.7]{stiegler_age_2019}
His sense of epok\={e} is \blockquote{such that it lies at the origin of a
  \textit{conversion of the gaze} of a \textit{change in the way of thinking},
  and, through that, of a transformation of what Heidegger called
  \textquote{the understanding that there-being (\textit{Dasein}) has of its
  being} (which, as we will see, consists in the individual and collective
  production of \textquote{circuits of transindividuation})---this philosophical
  and more generally noetic \textit{epokh\={e} (produced by a new form of
    thinking in general) is \textit{always the outcome of a techno-logical
    upheaval}, itself derived from what Bertrand Gille described as a change in
  the technical system}\footcite[ch. 2.7]{stiegler_age_2019}
  A new epoch forms around a new epoke\={e}, or reorientation to technological
  disruption, such that there are new ways of thinking, new ways of
    doing and new ways of living take shape, which are \textquote{new forms of
    life} in Georges Conguilhem's sense on the basis of precursors
    \textit{reconfiguring the retentions inherited from the earlier epoch into
  so many new kinds of protentions}}\footcite[ch. 2.8]{stiegler_age_2019}
While the digital age can have arguably introduced an upheaval, Stiegler holds
that it has undermined and interrupted the process of cultural reorientation
such that the epokh\={e} does not occur; hence, epoch without epokh\={e}.
We have, then, a better understanding of what Stiegler thinks is at stake in
breaking the time barrier.

Stiegler attempts to pump our intuition that this is possible by suggesting that
it is already the case.
To do so, he provides as example the statement of a fifteen year-old named
Florian: \blockquote{You really take no account of what happens to us. When I
  talk to young people of my generation, those within two or three years of my
  own age, they all say the same thing: we can no longer have the dream of
  starting a family, of having children, or a trade, or ideals, as you
  yourselves did when you were teenagers. All that is over and done with,
  because we're sure that we will be the last generation, or one of the last,
before the end.}\footcite[ch. 1.5]{stiegler_age_2019}
This, then, is evidence of the absence of epoch, the failure for cultures to
reorient and reconfigure around a collective future. \textquote{In the horizon
  of \textit{becoming}, Florian sees no possible \textit{future} for his
  generation---which is also to say, for the human species. He formulates in
  clear, simple and terrifying terms what everyone thinks, but which
  repress---except for a few who hurtle into the Twin Towers by plane, or into
  mountains, or into Christmas markets, or through the window of a police
station after having killed or injured twenty-seven people (we should also
mention Columbine, Breivik, and many others---and it will be necessary to
discuss the Kouachi brothers)}\footcite[ch. 2.6]{stiegler_age_2019}
To put an even finer point on it, Stiegler exclaims \textquote{ we could say
  that for Florian, no positive collective protention is possible: there is no
  protention other than \textit{the end of all protention}, this is, \textit{the
end of all dreams and any possibility of realizing them}.}\footcite[ch.
2.6]{stiegler_age_2019}

Let's put aside Stiegler's inconsistency in dating the point in which the rate
of technological innovation outstrips the rate of cultural evolution (here, he
suggests it has already happened but elsewhere he talks of it as a future
possibility). It may seem that in drawing from Florian's testimony, Stiegler
presents an ideal-as-descriptive-model.
However, the relevant condition for a nonideal theory is whether there is
sufficient and appropriate regard to how the actual may deviate from the ideal.
While deploying evidence is a step in the right direction, we should ask whether
the strength of the evidence is sufficient to ground the inference.
Here, Stiegler takes Florian's testimony to be representative and sufficient to
speak for \textquote{\textit{an entire generation}}\footcite[ch.
1.5]{stiegler_age_2019}
To generalize from a sample size of one (or possibly a dozen or so if we include
Stiegler's psychologizing of terrorists) to however many millions, we are not
forming our credences around the strength or preponderance of the evidence but
around the confidence of our priors.
In fact, this suggests an ideal-as-idealized-model that only glances at the
empirical to confirm what was already postulated.

Consider the other possible methodological choices Stiegler did not choose to
pursue.
There is an abundance of opinion polling attempting to characterize and classify
each generation.
We are arguably far to obsessive in our preoccupation with generational
taxonomies.
Regardless, if we can, as Stiegler does, move from individual testimony to
generalization of a generation, then there is a wealth of such testimonies
already collected to consider.
We should wonder why Stiegler does not avail himself of the available evidence.
This is particularly the case here where the inference that Florian says what we
\textit{all} think is obviously contentious in the way all universals are.
In the Spring 2021 Harvard Youth Poll, which surveyed 2,513 young Americans,
56\% responded \textquote{hopeful} to \textquote{At this moment, would you say
that you are more hopeful or fearful about the future of
America?}\footcite{noauthor_spring_nodate}
In 2002, the  Ipsos Group replicated a pre-9/11 survey attempting to measure
hope.
As they note, \textquote{The most significant finding of the results is that
  despite a traumatic year for Americans, comparison made with the survey
  results conducted pre-9/11 using the same questions and methodology show about
almost identical findings in a post-9/11
environment.}\footcite{noauthor_measuring_2002}
Results of the replicated survey suggested that 61\% of Americans had the
highest hopes for the future as compared to 45\% of Australians, 43\% of
Canadians, 42\% of Britons, and---interestingly---29\% of
French.\footcite{noauthor_measuring_2002}
Given these statistics, it is unclear that we can safely assume that Florian's
account is sufficiently representative even if it lines up with our intuitions
or sympathies.

Another thing worth noting about how Stiegler elaborates on his account here is
his identification of the technology of concern: information technology.
Admittedly, he does no phrase it in this way.
However, the salient distinction between different aspects of information
technology is often lost in Stiegler's account.
As he attempts tho describe the role such technologies play in
\textquote{reticulated society}: \textquote{As such, it contains unprecedented
  power of automation and computation: it is literally \textit{faster than
  lightning}---digital information circulates on fibre-optic cables at up to
  two thirds of light speed, quicker than Zeus's lightning bolt, which travels
at only 100 million metres per second (one third of the sped of
light)}\footcite[ch. 1.4]{stiegler_age_2019}
The comparison to Zeus's lightning bolt is a call back to the end of
\textit{Technics and Time}.
This example of the fiber-optic has, at best, indirect connection to the speed
of automation and computation.
Those seem to be questions of processing speed rather that the speed of data
transfer over networks.

This, however, makes clear that computation or calculation is the essential
feature of Stiegler's conception of Technology with a capital T.
He remarks in \textit{Technics and Time} that \textquote{modern technics
  constitutes the \textit{Gestell} of nature and humanity through
calculation.}\footcite[p. 19]{stiegler_technics_1998}
For one thing, this sets up Stiegler's account as a direct response to
technological optimists and accelerationists by adopting their assumptions about
the exponential growth of computing technology.
There is, after all, little else that can be postulated to give the kind of
accelerating returns that Stiegler needs for his account---as a rule of thumb,
exponential growth is unsustainable and what behaviour we actually see is
logarithmic or, in other words, diminishing marginal returns.
While the \nth{20} century saw an explosion of new technologies and social
integrations, it could be argued that the golden age of new invention is over.
Much of the early work on the space race was done by people with the job title
of \textquote{computer}.
Conversely, while \nth{21} century and the \textquote{information age} is
characterized by the development of computing power---faster processors, denser
memory, and faster networks---and new software.
In other words, albeit a gross simplification, iteration rather than invention.
Despite this, it is this development that has been attributed with exponential
growth.
Famously, Gordon Moore, co-founder of semiconductor chip manufacturer Intel,
posited Moore's Law---the observation and projection of historical trends in
1965 that the number of components per integrated circuit would double every
year.
This was revised in 1975 to a rate of doubling in processor chip density---and,
hence, performance---to every two years.
This \textquote{law} is frequently cited by Stiegler's targets and, while the
structure of immanent critique is not lost on me, he seems to genuinely accept
this premise as well.

I will not go into depth on this point since more will be said on Stiegler's
abstraction of technology as computation later, but this represents an ideal
theory critique of another ideal theory.
Moore's law is dead or dying as we are already manufacturing semiconductor chips
with components measurable in double digits of silicon atoms and rapidly
approaching physical limitations.
Die shrink processes have slowed below the rate postulated by Moore as well:
Intel's CPU development roadmap from 2007 to 2016 followed a tick-tock model
with a die shrink every two years, as projected by Moore's law. However, their
fabrication was stuck on the 14nm process from 2015 to 2018 leading to the
adoption of a three-year cycle roadmap.
Moore's law may have been, originally, a descriptive model at the time of
coinage, but its use without regard to how the actual has since deviated from
the model is an ideal-as-idealized-model.
Insofar as either apologists and critics rely on this ideal-as-idealized-model,
both are vulnerable to the same methodological issues already noted.

For now, let's suppose for the sake of argument that the rate of technological
innovation will continue to accelerate.
We can now turn to the question of how technology disrupts cultural
reorientation such that we end up in an epoch without epokh\={e}.
Stiegler identifies the digital technical system as the culprit such that it is
\textquote{disruptive precisely in that \textit{it gives absolutely no place to
  the second moment}, nor therefore to \textit{any thinking}: it gives rise only
  to an \textit{absolute emptiness of thought}, to a knosis so radical that
Hegel himself would not have been able to anticipate it.}\footcite[ch.
2.8]{stiegler_age_2019}
To make sense of this conclusion, we have to turn to Stiegler's account of
transindividuation---how \textit{I}'s create a \textit{We}---and the
significance of retentions---or how transgenerational knowledge is communicated
and retained.
This is to say, we need to turn to his picture of social or cultural evolution.

For Stiegler, at the heart of cultural evolution is the process by which
individuals come together to form communities.
He terms this process \textquote{transindividuation} because he believes that
the process of coming to a sense of self is also coming to a sense of others
because: \blockquote{during a transindividuation process, the co-individuation
  of many individuals tends to converge, beyond the differences of perspective
  between each individual, towards an attractor around which a metastable state
  of shared significations forms. These shared significations become the
  supports of interpretation, that is, of the production of meaning---one and
  the same signification, implicitly or explicitly accepted, can become in its
  play with other significations the vector of a new meaning, which is also to
say, a transindividuation process for one or more new
significations.}\footcite[p. 67]{stiegler_what_2013}
However, for this to occur, there must be what Stiegler calls metastable objects
and practices.
It is this attribute of metastability that technology becomes involved:
\blockquote{The \textit{adoption} of a technique or technology by a society, and 
  by an \textit{epoch} of that society, is a phase in a process of collective 
  individuation, which occurs between psychic individuals, and through the 
  mediation of this technique or technology, that is, as individuation that is 
  collective as much as it is psychic, thereby metastabilizing a stage of a
  transindividuation process. And these are also the conditions of the 
  individuation of the technical system itself, which transforms and 
metastabilizes itself through this adoption.}\footcite[p.
101]{stiegler_what_2013}
However, this adoption implies the agency and intentionality of the agent or
agents doing the adopting.
This adoption draws upon Gille and Leroi-Gourhan's models of technological
evolution.
Here technology and culture are mutually constituting.
It is the sense in which a community can stabilize around a technical system and
come to form a shared future around it.
This is contrasted with adaptation wherein the technical system strips away the
intention and agency of individuals by forcing them to merely conform rather
than adopt.
As such, \textquote{This adaptive transindividuation process is formed through
short-circuits---whereas adoptive transindividuation forms long circuits. 
Short-circuits can only result in accidents\ldots}\footcite[p.
102]{stiegler_what_2013}
The distinction, in part, between adoption and adaption lies in the aftermath of
disorientation---the creation of long circuits or short-circuits.

This picture of long circuits or short-circuits, then, ties back into the
question of epokh\={e}: \textquote{An epoch is always a specific configuration
  of the libidinal economy, organized around the ensemble of tertiary retentions
  (that is, around the technical supports of collective retention) that form,
  through their arrangement, a new technical system, which is always also a
retentional system.}\footcite[ch. 2.8]{stiegler_age_2019}
Whereas primary memory is the originary presentation of an experience and
secondary memory is the re-presentation of the prior, tertiary memory is the
externalization of memory in the technical memory supports; it is the
epiphylogenetic history transmitted by technology
transgenerationally.\footcite[p. 248]{stiegler_technics_1998}
This is significant because Stiegler holds that \blockquote{in intergenerational
  processes of transmission, tertiary retentions radically condition the
  relationships between psychic individuals, and, through them, between
  collective individuals---between the mother and the \textit{infans}, between
  the child and his or her siblings as well as other children, between the
  adolescent and the social milieu, between adults, between adults and new
  generations, and hence between generations, and through generations, between
social groups, and so on.}\footcite[ch. 2.8]{stiegler_age_2019}
As a result, Stiegler holds that \textquote{there are collective protentions
only to the extent that there are collective retentions. The latter constitute
forms of knowledge.}\footcite[ch. 2.9]{stiegler_age_2019}
Hence, the accusation that the digital prevents any thinking.

We can end on three passages about how Stiegler describes the digital creates an
absence of epoch.
\blockquote{In the contemporary epoch  of the absence of epoch, the role of
  digital tertiary retentions in the intergenerational (non)relationship, and in
  the (non)formation of collective retentions and protentions, is both
  \textit{perfectly obvious and totally escapes} comprehension---because there
  is no longer any adjustment between the new technical system and the social
  system. Far from adjusting the social systems by reshaping them to suit a
  \textquote{new epoch}, the technical system short-circuits them and,
ultimately, \textit{destroys them}.}\footcite[ch. 2.8]{stiegler_age_2019}
Her goes on later to elaborate on the perfectly obvious which totally escapes
comprehension:
\blockquote{Dasein's psychic retentions are made possible by tertiary retentions
  that are collective thanks to the very fact they are extemporize and
  spatialized. Dasein is thus able to share, with other psychic individuals,
  collective tertiary retentions that it apprehends as its \textit{own}
  retentions, and which belong to \textit{the same epoch} (and to the same
  \textquote{culture}) as those with whom this Dasein \textit{shares} these
  retentions. From this is follows, too, that individuals of the same epoch and
  the same culture have, if not the same expectations, at least a \textit{common
  horizon of the convergence of their expectations}, forming \textit{at
infinity} the common protention of a \textit{common future}---the undetermined
unity of a horizon of expectation---which is also ultimately the future of
humankind, that is, of noesis as worthy of being lived in a non-inhuman way.

We have seen, then, that such sharing constitutes the background or the
\textit{funds} of an epoch\ldots \textit{Digital tertiary retention, however,
  which constitutes the digital technical system, is disruptive because it takes
control of this sharing}. This is what I have called, in pursuing the
reflections of Gilles Deleuze and Felix Guettari, society of
hyper-control.}\footcite[ch. 2.1]{stiegler_age_2019}
However, Stiegler expands on this final point:
\blockquote{Digital retention may indeed bring with it new and unprecedented
  opportunities because it de-massifies the production of traces. Nevertheless,
  the disruption systematically explored and exploited by the new reticulation
  industry has in fact created a new, subtler stage of massification---that is,
  of the absence of epoch, giving rise to a new kind of barbarism, and doing so
  by creating a point of rupture, a breaking point.

  What is massified today is no longer the criteriology by which primary
  retentions are selected, which was achieved by standardizing secondary
  retentions: it is the formation of circuits between secondary retentions via
  intensive computing, capable of treating gigabytes of data simultaneously, so
  as to extract statistical and entropic patterns that short-circuit all genuine
  circuits of transindividuation---where the later would always be negentropic,
that is, singular, and as such incalculable: intractable.}\footcite[ch.
3.15]{stiegler_age_2019}
As such, we come back to the crux of the problem for Stiegler: computation.
This seems obvious and yet totally escapes comprehension: How is this an example
of technology disrupting culture, the fault of technology?
Throughout, Stiegler makes reference to computational capitalism.
We should wonder if the descriptive problem he sets out matches the idealized
model he presented.
He admits the possibility that digital retentions can instantiate unprecedented
protentional opportunities.
He identifies the relevant cause of disruption is not an intrinsic or essential
quality of digital technology, but the way in which the technical system is
deployed by societies of hyper-control.
If this is the case, it is puzzling why it is the Technology, computation, that
is short-circuiting the long circuits instead of the social situation,
capitalism, which elects to use technology to create short-circuits.
This, rather than technology, appears to be a case where the socioeconomic
structure is undermining the new opportunities for protention,
transindividuation, which Stiegler admits.

For not, this exegesis of Stiegler comes to a close on the hope that there is
some indication that what he offers is an ideal theory---an
ideal-as-idealized-model.
Indeed, This final puzzle may be a result of that methodological choice.
To draw out this claim, I will present as demonstration an alternative nonideal
approach at trying to make sense of how technology itself relates to the
transgenerational transmission of knowledge and, therefore, transindividuation.
Before then, a brief rejoinder to a possible reply: although Stiegler identifies
societies of hyper-control and their use of computation as the culprit in
short-circuiting transindividuation, we can still blame technology rather than
the social or cultural because it is that technology that enables the
possibility of socioeconomic culture of technocapitalism.
For now, this concedes that we have lost sight of the breaking of the time
barrier argument because it implies that the cultural evolution can keep pace,
albeit in an undesirable fashion.
However, a better reply can be given.
We should ponder Marx's methodological choice to turn to \textquote{scientific}
rather than \textquote{utopian} socialism.
It is notable that the Silesian Weavers' Uprising of 184 was one of the major
inspirations for Marx's break from the Young Hegelians.
In this incident, several thousand weavers in Silesia, still by and large feudal
with the dying remnants of the guild system, protested the introduction of the
autoloom.
They smashed the newly installed machines.
The Prussian government called in troops and brutally suppressed them.
Despite Marx's attraction and sympathy for this incident, his attention to
alienation, or his detailing in \textit{Das Kapital} of the struggle between
worker and machines as used by capitalism, we should wonder why he deliberately
chooses to call for workers to \textit{seize} the means of production rather
than to \textit{smash} them.
I suggest that in making the methodological choice in pursuing a nonideal theory
rather than an ideal one, he can appropriately recognize the distinction between
technology and the socioeconomic use of technology because they are mutually
constituting in both directions at the same time in different but significant
ways.

\section{The power of the pill}

The following section attempts to sketch out at least one way technology
interacts with transgenerational knowledge transmission by taking as a case
study the contraceptive pill as a piece of technology and the socioeconomic
dynamics involved. This account is not presented as an exhaustive account of
technology and culture.
Indeed, at the end, I argue that philosophy of technology requires a
comprehensive survey of technologies to fully appreciate the complex and
unpredictable dynamics involved. Nor should the following account be construed
as wholeheartedly contradictory to Stiegler: it is largely sympathetic although
there are some contradictory ramifications.
Rather, the following is an attempts to demonstrate that an
ideal-as-descriptive-model account of congruent concepts picked out by ideal
theory is possible, that ideal-as-descriptive-models of technology have better
explanatory and predictive power, and ideal-as-descriptive-models of technology
make more meaningful the understanding of technology by understanding
technologies.

The idea that technology disrupts and changes the social is no surprise: it is
often, but not always, liberating.
Suffragist Susan B. Anthony once remarked on the bicycle, \textquote{I think it
  has done more to emancipate women than any one thing in the world. It gives
  her a feeling of self-reliance and independence the moment she takes her seat;
and away she goes, the picture of untrammeled womanhood.}
Similarly, it would not be unreasonable to claim that the pill was a
contributing factor to the second- and third-wave feminist movements in the
latter half of the \nth{20} century. The pharmacological effects are well known:
\textquote{In comparison with the pill, therefore, the diaphragm is 60 times
  more risky in perfect use and seven times typically, whereas the condom is 30
times more risky when perfectly used and four times in typical use}\footcite[p.
731]{goldin_power_2002} However, the significance of the pill extend beyond that:
it was safe, discrete, and---most importantly---put the decision and use in the 
hands of the woman instead of her bumbling partner.

Moreover, the social effects of the contraceptive pill can be observed
empirically.
While the pill was approved by the Food and Drug Administration in 1960, it was
only legally accessible by married women.
It was only in the late 60s and early 70s when state law changes regarding the
age of majority and `mature minor' decisions, primarily in response to the
Vietnam War, created legal access for unmarried woman.
Claudia Goldin and Lawrence F. Katz present an econometric analysis of
cross-state and cross-cohort variation in pill availability to young, unmarried
women and conclude as follows:
\blockquote{The pill directly lowered the costs of engaging in long-term career
  investments by giving women far greater certainty regarding the pregnancy
consequences of sex. In the absence of an almost infallible contraceptive
method, young women embarking on a lengthy professional education would have pay
the penalty of abstinence or cope with considerable uncertainty regarding
pregnancy. The pill had an indirect effect, as well, by reducing the marriage
market cost to women who delayed marriage to pursue a career. With the advent of
the pill, \textit{all} individuals could delay marriage and not pay as large a
penalty. The pill, encouraging the delay of marriage, created a
\textquote{thicker} marriage market for career women. Thus the pill may have
enabled more women to opt for careers by indirectly lowering the cost of career
investment.}\footcite[p. 731]{goldin_power_2002}

The magnitude of these effects is remarkable.
There was a sharp break in the number of applications to major degree programs
around 1970.
As Goldin and Katz observe, \textquote{Throughout the 1960s the ratio of women
  to all students was around 0.1 in medicine, 0.04 in law, 0.01 in dentistry,
  and 0.03 in business administration. By 1980 it was 0.3 in medicine, 0.36 in
law, 0.19 in dentistry, and 0.28 in business.}\footcite[p.
749]{goldin_power_2002}
Unsurprisingly, the sharp increase of women in professional degrees lead to a
sharp increase of women in those professions: \textquote{The percentage of all
  lawyers and judges who are women more than doubled in 1970s (from 5.1 percent
  in 1970 to 13.6 percent in 1980) and was 29.7 percent in 2000. The share of
  female physicians increased from 9.1 percent in 1970 to 14.1 percent in 1980
  and was 27.9 percent in 2000. Similar patterns are apparent for occupations
  such as dentists, architects, veterinarians, economists, and most of the
engineering fields.}\footcite[p. 749]{goldin_power_2002}
It is also worth mentioning that the trend holds true for philosophy as well
(although, philosophy was the second worse in the humanities
overall only doing better that religion) with \textquote{three times as many
women receiving philosophy Ph.Ds in 1985 as in 1965.}\footcite[p.
683]{rorty_special_1987}

Of course, there were many factors driving social change at the time.
As Goldin and Katz are careful to point out: \textquote{No great social movement
is caused by a single factor.}\footcite[p. 767]{goldin_power_2002}
However, alternative explanations don't happen at quite the right time for the
effects they observe.
Of particular note is the resurgence of feminism in the in America.
While this is undoubtedly also a contributing factor, Goldin and Katz note,
\textquote{Although the Civil Right Act of 1964 covered discrimination by
  \textquote{sex}, the Equal Employment Opportunity Commission (EEOG) set up to
  investigate charges did little about sex discrimination until the early 1970s.
  Affirmative action in the form of Executive Order 11246 was amended in 1967 by
  Executive Order 11375 to cover women, but it took many years to put into
  effect. Title IX, guaranteeing women equal access to federally funded colleges
  and universities, was passed in 1972, but its guidelines were not formulated
  until 1975. Antidiscrimination legislation had effects complementary to that
  of the pill, but its timing appears to have been a bit too late for it to have
been the spark.}\footcite[p. 765-6]{goldin_power_2002}

While the expansion of functions which at least some women were reasonably
capable of pursuing as a result, in part, of the pill, the sea change in
demographics presents us with an interesting case of the dynamics of disruption
and reorientation.\footnote{If an intuition pump is preferred over the empirical
  data, consider Japan where the pill was only legalized in 1999 and has seen
  comparatively little increase in the economic status of women since the
1970s.}
The introduction of women's voices in these professions---as peers---left an
indelible mark on the social world.
If culture is understood as a vehicle for the transmission of intergenerational
knowledge, then the introduction of knowledge of the nonideal realities of
women's experiences---knowledge that had always been there but was
obfuscated---was certainly disorientating for members of the prior existing
culture.
We can take the history of the term \textquote{sexual harassment} as a poignant
example because of the disconnect to the history of sexual harassment itself.
While the act of sexual harassment has always been around, it was only given
language 1973 in a report \textquote{Saturn's Rings} by Mary Rowe who was then
working at MIT.
Afterwards, momentum behind the movement to redress systematic blindness of the
problem built up rapidly---primarily by female activists in universities like
Cornell.
Legal activist Catherine MacKinnon's work \textit{Sexual Harassment of Working
Women: A case of Sex Discrimination}---inspired in part from the
awareness-raising in Cornell---is credited for creating the legal claim for
sexual harassment as a form of sex discrimination under Title VII of the Civil
Rights Act of 1964.
The term, however, remained largely unknown outside academia and legal circles
until Anita Hill's testimony in the Clarence Thomas Supreme Court Nomination.
The EEOC has registered the number of sexual harassment cases growing from 10
cases per year prior to 1986 to 624 the following year and 2,217 in 1990 and
then 4,626 by 1995.\footcite{cochran_sexual_2004}

That this rapid social change happened contemporaneously with the surge of women
in long-term degrees and professions is not mere coincidence.
We can see this by attending to the testimonies of these women.
The power of having the term \textquote{sexual harassment} cannot be overstated.
By giving language to it, it became a concept that could be communicated where
there was none before.
Accounts like Susan Brownmiller's describe the experience as revelation:
\blockquote{\textquote{Lin's students had been talking in her seminar about the unwanted
  sexual advances they'd encountered on their summer jobs.} Sauvigne relates.
  \textquote{And then Carmita Wood comes in and tells Lin \textit{her} story. We
    realized that to a person, everyone of us---the women on staff, Carmita, the
    students---had had an experience like this at some point, you know? And none
    of us had ever told anyone before. It was one of those \textit{click, aha!}
moments, a profound revelation.}}\footcite[p. 281]{brownmiller_our_1999}

It is significant that this takes place in a university classroom. Miranda
Fricker argues that what has occurred is a hermeneutical injustice.
There had been a lacuna in the shared hermeneutical resources---a gap in the
knowledge of the epistemic community---that had prevented these women from
being able to render their own experiences intelligible to themselves much less
others.
That this was an injustice arises from the fact that this hermeneutical lacuna
arose from the exclusion of women in participating in the epistemic community as
peers.
As she argues, \textquote{Women's position at the time of second wave feminism
  was still one of marked social powerlessness in relation to men; and
  specifically, the unequal relations of power prevented women from
  participating on equal terms with men in these practices by which collective
  social meanings are generated. Most obvious among such practices are those
  sustained by professions such as journalism, politics, academia, and law---it
  is no accident that Brownmiller's memoirs recounts so much pioneering feminist
  activity in and around these professional spheres and their institutions.
  Women's powerlessness meant that their social position was one of unequal
  hermeneutical participation, and something like this sort of inequality
provides the crucial background condition for hermeneutical
injustice.}\footcite[p. 132]{fricker_epistemic_2011}
The injustice here should not be taken lightly.
A fundamental part of human being is the capacity for knowledge---both in
possessing and professing.
In this case both had been undermined thereby undermining these women as humans
\textit{qua} knowers.
The supporting role the pill had in instantiating the conditions for this to
finally take place---despite centuries of intelligent minds working on
ideal-as-idealized-models which presumably should have just been easily
\textquote{applied}---demonstrates the richness of the kind of disruption
technology can bring that was obscured in Stiegler.

The issue might be reframed in terms of ideology---in the descriptive sense
Sally Haslanger gives as \textquote{an element in a social system that
  contributes to its survival and yes is susceptible to change through some form
of cognitive critique.}\footcite[p. 75]{haslanger_but_2007}
She includes practical knowledge and rituals like habitual gestures in this
conception of ideology.
Moreover, this ideology, with a small \textquote{i}, maps onto different social
groups with their different experiences, beliefs, and frameworks for
understanding what actions and events mean.
Taking this together, it seems Haslanger's sense of \textquote{ideology} has the 
same, or sufficiently proximate, referent as Stiegler's sense of
\textquote{culture}.
If that is the case, then Haslanger gives us an alternative theory of disruption
and reorientation.

Haslanger begins from an examination of a mother and daughter disagreement about
whether crop-tops are cute.
If we are not rudely dismissive, then we might notice that there is some
interesting features of social knowledge at play here.
It seems that the daughter is right when she asserts crop-tops are cute and
girls who wear track suits are dorks because the truth or falsity of the
assertion indexes to the social structures she is embedded in---or social
milieus.
The practices of these social structures as a \textquote{product of schema (a set
  of dispositions to respond in certain ways) and resources (a set of tools and
material goods)} is not subjective but real constraints of the social 
world.\footcite[p. 79]{haslanger_but_2007}
In this way they instantiate social meaning.
The daughter says something true insofar as she accurately describes the
practices of the social milieu she lives, in part, in.
\textquote{Plausibly, cuteness and dorkiness are features that must be judged
from the social milieus because they are partly constituted by those
milieus.}\footcite[p. 81]{haslanger_but_2007}
Conversely, the mother says something true when she asserts the opposite because
it is true indexed to her social milieu.
Both assert the truth and yet both contradict each other.
In light of this, Stiegler's intuition that cultural change disrupts knowledge
is better clarified.

However, if we take this as a clarification of Stiegler's disruption, then
Haslanger gives us insight into reorientation as well.
As she notes, \textquote{There is something tempting about the idea that we live
  in different social worlds (or milieus); that what's true is one social world
  is obscure form another; that some social worlds are better for its
  inhabitants than others; and that some social worlds are based on illusion and
distortion.}\footcite[p. 81]{haslanger_but_2007}
If we take this position then we have two ways to address disagreement between
social milieus: understanding and critique. These dimensions map onto how other
social structures may be more or less accessible or more or less in harmony to
our own respectively. As such, the genuine disagreement between the mother and
the daughter is a function of the accessibility and harmony of the milieus.

This suggests in turn the picture of reorientation through adjudicating social
truths across milieus. Haslanger rejects, tacitly, the kind of conception which
Stiegler seems gesture at: \blockquote{A simplistic hypothesis might be that once
  one is exposed to a different social reality by engaging with assessors from
  another milieu, one will come to see the weaknesses of one's own milieu. On
  this view, the very exposure to another milieu, even in one's current
  (inadequate) milieu and provide opportunities for improvement. Critique,
  strictly speaking, is not necessary; one need only broaden the horizons of
  those in the grip of an unjust structure and they will gain
\textquote{consciousness} and gravitate to liberation.}\footcite[p.
81]{haslanger_but_2007}
She rejects this, however, on the grounds that it tends to smuggle in an
objectivist view about moral and epistemic value that is itself relative to
milieus.
If the claim is to privilege privileged milieus in determining whose horizons
need to be broadened and who already have \textquote{consciousness}, then there
needs to be an objective standard that is not milieu-relative.
This, however, demands a kind of ideal theory that our ideal-as-idealized-models
have never lived up to.

Instead, Haslanger proposes a notion of critique wherein \blockquote{the
  assessor's claim is a genuine critique of a speaker's only if there is some
  common ground (factual, epistemic, or social) between the speaker's milieu and
  the assessor's milieus, and the assessor's claim is truth relative to the
  common ground. To say that a critique is genuine, in this sense, is not to say
  that it is the final word; rather, it is to say that a response is called
for.}\footcite[p. 87]{haslanger_but_2007}
This illuminates the frustration between the mother and daughter.
To simply give a flat denial to the other is, at least, incomplete or, at worst,
pointless.
Rather, they ought to engage in the dialogue necessary to find the common ground
they both accept (is accessible and harmonious to both starting points) from
which they can assess the truth-value of the claim indexed to the common ground.
As Haslanger asserts, \textquote{An advantage of this notion of critique is that
  it would help make sense of the idea that ideology critique is transformative.
  If critique isn't just a matter of reasoned disagreement, but is a matter of
  forming or finding a common milieu, then because a milieu is partly
  constituted by dispositions to experience and respond in keeping with the
  milieu, then possibilities for agency other than those scripted by the old
milieu become socially available.}\footcite[p. 87]{haslanger_but_2007}

This is, moreover, significant in connection with Fricker's account.
The gap between milieus may be especially wide if they represent different
epistemic communities with different shared hermeneutical resources.
If this is the case, the problem with the prior conception---that critique is
not necessary---fails to account this social epistemological gap.
It may be the case that simply \textquote{broadening horizons} to gain
\textquote{consciousness} trivializes the problem of a hermeneutic lacuna and
fails to recognize the problems with historic structural testimonial injustice.
To be able to communicate knowledge between milieus across such a social
epistemological gap requires that the testimony of others be extended due regard
and entered into the shared epistemic and hermeneutic community.
Doing this entails the kind of ideology critique described by
Haslanger---dialogue to find common ground, to form a common milieu.
We might even describe it as epistemic transindividuation.

This, after all, is a picture of how epistemic and hermeneutic
resources---retentions in Stiegler's terms---are passed from one milieu to
another, including across generations.
Moreover, it describes an involvement of technology in the process that is
missing in Stiegler's ideal theory: the social epistemological power relations.
In that case, it is the social and cultural dynamics preventing
transindividuation.
The historic structural testimonial injustice of excluding women from full and
equal participation as epistemic peers lead to a hermeneutical lacuna.
Despite the work of millennia of the most intelligent philosophers constructing
ideal theories on ethics and politics, this clear case of wrongdoing went
under theorized.
This may contributed, knowingly or unknowingly, by either directly or by
omission, to the prevailing ideologies which construed \textquote{Baby it's cold
outside} as romantic and \textquote{One of these days, Alice, pow, right in the
kisser!} or \textquote{Bang! Zoom! To the moon Alice, to the moon!} as comedy.
That there was a sea change in the 1970s as a result of a significant increase
of women entering long-term professional graduate programs, and afterwards, the
professions themselves, represents a transindividuation as a result of changing
social epistemological power dynamics.
Now, women had the relevant states and credentials that forced an extension of
moral regard that admitted them into the epistemic community as peers.
As a result of their testimony finally being elevated to this level, they were
able to engage in cross-milieu dialogue to form new epistemic communities with
new homework resources such that they were finally able to share
knowledge---retentions.
New protention were created as people came together to a shared future aiming
towards righting historical gender inequalities and injustices.
All this, in part, instantiated by a piece of technology---the pill.
We can see here how technology can change the social epistemological power
relations by liberating previously unheard voices.
All this may be compatible with Stiegler's epokh\={e} but it gives unarticulated
in his account.

This poses problems because it raises the issue that the problem he attributes
to technology---disruption such that there is an epoch without epokh\={e}---can
be, in fact, caused by the cultural and rectified by the technological.
Again, the case being made here is not presented as exhaustive.
It is in no way suggested that we can extrapolate from this case study that we
can generalize all technology, or Technology with a capital T, as being
emancipatory.
Rather, if we admit at least this one case, then it looks like the actual
dynamics are sufficiently complex to call into question Stiegler's methodology.
He creates an ideal-as-idealized-model of those dynamics such that it is the
pace of technological innovation outstripping cultural evolution that creates
the time shock of an epoch without epokh\={e}.
He asserts that we are currently in one right now as a result of digital
computing.
However, when he describes the mechanics of how digital communication relates to
transindividuation, he concludes the disruption of transindividuation comes from
the technological despite identifying the potential in digital communication to
form unprecedented protentions and, ultimately, the society of hyper-control's
exploitation of computing capacity to monopolize control of what is shared
through these technological means.
Given the case above, it is unclear why it goes unthought that the fault instead
lies in the cultural---the sound epistemological power relations.
This suggests, through abduction, the fault lies in the ideal-as-idealized-model
methodology.
The deviation of the actual from the idealized model was not sufficiently
considered.

Given the multiply realizable ways in which the dynamics between technology and
culture interact, we should instead make the methodological choice to examine
technology through technologies.
This is not to say we cannot abstract a theoretical model at all, but that such
a model should always be cognizant of the way in which the actual may deviate
from the model.
As has been hinted above, I take Marx's theory, or at least G.A. Cohen's
analytical Marxist interpretation, as a pertinent example.
His model allows for a richer complexity of technology in understanding how it
interacts with cultures.
In describing the material condition for a socialist revolution, he draws out
have technology and culture are mutually constitutive in both directions at the
same time.

Cohen does this by noting the dynamics of Marx's three part model of the
socioeconomic productive forces, relations of production, and the political
superstructure.
The productive forces are those factors directly involved in the generation of
commodities.
This represents both the means of production and the labor force.
The relations of production are the arrangements of power of the relevant agents
over the productive forces.
These are the relations of effective control between a person and a productive
forces and the relations of effective control between a person and another
person qua productive force.
The aggregate of a society's relations of productions constitute the economic
structure as a whole and the two are interchangeable and equivalent.
The superstructure consists of a certain set of society's institutions which
relate to the economic structures.
This roughly circumscribes any institution relevant to productive society
ranging from the economic to the social.

G.A. Cohen notes the following interactions between these socioeconomic
mechanisms.
First, the level of development of the means of production, and therefore,
productive forces, tends to increase over time.
This is due to a disposition to reduce the historical condition of scarcity.
Second, the relations of production explain the level of development and the
level of development explain the relations of production.
Marx simply suggests there is some kind of \textquote{correspondence} at work
here.
However, Cohen makes the distinction that the relations of production
\textit{causally} explain the corresponding level of development and the level
of development \textit{functionally} explains the corresponding relations of
production.

To elaborate on what is meant by a functional explanation, suppose X causes Y.
A causal explanation describes the occurrence Y in terms of its cause X.
A functional explanations describes occurrence X in terms of effect Y.
This appears to be a violation of the rules of time order: we cannot suggest
\textquote{X occurred because Y occurred} or \textquote{X occurred because it
caused Y}. 
This seems impossible since it claims a later occurrence, before it has occurred
and can manifest anything, causes an earlier occurrence.
As such, Cohen suggests that a functional explanation must appeal to something
outside cause X and effect Y---the context.
It might be the case that situation Z creates a predisposition or affinity for a
certain effect to occur.
As such, we might say, \textquote{In situation Z, X occurs because there is a
predisposition for Y to occur and X causes Y.}
This is the structure of a functional explanation.\footcite{cohen_karl_2001}

The relation of production causally explain the levels of development of the
productive forces.
This can be described as the very technological development is directed
according to the arrangements of power over the productive forces.
Since labor power is effectively a constant, long term development of productive
forces is primarily driven by scientific and technological innovation.
However, the means of production are subject to the arrangement of control in
the relations of production.
Those who control the means are the executors by which new developments are
sought.
These motivations and desiderata are shaped by the character of the productive
relations: those in control will rationally use their power to maximize their
own benefit.
And so, the relations of production direct the technological development and,
therefore, determine, at least in part, the make-up of the productive forces and
consequently the level of development.

The level of development functionally explains the relations of production.
In order to make this claim, we must show that there is a functionally
justifying context in place.
The first interaction provides this.
The historical tendency to develops productive forces to address scarcity
creates a situation where there is an affinity for the effect of the advancement
of the productive forces to occur.
However, in attempting to resolve this situation, new technologies come about
that alter the characteristics of the relations of production in order to best
use them.
For example, under the steam engine, the factory had to be organized around the
logic of the driveshaft in order to evoke the best use of the steam engine.
It was most efficient to have one large engine and, therefore, one large
driveshaft to which all machines were connected to.
This created the conditions of those dark Satanic mills of William Blake where
work was dictated by the pace of the steam engine in dark, cramped, and lethal
conditions.
Consequently, labor was a question of quantity rather than quality.

The invention of the electric engine did not result in productive gains for
about 50 years because use largely consisted in direct substitution of the steam
engines.
However, the advantages of the electric motor when realized when factories were
built around the logic of the electric motor rather than a central driveshaft.
The electric motor instead allowed for many smaller driveshafts in every tool.
Work was no longer dictated by the constantly running steam engine, but by the
operator choosing to turn on the tool as they could be operated independently of
others.
Consequently, this enabled the logic of the production line which allowed for
space and light and, thereby, cleaner and safer.
Additionally, since workers set the pace of their own work, they had more
autonomy which changed the way they were recruited, trained, and paid had to
change towards quality rather than quantity.
A similar dynamic can be seen in the adoption and deployment of the
computer as described by Paul David.\footcite{david_dynamo_1990}
Consequently, due to the tendency for the level of productive forces to increase
over time to address scarcity, the level of development functionally explains the
relations of production.
The productive relations that do not facilitate a rise in productive forces do
not obtain because they do not create the effect of increasing the level of
productive forces.

Third, the economic base---the aggregate of the relations of production---and
the superstructure interact in a parallel way as the productive forces and
relations of production.
The superstructure causally explains the economic base and the economic base
functionally explains the superstructure.
Finally, we can observe that level of development of the productive forces---that
is to say, technological innovation---has explanatory priority in these dynamics
despite it only functionally explaining the other levels.
This is due to the fact that productive forces are the prime movers in this
model.
This is contrary to historical attempts of accelerationism to instantiate change
from the superstructure down and explains the failure of accelerationism.
Let us say we have conceived the perfect society and have recognized and
implemented the corresponding changes in the institutions of the superstructure.
The superstructure then accordingly recognizes, undoubtedly through coercive
means, the shape of the economic base and thereby the relations of production.
The production relations will then attempt to determine the level of development
of productive forces.
However, we do not chose the historical level of development we have been thrown
into.
If a lower level of development is asked for, then the intentional and
recognized regression of productive power, and thereby an increase in scarcity,
will be resisted because we have needs.
If a higher level of development is required for the sustainability of the
economic base and superstructure, as was the case with the Soviet Union, then we
will find that new means of production are not manifested nearly quickly enough
by sheer will power and coercion alone.
This is all to say that society does not fully choose its own material realities.

Moreover, Cohen argues these dynamics give us the conditions of epochal change,
including socialist revolution.
The conditions are technological rather than social or political.
Such a revolution becomes possible in a state of fettering: when the
contradiction between the productive forces and the relations of power become
sufficiently non-optimal.
This occurs when, as the means of production develop, they instantiate new
arrangements of the production relations to facilitate and benefit from this
growth.
This technological innovation motivates a social and cultural reorganization.
However, as this continues, it reaches a point where the correspondence of the
production relations requires a sufficiently radical reorganization such that it
would entail a recognizably different character of the power relations across
the economic base as a whole.
The class who controls the development of the productive forces, as determined
by the production relations, is usually encouraging or agreeable to
technological innovation as it generally allows them to maximize their own
benefit.
At this point of fettering, however, the level of development depends on a new
arrangement of production relations such that it would constitute a detriment to
the share of the benefits they previously enjoyed.
As such, they exert their power to suppress the advance of productive forces and
conserve the current shape of the relations of production.
In other words, the socioeconomic culture becomes a fetter for technological
development because of the ruling class's conservatism.
In this state, epochal change---of the type described by Marx such as the change
from feudalism to capitalism or capitalism to socialism---becomes materially
possible.
To put it another way, revolution is simply the activation of the explanatory
priority of the productive forces as it pushes against the inertia of the
economic base and the superstructure.

Granted, this model is not directly applicable to the concerns of
Stiegler---although I see no reason, in principle, why it could not be
generalized out as a description of relevant power dynamics of technical
innovation and recognizing the communication of culture generally.
Arguably, Cohen's state of fettering better describes the society of
hyper-control's exploitation of computation to disrupt the positive potential in
digital communications to create new transindividuation circuits.
The point was to illustrate a model that is responsive to the to the texture of
technology and culture and how they interact in a variety of ways is possible.
Note that while the model generalizes trends, it allows for the specificities of
a technology's attributes to play a role.
Not all technologies---indeed, most---will not have the characteristics which
require the kind of social reorganization that instantiates a state of fettering.
Moreover, it also explains the way in which socioeconomic forces direct the
development and deployment of technology to maintain certain arrangements of
power.
Admittedly, this Marxist model does not touch on the features of
transindividuation and transgenerational knowledge transmission.
But, then, it is not obvious that Stiegler's account satisfactorily connects that
model to the phenomenon described of societies of hyper-control either.
Plausibly they are two different phenomenon that require two different models.
Although we can note, again, that the above provides a hermeneutical lens by
which we can look for congruence in how technology affects power dynamics
generally including in social epistemology.
There certainly seems to be similar dynamics surrounding the pill.

\section{Can we break the time barrier?}

If a philosophy of technology is an examination of the nature of technology and
its social effects, then I assert the above nonideal approach qualifies as a
philosophy of technology.
It simply represents the methodological choice to understand technology from an
examination of technologies and their social effects rather than proceeding from
an attempt to propose some abstracted conception of Technology with a capital T
which defines the phenomenon as a whole according to some essential feature.
It might be suggested that the kind of analysis may be apt for a gender studies
context and may have virtue in generating sociological knowledge, but that it
does not address how we can make sense of social and technological progress.
This misconception rests on the methodological error of presuming an epistemic
confidence in attempting to directly grasp culture and technology as abstract
ideal forms.
To motivate the main methodological argument presented in this paper and drawn
from Mills, the ideal-as-idealized-model proceeds on inattentiveness to how the
actual may deviate from the idealized.
As a result, it bears the risk that it is flawed in its explanatory and
productive power because it, knowingly or not, fails to consider significant
evidence that may lie outside the immediate perception of the theorizer's
standpoint.
Given this is the case, we ought to exercise epistemic humility and proceed on
an ideal-as-descriptive-model that surveys the rich texture of technologies to
fully appreciate technology and how it interacts with culture.
To make this point explicit, in this final section, I will respond to possible
objections.

The first objection claims that the example nonideal case study does not
disagree with Stiegler's account.
As this account was prefaced, the account was not intended to be a
contradictory one but to be an example of how alternative methodologies can
conceptualize similar phenomenon from a nonideal direction.
That being the case, we might wonder why we could not accept the non ideal model
as supplementary modification to Stiegler's ideal theory.
After all, most would not suggest that ideal theory, by itself, is sufficient.
Herein, we see a case of the ideal being applied to the nonideal.

It is not entirely obvious to me that we can appropriately simplify the
presented methodology as such.
Stiegler's theory merely articulates that there are retentions and that they are
passed from generations until a technological disruption after which new
protentions must be formed.
Haslanger's social epistemology, however, gives a more detailed explanation of
cross-milieu knowledge transfer and how it is apparently interrupted arguably in
virtue of beginning from the puzzle about the truth value of \textquote{crop-tops
are cute}.
Putting that aside, though, this objection is in fact a concession to the main
methodological argument made defended here: that a nonideal theory is necessary
as a result of the methodological weaknesses inherent to ideal theory.
It is not my thesis that all ideal theory should be thrown out.
I merely argue that we should make the methodological choice to intentionally do
the work of nonideal theory that has historically been failed to be done.
In which case, supposing that we should supplement or modify an ideal theory with
a nonideal approach concedes the need for such a methodology.
It admits that it is nontrivial that such an account goes unarticulated in
Stiegler even if it is logically compatible.
While he gestures at the idea that technology is a \textit{pharmakon}, both cure
and poison, he focuses almost exclusively on negative aspects of technology.
This leaves open a hermeneutical lacuna about the positive aspects of
technology.
If the aim of an ethics of technology is to better guide our behavior towards
technology, how can we possibly come to a full understanding if our attention is
focused on the negative to the exclusion of the possible good?
The claim here is not that all technology is beneficial.
The claim is, rather, that if technology is \textit{pharmakon} and we aim
towards both diagnosis and prescription, then we must take into account the
holistic picture such that we can come to understand when and where the
\textit{pharmakon} is appropriately used.
We have to understand both the element of potential liberation and potential
alienation.
To do that, however, we must take up a nonideal examination in all the ways the
\textit{pharmakon} has been used appropriately and inappropriately.
It is true that this is logically consistent with Stiegler and that it can be
subsumed.
However, this is also a concession that his ideal theory has an explanatory gap.
The distinction between potential liberation and potential alienation and how
these potentials may be realized in both a necessary one and one that must be
made on a nonideal methodology.
This objection admits that such a distinction is not satisfactorily articulated
in Stiegler's ideal theory.

The second objection pushes back on the claim that there is better explanatory
power.
While Haslanger goes into detail on on cross-milieu knowledge transmission, her
account does not give a picture of transindividuation like Stiegler's does.
As such, it is not obvious that the example nonideal account has more
explanatory power.
The response to this is brief.
This attempts to substitute breadth in place of depth.
As noted, the nonideal model is not presented as exhaustive.
There is, in principle, no reason similar case studies cannot be made about
other features of the interaction of technology and culture.
Indeed, the methodology requires that we do conduct them.
As such, it is feasible that a nonideal approach to transindividuation can be
made.
Moreover, if there is some deviation of the actual experiences of
transindividuation to the idealized model of it, then an appropriately conducted
ideal-as-descriptive-model should take such deviations into account.
Thus, it should fill in an explanatory gap should one exist.
We cannot, prima facie, argue from ideal theory that there is no such deviation
and therefore explanatory gap because a sound argument to that effect would be a
nonideal methodology.
As such, we should admit the need for nonideal theory.

Thirdly, it might be objected that the nonideal account presented is not the
kind of phenomenon Stiegler is interested in.
He is theorizing about a disruption caused by technological innovation such that
there can be no cultural reorientation afterwards---the breaking of the time
barrier.
The pill example presents a short-circuiting of a long circuit from which we have
managed to form new long circuits.
Stiegler is ultimately interested in a short-circuiting so pervasive no long
circuit can exist thereafter.
As such, the nonideal approach demonstrated fails to address this concern.

This objection is the most significant one.
On one hand, we can respond that Stiegler does not give an adequate job of
providing a case either.
As has already been noted, there is an incongruency between how his theoretical
framework postulates the breaking of the time barrier and his disruption and
attribution of what is ostensibly such an example in the case of digital
communications technology.
On one hand, the claim is that the pace of technological innovation is supposed
to proceed so quickly that it interrupts any transmission of retentions and
thereby preventing a new culture being formed around a stable technical system.
On the other hand, he identifies digital communications as a relevant example
because it has caused an epoch without epokh\={e}.
However, when he describes this example he does not describe the mechanism kind
out in his theoretical framework.
Stiegler admits that, intrinsic to digital communications technologies, there
lies the potential for unprecedented new protentions and, therefore, retentions
and transindividuation.
In short, that the technology's characteristics seem to allow for cultural
evolution.
The disrupting force which Stiegler identifies is the society of
hyper-control---the monopolization of power over the development and direction
of digital communication---which leverages computational power to interrupt
cultural evolution to its own ends.
In short, what he describes to be the problem is capitalism or
\textquote{computational capitalism} in his terms.
However, this cause of an epoch without epokh\={e} is arguably social rather
than technological.
It is certainly the case that it is not the kind of technological disruption
described by the theoretical framework centered on time.

It might be responded that the epoch without epokh\={e} described in \textit{The
Age of Disruption} is not a case of breaking the time barrier.
We can assert this is a problem of epistemic overconfidence in
ideal-as-idealized-models when attempting to apply them to the disruption.
There is, however, a more interesting methodological point here.
Let's suppose that, in fact, the kind of technological disruption Stiegler is
interested in is a future event.
This, then, is a speculative project.
It is a projection of what may come to be gleamed from extrapolation.
It might be argued that if nonideal theory attempts to proceed from
ideal-as-descriptive-models then it cannot create such a model because the
future phenomenon is not actual yet, thereby preventing the formation of an
ideal-as-descriptive-model.
Presumably, then, this suggests that such speculative projects must be the
domain of ideal theories.

I argue otherwise.
The distinction between kinds of technological disruptions relies on a nonideal
methodology.
(it is true that in such speculative projects, a nonideal methodology cannot
proceed from a descriptive account oft he not yet actual.
However, it is also the case that nonideal methodology entails idealization in
the sense that the construction of any theoretical model requires.
This distinction to ideal idealization is that there is not an attempt to define
an ideal exemplar---the abstract picture of what is claimed to be the true
nature and mechanism---because there is attention to how the actual may deviate
from such a definition.
As such, nonideal theory an operate on speculative projects by attending to the
actual in the past and present and extrapolating from such a survey a model that
can be projected into the future.
The idea here is intuitive.
If you have more explanatory power, then, as a consequences, we would expect
there to be correspondingly more predictive power.
As such, a nonideal model of technology is able to attend the complex and
diverse ways into which technology and culture interact.
In principle, this should allow us to identify likely characteristics of
technology or its application which would instantiate such a disruption that 
would break the time barrier.
This is something that Stiegler does not attempt.

Hitherto, I have attempted to make the positive case for nonideal theory's
explanatory power.
It might be asked why ideal theory cannot do the same.
As such, I will conclude on the negative case against ideal theory's explanatory
and predictive power.
The first point has been gestured at throughout---technology is too complex and
diverse a phenomenon to be abstracted, essentialized, generalized into a concept
of Technology with a capital T.
We can see this in the way Stiegler idealizes automation.
As noted before, given the investment of intentionality into automated machines.
Stiegler draws from Simondon the conclusion that \textquote{hitherto, the human
  was a bearer of tools an was itself a technical individual. Today, machines
  are the tool bearers, and the human is no longer a technical individual; the
  human becomes either the machine's servant or its assembler: the human's
relation to the technical object proves to have profoundly changed.}\footcite[p.
23]{stiegler_technics_1998}
As a result of automation, the human becomes alienated from their own activity.

It certainly seems like this provides insight in some instances of automation.
Given the exclamation that automation should not become cybernetics,\footcite[p.
78]{stiegler_technics_1998} we might be concerned with the introduction of such
devices like the Jennifer unit.
In warehouses, a picker may be given a wireless receiver and an earpiece.
This earpiece, one such given the branding \textquote{Jennifer}, directs the
warehouse picker to go down, say, three aisles, down four bins on the right and
two from the bottom.
It will then direct them to pick from that bin \textquote{three items} and then
\textquote{three items} and then {three items} and then again {three items}.
These instructions are given on the principle that it is less cognitive
effort---and therefore less error prone---to pick three items four times than it
is to count out twelve of them once.
Such technology seems to strip already menial work the last dregs of human
cognition.
It is a nightmare turn for cyborgization.
In this instance, it seems that Stiegler's ideal-as-idealized-model of
automation gets it right: the human has become the tool of the machine.

The argument, however is not that ideal theories get it completely wrong.
They are made by genuinely intelligent people with real insights into their
inquiry.
The issue is whether or not that account considers all the relevant evidence and
considerations, particularly the ones that lie outside their standpoint.
In contrast to the Jennifer unit we can to turn to VisiCalc and its successors
Lotus 1-2-3 and Microsoft Excel.
It's hard to imagine a more completely automated job that the accounting clerk.
Such clerical workers had to tabulate numbers into a ledger, the spread of two
pages being a spreadsheet, whose outputs were transferred into a master ledger.
Any adjustments or new calculation meant going through by hand with a pencil,
eraser, and desk calculator across potentially several spreadsheets and several
ledgers to make all the relevant changes.
It Should be obvious why the first computer spreadsheet program, VisiCalc for
the Apple II, was hugely successful.
It has been called the first killer app---a piece of software so valued by its
users that they buy the hardware just to use it.
It's successor, Lotus 1-2-3, similarly, was the killer app for the IBM PC.
Unsurprisingly, the number of people employed as accounting clerks dropped
dramatically.
What is significant, though, is the number of accounting analysts had an even
greater increase.
This makes sense given the data and manipulation of that data necessary for
complex statistical analysis went from being time-consuming drudgery to being
processed by a computed quickly and effortlessly.
The main point here isn't that there were more analysts gained than clerks lost.
The way in which automation, in this instance, changed the work does not seem to
line up with Stiegler's ideal-as-idealized-model.
Distaste for computational capitalism aside, it would be strange to assert that
the analyst, who is engaged in cognitively stimulating work, is no longer the
technical individual and is now either machine's servant or assembler in a way
the accounting clerk, engaged in the mind-numbing tedium of maintaining the
ledgers, is not.
It is strange to claim that the digital spreadsheet and the clerk are the true
tool bearers and the analyst, who manipulates the data sets to construct models,
is not.
This seems, prima facie, absurd.

Automation is certainly transformative.
But the transformation it brings is rarely as simple as \textquote{Robots took
my job.}
Rather, such technologies chunk out the easily automated part of the work.
This leaves humans to do the rest which may be cognitively more demanding or
more alienating.
The point being her is that the social effects of automation is multiply
realizable in a way that is not straightforwardly a priori or generalizable.
It depends on the specificities of the technologies in question and the context
of its use.
As such, this makes an ideal theory difficult.
In examining the nonideal we find a diversity of experiences that indicated a
complex interaction of various factors.
Idealizing with no regard to these descriptive facts calls into question the
explanatory power of the ideal-as-idealized-model given that it has potentially
excluded, abstracted away, or overgeneralized these significant and relevant
considerations.
In zooming out from automation, the problem for ideal theory is the potential
that the methodology fails to recognize possible deviations of the actual from
the idealized.
If there are deviations then ideal theory will lack explanatory power because it
does not account what is lost in idealization.
If we do attempt to account for the potential deviation of the actual to the
idealized model, which has to be done a posterori rather than a priori, then it
looks like we are attempting a nonideal methodology.
Ideal theory rests on a uniformity of nature that is, inconveniently, not there
or not safe to assume to be there.

We can now turn to the issue of predictive power.
To draw out the claim that more explanatory power is conducive to more
predictive power, we can look at the problems of Stiegler's breaking of the time
barrier argument.
If we are sympathetic to Haslanger's model of how knowledge can be communicated
across social milieus through what sh terms ideology critique, then the
shortcomings in explanatory power in Stiegler's model becomes evident in the
comparison.
Stiegler asserts, in the theoretical framework in which the breaking of the time
barrier is articulated, that digital technologies disrupts the transmission of
knowledge across generations.
However, he does not adequately explain the mechanism by which that occurs.
As already discussed, the example Stiegler provides of digital technology does
not show clearly his theoretical framework in action.
As such, we are left to our intuition and imagination.
Perhaps I lack one or the other, but it is not obvious to me what would be
entailed outside the big picture.

This is relevant given that a speculative project is not merely about what is
imaginable but what is feasible.
There are all sorts of interesting things that we can imagine that are, in
principle, not possible.
Moreover, they can be edifying or illuminating.
We need only to turn to the wealth of fiction.
However, insofar as philosophy is interested in understanding the actual world,
then, despite the edification or inspiration we might draw from our
imaginations, we have to be constrained by feasibility.
It may be true that reality is in excess of our attempts to grasp it, but our
imaginations are by far in excess of reality.
Any attempt to interpret this world should reckon with what is feasible.
This is, by and large, in the same spirit of Mills critique of ideal theory.

In which case, if a speculative project should consider feasibility, we can turn
to how the frameworks can suggest that.
Given this is a speculation, though, entails that is an extrapolation, or
idealization---in short, a model.
Hence, the central methodological concern described in this paper: do we proceed
from an ideal-as-descriptive-model or an ideal-as-idealized-model?
An ideal-as-descriptive-model ought to be preferred on evidential grounds.
If the extrapolated model can be show to be based on phenomenon already actually
present, then it ought to motivate and lend credence to the feasibility of the
model.
If the extrapolated model is an interpretation of intuition and insights of the
author, then all we have to go on is an epistemic confidence in the author's and
our own intuitions and insights.
However, given all the work done on the limitations of human cognition,
rationality, and epistemology, we ought to exercise intellectual humility.
Taken together, we should give more weight to the kind of evidence an
ideal-as-descriptive-model brings even in a speculative project.

As such, we can note a couple of issues with Stiegler's model towards the time
barrier argument.
The central issue is that technical innovation will prevent us from being able
to communicate the kind of knowledge, retention, that are necessary to form a
community around a shared future, protentions.
Moreover, it seems to be the case that the disruption comes from, at least in
part, the pace of technological innovation.
Taken together, the claim is that the medium of communication would change so
frequently that it would constantly disrupt the meaningful communication
necessary fro transindividuation.
Noting that this is not what is at work in Stiegler's examples.
We can then ask whether there is present evidence that speaks to the feasibility
of this claim.

A first point is that this appears to contradict already observable human
behaviour.
Paul David gives a short, enjoyable history of \textquote{QWERTY-nomics}, or the
kinds of human behaviors that can be seen in the history of keyboard
arrangements.
An apparently trivial example, David explains: \blockquote{Cicero demands of 
  historians, first, that we tell true stories. I intend fully to perform my
 duty on this occasion, by giving you a homely
 piece of narrative economic history in which
 \textquote{one damn thing follows another.} The main
 point of the story will become plain enough:
 it is sometimes not possible to uncover the
 logic (or illogic) of the world around us
 except by understanding how it got that way.
 A \textit{path-dependent} sequence of economic
 changes is one of which important influences
 upon the eventual outcome can be exerted by
 temporally remote events, including happenings dominated by chance elements 
 rather
 than systematic forces. Stochastic processes
 like that do not converge automatically to a
 fixed-point distribution of outcomes, and are
 called \textit{non-ergodic}. In such circumstances
 \textquote{historical accidents} can neither be ignored,
 nor neatly quarantined for the purpose of
 economic analysis; the dynamic process itself
 takes on an essentially historical character.}\footcite[p.
 332]{david_clio_1985}
In brief, he observes that the QWERTY layout emerges from the development of the
up-stroke typebar typewriter to obviate the technical problem of typing too
rapidly causing the arms to jam.
As such, the QWERTY layout is, in some sense, intentionally designed to slow the
typist down.
Moreover, David notes that, as history and technological development progress,
there is increasingly less technical reasons to necessitate the QWERTY layout.
Typewriters operating without typebar mechanisms rearranged the homerow to be
DHIATENSOR which can compose over 70\% of words in the English
language.\footcite{david_clio_1985}
Moreover, with the introduction of the computer, the user could easily enable
alternatives: the Apple IIc had a built-in switch that enabled a virtual Dvorak
Simplified Keyboard with advertising copy that DSK \textquote{lets you type
20--40\% faster}\footcite[p. 352]{david_clio_1985}
While the empirical studies are contentious, the world's fastest typists use
Dvorak.
Moreover, the ease of reprogramability has allowed users to introduce their own
layouts organized on various criteria such as key travel: Colemak, Workman,
CarpalX, and so on.
Despite these innovations, we can see, as David characterizes it,
\textquote{competition in the absence of perfect futures markets drove the
  industry prematurely into standardization \textit{on the wrong system}---where
decentralized decision making subsequently has sufficed to hold it.}\footcite[p.
336]{david_clio_1985}

David gives three mechanisms for causing a path-dependence that results in
\textit{lock-in} on QWERTY.
First is technical interrelatedness, or \textquote{the need
 for system compatibility between keyboard
 \textquote{hardware} and the \textquote{software} represented
 by the touch typist's memory of a particular
 arrangement of the keys, meant that the expected present value of a typewriter 
 as an instrument of production was dependent
 upon the availability of compatible software
 created by typists' decisions as to the kind of
 keyboard they should learn}\footcite[p. 334]{david_clio_1985}
In brief, as increasing number of typists were trained to touch type on QWERTY,
the user costs would tend to decrease.
This results in the second mechanism, system scale economies.
Even if buyers of typewriters have no preference regarding keyboard layouts, if
typists have preference \textquote{with unbounded decreasing costs of selection, 
  each stochastic decision in favor of QWERTY would raise
 the probability (but not guarantee) that the
 next selector would favor QWERTY.}\footcite[p. 335]{david_clio_1985}
As a result, early historical accidents can have outsized downstream effects
leading to lock-in and path dependence.
Finally, the quasi-irreversibility of investments, or the high costs of
relearning specific touch typing skills creates an asymmetry which has
reinforced lock-in.
While the cost of hardware conversion went down to nearly nothing, the costs of 
software conversion---retraining typists---remains high.\footnote{Anecdotally,
  it took me one month to be proficient with Workman from QWERTY and several
  months to come back up to 80 WPM. While this appears to be worth the
  benefits---less strain and now typing up to 110 WPM---compounding this kind of
cost out to multiple typists is obviously significant.}
David concludes that these three mechanisms of human behavior results in being
locked-in to QWERTY despite later innovations that are potentially superior.

As David notes, there are other phenomenon where we can find similar lock-in
effects as QWERTY.
Arguably, communication technologies and platforms demonstrate this kind of
lock-in to a lesser degree.
Most significantly, the value of communication mediums depends heavily on
economies of scale and quasi-irreversibility of investments.
The value of a communication technology comes from its enabling of communication.
Consequently, there must be others with which you can communicate.
As such, the utility of communication technologies is significantly determined,
in part, by the network effects.
That is to say, the more users on a network results in a increasing positive
externality for each individual user.
Attempting to a different platform, consequently, imposes the switching costs
associated from losing those network effects.
You may think some new social media platform has superior features than a
well-established one, but, if there are not enough users on the new network, the
cost outweighs possible benefits.
As such, these two mechanisms divorce the rate of adoption from the rate of
innovation.
There are many attempts to disrupt lock-in, but since it is difficult for a new
network to achieve the critical mass for a bandwagon effect, lock-in prevails.

Given these observed behaviors, Stiegler's claim seems counter-intuitive.
It seems to presume that the rate of adoption matches the rate of innovation.
However, we don't observe this to be the case.
Moreover, it suggests a model of rationality that suggests otherwise.
If switching costs result in a loss of utility, then on most accounts of human
behavior---even ones that do not suppose perfect rationality---suggests that
this is not the expected decision which people would make.
Given that transindividuation is a intrinsic human value, it is
counter-intuitive to suggest that people will adopt new innovations at
significant cost, when they can stay on the old, just because something new
comes along.
As such, Stiegler is right in diagnosing a constant disruption as undesirable.
However, we have evidence on human behavior that suggests that is not lost on
people.
As a result, we choose to not constantly disrupt ourselves and instead become
locked-in, for better or for worse.
Finally, we might note that locked-in results in the accumulation of power which
Stiegler criticizes as societies of hyper-control.
But, again, note that this is a criticism of the social arrangement rather than
technological innovation---or, rather, the obstruction of technological
innovation.

Moreover, we might wonder about the assumption that the rate of human cultural
evolutionary limit.
This is particularly concerning because, given the account of Haslanger,
suggests a limit to the human power to extend moral and epistemic regard.
Putting that aside, to ascertain the feasibility of the rate of technological
innovation outpacing cultural evolution, we need some better idea of what that
implies.
Stiegler gives the picture of a techno-logic such that it is a kind of genetic
knowledge embedded in technology that only needs to be remembered.
Moreover, this is a reframing of innovation such that it is no longer the act of
human invention but the act of humans merely exhuming the external logic of
technology.
Taken with the focus on memory support retentions, this suggests another way in
which an accelerating pace of innovation can break the time barrier.
If technology produces an increasingly large amount of knowledge at an
increasingly faster pace, then the volume of new knowledge may be such that it
exceeds human capability to process and integrate into an understanding of the
world.
The breaking of the time barrier represents the point when we can simply no
longer keep up with knowledge generation and integration and, as a result, left
in a state of perpetual disorientation.

Pointedly, this cannot be merely about the generation of information---even true
information.
There is already enough information in nature and in math to outstrip human
comprehension.
To violate it, we would simply have to count to infinity.
Moreover, it cannot be interesting or relevant information either.
Again, this would seem to be overly fragile.
If I wanted to bring catastrophe, it would seem to suffice that I create a
program, let's call it monkeytypewriter, that publishes to the internet a string
of random characters.
While most of it will be nonsense, it will, if let run forever, publish an
infinite number of true propositions.
Indeed, there is already a website, www.elsewhere.org/pomo/, that randomly
generates philosophy papers.
There is no reason, in principle, that given enough iteration it produces a
golden needle in a haystack of Sokal-esque bullshit.\footnote{As per usage
defined by Harry Frankfurt in \textquote{On Bullshit}: speech intended to
persuade without regard for truth or even falsity.}
But as noted on the website: \textquote{The essay you have just seen is
completely meaningless and was randomly generated.}
(more can be read into this than was perhaps intended, but it indicates the
relevant kind of epistemic content---meaning, knowledge, or understanding.
These are all elements of belief and doxastic states.
It is the comprehension and attribution of significance as embodied in a human
(or alternative) intelligence that matters here.
Retentions and protentions must be doxastic states rather than merely
propositional content.

However, since the nonideal account enables us to make sense of how transmission
of such understandings can be made across different social milieus, that is to
say, different epistemic communities with their own hermeneutical resources,
then it seems that as long as that understanding is embodied in some human or
set of humans and so long as they can communicate, that is to say, can
participate in the epistemic community, then such generation and integration of
knowledge appears to operate within the mechanism of cultural evolution.
That is to say, the cultural rate of evolution seems to be some function of
knowledge as it is embodied in human beings.
So long as some piece of meaningful understanding is generated and integrated by
at least some member of the epistemic community and they can communicate, then
it is principle a transmissible retention.
After all, it cannot be the claim that the limitation of cultural evolution is
that all retention has to be able to be known by an individual member.

As such, for technology to overcome the cultural limit, technology itself must
generate new understandings, new meanings.
But the feasibility of such an occurrence is, at the moment, dubious.
It might be that, given Stiegler's characterization of technology as computation
can push back.
The turn here is clear.
Like the accelerationists he critiques, presumably the acceleration of
calculation or computation is the acceleration of knowledge generation.
However, given that such knowledge cannot be mere information but a doxastic
state and must be embodied in technology external to humanity, then it seems
the time barrier argument hinges on artificial intelligence.
Or, in Stiegler's words, the creation of automatic protentions which, given the
necessary doxastic states, implies a consciousness.
A strong AI thesis---a technological mind---would presumably be able to create
such volumes of significant meanings that, in trying to comprehend it, would
outstrip out human cognitive ability.
As such, there would be no stability for retentions or protentions since an
external force automatically and perpetually generates new ones forcing us to
instead adapt.
Plausibly, this would be the picture of the rate of technological innovation
outstripping cultural evolution.

However, the issue for that reply is that it rests on computation: that is to
say, software.
In briefs, a dominant position in the philosophy of AI forcefully argues that the
obstacle to strong AI is not just a question of sufficient computational power
or speed.
For a weak AI thesis, that we can construct programs that resemble the behaviors
of minds, computational power and speed are relevant factors.
However, for a strong AI thesis, that we can construct programs that actually
instantiate minds, then the picture is a lot more complicated than what the
Turing test would suggest---that is to say, mere behavior that is
indistinguishable to human perception.
Since the kind of knowledge generated by technology that is relevant to going
beyond cultural evolution is the kind of doxastic state that requires
embodiment in some kind of mind, we are interested in the strong AI thesis.

In brief, one classic argument in the philosophical literature in response to
the Turing test is John Searle's Chinese room thought
experiment.\footnote{However, there is a substantial body of literature to
  similar effect such as Ned Block's Blockhead machine in
\textquote{Psychologism and Behaviorism}.}
Suppose that a man who knows no Chinese is locked in a room and given a batch of
Chinese writing.
Moreover, he is given a rule book in English that gives instruction on how to
correlate one set of formal characters with another.
Formal here just means that the characters are identified by their shapes.
As a result, the man is able to use the rule book and the batch of Chinese
characters to create a second batch which is given back to those outside tho
room. Unbeknownst to the man, the Chinese characters represent a dialogue and
the rules are sufficiently sophisticated that they would pass as human
responses.
Suppose even further that the man becomes so proficient at manipulating these
symbols that---from outside the room---the answers are indistinguishable from
that of a native Chinese speaker in either content or response time.
Compare this to the case where the man is locked in a room and given batches of
English and writes responses, assuming he understands English.
The conceit of the thought experiment is clear: \blockquote{In the Chinese case
  I have everything that
artificial intelligence can put into me by way of a program,
and I understand nothing; in the English case I understand
everything, and there is so far no reason at all to suppose that
my understanding has anything to do with computer
programs, that is, with computational operations on purely
formally specified elements. As long as the program is
defined in terms of computational operations on purely
formally defined elements, what the example suggests is that
these by themselves have no interesting connection with
understanding. They are certainly not sufficient conditions,
and not the slightest reason has been given to suppose that
they are necessary conditions or even that they make a
significant contribution to understanding.}\footcite[p. 418]{searle_minds_1980}

Note that from outside the room, the behavior which replicates the behaviour of
a computer would pass the Turing test.
Despite this, Searle argues, \textquote{I have inputs and outputs that are
  indistinguishable from those of a Chinese speaker, and I can have any formal
program you like, bu I still understand nothing.}\footcite[p.
418]{searle_minds_1980}
Moreover, since the man in the Chinese room is, in essence, a computer program,
then the fact that he understands nothing implies that a computer program---even
one that might be called an artificial intelligence---understands nothing.
The issue lies in the nature of a program---of a construction of computation and
calculation---\textquote{such symbol manipulation by itself wouldn't be
  sufficient for understanding Chinese in any literal sense because the man
  could write \textquote{squoggle squoggle} after \textquote{squiggle squiggle}
without understanding anything in Chinese.}\footcite[p. 419]{searle_minds_1980}
To put it another way, syntactic manipulation of formal symbols alone does not
instantiate semantic understanding.
As Searle notes, we often attribute understanding by analogy and metaphor to
machines however: \textquote{The reason we
make these attributions is quite interesting, and it has to do
with the fact that in artifacts we extend our own intentionality;
 our tools are extensions of our purposes, and so we find
it natural to make metaphorical attributions of intentionality
to them; but I take it no philosophical ice is cut by such
examples.}\footcite[p. 419]{searle_minds_1980}
That is to say, they don't actually have the relevant mental properties to
instantiate doxastic states like knowledge, understanding, or meaning-making
even if we can usefully use metaphorical language.

Suppose I present the following phrase to a human English speaker: 
\textquote{I like the color of the leaves on the tree I'm looking at.} Given our 
psychology, our hardware,  we understand this both syntactically, the grammar, 
and semantically, the meaning.
We can even note that while there is no syntactic ambiguity, there is semantic
ambiguity.
It could be green, yellow, orange, red, brown, or whatever depending on the
intended referent.
But this intentionality is semantic.
Moreover, we cannot help but to think of some semantic content when looking at
the formal syntax even if there is ambiguity.
To the computer I typed this on, however, it has no such psychology of semantic
understanding.
This sentence is entered by my keyboard as ASCII codes, \textquote{73 32 108 
  105 107 101 32 
  116 104 101 32 99 111 108 111 114 32 111 102 32 116 104 101 32 108 101 97 118 
  101 115 32 111 110 32 116 104 101 32 116 114 101 101 32 73 39 109 32 108 111 
111 107 105 110 103 32 97 116 46}, which is stored as binary bits to the memory,
\textquote{01001001 00100000 01101100 01101001 01101011 01100101 00100000 
  01110100 01101000 01100101 00100000 01100011 01101111 01101100 01101111 
  01110010 00100000 01101111 01100110 00100000 01110100 01101000 01100101 
  00100000 01101100 01100101 01100001 01110110 01100101 01110011 00100000 
  01101111 01101110 00100000 01110100 01101000 01100101 00100000 01110100 
  01110010 01100101 01100101 00100000 01001001 00100111 01101101 00100000 
  01101100 01101111 01101111 01101011 01101001 01101110 01100111 00100000 
01100001 01110100 00101110}.
The electric valence of silicon representing these 0s and 1s is all that
presents to the computer's \textquote{psychology}.
The issue is not a question of mere language.
While the programmer sees instructions mediated by more or less human
understandable code and sees in the inputs and outputs meaningful referents,
this semantic meaning is being generated and interpreted by the programmer, not
the computer which compiles the human understandable code into some binary.
All the computer \textquote{see} are bits and bytes: 0s and 1s arranged just so
and manipulated according to the instruction set written in the processor's
assembly code into other arrangements of 0s and 1s.
This becomes even more evident if we look back through the history of computer
interfaces from the graphical user interface to the command line to the punch
card to little bits of iron embedded in a grid of copper wire and so on.
As we strip innovations to mediate interactions between user and computer, it
becomes increasingly less intuitive to make metaphorical attributions of
intentionality.
I'm currently looking at a punch card with 12-rows and 80 columns with neat
rectangular holes arranged formally to represent some program or memory: it is
utterly baffling.

As such, this poses a significant problem to projections of technology that,
explicitly or implicitly, rely on AI.
Given the increasingly broader senses of the term, we should be clear what,
specifically, is the referent.
If the referent is a strong AI thesis, which I argue has to be here, then the
evaluation of feasibility is much different than the feasibility of weak AI.
Strong AI, it seems, is perpetually only 20 years away from now.
There is an issue if we conflate the clear progress in weak AI as evidence of
progress towards strong AI.
This is because the advances in developing better machine learning algorithms
do not address the major obstacle to the possibility of strong AI.
That is to say, if syntactic manipulation of formal elements does not
instantiate a semantic manipulation on intentionality, reference, concepts, or
meaning, then increasingly sophisticated syntactic manipulation can only create
behaviour with resemblance and not psychology.
How to construct machines that can think both syntactically and semantically,
however, remains a hard problem on the level of the mind-body problem.
This is not to imply that only human brains count---it is completely possible
that there are other possibilities like animals or aliens.
We are, however, not remotely close to understanding what they are and, more
importantly, how to make them.
What we do have good reason to believe, thought, is that software
programs---constructs of calculation and computation---cannot by itself be
minds.
As a result, they cannot create retentions or protentions as Stiegler supposes
and, thereby, cannot create the conditions of adaptation to some external thing
to cultural evolution.

As a result, this seems to foreclose on another possibility of how to make sense
of Stiegler's speculation that technology will break the time barrier.
Even if computing power with weak AI produces increasingly large amounts of
information, it has to be mediated through human intelligence for it to become
the kind of relevant meaning-making for retention and protention.
Again, it cannot be merely the generation of information at all, external to
human beings, because this wold be too fragile.
Therefore, understanding and meaning-making remains embodied in human and
therefore within the limits of cultural evolution.
Since transindividuation depends essentially on communication, and it is worth
noting Stiegler's preoccupation with digital forms of communication, we might be
concerned that technological innovations would result in a constant disruption
in our communication mediums.
However, this presumes that adoption follows innovation.
Given economic phenomenon like path dependence and lock-in, there is good reason
to think communication technologies are inherently resistant.

Finally, Stiegler himself does not give as a satisfactory elaborations of what
is entailed when the rate of technological innovation overcomes the limits of
cultural evolution.
He certainly gives us arguments and examples which criticize computational
capitalism and corporate power monopolies, but these are arguments about the
social.
While there is some interaction between technology and the social, political,
economic, and cultural, that is not enough to pin phenomenon emerging out of the
latter as ultimately being caused by the former.
While these are supplementary arguments against technological accelerationism,
they should no be conflated with the time barrier argument.
The logical distinction should be evident.
Arguments against the reckless costs of accelerationism do not hinge on a
premise that it is feasible that accelerationism will work.
Regardless of the potential benefit, all that needs to be true is that the costs
are too high to bear.
That the rate of technological innovation will accelerate such that it outstrips
the rate of cultural evolution resulting in an epoch without epokh\={e},
however, does logically rely on the premise that accelerationism can feasibly
achieve its goals.
All that is being addressed here is the former: the latter is an argument I am
sympathetic with although I hold that it is strengthened by a nonideal critique
from Cohen's framework that shows accelerationism is not feasible and, thus,
unquestionably too costly given historical experiments like the USSR.

Weighing this all together, it should be evident why a nonideal approach is
preferable to the ideal.
It is not obvious why, given the nonideal observation and lack of ideal
illumination, we should think Stiegler's prediction of the breaking of the time
barrier is feasible beyond our own, obviously fallible, intuitions.
This argument has been made on the basis of the advantage in explanatory and
predictive power of the nonideal approach.
That this should be the case should not be surprising given the methodological
critique levied by Mills: it is ultimately a critique of the poverty of evidence
ideal theory rests on.
Of course, this is not to say that ideal theory must necessarily get it all
wrong.
As noted above, whether an ideal theory should be rejected whole cloth or simply
modified is outside the scope of this paper.
It ultimately depends on how far it deviates from the actual.
In either case, the move concedes the main thrust of this argument, that
nonideal theory is a necessary methodology that we ought to be intentionally
engaged in.
The introduction of the kind of distinctions that attempt to save
Stiegler---for example, between different kinds of effects of technology or
between different kinds of disruption---can not be made soundly from an
ideal-as-idealized-model.
Vallor notes accusingly:\blockquote{Now, it is a
common habit of many academics to roll their eyes at the first hint of a 
suggestion that the human situation has entered some radically new phase. As a 
prophylactic against overwrought claims of this kind, these sober-minded 
individuals keep on hand an emergency intellectual toolkit (which perhaps should
be labeled ‘Break Glass In Case of Moral Panic’) from which they can readily 
draw a litany of examples of any given assertion of transformative social change 
being trumpeted just as loudly a century ago, or five, or ten. This impulse is 
often well-motivated: libraries worldwide are stocked with dusty treatises by 
those who, either from a lack of historical perspective or an intemperate desire
to sell books, falsely asserted some massive seismic shift in human history that
supposedly warranted great cultural alarm.}\footcite[p.
9]{vallor_technology_2018}
As she goes on to note, usually this move is legitimate but sometimes things
really do change. I do not take this paper to me moving for that same toolkit.
Like Vallor, I believe there is something of genuine concern going on with
technology and how it interacts with society. Rather, this paper is an attempt
to suggest there is a better methodological approach than moral panic through a
considered observation and analysis of the problem as it actually manifests.
As such, we should proceed on ideal-as-descriptive-models because we simply
cannot understand technology, metaphysically or otherwise, without understanding
technologies.
That this critique has drawn on feminism, race theory, economics, economics,
social epistemology, and philosophy of AI is by no accident.
Technology brings to bear a complex web of mechanisms and interactions that have
far ranging ramifications.
In recognizing this, we should take the hard problem seriously and make the
appropriate methodological choice.


\nocite{searle_minds_1980}
\nocite{block_psychologism_1981}

\printbibliography

\end{document}
