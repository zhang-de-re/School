\documentclass[letterpaper,notitlepage,12pt]{article}
\usepackage[
  letterpaper,tmargin=1in,bmargin=1in,lmargin=1in,rmargin=1in]
{geometry}
\usepackage[backend=biber]{biblatex-chicago}
\usepackage{csquotes}
\addbibresource{workscited.bib}
\usepackage{setspace}
\usepackage[super]{nth}

\title{Equipoise or Evidentialism: Ethics of Uncertainty in Randomized Clinical
Trials}
\author{Alexander Zhang}
\date{}

\doublespacing

\begin{document}

\maketitle

The principle of equipoise has become widely established as the moral framework
that permits randomized clinical trials.
It is often seen as a necessary condition for a trial to be conducted or not.
That equipoise has become so entrenched is readily understandable when
considering the moral dilemma equipoise is meant to safeguard us from. We are,
at after all talking about human experimentation.
Whatever benefits of scientific knowledge we might value in human trials, it is
hopefully obvious that they do no justify the costs of abuse of the
participants.
Moreover, this concern seems to effect all human trials.
To be clear, the abuse in question is not that of the Nazi experiments but that
of Tuskegee.
The moral issue there is evident: that widely available care for syphillus, and
even the knowledge of it, was withheld from unconsenting participants.
The wrongness of the act can be cashed out in various ways, but what is
significant for us is the structure of the act: that clinicians ignored the
interests of the patient in front of them by intentionally withholding the best
available treatement from them.
This is seen as a violation of the \textquote{the fundamental principle that
the physician-investigator's primary responsibility is to his
patient.}\footcite[p. 487]{shaw_ethics_1970}
Shaw and Chalmers formalize the ethical principle as follows:
\blockquote{If the clinicians, or has good reason to believe, that a new
  therapy (A) is better than another therapy (B), he cannot participate in a
  comparative trial of Therapy A versus Therapy B. Ethically, the clinician is
  obligated to give Therapy A to each new patient with a need for one of these
  therapies.\footcite[p. 487]{shaw_ethics_1970}

What's worth noting is that there are widespread ramifications of the
underlying concern here.
Consider that, in the best case scenario for the researcher in a randomized
controlled trial, the control entails that if one course of treatment was better
than the alternatives, some portion of the participants were not given the best
intervention available.
In this way all randomized clinical trials have this in common with what went
wrong in Tuskegee.

However, there is something not quite right about this picture.
I agree that the principle of equipoise is epistemologically misguided and, as a
result, morally irrelevant.
In its place, if we are concerned with uncertainty and the permissibility of
rnadomized clinical trials, we ought instead to be evidentialist, at the very
least, in the case of medicine given the high stakes involved when medical
authorities assert to \textquote{know}.
To this end, I give a picture of why the relationship of care between the
clinician and the patient is in jeopardy if we do not recognize our doxastic
duty to inquire.

Hopefully this project is seen as a friendly nudge back onto the right
direction.
I more or less agree with the basic intuition---that what is morally relevant is
knowledge and uncertainty---behind the principle of equipoise.
To make sense of the criticism, then, I will briefly survey the dominant but
competing articulations of equipoise.
This will hopefully draw out what I see as the problematic epistemology shared
between these distinct theories.

The contemporary use of equipoise was instroduced by Charles Fried.
He is primarly concerned with arguing for a patient-centered relationship
between the clinician and patent wherein the clinician has a duty to consider
the spcefic particularities of the patient when reaching a medical decision.
On this view, Fried thought RCTs posed an ethical dilemma, in part because at
the time the requirement of informed consent in clinical research was still
contentious.
Fried coins equipoise as a possible objection to his argument.
An unfortunate consequence is that he does not give a formal definition of what
he means by equipoise which is now more salient given today's common acceptance
of informed consent.
As he puts it, \textquote{The argument is frequently made that where the balance
  of opinion is truly in equipoise there is no sense to the accusation that the
  prescribing of one or the other of the equally eligible treatments can
  constitute donig less than one's best (the alternative being no better). And
so no one sacrifices any body to anything.}\footcite[p. 58]{fried_medical_2016}
The classic characterization of equipoise, etymollogically something like
\textquote{equal weight}, is set down here: that there is a balance of opinion
and, as a result, the best is not known.

At the very least, this is characteristic of how Fried's concept of equipoise
has been interpreted since.
Freedman attributes to Fried that \textquote{it is necessary that the clinical
  investigator be in a state of in a state of genuine uncertainty regarding th
ecomparative mertis of treatements A and B for poulation P.}\footcite[p.
141]{freedman_equipoise_1987}
Similarly Shaw and Chalmers propose that \textquote{If the physician (or his
  peers) has genuine doubt as to which therapy might be worse than the old. Each
  new patient must have a fair chance of receiving either the new, and,
hopefully, better therapy or the limited benefits of the old
therapy.}\footcite[p.487]{shaw_ethics_1970}
Schafer describes equipoise is when \textquote{The physician-investigator is
  able honestly to say to a patient-subject, \textquote{You will be receiving
  the best \textit{known} therapy,} because prior to completing the trial there
  is not \textit{scientifically validated} reason to predict that therapy A will
  be superior to therapy B (and there is no further alternative, C, which is
\textit{knownk to be better than A or B})}\footcite[p.
4]{schafer_commentary_1985}
We can see that these articulations of the principle of equipoise derived
from Fried all shaer an underlying intuition.
What matters morally is what we know and when we know it.

This, after all, is what distinguishes the harm from the wrongful harm.
It is true that is a randomized clinical trial, unless all arms are equally
efficacious, that some patients did not receive the best intervention they could
have received.
Insofar as this is against their interests, which seems safe to say in general,
this assuredly represents a harm.
But not all harms are wrong.
It is against my interest when another job candidate receives the position
instead of me.
I may lose out substantially depending on my circumstances and the benefits and
compensation.
However, despite what harm I may feel, we would not say I was wronged or what
the hiring decision-maker did was morally wrong in most circumstances.
The exceptions here prove the rule.
For it to be reasonably accused of being morally wrong, there must be some
additional factor over and above the mere harm.
For example, if there is something like nepotism or prejudice behind the
decision, then it makes sense to say that there was a wrongful harm.
The relevant difference lies in the intension or the will in which the action
was committed.
In the case of the randomized clinical trial, the harm becomes wrongful if it
was done so intentionally---or knowingly.
So, which the harm of withholding the best available treatment is necessary for
wrongful harm, it is not sufficient.
When accompanied with knowledge what the best treatment is and that it is
available, then these conditions are necessary and sufficient for moral
wrongdoing.

If this is the case, the appeal of the principle of equipoise is obvious.
If being in a state of equipoise entails that you do not know then you are not
wrongfully harming anyone.
Still, while this intuition seems correct, there is something epistemically
suspect.
Freedman asserts that there are problems that \textquote{are predicated on a
faulty concept of equipoise itself.}\footcite[p. 141]{freedman_equipoise_1987}
He argues that the concept described above, what he terms theoretical equipoise,
is \textquote{both conceptually odd and ethically irrelevant.}\footcite[p.
429]{freedman_equipoise_1987}
Theoretical equipoise is too tenious and unstable a state to conduct a full
clinical trial.
On that framework, a trial may only be conducted when the researcher gives equal
weight to all arms of the experiment.
The nitty gritty of the epistemology entailed by that claim is important.

Freedman ebserves that the way theortical equipoise describes uncertainty is in
terms of the individual researche's subjective uvaluations of the options being
weighed.
He is, by and large, right.
While the princple is often phrased in terms of genuine doubt, knowledge, or
uncertainty, the way these states are treated is significant.
Fried is concerned with a \textquote{balance of opinion} or or even a
\textquote{posture of doubt}.
Shaw and Chalmers assert that \textquote{Clinicians who believe that they know
hich therapy is best should not participate in a comparative trial.}\footcite[p.
494]{shaw_ethics_1970}
This state of belief is also frequently describes in terms of opinions.
Schafer discusses treatment preference which \textquote{falls well short of
knowledge, even when it consists of an intuitive hunch or is based upon
uncontrolled cilincal experience or data from poorly designed earlier
trials.}\footcite[p. 5]{schafer_commentary_1985}
These characterizations make senes if we are talking about equipoise in an
individual.
After all, if equipoise is a state of a balance of evidence, or how we weigh the
evidence, this all represents evidence that must be weighed up after a fashion.

Freedman's critique, however, hinges on theoretical equipoise's requirement of
equivalent evidence rather than the quality or sufficienty of the evidence.
The problem he is concerned with is how researcher's maintain their balance on a
knife's edge over the course of a clinical trial.
After all, in the lengthry conduct of a trial, researchers will be exposed to
lengthy results or the data they themselves have collected and must rationally
update their credences in light of the new evidence.
This is a problem also recognized by Shaw and Chalmers.
Their solution is two-fold: first is that randomization should begin as early as
possible and second \textquote{is to keep the results confidential from the
  participating physician until a peer review group says that the study is
over.}\footcite[p. 493]{shaw_ethics_1970}



\printbibliography

\end{document}
