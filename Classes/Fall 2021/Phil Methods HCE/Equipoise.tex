\documentclass[letterpaper,notitlepage,12pt]{article}
\usepackage[
  letterpaper,tmargin=1in,bmargin=1in,lmargin=1in,rmargin=1in]
{geometry}
\usepackage[backend=biber]{biblatex-chicago}
\usepackage{csquotes}
\addbibresource{workscited.bib}
\usepackage{setspace}
\usepackage[super]{nth}

\title{Equipoise or Evidentialism: Ethics of Uncertainty in Randomized Clinical
Trials}
\author{Alexander Zhang}
\date{}

\doublespacing

\begin{document}

\maketitle

The principle of equipoise has become widely established as the moral framework
that permits randomized clinical trials.
It is often seen as a necessary condition for a trial to be conducted or not.
That equipoise has become so entrenched is readily understandable when
considering the moral dilemma equipoise is meant to safeguard us from. We are,
at after all talking about human experimentation.
Whatever benefits of scientific knowledge we might value in human trials, it is
hopefully obvious that they do no justify the costs of abuse of the
participants.
Moreover, this concern seems to effect all human trials.
To be clear, the abuse in question is not that of the Nazi experiments but that
of Tuskegee.
The moral issue there is evident: that widely available care for syphillus, and
even the knowledge of it, was withheld from unconsenting participants.
The wrongness of the act can be cashed out in various ways, but what is
significant for us is the structure of the act: that clinicians ignored the
interests of the patient in front of them by intentionally withholding the best
available treatement from them.
This is seen as a violation of the \textquote{the fundamental principle that
the physician-investigator's primary responsibility is to his
patient.}\footcite[p. 487]{shaw_ethics_1970}
Shaw and Chalmers formalize the ethical principle as follows:
\blockquote{If the clinicians, or has good reason to believe, that a new
  therapy (A) is better than another therapy (B), he cannot participate in a
  comparative trial of Therapy A versus Therapy B. Ethically, the clinician is
  obligated to give Therapy A to each new patient with a need for one of these
therapies.}\footcite[p. 487]{shaw_ethics_1970}

What's worth noting is that there are widespread ramifications of the
underlying concern here.
Consider that, in the best case scenario for the researcher in a randomized
controlled trial, the control entails that if one course of treatment was better
than the alternatives, some portion of the participants were not given the best
intervention available.
In this way all randomized clinical trials have this in common with what went
wrong in Tuskegee.

However, there is something not quite right about this picture.
I agree that the principle of equipoise is epistemologically misguided and, as a
result, morally irrelevant.
In its place, if we are concerned with uncertainty and the permissibility of
rnadomized clinical trials, we ought instead to be evidentialist, at the very
least, in the case of medicine given the high stakes involved when medical
authorities assert to \textquote{know}.
To this end, I give a picture of why the relationship of care between the
clinician and the patient is in jeopardy if we do not recognize our doxastic
duty to inquire.

Hopefully this project is seen as a friendly nudge back onto the right
direction.
I more or less agree with the basic intuition---that what is morally relevant is
knowledge and uncertainty---behind the principle of equipoise.
To make sense of the criticism, then, I will briefly survey the dominant but
competing articulations of equipoise.
This will hopefully draw out what I see as the problematic epistemology shared
between these distinct theories.

The contemporary use of equipoise was instroduced by Charles Fried.
He is primarly concerned with arguing for a patient-centered relationship
between the clinician and patent wherein the clinician has a duty to consider
the spcefic particularities of the patient when reaching a medical decision.
On this view, Fried thought RCTs posed an ethical dilemma, in part because at
the time the requirement of informed consent in clinical research was still
contentious.
Fried coins equipoise as a possible objection to his argument.
An unfortunate consequence is that he does not give a formal definition of what
he means by equipoise which is now more salient given today's common acceptance
of informed consent.
As he puts it, \textquote{The argument is frequently made that where the balance
  of opinion is truly in equipoise there is no sense to the accusation that the
  prescribing of one or the other of the equally eligible treatments can
  constitute donig less than one's best (the alternative being no better). And
so no one sacrifices any body to anything.}\footcite[p. 58]{fried_medical_2016}
The classic characterization of equipoise, etymollogically something like
\textquote{equal weight}, is set down here: that there is a balance of opinion
and, as a result, the best is not known.

At the very least, this is characteristic of how Fried's concept of equipoise
has been interpreted since.
Freedman attributes to Fried that \textquote{it is necessary that the clinical
  investigator be in a state of in a state of genuine uncertainty regarding th
ecomparative mertis of treatements A and B for poulation P.}\footcite[p.
141]{freedman_equipoise_1987}
Similarly Shaw and Chalmers propose that \textquote{If the physician (or his
  peers) has genuine doubt as to which therapy might be worse than the old. Each
  new patient must have a fair chance of receiving either the new, and,
hopefully, better therapy or the limited benefits of the old
therapy.}\footcite[p.487]{shaw_ethics_1970}
Schafer describes equipoise is when \textquote{The physician-investigator is
  able honestly to say to a patient-subject, \textquote{You will be receiving
  the best \textit{known} therapy,} because prior to completing the trial there
  is not \textit{scientifically validated} reason to predict that therapy A will
  be superior to therapy B (and there is no further alternative, C, which is
\textit{knownk to be better than A or B})}\footcite[p.
4]{schafer_commentary_1985}
We can see that these articulations of the principle of equipoise derived
from Fried all shaer an underlying intuition.
What matters morally is what we know and when we know it.

This, after all, is what distinguishes the harm from the wrongful harm.
It is true that is a randomized clinical trial, unless all arms are equally
efficacious, that some patients did not receive the best intervention they could
have received.
Insofar as this is against their interests, which seems safe to say in general,
this assuredly represents a harm.
But not all harms are wrong.
It is against my interest when another job candidate receives the position
instead of me.
I may lose out substantially depending on my circumstances and the benefits and
compensation.
However, despite what harm I may feel, we would not say I was wronged or what
the hiring decision-maker did was morally wrong in most circumstances.
The exceptions here prove the rule.
For it to be reasonably accused of being morally wrong, there must be some
additional factor over and above the mere harm.
For example, if there is something like nepotism or prejudice behind the
decision, then it makes sense to say that there was a wrongful harm.
The relevant difference lies in the intension or the will in which the action
was committed.
In the case of the randomized clinical trial, the harm becomes wrongful if it
was done so intentionally---or knowingly.
So, which the harm of withholding the best available treatment is necessary for
wrongful harm, it is not sufficient.
When accompanied with knowledge what the best treatment is and that it is
available, then these conditions are necessary and sufficient for moral
wrongdoing.

If this is the case, the appeal of the principle of equipoise is obvious.
If being in a state of equipoise entails that you do not know then you are not
wrongfully harming anyone.
Still, while this intuition seems correct, there is something epistemically
suspect.
Freedman asserts that there are problems that \textquote{are predicated on a
faulty concept of equipoise itself.}\footcite[p. 141]{freedman_equipoise_1987}
He argues that the concept described above, what he terms theoretical equipoise,
is \textquote{both conceptually odd and ethically irrelevant.}\footcite[p.
429]{freedman_equipoise_1987}
Theoretical equipoise is too tenious and unstable a state to conduct a full
clinical trial.
On that framework, a trial may only be conducted when the researcher gives equal
weight to all arms of the experiment.
The nitty gritty of the epistemology entailed by that claim is important.

Freedman ebserves that the way theortical equipoise describes uncertainty is in
terms of the individual researche's subjective uvaluations of the options being
weighed.
He is, by and large, right.
While the princple is often phrased in terms of genuine doubt, knowledge, or
uncertainty, the way these states are treated is significant.
Fried is concerned with a \textquote{balance of opinion} or or even a
\textquote{posture of doubt}.
Shaw and Chalmers assert that \textquote{Clinicians who believe that they know
hich therapy is best should not participate in a comparative trial.}\footcite[p.
494]{shaw_ethics_1970}
This state of belief is also frequently describes in terms of opinions.
Schafer discusses treatment preference which \textquote{falls well short of
knowledge, even when it consists of an intuitive hunch or is based upon
uncontrolled cilincal experience or data from poorly designed earlier
trials.}\footcite[p. 5]{schafer_commentary_1985}
These characterizations make senes if we are talking about equipoise in an
individual.
After all, if equipoise is a state of a balance of evidence, or how we weigh the
evidence, this all represents evidence that must be weighed up after a fashion.

Freedman's critique, however, hinges on theoretical equipoise's requirement of
equivalent evidence rather than the quality or sufficienty of the evidence.
The problem he is concerned with is how researcher's maintain their balance on a
knife's edge over the course of a clinical trial.
After all, in the lengthry conduct of a trial, researchers will be exposed to
lengthy results or the data they themselves have collected and must rationally
update their credences in light of the new evidence.
This is a problem also recognized by Shaw and Chalmers.
Their solution is two-fold: first is that randomization should begin as early as
possible and second \textquote{is to keep the results confidential from the
  participating physician until a peer review group says that the study is
over.}\footcite[p. 493]{shaw_ethics_1970}
This is rejected by Freedman as an unethical agreement in a patient-centered
understanding of the obligations clinicians have.
Instead he proposes what the terms clinical equipoise.
He hopes to make the principle more stable by taking it out of the individual's
psychology.
What mattes, to his mind, is disagreemnt in the medical community over the
preferred treatement.
Accordingly, Freedman holds clinical equipoise exists when \textquote{There is
  no consensus within the experct clinical community about the comparative
merits about the alternatives to be treated.}\footcite[p.
430]{freedman_equipoise_1987}
Given this embodies equipoise in the medical community at large, it is not
vulnerable to the problems of theoretical equipoise.
The opinions and preferences of involved individual clinicans do not defease the
lack of social consensus by their own strength alone.
Moreover, clinical equiopise is not disturbed by the slightest accretion of new
evidence.
It is only when there is evidence \textquote{convincing enough to resolve the
dispute among clincians} that we exit a state of equipoise.\footcite[p.
430]{freedman_equipoise_1987}

However, I believe Freedman's critique does no go far enough.
Rather, what we ought t osay is that equipoise itself is a faulty concept.
If what is morally relevant is the fact of knowledge or uncertainty, it is odd
to turn to the concept of equipoise.
It may be a salient distinction given the way it is set up by the moral dilemma
of randomized clinical trials.
Nevertheless, it is an epistemological red herring.
Equipoise underdetermines uncertainty.

It is a problem if we do not attend to the distinctions between different
doxastic attitudes.
If the mora lframework underlying the permissibility of randomized clinical
trials rests on the epistemic state of knowledge or uncertainty, we ought to be
clear-eyed about what constitutes these doxastic attitudes.
We have all kinds of beliefs, some kind of attitud towards the truth or falsity
of a proposition, of varying levels of confidence or credence.
That confidence or credence can be more or less reflective of the rational
evaluation of the body of evidence regarding the truth value of that
proposition.
What bears emphasizing is that knowledge is a very particular kind of belief.
Traditionally, knowledge is defined as justified true belief.\footnote{For the
  sake of parsimony, I will frame the following discussion in terms of a
  justified true belief account of knowledge. What is functionally relevant for
  my account is that knowledge is a doxastic attitude with a more rigorous
  epistemic standard. I take it to be the case that what is described below can
  be reached in alternative accounts of knowledge. If you are persuaded by
  Gettier, I suspect the responses would accept this argument. I am partial to
  Nozickean truth-tracking. If you are a Baysian probablist, then it suffices
  that only some particular threshold, say 95\% confidence for example, can be
  appropriately described as \textquote{knowledge} while something like, say,
  5\% confidence cannot be. The vagueness or aritrariness of that line of
distinction shouldn't be an insurmountable problem.}
All theese features are seen to be necessary for the existence of knowledge.
It would seem contradictory to claim that we know but don't belief that Saint
Louis is in Missouri.
It also seems strange to say that we know Saint Louis is in Missouri if it
actually was the case that it was on Mars.
For our own current inquiry, neither of these two conditions are
relevant.\footnote{Even if you are an infallibalist, it may seem like the
  stipulation that knowledge be true is too demanding for a principle justifying
  the permissiblity of RCTs since we are rarely, if ever, going to be 100\%
  confident about the conclusions drawn from the collected data. Sffice it to
  say, for now, that I don't take it to be the case that we know that we know
  for the standard of permissiblity. I presume that a commitment to tracking th
  etruth and some account of sufficient proximity to knowledge is good enough.
  This, atfre all, is the bset science can strive for.}
What is relevant is that knowledge must be justified.
That is to say, we must have the right evidence or reason behind that belief for
it to constitute knowledge.
Notably, this bars true belief that is not well justified from being knowdlege.
Getting it right is not enough: we must also show our work.
After al, true beliefs might just be a luck guess.
If I decide to belief that Saint Louis is in the state where a dart thrown at a
map lands on and it just surrpenditiously lands on Missouri, it would be strange
to assert that I have genuine knowledge about the location of the city despite,
or rather becase of, mere coincidence.
Moreover, mere confidence is insufficient for knowledge.
No matter how sincere or deep our conviction that some proposition is true, it
does not rise to the level of knowdlege without the right justification.
In fact, given the many ways in which our judgement can be biased, we ought to
be skeptical of assertions made off the back of mere confidence alone.

Finally, when we talk of uncertainty, at least in theis instance, it is evident
that what is significant is the absence of knowledge.
That is to say, what is morally relevant in this case is that we lack certainty,
or knowledge, about the relative merits of the experimental candidates.
Notably, this conception does not entail that we have no confidence or credence
at all.
We, in facet, may have all kinds of doxastic attitudes, appropriately or
inappropriately, short of knowledge and be in a position of genuine
uncertainty.\footnote{Again, if this seems overly demanding, see the weaker
thesis in the prior footnote.}
To make sense of this position, imagine we are entertaining putting a bet on who
will win the next F1 grand prix.
If I want to get a payout, I should choose a driver I belief will, or at least
can, win.
As such, (i weigh up various reasons to set my credences to.
I might pick Lewis Hamilton for subjective but sincerly held convictions---I
could just be a fan or I could just have a deep antipathy for his rival Max
Verstappen.
I could also have more objective reasons: Hamilton is statistally the most
successful driver of all time, his team, Mercedes, has won the last seven
championships, the car seems well suited for the trac, and so on.
The point is, I could rationally have confidence based on the evidence that I
have in Hamilton being a safe, or at least reasonable, bet which is born out by
the payouts set by the bookies.
Despite this reasonable and sincere conviction, it would be strange to ascribe
that I know or that I am certain. 
Otherwise, the rational thing to do, if I have certainty, it to arbitrage all
the money I can get a hold of.
Clearly, however, this would be gambling.
This doesn't seem prudential because we recognize that there is uncertainty.
For all my confidence in Hamilton, rational or otherwise, we recognize that the
evidence available to me still falls short of the kind of justification
necessary for knowledge.
I can only know when the chequered flag is waved.\footnote{Verstappen won.}

Understanding this, the epistemic mistep becomes apparent.
If what is morally significant is knowledge, or uncertainty, a state of
equipoise is irrelevant despite its salience.
Ultimately what is in question, epistemologially, is whether we have sufficient
justification or not.
In light of this, the epistemic issues of the various theories of the principle
of equipoise emerge.
The kind of subjective reasons grounding the various doxastic attitudes other
than knowledge like opinion, preference, or mere belief fall short of the kind
of good evidence we need to know something like the efficacny of a
pharaceutical.
The various kinds of things that might disrupt theoretical equiopise,
\textquote{including data from literature, uncontrolled experience,
  considerations of basic science and fundamental physiological processes, and
  perhaps a `gut feeling' or `instinct' resulting from (or superimposed on)
other considerations}\footcite[p. 429]{freedman_equipoise_1987} do not unsermine
uncertainty in this case since the relevant empirical findings that arise out of
a randomized clinical trial itself.
Clinicians who believe that they know the efficacy of an intervention before any
Phase II trials are simply mistaken.
We cannot know, or have justified true belief, prior to the completion of the
processes that gives us the relevant empiric justification that would instntiate
knowledge in the first place.
It is an epistemological mistake to think that these thengs move us out of the
relevant kind of uncertainty.

Clinical equipoise is also problematic on this account.
While Freedman moves closer to the right account, a socially embodied conception
of equipoise is at best an indirect way of tracking what really matters.
It is true that epistemic peer disagreement should indicate a lack of knowledge,
or at least, unceratinty as all involved suspend their judgement.
It is also the case that we would hope that the kind of evidence that would
disturb clinical equipoise is the kind of justification sufficient for
knowledge.
However, social consensus itself is not a reliable correlate to knowledge.
Medical history over the past centuries is marked, by and large, by consensus on
beleifs that are neither true nor justified.
For some three decades, the sanitary practice of hand washing in a clinical
setting was rejected by the medical community despite empirical findings
demonstrating its efficacy.
Consensus is ultimately the aggergation of less than perfectly rational agents
weighing up the same kind of subjective and objective, sound and unsound,
reasons above.
Given the real possibility of false positives, consensus is not a reliable
indicator of knowledge.\footnote{I will note that practically speaking we may at
  time have to settle for consensus in pragmatic affairs such as legal concerns.
  Freeman notes these kinds of considerations in justifying clinical equipoise.
  Still, it is a problem to conflate nonideal practicalities with the actual
  epistemology. Nor do I think this poses a problem. The account described below
doesn't seem to bme to be any less practible than clincal equipoise.}
I agree with Freedman that what matters is genuine uncertainty.
The asnwer, however, is not clinical equipoise.

Moreover, we should questoin the need for a conception of equipoise at all.
If equipoise is supposed to capture this epistemic state where we as individuals
or we as a community give equivalent weightiness to all the possiblities in a
randomized clinical trial, then we should recognize that it is not a necessary
condition for uncertainty at all.
It's sufficient.
Ift I am in a state of equipoise, then it is reasonable to infer that I do not
know.
But equipoise underdetermines uncertainty.
As described above, I can have all kinds of doxastic beliefs and still be
uncertain if I lack the necessary justification for knowledge.
Equipoise is salient given that it describes a kind of idealized form of
experimental trial.
However, it is sufficient only insofar as it entails the lack of relevant
justification, not because of the particulare epistemic state of giving equal or
equivalent weightiness to all the relevant opitions.
Indeed, we shouldn't expect it, given the costs of getting a human trial and the
costs of conducting one.
It seems pefectly reasonable to have some level of confidence that your
candidate will work.
To maximize the chance of a payout and avoid dead losses, it is reasonable to do
what you can to find the safe bet.

If the relevant condition for knowledge is justification, then we ought to be
attending to justification itself.
Remember, though, that we are concerned with the conditions necessary for the
moral permissibility of randomized clinical trials.
While epistemically sound, we might be worried that this account sacrifices the
obligations we have to the patient for that of the advacnement of scientific
knowledge.
The kind of epistemic concenrs described might be seen as motivating
intellectual or even prudential duties to inquire.
Some critics of the princpile of equipoise make that case.
Meier concludes that \textquote{there can be no experimenting unless we are
  prepared to yield some expected gain for some of the subjects of the
  experiment. We will do a better job, I think, if we face up to that fact, and
  try to take such costs into account explicitly, rather than, as at present,
  accounting for them internally and informally, to no one's advantage that I
can see.}\footcite[p. 640]{meier_terminating_1979}
Similarly, Schafer suggests, \textquote{It is at least arguable that potential
  benefits to society and to future generations of patients require the
  subordination or sacrifice of some patien rights. Perhaps traditional
  physicians ethics, with its highly individualistic commitment to patient
welfare needs to be modified.}\footcite[p. 6]{schafer_commentary_1985}
However, it is not clear to me that we have to bite this bullet.
While equipoise might be epistemologically faulty, the underlying intuition
seems right.
Moreover, if we are concerned with what we are ethically permitted to do in a
state of uncertainty, then we can go further.
Given what is at stake in medicine, we have both a moral and intellectual duty
to inquire and, thus, conduct appropriate randomize clinical trials.
In place of a principle of equipoise, we should adopt of a principle of
evidentialism.

William Clifford gives the famous example of the ship owner.
The ship is about to sail laden with emigrants.
The owner considers the journey, recognizing the ship is old with much use and
has often needed repairs.
Doubts arise regarding the ships seaworthiness.
He considers having it, at great cost, overhauled and refitted.
But the ship owner puts away these unhappy thoughts: the ship has sailed many
times and weathered many storms.
It has been safe and seaworthy in the past and it is ungenerous to suspect the
character and competence of the shipwrights.
Suppose the ship owner acquires a sincere conviction that all is well.

Suppose, then, the ship sinks.
Clifford asserts, \textquote{What should we say of him? Surely this, that he was
verily guilty of the death of these men.}\footcite[p. 339]{clifford_ethics_1886}
His sincere belief is not exculpatory because what is morally relevant is that
\textquote{\textit{he had not right to believe on such evidence before
him.}}\footcite[p. 340]{clifford_ethics_1886}
We might be tempted to call this mere negligence.
However, Clifford suggests that what is at issue is over and beyond the fact
that the ship owner failed to exercise sufficient prudence.
Suppose instead that the ship happily makes it to her destination safe and
sound.
Clifford maintains that there is moral wrongdoing: \textquote{Will that diminish
  the guilt of her owner? Not one jot. When an action is once done, it is right
  or wrong forever; no accitental failure of its good or evil fruits can
possibly alter that.}\footcite[p. 340]{clifford_ethics_1886}
Regardless of the outcomes, the ship owner was in a state of objective
uncertainty and subjected others to bear the risks without doing his due
epistemic diligence.

This may appear to be operating on some sort of utilitarian sacrifice of
patient-centered obligations for the sake of scientific benefit.
However, Clifford holds that a fidelity to well-formed beliefs---that is to seek
and appropriately consider the relevant evidence---is itself a moral duty.
There are a couple of waysin which he motivates this claim.
The first is that to do otherwise is a wrongful act of harm to others.
Our beliefs, especially assertions of knowledge, are not merely private but are
entered into a shared epistemic community.
As a result, we have a responsibility to he members of that community to not
accept beliefs on insufficent evidence.
As Clifford puts it, \textquote{That duty is to guard ourselves from such
  beleifs as from a pestilence, which may shortly master our own body and then
  spread to the rest of the town. What would be thought of ne who; for the sake
  of a sweet fruit, should deliberately run the risk of brining a plague upon
his family and his neighbors.}\footcite[p. 344]{clifford_ethics_1886}
I think this is a promising line of argument and one I hope to develop further
later on.
More relevant for our current line of thought is that Clifford also argues that
trust itself requires us to be evidentialist.

If this is the case, then what is at stake in how we handle the ethics emerging
from uncertainty in medicine is not just scientific advancement and the
instrumental value derived from it but the moral soul of health care itself.
This is counter to how the argument regarding randomized clinical trials is
often described---as \textquote{an apparent conflict between the obligation to
patients and the obligations to advance scientifically validated
knowledge.}\footcite[p. 4]{schafer_commentary_1985}
The moral significance of medicine is taken to be these obligations to the
patien.
As Pellegrino and Thomasasma identify, \textquote{\ldots the nature of the
  healing relationship is itself the foundation for the special obligation of
physicians as physicians.}\footcite[p. 40]{pellegrino_virtues_1993}
This relationship is not merely a professional one: it is a relationship of
care.
The nature of illness both instantiates and defines this relationship.
Pellegrino and Thomasma give five characteristics of the clinician-patien
relationship: the inequality of the medical relationship, the fiduciary nature 
of the relationship, the moral nature of the medical decision, the nature of 
medical knowledge, and the inerradicable moral complicity of the physician in
whatever happens to the patient.
The two which pertain to this discussion is the inequility and tiduciary
nature of the medical relashionip.
They highlight that at the foundation f the physician-patient relationship, as
with any relationship of care, the must be trust.
However, this trust is shaped by the power dynamics involved.
Given the fact of sickness, \textquote{n this state, the sick person is forced
  to consult another person who professes to hold the needed knowledge and skill
and who, therefore, has power over the sick person.}\footcite[p.
42]{pellegrino_virtues_1993}
This inequality is exacerbated further beyond the vulnerable but also the
asymetric acess to the resources to address that vulnerablitly---particularly,
medical knowledge.
This creates a trusting patient and a trusted clinician.
After all, \textquote{To seek professional help is to trust that physicians
  posses the capacity to help and heal. From the very first moment, the patient
  perfoms an act of trust: first, in the existence and utility of medical
  knowledge itself, and then in its possession by the one who is being
consulted.}\footcite[p. 68]{pellegrino_virtues_1993}
So, then, if trust lies at the heart of a patient-centered ethic, we should ask
what it demands.

Critics of the principle of equipoise argue that its demands are incompatible
with randomized clinical trials.
Typically they take it that the clincian-patien relationship puts obligations on
the clinician to exercise their judgement to best serve the interests of the
particular patient in front of them.
Randomized clinical trials are problematic, then, because they take the
treatement decision out of the hands of the clinician as a consequent of
experimental design. As suchk they hold that \textquote{it may not be true, in
  fact, that the not-yet-validated beliefs of physicians are as likely to be
  wrong as right. The greater certainty obtained with a randomized clinical
  trial is beneficial, but that does not mean that a lesser degree of certainty
  is without value. Physicians can acquire knowledge through methods other than
  the randomized clinical trial. Such knowledge, acquired over time and less
  formally than is required in a randomized clinical trial, may be of great
value to a patient.}\footcite[p. 1587]{hellman_mice_1991}
Scitifically sound evidence is nice, but if the clinician's judgement and ad hoc
art or craft passed down on the authority of tradition is sufficent, then care
seems to overide the need for the kind of empirical justification necesarry for
knowledge.
Rather, if the relationship of care and antecedent obligation emerge from trust,
then evidentialism suggests that the pursuit of truth is necessary to
instantiate genuine trust, and therfore a relationship of care in the first
place.

Evidentialism recognizes that the kind of trust at work in the clinician-patient
relationship bears, in part, an epistemic component.
As Pelegrino and Thomasma notes, the trust of a patien begins with a faith that
the clinician has medical knowledge.
We should not, however, suppose that the good judgement of the clinician, honed
by the craft passed down by generations and guided by good will, is sufficient.
The kind of trust in question, after all, is a trust in authority.
For that trust to be well placed, that authority must be legitimate.
Clifford notes these requirements: \textquote{In order that we may have the
  right to accept his testimony as ground for believing what he says, we must
  have reasonable grounds for trusting his \textit{veracity}, that he is really
  trying to speak the truth so far as he knows it, his \textit{knowledge}, that
  he had the opportunities of knowing the truth about the matter; and his
  \textit{judgment}, that he has made proper use of these opportunities in
coming tho the conclusion which he affirms.}\footcite[p.
348]{clifford_ethics_1886}
These are all necessary conditions for turst in the assertions of authority to
ge genuine.
In considering evidentialism, it is apparten that insufficent attention as been
put on the requirement of knowledge for trust.

Moreover, we are given the framework to understand why there must be a science
of medicien over and above an \textquote{art}.
For one, as reliable as we are, today, we must recognize that that reliability
stands on the shoulders of millennia of misguided medicine.
The science of medicien that we take for granted is still recent in the grand
scheme of thins.
The American medical community, entrenched in the miasma theory of disease, was
resistant to the practice of washing their hands and the germ theriy behind it
up until the turn of the \nth{20} century.
Moreover, given the cognitive science of the past few decades, we sohuld wonder
if we can put faith in informal and anecdotal experimentation conducted by less
than perfectly rational human beings with a vested interests of the sort
suggested by Hellman and Hellman.
The claim here is not to doubt the good intentions of those involved, but to
note that experimental design is as it is in order to safeguard from human
biases.
\textquote{It is hardly in human nature that a man should quite accurately gauge
  the limits of his own insight; but it is the duty of those who profit by his
  work to consider carefully where he may have been carried beyond it. If we
  must needs embalm his possible errors along with his solid achievements, and
  use his authority as an excuse for believing what he cannot have known, we
make of his goodness an occasion to sin.}\footcite[p. 352]{clifford_ethics_1886}
In appealing to the tradition of medicine, we must recognize that we also refer
to its long history sopping with bile, phlegm, and blood.

Given that this is the case, the need for a science based approach in medicine
is clear.
Aside from the practical fruits, genuine medical authority depends upon it.
\textquote{Thus it is to be observed that his authority is valid because there
  are those who question it and verify it; that it is precisely this process of
  examining and purifying that keeps alive among investigators the love of that
  which shall stand all possible tests, the sense of public responsibility as of
  those whose work, if well done, shal remain as the enduring history of
mankind.}\footcite[p. 354]{clifford_ethics_1886}
Without the existence of the empirical evidence necessary for justified true
beliefs, clinicians cannot, in good faith, lay a claim to medical knowledge.
Without this ground for their authority, the faith patients put in their
clincians does not give rise to genuine trust.
Without genuine trust, then the relationship between clinician and patien is not
one of true care.
It may be an amicable one, of faith and good will.
It may be a healing one, based on true belief.
But given the intrinsic inequality of the clinician-patient relationship, the
power the clinicain imposes on the patient, the obligations of care require a
pursuit of trust.
It is not enough to suggest clinicians be more upfront about what they know and
don't know.
Ultimately, they use their power to make assertions of knowledge (diagnosis and
  prognosis).
Moreover, most patients labor uunder the pretense that this is the case anyways.
To make good on this, we should be committed to seeking empirical evidence as
appropriate.\footnote{This obviously does not endorse wanton human
  experimentation. There must be an appropriate balance of considerations with
this being one of them.}

If we are concerned with the ethical permissibility of randomized clinical
trials, then we ought to be evidentialist because it gets right the ethics of
uncertainty.
While the underlying intuition is right, the princpile of equipoise is
epistemologically imsguided.
Evidenditalism, however, preserves that intuition through a greater fideltiy to
the epstimelogy.
It more accuratly describes when we have knowledge and when we are uncertain and
by, by doing so, provides a framework for the permissibility of randomized
clinical triilas along the same reasoning: that wo do not do wrongful harm
because in the absence of knowledge, it is not intentional withholding.
Moreover, evidentialism gives us a moral principle to pursue randomized clinical
trials on the basis of the relationship of care between clinician and patien.
If what we know matters, then what it is to know matters.

\printbibliography

\end{document}
