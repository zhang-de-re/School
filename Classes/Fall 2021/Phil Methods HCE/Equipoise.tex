\documentclass[letterpaper,notitlepage,12pt]{article}
\usepackage[
  letterpaper,tmargin=1in,bmargin=1in,lmargin=1in,rmargin=1in]
{geometry}
\usepackage[backend=biber]{biblatex-chicago}
\usepackage{csquotes}
\addbibresource{workscited.bib}
\usepackage{setspace}
\usepackage[super]{nth}

\title{Equipoise or Evidentialism: Ethics of Uncertainty in Randomized Clinical
Trials}
\author{Alexander Zhang}
\date{}

\begin{document}

\maketitle

\section{Introduction}

% insert 1
It is a common intuition that if we can prevent some evil, unnecessary harm and
suffering, then we morally ought to do so.
Intuition, of course, only gets us so far before a more considered, analytical
methodology is required.
How we ought to prevent evil, for example, is partly determined by our actual
capabilities to intervene.
This is well explained in the general moral philosophy literature.
Somewhat less well explored, in the bioethics literature, is the epistemological
element particularly as it intersects with medical research: if we
\textit{believe} we can prevent some evil, then are we morally compelled to
prevent it?

Certainly, the bioethics field, by and large, shares the intuition that if we
\textit{know} we can prevent evil through some medical intervention, then we
ought to barring some competing interest.
It is a clear abuse to harm a patient's interests by intentionally withholding
the best known treatment available.
There is, however, an epistemological gap between knowing and believing.
This distinction is pertinent to the application of this moral intuition to
medical research.
Take, for example, the moral controversy regarding the Phase II trial design for
vaccine candidates for the Ebola virus during the West African epidemic of
2013--2016.

This large-scale international collaboration to respond to the epidemic was
largely well regarded with one such trial, the Ebola \c{C}a Suffit-Guinea
vaccine trial, widely reported to have shown the candidate gave \textquote{100
percent protection.}\footcite{NYT}
The candidate, however, was unproven and was distributed through a clinical
trial seeking to determine its efficacy.
The design of this trial was subject to heated debate from strong opposition to
a randomized controlled trial.
Representatives of M\'{e}dicins San Fronti\`{e}res asserted, \textquote{Studies
  on efficacy in affected countries and more so in at-risk populations should
  not have a placebo or active control arm as this cannot be defended
ethically}\footcite{Science}
Ultimately, a stepped-wedge scheme was chosen for the trial design.
Notably, however, one such stepped-wedge trial, the Ebola \c{C}a Suffit-Guinea
trial, preemptively ended the delayed vaccination arm, the element of the
stepped-wedge design that substituted the control arm, after the interim
results.
The Data Safety Management Board concluded it \textquote{would be unethical to
  deny people access to this life-saving intervention when the interim analysis
showed that rVSV-ZEBOV is both safe and efficacious.}\footcite{UF}
We should note, however, that on an intention-to-treat analysis, the gold
standard, the final results found a vaccine efficacy of 65 percent with a 95\%
confidence interval of -47--91\% with a p-value of 0.344.\footcite{HR2016}
In other words, statistically speaking, it is feasible on the grounds of this
analysis of the data that the vaccine candidate had zero or even negative
efficacy.
The quoted 100 percent protection was derived from an on-treatment
analysis---not protected from bias by randomization---which had a 95\%
confidence interval of 69--100\%.\footcite{HR2016}
Significant for us, there are two points at which it was declared unethical to
withhold an unproven experimental vaccine on the basis of the aforementioned
moral intuition and the strength of belief: before any empirical evidence
regarding efficacy was collected at all and before the empirical evidence rose
to the level of sufficient statistical significance.
In either case, however, it would have or did prevent collecting the strong
empirical evidence necessary to determine the efficacy of the candidate in
humans.
As such, we can note that acting on the intuition \textquote{If we \textit{believe} we
can intervene, we should} can subvert the conditions necessary to act on the
intuition \textquote{If we \textit{know} we can intervene, we should.}

The principle of equipoise has become widely established as an essential
epistemic feature of the moral framework that permits randomized clinical
trials.
Shaw and Chalmers formalize the intuition principle as follows:
\blockquote{If the clinicians, or has good reason to believe, that a new
  therapy (A) is better than another therapy (B), he cannot participate in a
  comparative trial of Therapy A versus Therapy B. Ethically, the clinician is
  obligated to give Therapy A to each new patient with a need for one of these
therapies.\footcite[p. 487]{shaw_ethics_1970}}
What's worth noting is that the concern here has widespread ramifications.
Consider that, in the best case scenario for the researcher in a randomized
controlled trial, the control entails that if one course of treatment was better
than the alternatives, some portion of the participants were not given the best
intervention available.
In this way all randomized clinical trials have this in common.

% insert 2
There is something not quite right with this picture, however.
The issue rests on the distinction between the propositional attitudes of belief
and knowledge.
The function of the state of equipoise is appeal to uncertainty as exculpatory.
However, as I will argue, the principle of equipoise is epistemologically
misguided because it misapprehends uncertainty in terms of an absence of belief
rather than in terms of an absence of knowledge.
As a result, equipoise is a sufficient but not a necessary condition for
uncertainty and, thereby, morally irrelevant.
I appeal, in place of of equipoise, to a principle of evidentialism as set out
by William Clifford that \textquote{It is wrong, always, everywhere, and for
anyone to believe anything on insufficient
evidence.}\footcite[p. 346]{clifford_ethics_1886}
While I do not argue for a universal evidentialism, I do take it that Clifford's
arguments regarding epistemic authority and the ethics of belief points to a
more robust appeal to uncertainty in terms of knowledge and sufficient evidence.
Finally, I argue that we ought to be evidentialist---at least with regards to
medical knowledge, like the efficacy of pharmaceuticals, which depend on the
kind of empirical evidence produced by RCTs---because of the high stakes
involved.
Tho this end, I give an account of why the relationship of care between the
clinician and the patient is in jeopardy if we do not recognize the epistemic
duty of researchers to inquire into the facticity of such propositions via
RCTs.\footnote{I use clinician and researcher here to refer to individuals
  working in a clinical setting outside of research and individuals working in a
  research setting that may include clinical work. Quite frequently, in clinical
  trials, researchers are asked to wear both hats. Indeed, this is in part where
  the moral dilemma has been placed. I should note that some defenses of RCTs
  attempt to argue that the roles of researcher and clinician should be more
  distinct or mutually exclusive in order to prevent any ethical conflicts
  in attempting to fulfill both roles simultaneously. I do not address this line
  of argument here because it does not seem to interact with the argument
  presented. There, I think, some possible interactions between Clifford's
  social epistemology account and that line of defense, but that will not be
explored here.}

\section{Equipoise}

It will be helpful, then, to give a brief summary on the principle of equipoise.
The contemporary use of equipoise was introduced by Charles Fried.
He was primarily concerned with arguing for a patient-centered relationship
between the clinician and patent wherein the clinician has a duty to consider
the specific particularities of the patient when reaching a medical decision.
Fried thought RCTs posed an ethical dilemma. 
The requirement of informed consent in clinical research was still
contentious at the time and so there was concern, and even support, for
clinicians to make patients subjects without their knowledge or consent.
Fried advanced equipoise as a possible objection to his argument thereby
obviating Fried's primary concern of patients being sacrificed without consent
for the sake of science.
Unfortunately he does not give a formal definition of what
he means by equipoise which is now more relevant given today's common acceptance
of informed consent thereby obviating Fried's primary concern of patients being
sacrificed without consent for the sake of science.
As he puts it, \textquote{The argument is frequently made that where the balance
  of opinion is truly in equipoise there is no sense to the accusation that the
  prescribing of one or the other of the equally eligible treatments can
  constitute doing less than one's best (the alternative being no better). And
so no one sacrifices any body to anything.}\footcite[p. 58]{fried_medical_2016}
The classic characterization of equipoise, etymologically something like
\textquote{equal weight}, is set down here: that there is a balance of opinion
within an individual and, as a result, the best option is not known.

At the very least, this is characteristic of how Fried's concept of equipoise
has been interpreted since.
Freedman attributes to Fried that \textquote{it is necessary that the clinical
  investigator be in a state of in a state of genuine uncertainty regarding the
comparative merits of treatments A and B for population P.}\footcite[p.
141]{freedman_equipoise_1987}
Similarly Shaw and Chalmers propose that \textquote{If the physician (or his
  peers) has genuine doubt as to which therapy might be worse than the old. Each
  new patient must have a fair chance of receiving either the new, and,
hopefully, better therapy or the limited benefits of the old
therapy.}\footcite[p.487]{shaw_ethics_1970}
Schafer describes equipoise is when \textquote{The physician-investigator is
  able honestly to say to a patient-subject, \textquote{You will be receiving
  the best \textit{known} therapy,} because prior to completing the trial there
  is not \textit{scientifically validated} reason to predict that therapy A will
  be superior to therapy B (and there is no further alternative, C, which is
\textit{known to be better than A or B})}\footcite[p.
4]{schafer_commentary_1985}
We can see that these articulations of the principle of equipoise derived
from Fried all share an underlying intuition.
What matters morally is what did we know and when did we know it.\footnote{As
well as what we don't know, but that muddies the witty Watergate reference.}

This is what distinguishes harm from wrongful harm in the context of RCT design,
vis a vis withholding
In a randomized clinical trial, unless all arms are equally
efficacious, some patients do not receive the best intervention they could
have received.
Insofar as this is against their interests, this is a harm.
But not all harms are wrong.
It is against my interest when another job candidate receives the position
instead of me.
I may lose out substantially depending on my circumstances and the benefits and
compensation.
However, despite what harm I may feel, we would not say I was wronged or what
the hiring decision-maker did was morally wrong in most circumstances.
The exceptions here prove the rule.
To reasonably assert this is morally wrong, there must be some
additional factor over and above the mere harm.
For example, if there is something like nepotism or prejudice behind the
decision, then it makes sense to say that there was a wrongful harm.
The relevant difference lies in the intention or the will in which the action
was committed.
In the case of the randomized clinical trial, the harm becomes wrongful if it
was done so intentionally---or knowingly.
So, while the harm of withholding the best available treatment is necessary for
wrongful harm, it is not sufficient.
When accompanied with knowledge what the best treatment is and that it is
available, then these conditions are necessary and sufficient for moral
wrongdoing.\footnote{There may, of course, be cases where researchers are
  culpably ignorant. One of Fried's concerns regarding equipoise is that it
  would result in researchers adopting a posture of doubt disingenuously. I will
  not address this here because I suggest that it is an issue for equipoise but
  not evidentialism. In short, because a state of equipoise is pinned to
  subjective propositional attitudes which may not even have a mind-to-world
  direction of fit (arguably, the objection of MSF to randomization was
  predicated on hope and desired and, thusly, a world-to-mind direction of fit,
  given the lack of any empirical evidence regarding human efficacy at the
  time), it may pick out a mental state that is contingent but not relevant or
  even blameworthy. However, since evidentialism evidentialism is concerned with
  knowledge and the ethics of belief, it is straightforward to note that
  culpable ignorance is wrong because it fails to discharge our epistemic
duties.}

If this is the case, the appeal of the principle of equipoise is obvious.
If being in a state of equipoise entails that you do not know then you are not
wrongfully harming anyone by knowingly withholding treatment.
Still, while this intuition seems correct, there is something epistemically
suspect.
Freedman asserts that there are problems that \textquote{are predicated on a
faulty concept of equipoise itself.}\footcite[p. 141]{freedman_equipoise_1987}
He argues that the concept described above, what he terms theoretical equipoise,
is \textquote{both conceptually odd and ethically irrelevant.}\footcite[p.
429]{freedman_equipoise_1987}
On that framework, a trial may be conducted when the researcher gives equal
weight to all arms of the experiment.
This results in theoretical equipoise being too tenuous and unstable a state to conduct a full
clinical trial.

The details of the epistemology entailed by that claim are important.
Freedman observes that the way theoretical equipoise describes uncertainty is in
terms of the individual researcher's subjective evaluations of the options being
weighed.
He is, by and large, right.
While the principle is often phrased in terms of genuine doubt, knowledge, or
uncertainty, the way these states are treated is significant.
Fried is concerned with a \textquote{balance of opinion} or or even a
\textquote{posture of doubt}.
Shaw and Chalmers assert that \textquote{Clinicians who believe that they know
which therapy is best should not participate in a comparative trial.}\footcite[p.
494]{shaw_ethics_1970}
This state of belief is also frequently described in terms of opinions.
Schafer discusses treatment preference which \textquote{falls well short of
knowledge, even when it consists of an intuitive hunch or is based upon
uncontrolled clinical experience or data from poorly designed earlier
trials.}\footcite[p. 5]{schafer_commentary_1985}
These characterizations make sense if we are talking about equipoise in an
individual.
After all, if equipoise is a state of a balance of belief or opinion, or how we
subjectively evaluate the
options, this all represents evidence that must be weighed up after a fashion.

Freedman's critique, however, hinges on theoretical equipoise's requirement of
equivalent evidence rather than the quality or sufficiency of the evidence.
The problem he is concerned with is how researcher's maintain their balance on a
knife's edge over the course of a clinical trial.
After all, in the lengthy conduct of a trial, researchers will be exposed to
interim results or data they themselves have collected and must rationally
update their credences in light of the new evidence.
We see this with the Ebola \c{C}a Suffit-Guinea Trial
This is a problem also recognized by Shaw and Chalmers.
Their solution is two-fold: first is that randomization should begin as early as
possible and second \textquote{is to keep the results confidential from the
  participating physician until a peer review group says that the study is
over.}\footcite[p. 493]{shaw_ethics_1970}
Freedman rejects this as an unethical agreement in a patient-centered
understanding of clinician obligations.
Instead he proposes what he calls clinical equipoise.
He hopes to make the principle of equipoise more stable by taking it out of the individual's
psychology.
What matters, to his mind, is disagreement in the medical community over the
preferred treatment.
Accordingly, Freedman holds clinical equipoise exists when \textquote{There is
  no consensus within the expert clinical community about the comparative
merits about the alternatives to be treated.}\footcite[p.
430]{freedman_equipoise_1987}
Since this reflects uncertainty in the medical community at large, it is not
vulnerable to the problems besetting theoretical equipoise.
The opinions and preferences of involved individual clinicians do not defease the
lack of social consensus by their own strength.

% insert 4
Although Freedman does not frame clinical equipoise in this way, we can
understand it as a practical application of the epistemology of peer
disagreement.
That is to say, the fact of the matter that there is disagreement among a group
we might reasonably presume to be epistemic peers---more or less equally
competent as experts---is a good reason in of itself to adjust our confidence in
our belief or disbelief in a proposition to either an equal weight view or a
suspension of judgment.
Moreover, Freeman appears to hold that the fact of peer disagreement is
sufficiently strong that we should not be steadfast or, if we are, at the very
least lower our norm of assertion from knowledge to belief or even lower.
As such, clinical equipoise, or the fact of expert peer disagreement, appeals
again to a state of uncertainty to ground the permissibility of RCTs.

Moreover, clinical equipoise is not disturbed by the slightest accretion of new
evidence.
It is only when there is evidence \textquote{convincing enough to resolve the
dispute among clinicians} that we exit a state of equipoise.\footcite[p.
430]{freedman_equipoise_1987}
% insert 5
While this last point seems to look to evidential merit as a way out, note that
it is still defining uncertainty in terms of contingent subjective states: the
consensus of the medical community.
Freedman sets the bar for evidential merit to convincingness or persuasiveness
rather than knowledge.
While the two might hopefully be coincidental with evidential merit as
contingent on evidential merit, often times they are not.
As such, even clinical equipoise appeals to uncertainty in an epistemologically
dubious way.

\section{Uncertainty}

We should go further than Freedman and say that equipoise itself is a faulty 
concept.
If what is morally relevant is the fact of uncertainty, it is odd
to turn to the concept of equipoise and the way it attempts to characterize a
state of uncertainty.
It may be a salient distinction given the way it is set up by the moral dilemma
of randomized clinical trials.
Nevertheless, it is an epistemological red herring.
To make sense of this, it will be helpful to first discuss uncertainty.

If the moral framework underlying the permissibility of randomized clinical
trials rests on the epistemic states of uncertainty, we ought to be
clear about what constitutes uncertainty.
% insert 6
First, it will be helpful to elaborate on propositional attitudes.
In brief, these are the kind of cognitive, conative, or even affective ways we
might regard some proposition.
These propositional attitudes can range from believing, hoping, desiring,
predicting, wishing, fearing, loving, suspecting, expecting, and so on.
Worth noting is that some of these attitudes are more belief-like and some are
more desire-like.
This is to say, propositional attitudes like believing or knowing have a
mind-to-world direction of fit wherein our attitudes are meant to conform or
reflect what the world is actually like.
Conversely, some like desiring or hoping have a world-to-mind direction of fit
such that the world is supposed to conform to the attitudes (ideally, for the
one who holds the attitude).
Given these kinds of propositional attitudes, we should note that there are
different ways in which we assign confidence in the truth or falsity of a
proposition from the cognitive to the conative.
That confidence can be more or less reflective of the rational evaluations of
the body of evidence regarding the truth value of that proposition.

What bears emphasizing is that knowledge is a very particular kind of
propositional attitude.
Traditionally, knowledge is defined as justified true belief.\footnote{For the
  sake of parsimony, I will frame the following discussion in terms of a
  justified true belief account of knowledge. What is functionally relevant for
  my account is that knowledge is a doxastic attitude with a more rigorous
  epistemic standard. I take it to be the case that what is described below can
  be reached in alternative accounts of knowledge. If you are persuaded by
  Gettier, I suspect the responses would accept this argument. I am partial to
  Nozickean truth-tracking. If you are a Bayesian probabilist, then it suffices
  that only some particular threshold, say 95\% confidence for example, can be
  appropriately described as \textquote{knowledge} while something like, say,
  5\% confidence cannot be. The vagueness or arbitrariness of that line of
distinction shouldn't be an insurmountable problem.}
All these features are seen to be necessary for the existence of knowledge.
It would seem contradictory to claim that we know but don't believe that Saint
Louis is in Missouri.
It also seems strange to say that we know Saint Louis is in Missouri if it
actually was the case that it was on Mars.
For our own current inquiry, neither of these two conditions are
relevant.\footnote{Even if you are an infallibilist, it may seem like the
  stipulation that knowledge be true is too demanding for a principle justifying
  the permissibility of RCTs since we are rarely, if ever, going to be 100\%
  confident about the conclusions drawn from the collected data. Suffice it to
  say, for now, that I don't take it to be the case that we know that we know
  for the standard of permissibility. I presume that a commitment to tracking
  the truth and some account of sufficient proximity to knowledge is good enough.
  This, after all, is the best science can strive for.}
What is relevant is that knowledge must be justified.
While this last point seems to look to evidential merit as a way out, note that
it is still defining uncertainty in terms of contingent subjective states: the
consensus of the medical community.
Freedman sets the bar for evidential merit to convincingness or persuasiveness
rather than knowledge.
While the two might hopefully be coincidental with evidential merit as
contingent on evidential merit, often times they are not.
As such, even clinical equipoise appeals to uncertainty in an epistemologically
dubious way.
That is to say, we must have the right evidence or reason behind that belief for
it to constitute knowledge.
Notably, this bars true belief that is not well justified from being knowledge.
Getting it right is not enough: we must also show our work.
After all, true beliefs might just be a lucky guess.
If I decide to believe that Saint Louis is in the state where a dart thrown at a
map lands on and it just serendipitiously lands on Missouri, it would be strange
to assert that I have genuine knowledge about the location of the city despite,
or rather because of, mere coincidence.
Moreover, mere confidence is insufficient for knowledge.
No matter how sincere or deep our conviction that some proposition is true, it
does not rise to the level of knowledge without the right justification.
In fact, given the many ways in which our judgment can be biased, we ought to
be skeptical of assertions made off the back of mere confidence alone.

Finally, when we talk of uncertainty, at least in context of RCTs, it is evident
that what is significant is the absence of knowledge.
That is to say, what is morally relevant in this case is that we lack certainty,
or knowledge, about the relative merits of the experimental candidates.
Notably, this does not entail that we have no confidence favoring one option
over the other at all.
We may have all kinds of propositional attitudes, appropriately or
inappropriately, short of knowledge and be in a position of genuine
uncertainty.\footnote{Again, if this seems overly demanding, see the weaker
thesis in the prior footnote.}
To make sense of this position, imagine we are entertaining putting a bet on who
will win the next Formula 1 grand prix.
If I want to get a payout, I should choose a driver I believe will, or at least
can, win.
As such, (I weigh up various reasons to set my credences to.
I might pick Lewis Hamilton for subjective but sincerely held convictions---I
could just be a fan or I could just have a deep antipathy for his rival Max
Verstappen.
I could also have more objective reasons: Hamilton is statistically the most
successful driver of all time, his team, Mercedes, has won the last seven
championships, the car seems well suited for the track, and so on.
The point is, I could rationally have confidence based on the evidence that I
have in Hamilton being a safe, or at least reasonable, bet which is born out by
the payouts set by the bookies.
Despite this reasonable and sincere conviction, it would be strange to ascribe
that I know or that I am certain. 
Otherwise, the rational thing to do, if I have certainty, it to arbitrage all
the money I can get a hold of.
Clearly, however, this would be gambling.
This doesn't seem prudential because we recognize that there is uncertainty.
For all my confidence in Hamilton, rational or otherwise, we recognize that the
evidence available to me still falls short of the kind of justification
necessary for knowledge.
I can only know when the checkered flag is waved.\footnote{Verstappen won.}

% insert 7
In light of the above discussion of propositional attitudes, we can make better
sense of the kind of uncertainty the intuition underlying equipoise appeals to.
If the question is uncertainty regarding the best course of intervention, then
the kind of propositional attitude we must be interested in are the cognitive,
belief-like ones with a mind-to-world fit.
What is morally relevant is that we are not certain which interventions is
question will be most effective and, therefore, are not certain which
intervention is wrong to withhold.
These concerns hinge on the actual fact of the matter.
As a result, the kind of uncertainty in question does not hinge on the conative
desire-like propositional attitude.

This is drawn out clearly in the two instances in which it was asserted that
the Ebola vaccine trial was unethical.
In the first, MSF's objection to a randomized control arose before any empirical
evidence regarding known efficacy was collected.
This doesn't seem to appeal to the correct kind of uncertainty---or lack of
uncertainty---because there is clearly uncertainty in the cognitive sense given
the lack of any empirical evidence at the time regarding human efficacy.
There was insufficient evidence in that regard to make a sound judgement one way
or the other: we did not know.
Rather, the objection appeals to a kind of conative certainty with a
world-to-mind fit.
The crisis is dire and we want this to work: we hope that it is better than
nothing.
The problem is that this leaves actual efficacy of the candidate to moral and
epistemic luck.
We don't actually know yet if we can help.

In the second, the Ebola \c{C}a Suffit-Guinea trial's suspension of the randomly
delayed vaccination arm does seem to be cognitive rather than merely conative
compared to the MSF case.
The DSMB had based this decision after the analysis of the promising interim
results.
The question in then whether we went from a state of uncertainty to certainty at
that point.
This does not appear to be the case.
Despite misleading reporting, which is in of itself an ethical issue, the result
of the that trial by itself provide sufficient justification, on the basis of
the available evidence, that the candidate was efficacious.
Indeed, approval of the vaccine by the FDA and its equivalents only came after
subsequent trials conducted during the 2018--2020 epidemic in the Democratic
Republic of Congo.
As such, it seems again that at that point in time there was still insufficient
to dispel uncertainty: we did not know.
Certainly, there was enough evidence such that credence levels of beliefs ought
to have been updated in favor of efficacy.
However, it is significant that it fell short of justified true belief.
This is important despite rVSV-ZEBOV's efficacy because if we happened to come
to a true belief on insufficient justification, then the decision was just
epistemically lucky.
In that case, there was still uncertainty because there was an absence of
knowledge even in the presence of other propositional attitudes including
true belief.
This is still morally problematic given, as discussed later on, epistemic luck 
in this context entails moral luck.

% redraft section.
% elaborate on various reasons why need empirical evidence
% add in defeasible application - exception like animal testing of anthrax
% instead of human challenge trials
If the kind of uncertainty we are interested in is the absence of knowledge,
then what is epistemologically in question is whether whether we have sufficient
justification.
After all, if we can have true beliefs regarding the facticity of
a proposition but not have knowledge, then the only distinction is sufficient
justification.
A brief word on what kind of justification is in question here will be helpful.
Since what is fundamentally at issue is the permissibility of RCTs, then we can
assume the kind of the kind of knowledge we are interested in here is the kind
that is instantiated by the kind of empirical evidence sought after through the
methodology of randomized trials.
This move shouldn't be seen as circular but a scoping out of the kind of
propositions that are not relevant to this discussion.
Presumably, if we have no reason to think the kind of empirical methodology of a
RCT will provide the kind of justification for the proposition in question, then
it seems prima facie that the RCT is unnecessary and, therefore, unjustified.
We wouldn't be tempted to justify a randomized trial to determine if murder is
wrong or if other people are p-zombies.
The argument in this paper is not concerned with even general medical knowledge
or any bioethical issue relating to evidence and belief such as whether a
pro-life physician's sincerely-held belief is sufficient for conscientious
objection.
These are either knowable without experimentation or not knowable through
experimentation.
As such, the relevant kind of justification here---the kind that we would want a 
RCT for in the first place---is the empirical evidence that we can reasonably
hold to be needed to form knowledge and not just mere belief.
In other words, the kind that would dispel uncertainty.\footnote{It's worth
  noting that the permissibility of RCTs is defeasible, even on the
  evidentialist account, in instances when we can morally tolerate uncertainty.
  The approval of, for example, Anthrax vaccines on the basis of animal
  testing rather than human testing defeases the need for human RCTs because the
  only alternative seems to be human challenge trials---intentionally exposing
  healthy individuals to Anthrax and randomly not giving them the vaccine. The
  remaining uncertainty of human efficacy inferred from animal testing is, in
  light of that, clearly tolerable. It may be argued that Ebola and AIDS are
  other possible cases of this. The line in the sand is not going to be explored
  here. All that needs to be supposed is, first, exceptions are relatively rare
  and, second, these exceptions prove the rule. Or, at the very least, they do
  not show that any and all RCTs are impermissible even in less extreme and
tolerable cases.}

In light of this, the epistemic issues of the aforementioned accounts
of equipoise emerge.
The kind of subjective reasons grounding the various propositional attitudes other
than knowledge like opinion, preference, intuition, hunches, hope, or mere 
belief fall short of the kind
of good evidence we need to know something like the efficacy of a
pharmaceutical.
The various kinds of things that might disrupt theoretical equipoise,
\textquote{including data from literature, uncontrolled experience,
  considerations of basic science and fundamental physiological processes, and
  perhaps a `gut feeling' or `instinct' resulting from (or superimposed on)
other considerations}\footcite[p. 429]{freedman_equipoise_1987} do not undermine
uncertainty in this case since the relevant empirical findings arise out of
a randomized clinical trial itself.
Clinicians who believe that they know the efficacy of an intervention before any
Phase II trials are simply mistaken.
We cannot know, or have justified true belief, prior to the completion of the
processes that gives us the relevant empiric justification that would instantiate
knowledge in the first place.
It is an epistemological mistake to think that these things move us out of the
relevant kind of uncertainty.

Clinical equipoise is also problematic on this account.
While Freedman moves closer to the right account, a socially embodied conception
of equipoise is at best an indirect way of tracking what really matters.
It is true that epistemic peer disagreement should indicate a lack of knowledge,
or at least, uncertainty as all involved suspend their judgment.
We would hope that the kind of evidence that would
disturb clinical equipoise is the kind of justification sufficient for
knowledge.
However, social consensus itself is not a reliable correlate to knowledge.
Medical history over the past centuries is marked, by and large, by consensus on
beliefs that are neither true nor justified.
For some three decades, the sanitary practice of hand washing in a clinical
setting was rejected by the medical community despite empirical findings
demonstrating its efficacy.
Consensus is ultimately the aggregation of less than perfectly rational agents
weighing up the same kind of subjective and objective, sound and unsound,
reasons.
Given the real possibility of false positives, consensus is not a reliable
indicator of knowledge.\footnote{I will note that practically speaking we may at
  time have to settle for consensus in pragmatic affairs such as legal concerns.
  Freeman notes these kinds of considerations in justifying clinical equipoise.
  Still, it is a problem to conflate nonideal practicalities with the actual
  epistemology. Nor do I think this poses a problem. The account described below
doesn't seem to be to be any less practicable than clinical equipoise.}
I agree with Freedman that what matters is genuine uncertainty.
The answer, however, is not clinical equipoise.

Moreover, we should question the need for a conception of equipoise at all.
If equipoise is supposed to capture the epistemic state where we as individuals
or a community give equivalent weightiness to all the arms in a
randomized clinical trial, then it is not a necessary
condition for uncertainty at all.
It's sufficient.
IF I am in a state of equipoise, then it is reasonable to infer that I do not
know.
As described above, I can have all kinds of propositional attitudes and still be
uncertain if I lack the necessary justification for knowledge.
Equipoise is salient given that it describes a kind of idealized form of
experimental trial.
However, it is sufficient only insofar as it entails the lack of relevant
justification, not because of the epistemic state of giving equal or
equivalent weight to all the relevant options.
Indeed, we shouldn't expect a state of theoretical equipoise, at the very least, 
given the costs of getting a human trial and the
costs of conducting one.
It seems perfectly reasonable to have some level of confidence that your
candidate will work.
To maximize the chance of a payout and avoid dead losses, it is reasonable to do
what you can to find the safe bet.
% add in wrap up
It seems, then, that equipoise undetermines uncertainty.

\section{Evidentialism}

To draw out what a principle of evidentialism regarding the moral permissibility
of RCTs, ti will be first helpful to note some standing objections made against
the principle of equipoise and RCTs in general.
After which, I will elaborate on what the evidentialist principle is and
conclude on a note of how it defends against criticism of equipoise given that
the epistemic duty of researchers to inquire is necessary to instantiate care 
between patient and clinician at all.

Equipoise, as the predominant framework for the permissibility of RCTs, is often
the target for ethical critique of RCTs full stop.
Frequently, these critiques characterizes the disagreement as Schafer does:
\textquote{This argument represents an attempt to resolve an apparent conflict
  between the obligation to patients and the obligation to advanced
  scientifically validated knowledge\ldots There is an unresolved (and perhaps
irresolvable) ethical conflict.}\footcite[p. 4--5]{schafer_commentary_1985}
This tension is sometimes characterized as a kind of utilitarian sacrifice:
\textquote{the randomized clinical trial routinely asks physicians to sacrifice
  the interests of these particular patients for the sake of the study and that
of the information that it will make available for the benefit of
society.}\footcite[p. 1566]{hellman_mice_1991}

If this is the case, then what is at stake in how we handle the ethics of belief
emerging from uncertainty in medicine is not just scientific advancement and the
instrumental value derived from it but the moral soul of health care itself.
The moral significance of medicine is taken to reside in obligations to the
patient.
As Pellegrino and Thomasma identify, \textquote{\ldots the nature of the
  healing relationship is itself the foundation for the special obligation of
physicians as physicians.}\footcite[p. 40]{pellegrino_virtues_1993}
This relationship is not merely a professional one: it is a relationship of
care.
The nature of illness both instantiates and defines this relationship.
Pellegrino and Thomasma give five characteristics of the clinician-patient
relationship: the inequality of the medical relationship, the fiduciary nature
of the relationship, the moral nature of the medical decision, the nature of
medical knowledge, and the ineradicable moral complicity of the physician in
whatever happens to the patient.

Critics of RCTs like Hellman and Hellman argue that the demands of a
patient-centered ethics are incompatible with RCTs.
Typically they take it that the clinician-patient relationship puts obligations
on the clinicians to exercise their judgement to best serve the interests of the
particular patient in front of them.\footcite[p. 1585]{hellman_mice_1991}
RCTs are problematic, then, because they take the treatment decision out of the
hands of the clinician as a consequence of experimental design.
For some moderates, this simply means accepting the trade off and attempting to
lessen the moral cost by adjusting high standards of disclosure and informed
consent.
As Schafer suggests, \textquote{It is at least arguable that potential
  benefits to society and to future generations of patients require the
  subordination or sacrifice of some patient rights. Perhaps traditional
  physicians ethics, with its highly individualistic commitment to patient
welfare needs to be modified.}\footcite[p. 6]{schafer_commentary_1985}
Meier is willing to go further and concludes that \textquote{there can be no experimenting unless we are
  prepared to yield some expected gain for some of the subjects of the
  experiment. We will do a better job, I think, if we face up to that fact, and
  try to take such costs into account explicitly, rather than, as at present,
  accounting for them internally and informally, to no one's advantage that I
can see.}\footcite[p. 640]{meier_terminating_1979}
It is not clear, however, that we have to bite this bullet in response to
objections like Hellman and Hellman's.

The issue on either side is the failure to make the distinction between
different kinds of propositional attitudes and uncertainty.
Given that this criticism of equipoise is already discussed above, I will only
briefly address how the objections also make the same mistake.
Hellman and Hellman assert the following: \blockquote{The study may create a
  false dichotomy in the physician's opinions: according to the premise of the
  randomized clinical trial, the physician may only know or not know whether a
  proposed course of treatment represents an improvement; no middle position is
  permitted. What the physician things, suspects, believes, or has a hunch about
  is assigned to the \textquote{not knowing} category, because knowing is
  defined on the basis of an arbitrary but accepted statistical test performed
  in a randomized clinical trial. Thus, little credence is given to information
  gained beforehand in other ways or to information accrued during the trial but
  without the required statistical degree of assurance that a difference is not
due to chance.\footcite[p. 1586]{hellman_mice_1991}}
Moreover, they respond to the critique of the usefulness of the physician's
beliefs and opinions before validation in an RCT---as they characterize it, that
\textquote{these not-yet-validated beliefs are as likely to be wrong as
right}---with the following: \blockquote{it may not be true, in
  fact, that the not-yet-validated beliefs of physicians are as likely to be
  wrong as right. The greater certainty obtained with a randomized clinical
  trial is beneficial, but that does not mean that a lesser degree of certainty
  is without value. Physicians can acquire knowledge through methods other than
  the randomized clinical trial. Such knowledge, acquired over time and less
  formally than is required in a randomized clinical trial, may be of great
value to a patient.\footcite[p. 1587]{hellman_mice_1991}}

Again, this conflates propositional attitudes.
It misidentifies subjective confidence with certainty in claiming that there is
a false dichotomy.
There is some suggestion that the conative world-to-mind propositional attitudes
give sufficient certainty on their account: \textquote{Even though the
  therapeutic value of the new agent is unproved, if the physicians think that
  it has promise, are they acting in the best interests of their patients in
allowing them to be randomly assigned to the control group?}\footcite[p.
1586]{hellman_mice_1991}
It might be responded that their inclusion of cognitive propositional attitudes
like belief that, in their mind-to-world fit, increase reliability through
attending to alternative evidence sought outside the rigorous empirical
methodology of RCTs.
It is a mischaracterization, at least on my view, to think that all
propositional attitudes other than knowledge are equally likely to be wrong as
they are right.
Similarly, it is a mischaracterization to claim here that knowledge as
predicated on some statistical measure is completely arbitrary.
There is good reason to ask whether the standards of evaluating knowledge,
arbitrary or not, actually track the distinction between knowledge and mere
belief.
Similarly, there are good reasons not to put complete faith in scientific
methods merely in virtue of them being scientific.
However, this doesn't open the door to the idea that this is all simply a matter
of opinion.
It is a problem to think that certainty here is predicated on knowledge as much
as it is on suspicion, beliefs, hunches, intuition, anecdotes, opinions, or
preferences.
We should understand certainty and uncertainty to be based on the presence or
absence of knowledge---justified true belief-0--respectively.
As discussed later on, this is due to Hellman and Hellman's observation that we
give little credence to other propositional attitudes \textit{due to lack of
assurance that a difference is not due to chance.}

It might be asked why certainty should be predicated on knowledge and not mere
belief.
This is the turn to evidentialism which holds that it is our epistemic duty to
base our beliefs only on the relevant evidence, the evidence that bears on the
truth of the proposition, and that our degrees of belief are proportional to
that evidence.
Moreover, this is a turn to Clifford's \textquote{Ethics of Belief} which holds
it is morally wrong not to do so.
To illustrate the intuition behind these claims, let's briefly consider
Clifford's example of the ship owner.
The ship is about to sail laden with passengers.
The owner considers the long journey, recognizing the ship is old with much use and
has often needed repairs.
Doubts arise regarding the ship's seaworthiness.
The owner considers having it, at great cost, overhauled and refitted.
But he puts away these unhappy thoughts: the ship has returned from many voyages
times and weathered many storms.
It has been safe and seaworthy in the past and it is ungenerous to suspect the
character and competence of the shipwrights or to not put faith in Providence.
Suppose the ship owner acquires a sincere conviction that all is well.

Suppose, then, the ship sinks.
Clifford asserts, \textquote{What should we say of him? Surely this, that he was
verily guilty of the death of these men.}\footcite[p. 339]{clifford_ethics_1886}
His sincere belief is not exculpatory because what is morally relevant is that
\textquote{\textit{he had not right to believe on such evidence before
him.}}\footcite[p. 340]{clifford_ethics_1886}
We might be tempted to call this mere negligence.
However, Clifford suggests that what is at issue is over and beyond the fact
that the ship owner failed to exercise sufficient prudence.
Suppose instead that the ship happily makes it to her destination safe and
sound.
Clifford maintains that there is moral wrongdoing: \textquote{Will that diminish
  the guilt of her owner? Not one jot. When an action is once done, it is right
  or wrong forever; no accidental failure of its good or evil fruits can
possibly alter that.}\footcite[p. 340]{clifford_ethics_1886}
Regardless of the outcomes or his subjective confidence, the ship owner was in a state of 
uncertainty and subjected others to the risks without doing his due
epistemic diligence.

What matters, then, is uncertainty as predicated on knowledge rather than on
other kinds of propositional attitudes.
Moreover, if the relevant conditions for knowledge as justified true belief is
justification, then we ought to be attending to justification itself.
The accounts from both equipoise and against RCTs fail to do this in their
conflation of the subjective confidence arising out of other propositional
attitudes as evidence towards certainty.
The alternative principle from evidentialism emerges in light of these
contrasts.
Equipoise's intuitive appeal to uncertainty seems right and should be preserved.
What is at issue is the epistemological characterization of that uncertainty.
An evidentialist framework regarding RCTs holds that they are permissible when
researchers have insufficient evidence to instantiate knowledge, and therefore
are in a state of uncertainty, and reasonably believe that the kind of empirical
evidence resulting from the methodology of a RCT would bear significantly on the
facticity of the proposition in question and is, thus, the kind of justification
that might instantiate knowledge.
As corollary to this general principle from uncertainty, evidentialism holds
that researchers have an epistemic duty to ascertain the facticity of
significant medical knowledge.
This leads to the following two points.
First, that researchers are obliged to seek out already existing evidence,
confirm already existing evidence when reasonable or prudent, and generate new
evidence that bears on the facticity of the proposition through empirical
methods like RCTs when relevant or necessary and through alternative methods
when not.
Second, researchers and clinicians are morally culpable when they have undue
confidence or undue ignorance given what degree of belief is proportional to the
body of evidence reasonably available to them.
In short, \textquote{Inquiry into the evidence of a doctrine is not to be made
  once for all, and then taken as finally settled. It is never lawful to stifle
  a doubt; for either it can be honestly answered by means of the inquiry
already made, or else it proves that the inquiry was not yet
complete.}\footcite[p. 346]{clifford_ethics_1886}

This framework rests on Clifford's arguments that a fidelity to well-formed
beliefs---that is, to seek and appropriately consider the relevant evidence
through the appropriate means---is itself a moral duty.
He gives two arguments to the effect although this account will focus only on
the second.
Very briefly, the first is that to do otherwise is a wrongful harm to
the epistemic community writ large.
The second, and the one pertinent to this discussion, is that trust itself
requires us to be evidentialist.

Evidentialism recognizes that the kind of trust at work in the clinician-patient
relationship bears, in part, an epistemic element.
As Pellegrino and Thomasma notes, the patient's trust begins on faith that the
clinician actually has the relevant medical knowledge.
The kind of trust in question is a trust in the clinician's epistemic authority.
For a clinician to be trustworthy, that authority must be legitimate.
Clifford observes three epistemic requirements for trustworthiness: \blockquote{In order that we may have the
  right to accept his testimony as ground for believing what he says, we must
  have reasonable grounds for trusting his \textit{veracity}, that he is really
  trying to speak the truth so far as he knows it, his \textit{knowledge}, that
  he had the opportunities of knowing the truth about the matter; and his
  \textit{judgment}, that he has made proper use of these opportunities in
coming tho the conclusion which he affirms.\footcite[p.
348]{clifford_ethics_1886}}
These are all necessary conditions for trust in epistemic authority to be
genuine.
It is apparent that insufficient attention as been paid to the requirement of
knowledge for trustworthiness.

Consider again Pellegrino and Thomasma's five characteristics of the
clinician-patient relationship.
Two of which pertain to this discussion are the unequal and fiduciary nature of
the medical relationship.
At the foundation of the clinician-patient relationship, as with any
relationship of care, there must be genuine trust.
However, this trust is shaped by the asymmetric power dynamics involved.
Given the fact of sickness, \textquote{In this state, the sick person is forced
  to consult another person who professes to hold the needed knowledge and skill
who, therefore, has power over the sick person.}\footcite[p.
122]{pellegrino_virtues_1993}
This inequality is exacerbated beyond that vulnerability by the asymmetric
access to the resources to address it---particularly, medical knowledge.
This creates a trusting patient and a trusted clinician.
After all, \textquote{To seek professional help is to trust that physicians
  possess the capacity to help and heal. From the very first moment, the patient
  performs an act of trust: first, in the existence and utility of medical
  knowledge itself, and then in its possession by the one who is being
consulted.}\footcite[p. 68]{pellegrino_virtues_1993}
However, just because someone is trusted doesn't make them trustworthy.
If this fiduciary relationship lies at the heart of a patient-centered ethic,
then it must be one of genuine care such that the patient's faith is
reciprocated by the clinician's trustworthiness.
As such, we should ask what such trustworthiness demands.

What it demands is knowledge---justified true belief---or, at least, the
rigorous pursuit of it and appropriate formation of belief proportional to the
available evidence.
In short, certainty.
It is not enough to ask that clinicians be more upfront about what they know and
don't know.
Given the intrinsic inequality of the clinician-patient relationship, the power
the clinician imposes on the patient, the obligations of care requires a pursuit
of trustworthiness.
Ultimately, clinicians use their power to make assertions of knowledge
(diagnosis and prognosis).
Moreover, most patients labor under the pretense that this is the case anyways.
We are morally obligated to make good on this faith.

It is a problem, then, to misconstrue certainty and settle on lesser
propositional attitudes like mere belief or opinion.
This is not a question of arbitrary lines drawn in the statistical sand.
Again, as Pellegrino and Thomasma note, the patient-centered relationship is
characterized by the moral nature of the medical decision and the ineradicable
moral complicity of the physician in whatever happens to the patient.
The argument is not that these propositional attitudes are just as likely to be
wrong as they are right.
They may be more or less reliable, depending.
The issue is whether they are sufficiently reliable.
It is significant Hellman and Hellman concedes that the methodology of RCTs
provides statistical assurance that the difference is not due to chance.
We need not impute that the speculative views of clinicians are without value.
We only need to observe that without the relevant justification that
instantiates knowledge or certainty, these propositional attitudes, including
true beliefs, are subject to epistemic luck.
It is true scientific methods cannot completely eliminate uncertainty.
However, if they provide our best chance at reducing it and we knowingly choose
to rely on less reliable methodologies, then we intentionally choose tho expose
ourselves to an increased risk of epistemic luck.\footnote{This is in part why
Gettier problems are not a significant obstacle here.}
However, given the nature of medical decisions, the epistemic gamble is not made
int the vacuum of the physician's beliefs.
The stakes of the bet are paid by the patient.
As a result, the epistemic luck of the clinician's propositional attitudes
becomes the moral luck of the patient's health.
This, then, becomes the case of Clifford's ship owner.

This need for RCTs, then, rests on both the ethics of belief regarding
uncertainty and the patient-centered ethics of health care conjoined.
The good will of the physician, in their veracity and their judgment, is not
enough.
Given medical history, sopping with bile, phlegm, and blood, can be told through
a history of mishaps and false beliefs, we should approach our own reliability
with humility and skepticism.
What is at issue here is not merely increased social benefit or decreased risk.
Rather, it is whether the care relationship at the heart of a patient-centered
ethic can be instantiated in these conditions.
Arguably, it is not.
If the relationship of care and antecedent obligation emerge from trust, then
evidentialism suggests that the pursuit of knowledge and certainty is necessary
to instantiate trustworthiness and, therefore, a relationship of care in the
first place..
Aside from the practical fruits, the trustworthiness of the medical community
requires not just a commitment to veracity and judgment but also knowledge and
the opportunities to learn the truth about the matter.
The first two conditions cannot suffice without the third.
As such, \textquote{Thus it is to be observed that his authority is valid
  because there are those who question it and verify it; that it is precisely
  this process of examining and purifying that keeps alive among investigators
  the love of that which shall stand all possible tests, the sense of public
  responsibility as of those whose work, if well done, shall remain as the
enduring heritage of mankind.}\footcite[p. 354]{clifford_ethics_1886}
Without the necessary empirical evidence for justified true beliefs regarding
things like efficacy or possible side-effects, clinicians cannot in good faith
lay claim to such medical knowledge.
That is to say, they do not instantiate trustworthiness.
Without this, the faith patients put in their clinician, is not one of true
care.
It may be an amiable one, of faith and good will.
It will be a healing one, based on true belief.
It will, however, not be one of care if clinician's do not have access to the
relevant knowledge generated by appropriate RCTs.

If we are concerned with the ethical permissibility of randomized clinical
trials, then we ought to be evidentialist because it gets right the ethics of
uncertainty.
The principle of equipoise is epistemologically misguided but the underlying
intuition to appeal to uncertainty is right.
Evidentialism, however, preserves the intuitive appeal to uncertainty through a
greater fidelity to the epistemology of uncertainty as absence of knowledge.
Evidentialism more accurately describes when we have knowledge and when we are
uncertain.
By doing so, it provides us a framework for the permissibility of randomized
clinical trials along the same lines: that we do not do wrongful harm because,
in the absence of knowledge, it is not intentional withholding.
Moreover, evidentialism gives us a moral principle to pursue RCTs on the basis
of the relationship of care between clinician and patient.
If what we know matters, then what it is to know matters.

\printbibliography

\end{document}
