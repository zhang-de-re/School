\documentclass[letterpaper,notitlepage,12pt]{article}
\usepackage[
  letterpaper,tmargin=1in,bmargin=1in,lmargin=1in,rmargin=1in]
{geometry}
\usepackage[backend=biber]{biblatex-chicago}
\usepackage{csquotes}
%\addbibresource{references.bib}
\usepackage{setspace}

\title{Ebola, Equipoise, and Evidentialism}
\author{Alexander Zhang}

\doublespacing

\begin{document}

\maketitle

% efficacy and effectivenss are different! make relevant changes

% rework thesis to be limited to arguing for the need for an ethics of
% belief vis a vis medicine.

\section{Introduction}

At time of writing, the Covid-19 pandemic has just surpassed the death toll of
the 1918 Spanish flu pandemic.
I am struggling to conceive that I have lived in, if not through, the deadliest
disease vent in American history.
It's not worth belaboring the point, but in a time of crisis, it is also a time
when what we choose to believe has become intensely morally and politically
charged.
% ^ that needs to be rephrased
On a backdrop of conspiracy theory---what do you mean there are germ theory
deniers?---we ought to pose the question "What are the ethics of belief itself?"

This paper takes a small step in that direction by reflecting on the response to
the West African Ebola virus epidemic: specifically, the Ebola \c{c}a
Suffit-Guinea ring vaccination trial.
The trial and distribution of the rVSV-ZEBOV vaccine was hailed in the press as
a triumph of modern science and international cooperation.
However, in looking at its history, it is also noteworthy because it was also a
trial of our ethical concepts regarding the permissibility of randomized
clinical trials, that empirical process by which we are supposed to set our
credences.
In the following, I will discuss how two moral principles---clinical equipoise
and the fiduciary obligation of care---appear to have been applied in contention
to the trial design but were ultimately both not satisfied.
In place of those two positions, I will propose a third---evidentialism---and
sketch out two considerations from that standpoint which argue for not just the
permissibility of randomized clinical trials but the moral and epistemic virtue
in conducting them.

\section{Ebola \c{c}a Suffit!}

If we want to point to an exemplary case of what a good and effective response
to an epidemic looks like, it might be suggested that we consider that of the
2013-2016 outbreak of the Ebola virus in Guinea, Liberia, and Sierra Leone.
There is much to commend.
It was likely one of the largest collaborative efforts of the international
community with organizations spanning national governments (e.g. Canada, the
United States, and affected African countries); government agencies (e.g. PHAC,
NIH, and CDC); non-governmental organizations (e.g. M\'{e}dicins San 
Fronti\`{e}res); global public health entities (e.g. WHO, GAVI, and the Wellcome
trust); universities (e.g. University of Geneva and Dalhousie University); and
private sector companies (e.g. NewLink Genetics, GlaxoSmithKline, and Merck).
This culminated in a phase 2 trial in Guinea that was widely reported to be
100\% effective.\footcite{NYT "New Ebola Vaccine gives 100 percent protection}
As a result, a powerful prophylaxis was distributed to those at most risk
protecting them from a disease with an average mortality rate of 50\%.
Surely this represents a shining example of what can happen wen humankind comes
together to accomplish good.

Perhaps.

It is also worth noting that the rVSV-ZEBOV vaccine was still experimental and
distributed under the pretense of a phase 2 trial seeking to determine the yet
unknown efficacy of the vaccine.
This was not without moral controversy.
It was decided, for example, to hold phase 1 safety trials in first world
countries to preempt misgivings that vulnerable Africans were being exploited as
test subjects by the West.
More relevantly for us, the decision making for the design of the various
concurrent trials during the outbreak was subject to contentious moral debate.

For some, the best way forward was to hold a trial with a randomized control
arm.
Notably, this was advocated by representatives from GlaxoSmithKline (GSK) whose
vaccine candidate was in consideration with NewLink Genetics (whose rVSV-ZEBOV
vaccine would be later licensed to Merck).
They proposed that half of the participants be randomly selected to be given
some other, approved vaccine, thereby serving as an active control arm.\footcite
{Science "tough choices ahead in Ebola vaccine trials"}
% ^ this needs to be rephrased
This was suggested under the shadow of the WHO announcement on September 5 that
the urgency of the crisis required accelerating the process.
% rework to cite "Statement on the WHO Consultation on potential Ebola therapies
% and vaccines" and "WHO issues roadmap to scale up international response to
% the Ebola outbreak in west Africa"
The vaccines would move from phase 1 trials to phase 2 as soon as feasible---a
process that typically takes several years---and phase 3 trials would be
effectively dropped.
% "at this meeting it was agreed that Phase 1 trials would launch and that
% before Phase 1 trials were completed, efficacy trials in the affected
% countries would be initiated. This decision made it difficult for
% manufacturers, as they were unsure which dose would be required for Phase 2
% trails (Mohammadi, 2015)
% ^ look this up and integrate
As WHO assistant director-general Marie Paul Kierny
observed, \textquote{The attendees had a strong sense that there was no time to
wait...}\footcite{Science "tough choices ahead in Ebola
vaccine trials"}
Thus, a randomized control trial was attractive to some as it was seen as the
fastest and most statistically sound way to determine whether the vaccine
candidates work.
It was hoped this would hasten a wider deployment and thereby save more lives.
Additionally, there were many unknowns such as infection risk which was
estimated at 10\% per year spent in contact with Ebola patients. It was
estimated with that figure that, in a randomized controlled trial of 5000, if
\textquote{the vaccine works at least 80\% of the time, researchers could be
  \textquote{absolutely confident} about efficacy after 30 infections and within
  3 months... A vaccine that had 60\% efficacy could still yield an answer with
fewer than 60 infections.}\footcite{Science "tough choices ahead in Ebola
vaccine trials"}

However, not everyone involved were accepting of a randomized control trial.
As the representative from GSK, Ripley Ballou, commented, \textquote{Going into
  this meeting, we were told the idea of a controlled trial...was noting going
to be acceptable.}
Notably, representatives of M\'{e}dicins San Fronti\`{e}res strongly opposed
giving participants anything but an Ebola vaccine candidate.
As they asserted, \textquote{Studies on efficacy in affected countries and more
  so in at-risk populations should not have a placebo or active control arm as
this cannot be defended ethically.}\footcite{Science}
Given the high rate of mortality for those infected, it was deemed inhumane.
Part of that judgment rested on a calculation that the risk of the unknown
adverse effects of the experimental vaccines were smaller than the present
background risk of infection and then death from Ebola.\footcite[p. 114]{NAP}
Hence, it was a gamble worth taking.
Simply put a randomized control would entail that some significant portion of
participants would still be at extreme risk of infection to an often deadly
disease.
Moreover, they would be so as a deliberate act of the trial designers.
% repetitive?
This withholding of an available intervention was perceived to be morally
untenable.

A compromise was reached.
The accepted trial design was a step-wedge scheme.
Under a step-wedge design, time is leveraged as a control group.
Rather than have a randomly selected group that is not given the vaccine
candidate, the injection of the vaccine is instead delayed for a randomly
selected portion of participants.
By comparing the infection rates of those already vaccinated and those yet to
be, it is in theory possible to determine the candidate's efficacy.
While it was agreed upon that a trial was necessary, and that it did not make
sense to simply give healthy people an experimental vaccine on a compassionate
use basis, the step-wedge design was attractive because it entailed
distributing the vaccine, eventually, to all participants in the study.

However, it posed additional statistical challenges.
For one, it would be more difficult to determine efficacy because there would be
less control over confounding factors such as the limited window for the delayed
group to be at risk, differences in rates of new infections, behaviors, and
availability of personal protective equipment.
It also poses a difficulty in determining long term health effects or whether an 
infection-enhancing immune adverse response might be induced.\footcite[p. 116]
{NAP}
For another, a step-wedge design would take longer than a randomized controlled
trial.
While each cluster would be observed for the same amount of time, not all
clusters would take place concurrently.
As such, the staggered pace of clusters would mean the study would have to take
place over a lengthy period of time until there were a sufficient number to
statistically power the study. 
More over, it was noted that \textquote{the sample size must be inflated for the
  effect of clustering within rings as the members of a ring share a common
exposure to the index case and are not statistically independent}\footcite{ebola
ca suffit ring vaccintation trial consortium, 2015}

Despite this, the Ebola \c{c}a Suffit-Guinea vaccine trial used a stepped-wedge
ring vaccination design.
This trial was seen on some aspects as a success: \textquote{Among all of the
therapeutic and vaccine trials conducted in West Africa during the outbreak, the
ring vaccination trial came the closest to fulfilling the hope for a clinical
trial \textquote{home run}.}\footcite{NAP}
In a ring vaccination scheme, a known case is identified and then contact
traced.
Individuals who are socially or geographically connected around them (a ring)
are vaccinated in order to create a fence of immunity to prevent the spreak of
infection.
In the Guniea trial, the rings were randomized into two groups by whiche the
clusters for the stepped-wedge design: one group was immediately vaccinated and
the other was injected after a 21 day delay.
By all accounts, logistically, this design was effectively executed.
Such a study would have to really on the quality of the public health contact
tracing efforts.
Cases were identified within 3.9 days and clusters formed around them within 9.7
days on average.\footcite{HR 2016}
As noted by the Lancet, \textquote{That such a trial was even posssible is a
  testament not only to the skill of the research teams but also to the
  commitment of communication to defeating an epidemic that has devastated their
  nation. Over 90 percent of the study's staff was from Guinea. Before this
work, no clinical trial of this scale had ever been performed in the
country.}\footcite{lancet 2015}

However, the trial design was changed midway through the study.
In the interim report, which contained all the collected data prior to the
change, the researchers noted that \textquote{The results of this interim
  analysis indicate that rVSV-ZEBOV might be highly efficacious and safe in
  preventing Ebola virus disease and is most likely effective at the population
  level when delivered during an Ebola outbreak via a ring vaccination
strategy.}\footcite[p. 857]{HR 2015}
On the basis of the data and the fact that the incidence of new ring-defining
cases were rapidly decreasing, the Data Safety Management Board concluded it
\textquote{would be unethical to deny people access to this life-saving
  intervention when the interim analysis showed evidence that rVSV-ZEBOV is both
safe and efficacious.}\footcite{UF 2015} 
As a result, the delayed-immunization arm was terminated.
The stepped-wedge, the compromise from a randomized control that was to unsure
that both moral concerns were assuaged and that the efficacy of the vaccine
would still be determined in a statistically sound was was effectively
abandoned.

As a brief aside, it may be interesting to note that in 2015, the FDA published
\textquote{Product Development Under the Animal Rule: Guidance for Industry}.
The animal efficacy rule was enacted in 2002 following bioterrorism concerns.
It establishes a pathway for drugs and vaccines to be developed and approved
without human trials and relying instead on a sufficiently like animal model
trial.
In this document, the FDA \textquote{guidance provides information and
  recommendations on drug and biological product development when human efficacy
studies are not ethical or feasible.}\footcite{animal rule 2013}
At the time, connections were drawn with Ebola on the same ethical concerns as
raised in the trial design discussion.
I also bears remembering that several US defense agencies were involved in the
effort against Ebola virus.

Th final data that emerges drives home the point of the story.
It is clear, especially with the preponderance of evidence gathered from
concurrent and then later clinical trials, that the vaccine provides protection
from the Ebola virus.
The final report included all collected data including befroe the DSMB's
termination of the delayed arm and after. 
The researchers provided several different analyses of the data.
Significantly, different estimates of how effective the vaccine was depended on
the analytical method.
The gold standard is the intention-to-treat analysis.
Typically RCTs are evaluated on the dictum \textquote{once randomized, always
analyzed}.\footcite{Heenekens et al 1987}
On this metric, every randomized participant is analyzed according to
assignment.
It ignores anything that happens after randomization.
By doing so, it maintains the prognostic balance generated from the original
allocation, providing a conservative but unbiased estimate.
% get ITT def from EBM toolkit or the handbook
On an ITT analysis, the researchers reported inconclusive results for a vaccine
effectivness of 65 percent with a 95\% confidence interval of -47--91\%.
That is to say, \textquote{the range of values withing which we can be 95\% sure that the
true value lies for the whole population of patients from whom the study
patients were selected.
Wide confidence intervals indicated less precise estimates of
effect.}\footcite{ebm glossary}
So while 65\% is the median, on the basis of all the data as ITT, there is an
extremely wide range of values we could be just as confident in.
Particularly note that this range crosses zero.
Statistically speaking, it is feasible on grounds of this data set alone that
the vaccine was zero percent effective or even possibly harmful.
% include p-value of 0.344

Of course, on the other hand, the researchers presented, as the primary
comparison, an alternvative analysis.
On an on-treatment analysis where only those in the immediate vaccination
clusters who were actually vaccinated and the delayed vaccination clusters, the
trial showed efficacy of 100\% with a 95\% confidence interval of 69--100\%.
% cite HR for stats
However, on this sort of analysis, we would expect a bias towards vaccine
efficacy because it is no longer protected by randomization.
There is a potential and unmeasured bias due to the fact that thos ewoh received
the vaccine mya have had a different risk exposure regardless and that the small
proportion of clusters which reported Ebola cases makes comparability of risk
across clusters difficult to determine.\footcite[p. 129]{NAP}
Moreover, heading into the trial, as noted earlier, there needed to be
sufficient number of clusters in order to show statistical power of the study.
However, by terminating the stepped-wedge element early, this left the study
underpowered.
Finally, it should also be noted that early termination is possible to show a
positive bias as early on in a study with a smaller sample size, we can expect a
larger statistical variation before the data trends towards a regression to the
truth with a larger sample set.\footcite{handbook}
Again, this is not to say that the vaccine was not effective at all.
As the National Academy of Sciences notes, \textquote{We concur that, taken
  together, the results suggest that the vaccine most likely provides some
  protection to receipients---possibly \textquote{substantial protection,} as
  stated in the final report. However, we remain uncertain about the magnitude
  of its efficacy which could in reality be quite low or even zero as the
confidence limits around the unbiased estimate include zero.}\footcite[p.
128]{NAP}

Regardless, as suggested before, the number which caught the attention of the
world was 100\% effectiveness despite the fact that no vaccine is ever 100\%
effective given all the confounding variation across individuals.
Also frequently cited was the statistic that no infection was detected within 10
days after administration.
However, WHO estimated that infections occurred on average between 2--21
days.\footcite{Bosclay 2016}
In any case, by the time of the trial, the Ebola outbreak in Guinea was already
dying down.
A final note of significance for our story is that the rVSV-ZEBOV vaccine was
approved at the end of 2019.
This approval, however, only came after the support of additional clinical
trials that were conducted during the 2018--2020 outbreak in the Democratic
Republic of Congo.

\section{Reflection}

If the vaccine was a success---after all the rVSV-ZEBOV vaccine in the Guinea
trail was the same as the one that would be approved in 2019---then it might be
unclear why this case illustrates a need for an ethics of belief.
After all, it seems that we got it right---that the vaccine was effective and
that it was safe is vindicated by regulatory approval and that it very likely
saved lives in being distributed as part of a trial.
That there was uncertainty in the results of the trial is surely statistcal
handwringing.
However, it is that recognition of objective uncertainty itself that
neccesitates an ethics of belief.
If we were fully transparten about what we did, in fact, know and what we didn't
then there would be no need.
However, we are mired in the muddy waters of the various doxastic attitudes, or
epistemic positions we can hold towards a proposition, that we might adopt
besides knowledg.
To motivate this argument, however, I will start by evaluating how the current
convetional standards applied to this case fell short even if, ultimately, the
vaccine itself was successful.

First, let us consider clinical equipoise.
Roughly put, clinical equipoise holds that the moral permissiblity of of a
randomized clinical trial requires that it be conducted from an epistemic state
where the options evaluated, the candidtate or the null hypothesis, are valued
to be of equal weight or credence.
By stipulating this, the principle is supposed to provide the moral framework by
which a randomized clinical trial is not morally wrong for not providing an
available beneficial treatement to all the participants.
After all, if it is wrong to intentionally withhold the best treatement, then
reserachers in a state of equipoise are absovled since, in giving all options
equal weightiness, it cannot be intentional because they do not know.

More specifcally, there are a number of ways of a principle of clinical
equipoise has been articulated.
Most notable is that of clinical equipoise in an individaul and that of cilincal
equipoise of the medical community.
The first is perhaps best represented by the charatterization that \textquote{if

\end{document}
