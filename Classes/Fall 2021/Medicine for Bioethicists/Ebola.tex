\documentclass[letterpaper,notitlepage,12pt]{article}
\usepackage[
  letterpaper,tmargin=1in,bmargin=1in,lmargin=1in,rmargin=1in]
{geometry}
\usepackage[backend=biber]{biblatex}
\usepackage{csquotes}
\addbibresource{ebola.bib}
\usepackage{setspace}

\title{Ebola and the Need for an Ethics of Belief}
\author{Alexander Zhang}
\date{}

\doublespacing

\begin{document}

\maketitle

\section{Introduction}

At time of writing, the Covid-19 pandemic has surpassed the death toll of
the 1918 Spanish flu pandemic.
I am struggling to grasp that I have lived in, if not through, the deadliest
disease event in post-colonization American history.
It's not worth belaboring the point: it is a time of crisis.
But there is an inflection that casts a different pallor to the current
pandemic.
What we choose be believe in has become morally, emotionally, and politically
charged.
Of course, we can draw connections to the past like the Anti-Masking League of
San Francisco that opposed the city's Board of Health's measures to attempt to
curb the Spanish flu.
But we should also bear in mind that American physicians were highly resistant
to germ theory well up to the 1880s.\footcite{tomes}
Chicago's project to reverse the flow of the Chicago river into the Mississippi
to prevent future Cholera outbreaks, rationalized on the grounds of the miasma
theory of disease, was well within living memory.
What is different is that we are behaving this way with the advantage of the
body of medical science gleaned in the past century.
On a backdrop of conspiracy theory---what do you mean there are germ theory
deniers?---we ought to pose the question "What are the ethics of belief itself?"

This paper will attempt to establish the need for such a project by reflecting 
on the response to the West African Ebola virus epidemic: specifically, the 
Ebola \c{c}a Suffit-Guinea ring vaccination trial.
The trial and distribution of the rVSV-ZEBOV vaccine was hailed in the press as
a triumph of modern science and international cooperation.
However, in looking at its history, it is also noteworthy because it was also a
trial of our ethical concepts regarding the permissibility of randomized
clinical trials, that empirical process by which we are supposed to set our
credences.
After laying out the case, I will examine how two moral principles---clinical
equipoise and the fiduciary obligation of care---appear to have been applied in
contention to the trial design and how they were both ultimately not satisfied
nor satisfactory.
This will hopefully persuade that, drawing on evidentialism, there is both
fertile ground and a need for an alternative framework that emerges out of an
ethics of belief.

\section{Ebola \c{c}a Suffit!}

If we want to point to an exemplary case of what a good response
to an epidemic looks like, it might be suggested that we consider that of the
2013-2016 outbreak of the Ebola virus in Guinea, Liberia, and Sierra Leone.
There is much to commend.
It was likely one of the largest collaborative efforts of the international
community with organizations spanning national governments (e.g. Canada, the
United States, and affected African countries); government agencies (e.g. PHAC,
NIH, and CDC); non-governmental organizations (e.g. M\'{e}dicins San 
Fronti\`{e}res); global public health entities (e.g. WHO, GAVI, and the Wellcome
trust); universities (e.g. University of Geneva and Dalhousie University); and
private sector companies (e.g. NewLink Genetics, GlaxoSmithKline, and Merck).
This culminated in a Phase II trial in Guinea that was widely reported to be
100\% effective.\footcite{NYT}
As a result, a powerful prophylaxis was distributed to those at most risk
protecting them from a disease with an average mortality rate of 50\%.
Surely this represents a shining example of what can happen wen humankind comes
together to accomplish good.

Perhaps.

It is also worth noting that the rVSV-ZEBOV vaccine was still experimental and
distributed under the pretense of a Phase II trial seeking to determine the yet
unknown efficacy of the vaccine.
This was not without moral controversy.
It was decided, for example, to hold Phase I safety trials in first world
countries to preempt misgivings that vulnerable Africans were being exploited as
test subjects by the West.
More relevantly for us, the decision making for the design of the various
concurrent trials during the outbreak was subject to contentious moral debate.

For some, the best way forward was to hold a trial with a randomized control
arm.
Notably, this was advocated by representatives from GlaxoSmithKline (GSK) whose
vaccine candidate was in consideration with NewLink Genetics (whose rVSV-ZEBOV
vaccine would be later licensed to Merck).
They proposed that the trial have an active control arm wherein a random half of
the participants would be given some other, approved vaccine.\footcite
{Science}
It should be remember that
This was suggested under the shadow of the WHO announcements in the prior weeks  
that the urgency of the crisis required accelerating international response.
The vaccines would move from Phase I trials to Phase II as soon as feasible---a
process that typically takes several years.
As WHO assistant director-general Marie Paul Kierny
observed, \textquote{The attendees had a strong sense that there was no time to
wait...}\footcite{Science}

Thus, a randomized control trial was attractive to some as it was seen as the
fastest and most statistically sound way to determine whether the vaccine
candidates worked.
It was hoped this would hasten a wider deployment of the candidate.
If more people got it sooner, then more lives could be saved.
One concern was the infection risk which was estimated at 10\% per year spent in
contact with Ebola patients.
With an estimated infection rate of 10\% per year spent in contact with Ebola
patients, in a randomized controlled trial of 5000, 
if \textquote{the vaccine works at least 80\% of the time, researchers could be
  \textquote{absolutely confident} about efficacy after 30 infections and within
  3 months... A vaccine that had 60\% efficacy could still yield an answer with
fewer than 60 infections.}\footcite{Science}

However, not everyone was receptive to a randomized control trial.
As the representative from GSK, Ripley Ballou, commented, \textquote{Going into
  this meeting, we were told the idea of a controlled trial...was not going
to be acceptable.}
Representatives of M\'{e}dicins San Fronti\`{e}res strongly opposed
giving participants anything but an Ebola vaccine candidate.
As they asserted, \textquote{Studies on efficacy in affected countries and more
  so in at-risk populations should not have a placebo or active control arm as
this cannot be defended ethically.}\footcite{Science}
It was deemed inhumane given the high risk of mortality.

Part of that judgment rested on a calculation that the risk of the unknown
adverse effects of the experimental vaccines were smaller than the present
background risk of infection and then death from Ebola.\footcite[p. 114]{NAP}
It was a gamble worth taking.
Simply put a randomized control would entail that some significant portion of
participants would still be at extreme risk of infection to an often deadly
disease.
Moreover, they would be so as a deliberate act of the trial designers.
This withholding of an available intervention was perceived to be morally
untenable.

A compromise was reached.
The accepted trial design was a step-wedge scheme.
Under a step-wedge design, time is leveraged as a control group.
Rather than not give the vaccine to a randomly selected portion of participants,
the injection of the vaccine is delayed.
By comparing the infection rates of those already vaccinated and those yet to
be, it is in theory possible to determine the candidate's efficacy.
While it was agreed upon that a trial was necessary on the basis that it did not 
make sense to give healthy people an experimental and possibly harmful vaccine on 
a compassionate use basis, the step-wedge design was considered attractive because 
it guaranteed distributing the vaccine to all participants in the study.

However, a stepped-wedge posed additional statistical challenges.
For one, it would be more difficult to determine efficacy because there would be
less control over confounding factors: a limited window for the delayed
group to be at risk, differences in rates of new infections, behaviors, and
availability of personal protective equipment.
It also posed a difficulty in determining long term health effects or whether an 
infection-enhancing immune adverse response might be induced.\footcite[p. 116]
{NAP}
Moreover, a step-wedge design would take longer than a randomized controlled
trial.
It was noted that \textquote{the sample size must be inflated for the
  effect of clustering within rings as the members of a ring share a common
exposure to the index case and are not statistically
independent}\footcite{consortium}
Additionally, while each cluster would be observed for the same amount of time, 
not all clusters would take place concurrently.
The staggered pace of clusters would mean the study would have to wait
until there were a sufficient number to
statistically power the study. 

Regardless, the Ebola \c{c}a Suffit-Guinea vaccine trial used a stepped-wedge
ring vaccination design.
It was seen by and large as a success: \textquote{Among all of the
therapeutic and vaccine trials conducted in West Africa during the outbreak, the
ring vaccination trial came the closest to fulfilling the hope for a clinical
trial \textquote{home run}.}\footcite[p. 119]{NAP}
In a ring vaccination scheme, a known case is identified and then contact
traced.
Individuals who are socially or geographically connected around them (a ring)
are vaccinated in order to create a fence of immunity to prevent the spread of
infection.
By all accounts, logistically, this design was effectively executed.
Such a study would have to rely on the quality of the public health contact
tracing efforts.
Cases were identified within 3.9 days and clusters formed around them within 9.7
days on average.\footcite{HR2016}
The Lancet praised the study: \textquote{That such a trial was even possible is a
  testament not only to the skill of the research teams but also to the
  commitment of communication to defeating an epidemic that has devastated their
  nation. Over 90 percent of the study's staff was from Guinea. Before this
work, no clinical trial of this scale had ever been performed in the
country.}\footcite{lancet2015}

However, midway through the study, the trial was changed.
In the interim report, 
the researchers noted that \textquote{The results of this interim
  analysis indicate that rVSV-ZEBOV might be highly efficacious and safe in
  preventing Ebola virus disease and is most likely effective at the population
  level when delivered during an Ebola outbreak via a ring vaccination
strategy.}\footcite[p. 857]{HR2015}
On the basis of this data and the decreasing incidence of new ring-defining
cases, the Data Safety Management Board concluded it
\textquote{would be unethical to deny people access to this life-saving
  intervention when the interim analysis showed evidence that rVSV-ZEBOV is both
safe and efficacious.}\footcite{UF} 
As a result, the delayed-immunization arm was terminated.
The stepped-wedge design---the compromise intended to ensure both the morality
and scientific soundness of the trial---was abandoned.

It is clear, given the preponderance of evidence gathered from
concurrent and later trials, that the vaccine provides protection
from the Ebola virus.
However, the final results from the Guinea trial only bring the moral and
epistemic issue more into relief.
The final report included all collected data including before and after the 
DSMB's termination of the delayed arm.
The researchers provided several different analyses of this data.
Significantly, different estimates of how efficacious the vaccine was depended on
the analytical method.

The gold standard is the intention-to-treat analysis.
Typically RCTs are evaluated on the dictum \textquote{once randomized, always
analyzed}.\footcite{hennekens1987}
On this metric, every randomized participant is analyzed according to
assignment.
It ignores anything that happens after randomization.
By doing so, it maintains the prognostic balance generated from the original
allocation, providing a conservative but unbiased estimate.

On an ITT analysis, the researchers reported inconclusive results: a vaccine
efficacy of 65 percent with a 95\% confidence interval of -47--91\%.
A confidence interval of 95\% denotes the range of values within which we are
95\% sure that the true lies.
A wider range indicates less precise estimates of effect.
So while 65\% is the median value, on the basis of all the data as ITT, there 
is an extremely wide range of values we could be just as confident in.
Particularly note that this range crosses zero.
Statistically speaking, it is feasible on grounds of this data set alone that
the vaccine had zero percent efficacy or was possibly even harmful.

An alternative statistical measure is \textit{p}-value.
This represents the probability of obtaining similar results under an assumption
of a null hypothesis---that there is no effect or difference between the
compared groups.
In other words, the probability that an observed difference could have occurred
by random chance.
Lower \textit{p}-values indicated greater statistical significance.
The \textit{p}-value of the Ebola \c{c}a Suffit trial on an ITT analysis was
0.344.\footcite{HR2016}
Conventionally, only \textit{p}-values of 0.05 are considered sufficient for
acceptable statistical significance.

On the other hand, the researchers presented an alternative analysis as the
primary comparison to draw conclusions from.
On an on-treatment analysis where only those in the immediate vaccination
clusters who were actually vaccinated and the delayed vaccination clusters were
compared, the
trial showed efficacy of 100\% with a 95\% confidence interval of
69--100\%.\footcite{HR2016}
However, on this sort of analysis, we would expect a bias towards vaccine
efficacy because it is no longer protected by randomization.

The possible confounding factors on this analysis include the following.
There is a potential and unmeasured bias due to the fact that those who received
the vaccine may have had a different risk exposure regardless. 
Additionally, the small
proportion of clusters which reported Ebola cases makes comparability of risk
across clusters difficult to determine.\footcite[p. 129]{NAP}
Moreover, heading into the trial, as noted earlier, there needed to be
sufficient number of clusters in order to show statistical power of the study.
However, by terminating the stepped-wedge element early, this left the study
underpowered.
Finally, it should also be noted that early termination can show a
positive bias due to the fact that, with a smaller sample size, we can expect a
larger statistical variation before the data trends regresses to the 
truth.\footcite{handbook}
Again, this is not to say that the vaccine was not effective at all.
As the National Academy of Sciences notes, \textquote{We concur that, taken
  together, the results suggest that the vaccine most likely provides some
  protection to recipients---possibly \textquote{substantial protection,} as
  stated in the final report. However, we remain uncertain about the magnitude
  of its efficacy which could in reality be quite low or even zero as the
confidence limits around the unbiased estimate include zero.}\footcite[p.
128]{NAP}

Regardless, as suggested before, the number which caught the attention of the
world was 100\% effectiveness despite the fact that no vaccine is ever 100\%
effective.
In any case, by the time of the trial, the Ebola outbreak in Guinea was already
petering out.
A final note for our story is that the rVSV-ZEBOV vaccine was
approved at the end of 2019.
This approval, however, only came after the support of additional clinical
trials that were conducted during the 2018--2020 outbreak in the Democratic
Republic of Congo.

\section{Reflection}

If the vaccine was a success then it might be
unclear why this case illustrates a need for an ethics of belief.
After all, it seems that we got it right.
The vaccine was effective and safe and it very likely
saved lives as a result of being distributed as part of a trial.
That there was uncertainty in the results of the trial is surely statistical
hand wringing.

However, it is that recognition of objective uncertainty that
necessitates an ethics of belief regardless of outcome.
If we were fully transparent about what we do, in fact, know and what we don't
then there would be no need.
However, we are mired in the muddy waters of the various doxastic attitudes, or
epistemic positions we can hold towards a proposition, that we might adopt
besides knowledge.
To motivate this argument, however, I will start by evaluating how the current
conventional standards applied to this case fell short even if, ultimately, the
vaccine itself was successful.

First, let us consider clinical equipoise.
Roughly put, clinical equipoise holds that the moral permissibility of a
randomized clinical trial requires an epistemic state
where the options evaluated are thought
to be of equal weight or credence.
By stipulating this, the principle is supposed to provide the moral framework 
that demonstrates a randomized clinical trial is not morally wrong for not 
providing an available and beneficial treatment to all the participants.
After all, if it is wrong to intentionally withhold the best treatment, then
researchers in a state of equipoise are absolved since, in giving all options
equal weightiness, it cannot be intentional because they do not know.

There are a number of ways a principle of clinical
equipoise has been articulated.
Most common is that of clinical equipoise in an individual and that of clinical
equipoise of the medical community.
The first is perhaps best represented by the characterization that \textquote{if
  the clinician knows, or has good reason to believe, that a new therapy (A) is
  better than another therapy (B), he cannot participate in a comparative trial
  of Therapy A versus Therapy B. Ethically, the clinician is obligated to give
Therapy A to each new patient with a need for one of these
therapies.}\footcite{shaw1970}
If this is the case, then clinical equipoise is the inverse: if a clinician does
not know or have good reason to believe, they are not obligates to give that
therapy to each new patient with a need.
This puts it squarely in the mind and doxastic attitudes of the individual.
It might be an equal confidence on the basis
of the available data or an absence of treatment preference all things
considered.\footcite{fried1974}

This conception was criticized by Benjamin Freedman as psychologically
unfeasible as it was \textquote{balanced on a knife's edge.}\footcite[p. 143]
{freedman1987}
Since any new and relevant evidence ought to, rationally, update our credences,
such mental state would be too unstable to run a clinical trial.
Freedman suggested an alternative conception that pegged equipoise to the
doxastic attitudes of the medical community.
Under such a conception, clinical equipoise occurs, \textquote{if there is
  genuine uncertainty within the expert medical community---not necessarily on
the part of the individual investigator about the preferred
treatment.}\footcite{freedman1987}
This is taken to mean, in practice, that this state of genuine uncertainty is
identified via the consensus, or lack thereof, of the medical community.
Equipoise, then, appears to be socially constructed by the medical community in
a kind of inverted analogy of how some hold that the art-world determines what
is "Art".
We can see this in Fred Gifford's criticisms of Freedman.
As he argued, randomized clinical trials take so long that incoming interim
results are likely to change the medical communities opinion before the study is
completed.\footcite[p. 129]{gifford1995}
In which case, it is at least a common interpretation that clinical equipoise
embodied in the medical community rests upon opinions and consensus.

What we can observe is that the concept of equipoise is aimed at uncertainty.
However,
instead of pinning that to the objective absence of
knowledge, or justified true belief, it is characterized in terms of alternative
doxastic attitudes like beliefs, preferences, opinions, and social consensus.
This is important because of what is morally relevant are the epistemological
features.
If uncertainty---or lack of knowledge---is what is significant, then the
epistemology of knowledge lies at the heart of the matter.
If knowledge requires justification to be knowledge, then uncertainty can be
identified by the objective state of a lack of sufficient justification rather
than the subjective states of belief.
Accepting this, then clinical equipoise as it is conventionally interpreted, at
the very least, is epistemologically misguided.
What matters is the whether we possess empirical justification that instantiates
knowledge regarding the efficacy of an intervention.

This is especially apparent when equipoise was applied to the Guinea Ebola ring
vaccination trial.
After all, if regulatory approval is taken to be indicative of sufficient
proximity to certainty, then uncertainty persisted throughout and well after the
trial.
In spite of this, the stepped-wedge design aspect was terminated early.
This should be contradictory.
We cannot claim to know that a treatment works, if knowledge resembles anything
like justified true belief, before the relevant process which gets us to
empirical justification is complete.

However, if we do not make careful distinctions between our doxastic beliefs,
then it becomes apparent how we might think clinical equipoise no longer
justifies the clinical trial. 
This is because in misidentifying objective uncertainty with a subjective state 
of equipoise, it becomes easy to make conflations that are not epistemologically
relevant.
After all, our subjective beliefs are affected by all kinds of biases ranging
from the social, political, and our own preferences and hopes.
It is not surprise that, despite a contentious debate over the trial design, we
would be biased towards confidence in the candidate.
If we are beneficent and morally concerned regarding the crisis, we want the
vaccine to work.

This only drives home the fundamental conceptual flaw in equipoise: an
epistemic state of holding two competing propositions as equally weighty is an
underdetermination of uncertainty.
It may be sufficient. It is not, however, a necessary condition.
For example, I may be very confident, for good or bad reasons, that my favorite 
driver will win the next race.
This appears to be a case where I do not have equipoise.
It might even be the case that the majority of the racing community agree.
But this should not be conflated with certainty as, ultimately, I cannot know
the result until the checkered flag.
If a lack of equipoise was enough to establish knowledge or certainty, then the
rational thing to do would to take all my savings to a bookie.
Clearly, though, this is still a gamble regardless of how confident I am of a
sure thing.
As such, by being epistemologically misguided about uncertainty,
clinical equipoise failed to deliver us from uncertainty in the case of the
Guinea Ebola trial because it allowed insufficient empirical justification to
terminate the process that would have brought us closer to knowledge.

Alternatively, we might want to apply the principle of fiduciary obligation to
the clinical trial.
That is to say, that clinicians have an obligation to treat the patient in front
of them by both recognizing the specific interests and needs of the patient as
an individual and by exercising their cultivated judgment.
We might also be tempted to say that proponents of this argument were successful
in getting what they wanted.
As MSF desired, no control arm was implemented and, moreover, the comparative
increased risk exposure from a delayed vaccination was also terminated.
In the end, it turned out that the experimental candidate was safe and
effective.
We might claim no harm no foul.

However, What this position represents in this case is a disregard for the moral
features that emerge out of uncertainty and, hence, further entrenchment.
The fact of the matter is that we did not know and it is only moral luck that
things turned out fine regardless.
William Clifford encapsulates this well in the famous shipowner example.
Suppose that you own a ship that will sail carrying a great number of
passengers.
It is an old ship that has seen much use and often needed repairs.
Doubts arise regarding the ships seaworthiness.
Perhaps you should, at great cost, have it overhauled and refitted or, at the
very least, inspected.
However, you put aside these unhappy thoughts : the ship has safely sailed many
times and weathered many storms.
It is ungenerous to be suspicious of the builders.
You acquire a sincere conviction that the ship is safe and seaworthy as it has
been in the past.

Suppose, then, the ship goes down.
As Clifford observes, \textquote{What should we say of him? Surely this, that he
was verily guilty of the death of theses men.}\footcite[p. 339]{clifford}
The true sincerity of belief is not relevant \textquote{because \textit{he had
  no right to believe on such evidence as was before him.}}\footcite[p. 340]
  {clifford}
Suppose, on the other hand that the ship happily makes it to its destination
safe and sound.
Again, Clifford asserts, \textquote{Will that diminish the guilt of her owner?
  Not one jot. When an action is once done, it is right or wrong forever; no
accidental failure of its good or evil fruits can possibly alter
that.}\footcite[p. 340]{clifford}
% double check spelling of forever.
This is because, regardless of the outcomes, the shipowner was in a state of
objective uncertainty and subjected others to bear the risks without
doing his due diligence.

We might call this negligence.
But over and beyond prudential care, what the shipowner has failed to carry out
is both a moral and intellectual duty to inquire.
It may be the case that there are countervailing ethical considerations at work
in the Ebola vaccine case.
However, we still must attend to the considerations that emerge out of the
ethics of belief as well to have a holistic understanding of the moral dilemma.
We cannot know if we do not have the justificatory evidence to set our credence
to.
Uncertainty does not give rise to only intellectual duties.
This is not merely academic.
These are also moral obligations, particularly in the case of medicine because
the stakes involved.

I will conclude by sketching out two ways in which things can go wrong if we
don't attend to these moral and epistemic duties.
Hopefully this will suffice to persuade that there is a need for an ethics of
belief in our moral consideration regarding clinical research.

The first one is obvious.
It is problematic that the media latched on to the 100\% effectiveness figure when
there was questionable warrant for it.
As the NASEM report noted, this could have posed a hurdle for further
trials as it would have undermined the claim of clinical equipoise if it was in
the public consciousness that the vaccine was already found to be 100\%
effective.
As Clifford noted, we have a duty to guard from beliefs accepted on insufficient
evidence \textquote{as from a pestilence, which may shortly master our own body
and then spread to the rest of the town.}\footcite[344]{clifford}
In other words, we have positive and negative moral duties given that we are
part of a shared epistemic community.
As Miranda Fricker argues, we have a negative moral duty not to commit epistemic
injustice by prejudicially excluding others from that
community.\footcite{fricker}
But as clinical researchers, we are authorities in that epistemic community and
therefore have a moral and epistemic responsibility for the beliefs in that
shared community.

This might look like the second consideration.
If, in a clinical setting, a physician has a relationship of care with the
patient, then at the heart of that care relationship must be trust.
We speak of a fiduciary obligation a clinician has to exercise their judgment
to best serve the interests of the patient in front of them.
But for there to be trust, it is not sufficient, although it is obviously
necessary, for the clinician to merely act out of good will and to be sincere in
their judgment.
\textquote{The goodness and greatness of a man do not justify us in accepting a
  belief upon the warrant of his authority, unless there are reasonable grounds
  for supposing that he knew the truth of what he was saying.}\footcite[p. 353]
  {clifford}
For genuine trust to be instantiated, as the clinician asserts to the patient a
diagnosis and prognosis, the clinician needs to have access to the relevant
empirical evidence that is necessary for knowledge, or justified true belief.
This is, more or less, the fundamental intuition behind evidence-based medicine
as I understand it.
Otherwise, without knowledge, or something sufficiently proximal to knowledge,
trust is unwarranted and what relationship there is takes advantage of the faith
the patient puts in the clinician.
It may be the case that with good will, sincere judgment, sufficient
experience, and moral luck we tend to get things right most of the time.
But a relationship of care cannot be grounded on mere faith.
As such, health care has a moral obligation to seek the empirical justifications
for its assertions to \textquote{know}.

\printbibliography

\end{document}
