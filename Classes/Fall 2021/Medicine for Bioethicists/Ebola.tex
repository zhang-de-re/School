\documentclass[letterpaper,notitlepage,12pt]{article}
\usepackage[
  letterpaper,tmargin=1in,bmargin=1in,lmargin=1in,rmargin=1in]
{geometry}
\usepackage[backend=biber]{biblatex-chicago}
\usepackage{csquotes}
%\addbibresource{references.bib}
\usepackage{setspace}

\title{Ebola, Equipoise, and Evidentialism}
\author{Alexander Zhang}

\doublespacing

\begin{document}

\maketitle

\section{Introduction}

At time of writing, the Covid-19 pandemic has just surpassed the death toll of
the 1918 Spanish flu pandemic.
I am struggling to conceive that I have lived in, if not through, the deadliest
disease vent in American history.
It's not worth belaboring the point, but in a time of crisis, it is also a time
when what we choose to believe has become intensely morally and politically
charged.
% ^ that needs to be rephrased
On a backdrop of conspiracy theory---what do you mean there are germ theory
deniers?---we ought to pose the question "What are the ethics of belief itself?"

This paper takes a small step in that direction by reflecting on the response to
the West African Ebola virus epidemic: specifically, the Ebola \c{c}a
Suffit-Guinea ring vaccination trial.
The trial and distribution of the rVSV-ZEBOV vaccine was hailed in the press as
a triumph of modern science and international cooperation.
However, in looking at its history, it is also noteworthy because it was also a
trial of our ethical concepts regarding the permissibility of randomized
clinical trials, that empirical process by which we are supposed to set our
credences.
In the following, I will discuss how two moral principles---clinical equipoise
and the fiduciary obligation of care---appear to have been applied in contention
to the trial design but were ultimately both not satisfied.
In place of those two positions, I will propose a third---evidentialism---and
sketch out two considerations from that standpoint which argue for not just the
permissibility of randomized clinical trials but the moral and epistemic virtue
in conducting them.

\section{Ebola \c{c}a Suffit!}

If we want to point to an exemplary case of what a good and effective response
to an epidemic looks like, it might be suggested that we consider that of the
2013-2016 outbreak of the Ebola virus in Guinea, Liberia, and Sierra Leone.
There is much to commend.
It was likely one of the largest collaborative efforts of the international
community with organizations spanning national governments (e.g. Canada, the
United States, and affected African countries); government agencies (e.g. PHAC,
NIH, and CDC); non-governmental organizations (e.g. M\'{e}dicins San 
Fronti\`{e}res); global public health entities (e.g. WHO, GAVI, and the Wellcome
trust); universities (e.g. University of Geneva and Dalhousie University); and
private sector companies (e.g. NewLink Genetics, GlaxoSmithKline, and Merck).
This culminated in a phase 2 trial in Guinea that was widely reported to be
100\% effective.\footcite{NYT "New Ebola Vaccine gives 100 percent protection}
As a result, a powerful prophylaxis was distributed to those at most risk
protecting them from a disease with an average mortality rate of 50\%.
Surely this represents a shining example of what can happen wen humankind comes
together to accomplish good.

Perhaps.

It is also worth noting that the rVSV-ZEBOV vaccine was still experimental and
distributed under the pretense of a phase 2 trial seeking to determine the yet
unknown efficacy of the vaccine.
This was not without moral controversy.
It was decided, for example, to hold phase 1 safety trials in first world
countries to preempt misgivings that vulnerable Africans were being exploited as
test subjects by the West.
More relevantly for us, the decision making for the design of the various
concurrent trials during the outbreak was subject to contentious moral debate.

For some, the best way forward was to hold a trial with a randomized control
arm.
Notably, this was advocated by representatives from GlaxoSmithKline (GSK) whose
vaccine candidate was in consideration with NewLink Genetics (whose rVSV-ZEBOV
vaccine would be later licensed to Merck).
They proposed that half of the participants be randomly selected to be given
some other, approved vaccine, thereby serving as an active control arm.\footcite
{Science "tough choices ahead in Ebola vaccine trials"}
% ^ this needs to be rephrased
This was suggested under the shadow of the WHO announcement on September 5 that
the urgency of the crisis required accelerating the process.
% rework to cite "Statement on the WHO Consultation on potential Ebola therapies
% and vaccines" and "WHO issues roadmap to scale up international response to
% the Ebola outbreak in west Africa"
The vaccines would move from phase 1 trials to phase 2 as soon as feasible---a
process that typically takes several years---and phase 3 trials would be
effectively dropped.
% "at this meeting it was agreed that Phase 1 trials would launch and that
% before Phase 1 trials were completed, efficacy trials in the affected
% countries would be initiated. This decision made it difficult for
% manufacturers, as they were unsure which dose would be required for Phase 2
% trails (Mohammadi, 2015)
% ^ look this up and integrate
As WHO assistant director-general Marie Paul Kierny
observed, \textquote{The attendees had a strong sense that there was no time to
wait...}\footcite{Science "tough choices ahead in Ebola
vaccine trials"}
Thus, a randomized control trial was attractive to some as it was seen as the
fastest and most statistically sound way to determine whether the vaccine
candidates work.
It was hoped this would hasten a wider deployment and thereby save more lives.
Additionally, there were many unknowns such as infection risk which was
estimated at 10\% per year spent in contact with Ebola patients. It was
estimated with that figure that, in a randomized controlled trial of 5000, if
\textquote{the vaccine works at least 80\% of the time, researchers could be
  \textquote{absolutely confident} about efficacy after 30 infections and within
  3 months... A vaccine that had 60\% efficacy could still yield an answer with
fewer than 60 infections.}\footcite{Science "tough choices ahead in Ebola
vaccine trials"}

However, not everyone involved were accepting of a randomized control trial.
As the representative from GSK, Ripley Ballou, commented, \textquote{Going into
  this meeting, we were told the idea of a controlled trial...was noting going
to be acceptable.}
Notably, representatives of M\'{e}dicins San Fronti\`{e}res strongly opposed
giving participants anything but an Ebola vaccine candidate.
As they asserted, \textquote{Studies on efficacy in affected countries and more
  so in at-risk populations should not have a placebo or active control arm as
this cannot be defended ethically.}\footcite{Science}
Given the high rate of mortality for those infected, it was deemed inhumane.
Part of that judgment rested on a calculation that the risk of the unknown
adverse effects of the experimental vaccines were smaller than the present
background risk of infection and then death from Ebola.\footcite[p. 114]{NAP}
Hence, it was a gamble worth taking.
Simply put a randomized control would entail that some significant portion of
participants would still be at extreme risk of infection to an often deadly
disease.
Moreover, they would be so as a deliberate act of the trial designers.
% repetitive?
This withholding of an available intervention was perceived to be morally
untenable.

A compromise was reached.
The accepted trial design was a step-wedge scheme.
Under a step-wedge design, time is leveraged as a control group.
Rather than have a randomly selected group that is not given the vaccine
candidate, the injection of the vaccine is instead delayed for a randomly
selected portion of participants.
By comparing the infection rates of those already vaccinated and those yet to
be, it is in theory possible to determine the candidate's efficacy.
While it was agreed upon that a trial was necessary, and that it did not make
sense to simply give healthy people an experimental vaccine on a compassionate
use basis, the step-wedge design was attractive because it entailed
distributing the vaccine, eventually, to all participants in the study.

However, it posed additional statistical challenges.
For one, it would be more difficult to determine efficacy because there would be
less control over confounding factors such as the limited window for the delayed
group to be at risk, differences in rates of new infections, behaviors, and
availability of personal protective equipment.
It also poses a difficulty in determining long term health effects or whether an 
infection-enhancing immune adverse response might be induced.\footcite[p. 116]
{NAP}
For another, a step-wedge design would take longer than a randomized controlled
trial.
While each cluster would be observed for the same amount of time, not all
clusters would take place concurrently.
As such, the staggered pace of clusters would mean the study would have to take
place over a lengthy period of time until there were a sufficient number to
statistically power the study. 
More over, it was noted that \textquote{the sample size must be inflated for the
  effect of clustering within rings as the members of a ring share a common
exposure to the index case and are not statistically independent}\footcite{ebola
ca suffit ring vaccintation trial consortium, 2015}


\section{Equipoise or fiduciary obligation?}

\section{The case for evidentialism in evidence-based medicine}

\section{Conclusion}

\end{document}
