\documentclass[letterpaper,notitlepage,12pt]{article}
\usepackage[
  letterpaper,tmargin=1in,bmargin=1in,lmargin=1in,rmargin=1in]
{geometry}
\usepackage[backend=biber]{biblatex-chicago}
\addbibresource{wonder.bib}
\usepackage{setspace}
\usepackage[super]{nth}


\title{This is Wonder}
\author{Alexander Zhang}
\date{}

\begin{document}

\maketitle

\section{Introduction}

There is, I suspect, a point for every philosophy professor when they think
about teaching their class and ask themselves, ``What am a I even doing
here?''
It might be during new course preparation to comply with the latest
administrative-bureaucratic demands.
It might be in the middle of grading undergraduate papers or reading their
student--teacher evaluations.
It might just creep up on you whenever what you do for a living is mentioned in
public.
My father, an engineer, once told me a story.
He was in a meeting.
During his introduction, he proudly mentioned that his son was a philosopher.
A woman then audibly commented, ``How fucking useless.''
I still haven't figured out why he felt he had to mention that last bit.

Some might simply say that teaching is a requirement to get paid to research.
Others, hopefully, will believe that teaching philosophy is valuable in of
itself.
My initial response to to my father's story was to recall David Foster Wallace's
``This is Water'' commencement speech where he defends the value of a liberal
arts education and then I imagined myself giving an impassioned appeal to the
speech to this stranger presumably to uproarious applause and a standing 
ovation.
I suspect I am not alone in seeking benediction in Wallace's declaration that
``this is what the real, no bullshit value of your liberal arts education is
supposed to be about: how to keep from going through your comfortable,
prosperous, respectable adult life dead, unconscious, a slave to your head and
to your natural default setting of being uniquely, completely, imperially alone
day in and day out.''\footcite{david_foster_wallace_this_2005}
Something might be said here about a cave or a life not worth living.
However, there is a point worth considering in Wallace that isn't salient in
Plato.

After a didactic story about two young fish baffled by an older fish asking them
how's the water (``What the hell is water?''), Wallace proceeds to address the
other trope of a commencement speech: ``Of course the main requirement of
speeches like this is that I’m supposed to talk about your liberal arts
education’s meaning, to try to explain why the degree you are about to receive
has actual human value instead of just a material
payoff.''\footcite{david_foster_wallace_this_2005} 
I suggest that rather than consider the value of philosophy to a life as a
whole, in reflecting on Wallace, we consider the value of philosophy in the
context of a liberal arts general education some of us teach in.
In finding our purpose---both function and intention---in teaching a
philosophy class, it would be fruitful to consider the activity and purpose in
the context the activity is situated in.
That is to say, the question I want to answer is ``What are we doing when we
teach philosophy as part of a liberal arts general education?''

To do this, we have to also make sense about the point of a liberal arts
education is.
A common response educators like to give is that, rather than merely dispensing
knowledge, we are teaching students how to think.
Wallace proposes that what this actually means, in his words, is ``that the 
liberal arts clich\'e about teaching you how to think is actually shorthand for a
much deeper, more serious idea: learning how to think really means learning how
to exercise some control over how and what you think. 
It means being conscious
and aware enough to choose what you pay attention to and to choose how you
construct meaning from experience.  
Because if you cannot exercise this kind of choice in adult life, you will be
totally hosed.''
What I want to suggest is that Wallace's amendment is actually about having a
sense of wonder.
Moreover, I want to suggest that in contextualizing how some of teaching is not
merely in a liberal arts institution but a foundational component of a
liberal arts general education, the purpose of that kind of teaching is also to
inculcate a sense of wonder in a way philosophy is uniquely qualified to do.

Wonder, as an affective and conative attitude, is in contrast to learning
objectives in the cognitive domain like skills and knowledge.
While these can and should be outcomes in a philosophy class, they are secondary
and tertiary purposes.
We cannot appropriately understand primary purpose, as situated in the context
of providing in a liberal arts general education, in terms of learning
objectives in the cognitive domain alone.
This includes the misapprehension that our purpose is train philosophers or
teach how to do philosophy.
It may be asked what we kind of education are we providing if not an
education in academic philosophy given that we are philosophy professors.
I contend that is also a mistake.
Rather, in that specific context, we are public philosophers doing public
philosophy and not philosopher educators in the same sense as when we are
teaching in other contexts like high-level courses offered primarily to
philosophy program students.
In the context of providing classes that fulfill part of a liberal arts general
education, our primary purpose is most appropriately understood as
communicating, and hopefully inculcating, a sense of philosophical wonder.

To this end, I argue that given the liberal arts general education context in
which our introductory philosophy courses are situated in, we should
conceptualize these courses within that rather than in the context of academic
philosophy solely.
Given the likely facts of our students, we cannot appropriately hold that our
primary, but not exclusive, purpose is to train philosophers.
Moreover, it is similarly inappropriate from the broader liberal arts
perspective to conceive of such courses as a subject matter course readily
reducible to learning objectives in the cognitive domain only.
Rather, we should conceptually orient our general education classes towards the
purpose they are uniquely qualified to do: inculcate philosophical wonder.
This is rooted, in part, from the observation that such courses, and the
subjective experience of students, are not obviously aimed at the acquisition of
learning outcomes in the cognitive domain like philosophical skills or knowledge
of philosophy.
By making this purpose explicit to at least ourselves, we can hopefully better
facilitate a necessary and important epistemic and moral function by inculcating
the epistemic and moral emotion of wonder.

Finally, some brief notes on the following account.
I will be concerned primarily on the American liberal arts general education
system because that is what I have experience teaching in.
How the following applies to other national higher education contexts depends
upon the degree to which they relevantly resemble the American one.
That being said, it is my understanding that in places where higher education
consists primarily of courses within the subject of study, like the United
Kingdom, a something that resembles a liberal arts general education takes
place in secondary education rather than post-secondary.
To that end, what follows may be applicable, and even reason for, earlier
philosophy education.
I won't however, comment on this as it is outside my experience.
In light of that, the following should not be taken as making
assertions to the activity of teaching philosophy in of itself.
While it may be pedagogically or rhetorical to consider how we can encourage
wonder, I do not argue that this is the primary purpose in all philosophy
courses.
Graduate seminars, for examples, are appropriately understood in terms of
learning objectives in the cognitive domain because it is obvious that we are,
in that context, training philosophers.
Given the contingency on the context of the activity of teaching, I will rely in
part on autoethnographic reflections on teaching in this analysis.
In light of the contingent nature of the project, hopefully resonance on these
points rather than analytic or empirical truth suffices.
Finally, throughout I will refer to philosophy for the sake of brevity  but am 
open to it being understood in the sense of philosophy and adjacent subjects 
like bioethics.

\section{}

I will admit that it may sound peculiar to suggest our purpose in teaching
philosophy is not, first and foremost, student acquisition of learning outcomes
in the cognitive domain like knowledge or skills.
Some teachers may already be sympathetic to where I'm heading with this but I
still want to note that, for many, this may be counter-intuitive or sound like
schmaltzy sentimentalism.
I'd have probably felt this way when I started out.
Here's my concession to them: There is nothing intrinsic to philosophy itself
that entails it cannot be taught in terms of cognitive learning outcomes.
However, we do not teach philosophy in a vacuum.
Consequently, the nature of what we are doing, the kind of activity we are
engaged in, is molded by the context it is situated in.
Thus, the context should inform our activity.
It is only continent on the context of teaching philosophy as part of a liberal
arts general education.
In doing so, I contend, it becomes clear that while cognitive learning outcomes
are salient, they are not the primary purpose.

The claim here is not that cognitive learning outcomes do not exist in
philosophy or that they are intrinsically undesirable.
I do not advocate for striking learning outcomes and course objectives from your
syllabi.
There are, if nothing else, pragmatic reasons of a compliance with
administrative-bureaucratic demands.
The claim is simply that, when we consider our pedagogical goals within the
context of a liberal arts general education, it is inappropriate to put it
purely in terms of acquiring cognitive outcomes.
When I say teaching in a liberal arts general education, I mean it in a narrow
sense.
I am referring to teaching only those classes that are offered with the
intuition of serving that liberal arts general education.
This obviously includes courses that satisfy a compulsory coursework requirement
for a student's general education: introductory philosophy courses, introductory
ethics courses, and professional ethics courses.
It may also include ``fluff'' electives insofar as they share the relevant
characteristics of the prior.
It does not include upper-level courses intended for students committed to a
philosophy or philosophy-adjacent program of study: undergraduate majors or
minors and graduate students.
The liberal arts general education context referred to here is not that any and
all teaching at a school with a liberal arts program but only that teaching that
is intended as part of the general education of the general student body.

It was not immediately apparent to me that contextualizing teaching in this way
is significant.
I had long struggled with what my purpose was in this activity and how to square
that with assessments and other administrative-bureaucratic miscellany.
The puzzle was helpfully clarified for me in a conversation with a colleague.
We were discussing new prep to meet the school's new core requirements.
He had said, ``I don't understand why we have to teach these reading
requirements.
I hate teaching all this modern and contemporary stuff.
If had my way, I would just teach only Plato for the entire semester.''
My jaw hit the floor.
After prompting, he followed up with the following reasoning: 
``Everything in philosophy is a response to Plato. 
It doesn't make sense to teach later stuff before teaching Plato. 
It's like jazz. 
If you want to play jazz, you don't start with improvisation and stuff.
In order to learn to play jazz, you have to learn the fundamentals first.''
This seemed to be palpably not right.
It took me a while to realize why.

The difficulty this posed for me stemmed from the fact that I was not inclined
to think that he was wrong.
I narrowly avoided becoming a musician growing up when I realized how many hours
I would have to dedicate to practicing scales If I wanted to become more
competent.
I had gone to a music camp in the mountains.
I walked around seeing the good students and the professionals there to teach
stood around doing scales over and over again constantly.
This, teenaged me decided, was not how I was going to spend my time.
Nevertheless, I agree with the general principle because of this background.
Moreover, I'm inclined to think it's an apt analogy.
I did my undergraduate degree in the United Kingdom.
In my program, it was compulsory to take six specific introductory courses on
different aspects of philosophy.
One requirement for the first semester was \textit{Introduction to History of
Philosophy I} which consisted of only ancient Greek readings.
In other words, I was an example of what my colleague had prescribed.
That education was the foundation to whatever debatable competencies I have as a
philosopher, also debatable, today.
Consequently, I could not hold that his reasoning was wrong without having to
cast doubts on my past that would open the door to ever more impostor syndrome.
Yet, I had a conviction that what he had said was problematic in some way.

I contend that while his reasoning was not wrong per se but that it was
inappropriate.
It was inappropriate given the fact that it was uttered within the context of an
American liberal arts general education system.
In such a context, it is a mistake to think that what we are doing is like
teaching people how to play jazz.
If we believe that our primary purpose in a general education classroom is to
train philosophers, then we have simply misapprehended what we are doing.

The brute realities of our activity do not support this belief.
There are two particularly problematic issues here.
First, given the fact that such courses are offered as part of a liberal arts
general education, it is not the case that our students are philosophers.
These courses are typically compulsory---or, at least, an option within a
compulsory choice---for the general undergraduate student body.
Consequently, the students we teach in such classes are diverse ranging from the
humanities to medicine to STEM.
Only a small minority of them are likely to be philosophy students.
Anecdotally, I estimate the most I have had is somewhere in the region of
1--2\%.
At worst, in a professional ethics course, like business ethics or nursing
ethics, there are likely to be no philosophy students given the course is being
offered primarily to students of that particular professional program.
Given the demographics of these classes are dramatically different than other
philosophy courses offered primarily to philosophy students, it must surely be
the case that this should inform how we understand the activity we're engaged in
and, thereafter, the primary purpose we have in mind.

As such, it is patently the case that it is inappropriate to think that what we
are doing is training philosophers and, therefore, primarily interested in
cognitive learning outcomes relating to competence in philosophy. For one, it is
inappropriate because it is premised on a problematic relationship between the
students and the professor.
It is wrong to neglect the vast majority---possibly all---of the students in our
classroom.
It is neglect to conceive ourselves as primarily training philosophers and
putting cognitive learning outcomes relevant primarily or even only to academic
philosophers.
If fails to properly recognize, consider, and respond to the alternative
interests of those in the classroom who are not interested in pursuing academic
philosophy.
To put it another way, it is inappropriate because it is a failure of our
responsibility to care for our students.
It would be inconsiderate and callous not to make the class a worthwhile
experience for the students.
Coercion is perhaps too strong of a word.
The fact of the matter is that the vast majority of them were compelled to
enroll in our class contrary to what they would have chosen in the absence of
mandatory graduation requirements.
It is wrong to begrudge them of their compliance to a system they cannot change.
Moreover, it is a compelled decision for which they---or at least someone---is
paying a significant sum of money for.
This is not to say that our class should be understood as a commodity they have
bought and paid for.
Rather, that these are normative reasons to try to do good by them.
Call it fairness, justice, care, or whatever.
Obviously, I do not take this to mean what we provide should be understood in
financial terms---what is the financial value of wonder?
Rather, we should aim at outcomes that are worthwhile and valuable to everyone
generally generally the possible diversity of backgrounds and interests.

It might be argued that the acquisition of skills relevant to being a
philosopher has generalizable value.
After all, to be successful in philosophy, one must demonstrate competence in
logic, analysis, argument, rhetoric, reading complicated texts, writing,
communicating complicated content, problem solving, et cetera, et cetera.
I held this view for a while.
I still think these are valuable secondary purposes for us to pursue.
However, I am dubious to the extent we can be understood as sincere in this
regard.
To put it another way, I suspect that in most philosophy classes this is,
unintentionally, mere lip service.
What I suggest is that if we were sincerely committed to the notion that we are
primarily aiming to track the skills relevant to doing philosophy as
generalizably useful and not mere to train what few philosophers are in the
room, then we would be committed to explicit instruction of these skills as
generalizable.
I can note some philosophers that have done this.
But in them being notable, it suggests that they are exceptions to the rule
rather than representatives of it.

Certainly, on reflection of my own experiences, I realize now that this was not
the case.
If reading, comprehending, and analyzing philosophy texts is a transferable
skill, it is not obvious to me that I was given instruction in it besides
``sometimes it will take an hour to read and understand a single paragraph.''
As David W. Concepci\'on observes:
\begin{quotation}
The relative absence of appropriate ``How to Read'' material is peculiar. 
As years of listening to plaintive students teaches, intelligent and literate 
general studies and early major students lack the skills needed to read 
philosophy well.
Students are not familiar with the folkways of academic philosophy and are too
often left to learn them through trail [sic] and error. 
But we do our students a disservice if we let them flounder through with 
nothing but trial and error.
Philosophy professors should not ask students in introductory or early major
classes to spend three hours or more per week doing something they have never
done before (i.e., read like a philosopher) without telling them how to do it.
This is particularly true since we know they are likely to think reading
philosophy is just like reading anything
else.\footcite[p.351-2]{concepcion_reading_2004} 
\end{quotation}
That is to say, that explicit instruction on those skills ought to do  if we
expect students to use them.
Yet this is something that is notably neglected.
In light of this, it seems hard to sincerely say that we teach generally 
valuable skills in the course of doing philosophy if there is a dearth of
explicit instruction on that skill at all.

Or, consider writing.
The majority of my education in philosophy consisted in final papers as the sole
assessment.
Consequently, the only writing feedback I received was well after the course had
ended.
What competence in writing I acquired, if any, was a function of trial and error
over an entire degree program rather than in any course itself.
Further, the majority of my feedback was chiefly concerned with the
philosophical substance of the argument.
This is troubling in light of Jonathan Bennett and Samuel Gorovitz's observation
that ``If we only look to the substance, we encourage our students to undervalue
the importance of the quality of writing.''\footcite[p.
10]{bennett_improving_1997}
That is to say, if what feedback we do give is primarily in regards to the
quality of their philosophy then we are not likely to be helping students
improve the quality of their skills as writers.
They charge that ``too much academic writing, even in prestigious
journals and books, is `academic' in the worst sense.''\footcite[p.
10]{bennett_improving_1997}
In this light, it looks less than hopeful that we can sincerely say that we are
trying to teach transferable writing skills as part of our primary purpose in
these classes.
There is simply too little explicit instruction to this effect in most cases to
support that claim.

For the sake of brevity, I will only make a few comments in regards to the
suggestion that the skill we teach is how to think or how to be rational.
I hope that we can at least make a better case that we do this.
I also hope that we do a better job in this regard.
What I will say is that I find it difficult to gauge my confidence if it is the
case that there is doubt regarding the competencies of how well students provide
input into their critical thinking process, e.g. reading, and how well students
provide the outputs from that process, e.g. writing.
If they struggle at either end, I'm not sure how to assess the black box in the
middle.
I'm less than hopeful, however, that we give much of any explicit instruction on
how to whatever philosophical thinking skills they exercise can be transferred
outside the field with it's idiosyncrasies.
When I have mentioned to other teachers that I incorporate some superficial
details from computer science when I teach logical connectives, they generally
respond in either surprise or confusion.
This, to me, is the bare minimum I could do to show that the lesson has broader
connection and utility than in just analyzing philosophical arguments.
If even this is novel, I am gloomy that there is much explicit instruction on
how whatever way we teach students how to think is useful outside of our
classroom.

This focus on explicit instruction as the relevant evidence on the degree to
which we can sincerely say that teaching students these skills is part of our
primary purpose might be contentious.
For one, it might be said that this is a philosophy class, after all, we both do
not have the time nor the responsibility to provide explicit instruction on all
these things and teach about philosophy in any depth.
I agree.
That is why I suggest that the acquisition of these cognitive learning outcomes
is not our first and foremost purpose even though it is hopefully secondary or
tertiary.
It might otherwise be objected that these outcomes can be acquired without
explicit instruction: they are learned implicitly through philosophical
training.
I suspect this is how most of us think this is supposed to work.
We teach students to use these skills in philosophy and then they will just work
out how to apply them outside of philosophy on their own.
If they do well in philosophy, then we can assume they have become better
thinkers in general.
This seems problematic to me because I have a hard time distinguishing between
the degree to which this actually happens and surviorship bias of the trial and
error training that tends to exist in academic training.
If our classroom is sink or swim given the absence of explicit instruction, then
it seems odd to just throw them in the deep and point to the ones that haven't
sunk and say this is evidence we are teaching people to swim.

\section{}

% reframe as making distinct philosophy classes as something special

Alternatively, it might be argued that there is something intrinsically valuable
about acquiring knowledge about philosophy that is our primary purpose.
This, I think, is unlikely to be the kind of tack philosophers will take.
Rather, I suspect it is the kind of reasoning used by non-philosophers
justifying the requirement of philosophy in a liberal arts general education.
That is to say, that philosophy is simply a subject matter course students are
expected to have some familiarity with like english literature or history.
No offense intended.
This approach escapes the problem of the prior by situating the activity in its
context.
However, I do not take it to be a successful one because it attempts still to
conceptualize our purpose i just cognitive learning outcomes.
This is inappropriate because it is not at all obvious that a well-designed and
successful philosophy course has to do beyond a minimum.
To put it another way, philosophy is not appropriately measured by and then
asked to design around cognitive learning outcomes because we are doing
something outside the cognitive domain.

I take as salient example to this kind of position the
bureaucratic administrators who insist on faculty using the language of Bloom's
taxonomy.
Bloom's taxonomy is meant to be a standardized measure by which different
concerns from different institutions can be compared.
One problem with the measure, though, is that it's use focuses nearly
exclusively on knowledge acquisition.
Note, that in it's original proposal it was never suggest that it was an
exhaustive model.
Bloom originally proposed three domains captured in the model: the
cognitive, affective, and psychomotor.
Yet in actual practice, ``Bloom's taxonomy'' is near universally used to refer
to the taxonomy of the cognitive domain.
In the 2001 revised edition, only the cognitive domain is discussed in depth.
It offers the following hierarchical model of educational learning objectives:
Remember, Understand, Apply, Analyze, Evaluate, and Create.
Learning objectives are phrased in terms of a verb associated with one of these
levels and an object of knowledge or information.
This ranges evaluations of whether the student can recall the information,
explain it, use it in a new way, and make distinctions to things like whether
they can defend some view about the information or create some knew knowledge
from the evidence they have.
Notably, these are about either acquisition or deployment of either information
or knowledge.\footcite{armstrong_blooms_nodate}

The use of these measures is pervasive in education.
The difficulty is that they are treated in practice as exhaustive and imposed as
stipulations around which a course should be designed around.  As one university
resource notes, ``Section III of \textit{A Taxonomy for Learning, Teaching, and
Assessing: A Revision of Bloom’s Taxonomy of Educational Objectives}, entitled
“The Taxonomy in Use,” provides over 150 pages of examples of applications of
the taxonomy. Although these examples are from the K-12 setting, they are easily
adaptable to the university setting.''\footcite{armstrong_blooms_nodate}
It is presumed in use that the taxonomy can readily applicable to any course and
captures all possible, or all the important, pedagogical objectives.
Further, it is imposed as the language by which we establish our objectives and
deliver appropriate instruction and assessment.
I suspect quite a few philosophers, myself included, experience either
frustration or unease when volunteered to do this.
This, I contend, is a reasonable response because it responds to the
inappropriateness of trying to reduce our purpose into the cognitive domain
alone.
These do not exhaustively or aptly describe the kind of pedagogical goods a
successful philosophy course may be designed around.

Ironically, to see this we need only to turn to Plato.
We have in the Apology the case of Socratic ignorance---the awareness of our own
ignorance---as the explanation provided for the attribution of being the wisest
of all to Socrates.
More relevantly, this outcome is used as a pedagogical device in Plato's early
works as \textit{aporia}.
This is the state of puzzlement that occurs when faced with a philosophical
dilemma.
Significantly, aporia is used as a desirable outcome, not the starting point of
learning, in many of the early Socratic dialogues.
Socrates engages with an interlocutor who claims to have knowledge and offers
this to Socrates.
Socrates then considers this and then points out the flaws in the interlocutor's
account.
The interlocutor will then offer an alternative in response and this process is
repeated until, in the aporetic dialogues, Socrates's interlocutor admits that
they do not know and walks away.
Notably, in these texts, Socrates does not provide an answer he claims to be the
truth.
Significantly, what is demonstrated in these works is the acclaimed Socratic
method.
This well established pedagogical method is used from the starting point of
knowledge to arrive at a state of ignorance as an outcome.
I once explained this to one of those education people leading a compulsory
course for faculty when they asked me to describe my learning objectives purely
in terms of Bloom's taxonomy verbs. 
I stated I was unsure how to do this so I asked him how I would translate aporia
as a learning objective.
He wasn't happy.

Of course, what is really going on is the destruction of the appearance of
knowledge, rather than knowledge per se.
Still, it is not obvious that this is neatly categorized in Bloom's taxonomy.
This is especially not the way they want the taxonomy to be used: to be ale to
point at some stuff and say that the students walked out of the classroom with
more stuff in their heads than before.
However, what we do when we teach introductory philosophy courses is not
education in the conventional, or at least this, sense.
We can frame aporia in Bloom's verb-object formula if we're a little bit
creative.
Dispel bad beliefs.
Analyze and critique things they mistake to be knowledge.
But these are either impermissible verbs (use the ones listed here!) or
undesirable objects.
I suggested these in that mandatory workshop.
He, even more unhappily, asked me how I was going to objectively assess these
objectives and then told me to just give them some basic comprehension test.

It is, I think, obvious to many of us and our students that a philosophy class
can be a unique experience distinct from the other courses they may take.
The more successful classes I've seen have done this.
Classes based primarily on didactic lectures have tended to have difficulties.
Students tend to get lost and quickly check out.
They express their frustration and bafflement.
We can be quick to put the blame on them, but we should consider that what we're
asking them to do may be quite novel to them as students.
It is unclear to them what knowledge they are supposed to be memorizing and
regurgitating and, as a consequence, become disoriented.
Of course, this is not what we want them to do but this can be more or less
apparent to them.
The more successful have largely integrated dialogues.
While this is not always possible, I think it's worth noting that one thing that
this methodology can do implicitly is make the experience of the class
distinctive and, by association, help the students recognized that the purpose
is not the same as in other classes they may have taken.

This is important for the students to recognize because the way we tend to
incorporate aporia in our own pedagogy.
It seems to be the case that students in the primarily didactic classes
struggled with philosophy as an activity rather than as a subject.
They are perfectly capable of the more conventional educational tasks.
They can recall, digest, and explain what whoever said about whatever.
However, when asked to analyze and evaluate, they clam up.
I will admit in my first class I taught, I tended to be didactic.
However, there was one point when I had asked a question and a student had
spoken up frustratedly, ``But you're just asking for opinions, not facts. When
are we going to get to the answers?''
I realized then that I had screwed up.
I realized that the students had been misapprehending philosophy as a mere
subject matter course and not as an activity I was asking them to engage in.
They had grown up developing an aptitude as students their entire life in
courses where they learned about a thing rather than really doing it.
While the latter depends in part on the prior, I suggest that this is not really
the kind of activity we're engaged in.

In part, it is inappropriate to conceive of a philosophy course as a subject
matter course because an element of aporia is inevitable.
We raise questions and pose puzzles but we do not provide the answers to them.
The textbook does not have the answers in the back of the book.
If we teach different theories about the same subject, which presumably we would
do unless you do something like only teach Plato, then we present to the
students competing and usually contradictory things from week to week.
When we then ask them to evaluate what they've been presented with and to come
up with a position, then they are quite reasonably going to be puzzled and
bewildered.
The degree to which we don't make this explicit to the students can hinder them.
They will find the experience frustrating.
They may well give up or double down on their priors instead of engaging in
philosophy.
Insofar as didactic teaching obfuscates this by being methodologically and
Phenomenologically similar to other subjects where the instructor is the
definitive authority and font of facts from which the students drink and then
regurgitate.

I think successful philosophy classes are successful because they effectively
incorporate and utilize this inevitable aporetic element.
They do not encourage the misconception that the students are passive receivers
of the professor's wisdom.
As an MA student, an undergraduate student once confided in me that she dropped
her philosophy minor because, ``It's all made up. There's no right or wrong
answer.''
The problem here is that there clearly is a right or wrong answer.
We simply do not know what it is.
It is a mistaken expectation to think that the right answer will be presented as
such.
The accusation that philosophy is all made up is indicative of a
miscomprehension about what kind of activity we're engaged in.

I have never seen a successful business ethics class.
Admittedly, my only experience with business ethics is as a grader for one
professor.
That being said, I have heard similar frustration from others with more
experience.
The problem is not that they cannot or do not achieve the kind of cognitive
learning objectives conventionally set out by Bloom's taxonomy, or at least how
it's conventionally applied.
Moreover, they do decently on that kind of assessment as well.
They are smart, motivated students.
The problem is not whether or not they achieved the cognitive learning
objectives set out in the course.
The failure is that they do not engage with philosophy as an activity.
They can read, regurgitate, and explain the material.
The issue is that they are a self-selecting population with clear and strongly
held priors.
They are in business school to make money.
That is not to say that they are consequently unethical people.
Most of them state in the beginning that they are interested or even
enthusiastic about learning how to be ethical.
The issue is that an ethics course doesn't train students to be ethical. A
philosophical ethics class is not going to be a class on conduct: compliance
with the law, compliance with the corporate ``values'' statement, and so on.

It can't be because all the philosophers who just read that immediately thought
to themselves, ``Did this guy just assume that the rules of conduct are always
ethical? What an idiot!''
I used to think that the point was to teach people to think in a way that
cohered with at least some theory of ethics.
At present, I am only confident that we aim to teach people to just think about
ethics.
This poses an obstacle, however, because in thinking about ethics with any
seriousness, and especially equipped with a little knowledge about philosophy,
they will quickly reach an aporetic impasse.
Given the theoretical ambiguity, they will not have a clear answer or a clear
way to integrate these ethical considerations into their prior framework.
Thus, they will struggle to see how they will evaluate ethical considerations
alongside competing interests---namely, financial interests.
The vague unknown loses out to the clear and entrenched belief.
I suspect the business ethics courses I've observed failed because I was not
confident that it had a durable and lasting effect on the students such that
they would, when the time came, spontaneously pause and think about ethics in a
way the class hopefully equipped them to do.
It is not clear to me how to make sense of that experience in terms of the
cognitive domain alone.
It's like being dragged to an art lesson; receiving a set of water colors and
brushes; and then going home to stick them in the box at the back of the coat
closet next to the running shoes you bought years ago and wore twice.
It is, however, apparent to me how to account for that kind of issue in terms of
affective and conative attitudes.

A final comment before moving on.
The above accounts may paint a more pessimistic outlook than I actually have.
Philosophy courses in a liberal arts general education are often successful and
wonderful experiences.
I suggest that it is because they do something more than achieve cognitive
outcomes.
Rather, they inculcate the affective and conative attitude that can be
appropriately understood as our primary purpose.
Admittedly, I think this is often done unwittingly but the hope here is that by
making it explicit, at least to ourselves, we can more reliably succeed.

An example: recently, I asked a colleague teaching nursing ethics how it was
going and if she encountered analogous issues to what I saw in business ethics.
She had said that as a trained midwife and not a philosophy primarily, she
focused her class away from the theoretical side applied to issues but more
towards the anthropological side of exploring the diversity of standpoints
around the issue.
The positive student feedback she described to me was illuminating.
I remember particularly that one general theme was that the students said
something like the following: ''I never realized you could think about this
issue in so many different ways. I thought I already knew the ethics about this
but now I realized I don't. I enjoyed thinking about these issues in all these
different ways I had not realized before.''
I would be over the moon.
I take this as a clear indication of a successful professional ethics class.
She took a group of students who, I suspect, are self-selecting to have clear
and strong held priors---``I'm going to be a nurse to help people so there's no
question I'm ethical''---and left them with a durable and lasting inclination to
think about ethics spontaneously when the time comes.
Notably, this is not because they found a strong belief to replace the
appearance of knowledge that was lost.
What's relevant is the affective and conative reaction to the unexpected move
from certainty to uncertainty.

This, as I will argue, is wonder.
The point of the activity we are trying to get the students to participate in
part depends on the fact we don't know which answer is right or wrong.
That it's all ``made up'' is, in a figurative sense, the purpose as both
function and intention.
More precisely, it is the affective reaction to that state of uncertainty which
generates a conative drive to latch onto the question and speculate in the hopes
of coming closer to meaningful understanding.

\section{}

Facilitating this sense of wonder is a crucial role that philosophy is uniquely
qualified to serve.
Wonder has a long history of association with philosophy.
Socrates tells Theaetetus that ``this feeling of wonder shows that you are a
philosopher, since wonder is the only beginning of philosophy''\footnote{155d}
Descartes places it as the first of all passions from which all others
originate.\footnote[p. 350]{cottingham_passions_1985}
Adam Smith asserts that ``Wonder, and not an expectation of advantage from its
discoveries, is the first principle which prompts mankind to the study of
Philosophy''.\footcite[p. 190]{adam_smith_history_1811}
However, in considering David Foster Wallace's characterization of the
present day liberal arts, it becomes evident that insofar as the
liberal arts general education is aimed towards wonder, so to does philosophy in
it's service towards that general education.

It may contended that this is not what Wallace says.
After all, he never mentions wonder in the ``This is Water'' speech.
Despite this, I contend that what he describes is more aptly understood as
wonder.
He describes the project of a liberal arts education in a number of similar but
slightly different ways: ``choosing to do the work of somehow altering
or getting free of my natural, hard-wired default setting which is to be
deeply and literally self-centered and to see and interpret everything
through this lens of self''; ``being conscious and aware enough to choose what
you pay attention to and to choose how you construct meaning from experience'';
``conscious decision about how to think and what to pay attention to''; ``to
consciously decide what has meaning and what
doesn't''.\footcite{david_foster_wallace_this_2005}
It appears that Wallace thinks of whatever this is as being a strongly
volitional and cognitive act.

I argue that he is attempting to describe wonder by comparing the descriptions
from both ends to illustrate that they clearly refer to the same thing if only
through different sense.
What is clear is that Wallace holds that the real issue at stake is how to not
be ``slave to your head and to your natural default setting of being uniquely,
completely, imperially alone day in and day
out''\footcite{david_foster_wallace_this_2005}
The slave to your head bit belies the tension Wallace has with volition:
\begin{quotation}
Think of the old clich\'e about quote the mind being an excellent servant but a
terrible master.  This, like many clich\'es, so lame and unexciting on the
surface, actually expresses a great and terrible truth. It is not the least bit
coincidental that adults who commit suicide with firearms almost always shoot
themselves in: the head. They shoot the terrible master. And the truth is that
most of these suicides are actually dead long before they pull the
trigger.\footcite{david_foster_wallace_this_2005}
\end{quotation}
The degree to which we can exercise our will to what Wallace refers to is not as
volitional as his word choice initially suggests.
In his account of exercising this volition, it is evident it is not a simple
choice of will: ``Again, please don't think that I'm giving you moral advice, or
that I'm saying you are supposed to think this way, or that anyone expects you
to just automatically do it. Because it's hard. It takes will and effort, and if
you are like me, some days you won't be able to do it, or you just flat out
won't want to.''\footcite{david_foster_wallace_this_2005}
Another tension worth point out is how the volition to ``choose to think'' is in
tension with the ethics of it.
He stresses that what he describes is not a virtue or in any way moral advice.
And yet, he also asserts that what he describes ``involves attention and
awareness and discipline, and being able truly to care about other people and to
sacrifice for them over and over in myriad petty, unsexy ways every
day.\footcite{david_foster_wallace_this_2005}
It is hard, then, to see how this as not morally charged if it is both a choice
and way in which we relate and care about other individuals.

The volitional aspect may appear contradictory to wonder as wonder is frequently
associated with surprise.
This stems from Descartes's description of it as the first of all passions:
\begin{quotation}
  When our first encounter with some object surprises us and we find it novel,
  or very different from what we formerly knew or from what we supposed it ought
  to be, this causes us to wonder and to be astonished at it. Since all this may
  happen before we know whether or not the object is beneficial to us, I regard
  wonder as the first of all the passions. It has no opposite, for, if the
  object before us has no characteristics that surprise us, we are not moved by
  it at all and we consider it without passion.\footcite[p.
  300]{cottingham_passions_1985y
\end{quotation}
It is a matter of interpretation on how much of this hangs on the element of
surprise.
I suggest that surprise is simply a salient expression of wonder since it is so
distinctive and discrete a moment.
However, it is not a necessary condition but only a sufficient one.
Rather, what is necessary is the presence of uncertainty either as novelty or as
realization that moves us from certainty to uncertainty.
It can consist in the state of puzzlement as we recognize that our prior
appearance of knowledge is destroyed upon contradiction.
In a this sense, even when we have wonder towards something we had prior regard
or conception, the move towards uncertainty is as if we regard it for the first
time with new eyes.

In light of this, the phenomenological descriptions of wonder are notably similar
to what Wallace describes.
As Katherine Dean Moore puts it, ``Wonder is the opposite of boredom,
indifference, or exhaustion---the lapse into unseeing
familiarity...''\footcite[p. 289]{moore_truth_2005}
Sara Ahmed's description has even more resemblance:
\begin{quotation}
It is hence a departure from ordinary experience; or, by implication, the
ordinary is not experienced or felt at all. We can relate this non-feeling of
ordinariness to the feeling of comfort, as a feeling that one does not feel
oneself feel\ldots What is ordinary, familiar or usual often resists being
perceived by consciousness. It becomes taken for granted, as the background that
we do not even notice, and which allows objects to stand out or stand apart.
Wonder is an encounter with an object that one does not recognise; or wonder
works to transform the ordinary, which is already recognised, into the
extraordinary.\footcite[p. 179]{ahmed_cultural_2004}
\end{quotation}
Wonder and the way it transforms the ordinary as described here is remarkably
like Wallace's fish story---``What the hell is water?''---and parting
advice---``It is about the real value of a real education, which has almost
nothing to do with knowledge, and everything to do with simple awareness;
awareness of what is so real and essential, so hidden in plain sight all around
us, all the time, that we have to keep reminding ourselves over and over: `This
is water.' `This is water.'''\footcite{david_foster_wallace_this_2005}
It is evident, then, that what Wallace talks about and wonder have a large
degree of phenomenological overlap as it relates to the ordinary or
default setting.

In regards to the apparent cognitive and affective disagreement, I also contend
that the difference is not as great as it appears.
I suggest that Wallace is attempting to grasp---ever so slightly
unsuccessfully---for the concept of wonder by bringing it out from the negative
of the ordinary or default setting.
It is clear that Wallace holds that, in contrast to what he describes, ``The
alternative is unconsciousness, the default setting, the rat race, the constant
gnawing sense of having had, and lost, some infinite
thing.''\footcite{david_foster_wallace_this_2005}
This element of the default setting is used frequently throughout as the
constant by which he makes the juxtaposition.
Yet, it should be noted that what is describes in that juxtaposition is at times
described as a kind of thinking and at other times a kind of attention or
awareness.

Significantly, both are elements of wonder.
Wonder is not merely affective but also conative.
Moore defines wonder as ``an attitude of openness or receptivity that leads a
person from a preoccupation with self into a search for meaning beyond
oneself.''\footcite[p. 269]{moore_truth_2005}
Wonder is both thinking and awareness.
The affective response to surprise or uncertainty seamlessly leads into a
conative hunger to fix our attention to the apparently meaningful object in 
question and to make sense of it.
The conative element, as Ahmed puts it, is ``the desire to keep
looking''\footcite[p. 180]{ahmed_cultural_2004}
That being said, wonder is not altogether rational even though it is
action-guiding.
The dark side of wonder is evidenced in horror, loneliness, and desperation.
Surprise can be the grip of fear.
The desire to keep looking can become obsession.
The hunger for meaning can give us false hope or despair when it leads us astray
from the truth.
However, Jeremy David Bendik-Keymer argues that the subjective nature of wonder
is what is essentially special about it:
\begin{quotation}
\ldotsnamely, wonder’s capacity to consider things in an open way that forms
surprising connections. This capacity needs to be subjective. If I could not wonder
from A to Anything and from Anything back to B, thereby linking A to B in wonder
through a surprising connection, I would not really be using wonder. Wonder is
speculative and in that subjective.\footcite[p.
341]{bendik-keymer_reasonableness_2017}
\end{quotation}
It is this speculative feature of wonder that makes it an epistemic emotion.
Although it is not, by itself, rational and may be beguiled by appearance, it
motivates inquiry from a state of uncertainty.
The conative hunger leads to speculation---thinking---in order to satiate
itself.
In this sense, wonder is both awareness and thinking.

Hopefully at this point, the case has been made that what lies under the surface
of ``This is Water'' is actually wonder.
Additionally, it may now be clear why inculcating wonder is most appropriately
understood to be the primary purpose of teaching philosophy in a liberal arts
general education.
As an epistemic emotion, both affective and conative, it can generate a durable
and lasting receptivity to observe and inquire.
Moreover, it does so cohesively in our situatedness in the liberal arts general
education context.
It is both relevant to philosophy students in their training as philosophers
given it lies at the root of philosophical inquiry and it is relevant to general
education students because it is, as Wallace puts it, the real, no bullshit
value of a liberal arts education.
Inculcating a sense of wonder is what the liberal arts clich\'e about ``teaching
you how to think'' is actually shorthand for.
It is intrinsically valuable for everyone.
Moreover, contrary to the stranger in my father's story, it is both a
therapeutic and prophylactic.  
As Moore asserts, ``Wonder deepens lives that
might otherwise be shallow, probing depths of meaning and allowing a person
fully to experience the rich texture of a life.''\footcite[p.
272]{moore_truth_2005}

There are two possible objections to this.
First, briefly, is that, given wonder's nonrational nature, it is inappropriate
to hold it as the primary purpose of teaching philosophy, a presumably rational
field of inquiry.
Further, given the above discussion of aporia in the pedagogy of philosophy, why
not hold something like that as the primary purpose and allow rational inquiry,
rather than wonder's subjective conative fixation, to lead us out?
The answer is partly in the question.
When I say wonder is the primary purpose, I mean only first and foremost and not
in exclusion.
An ideal philosophy course would both inculcate wonder and provide the rational
tools to temper it.
Moreover, while rationality may lead us into aporia, and thus uncertainty, I
suggest that rationality alone may not be enough to lead us out of it.

To explain, we should understand why aporia alone is inappropriate.
While it rests on uncertainty, like wonder, it is purely distinctive.
While it is worthwhile to unmoor students from bad beliefs, we should recognize
that we cannot abandon them in those waters.
Too much uncertainty is corrosive to our being.
Theaetetus when declares he wonders, he also declares that it is also
disorientating.\footcite{155d}
Both wonder and aporia, predicated on uncertainty, can be disorientating.
But in destroying the appearance of knowledge, either through wonder or aporia,
we risk doing much more to our students than just having them exercise their
rationality.
As Concepce\'on observes, to read philosophy requires courage:
\begin{quotation}
  The experience of reading philosophy is often disquieting. When reading
  philosophy, the values around which one has heretofore organised one’s life
  may come to look provincial, flatly wrong, or even evil. When beliefs
  previously held as truths are rendered implausible, new beliefs, values, and
  ways of living may be required. This philosophical cut at one’s core beliefs,
  values, and way of life is difficult enough. What’s worse, philosophers
  admonish each other to remain unsutured until such time as a defensible new
  answer is revealed or constructed. Sometimes philosophical writing is even
  strictly critical in that it does not even attempt to provide an alternative
  after tearing down a cultural or conceptual citadel. The reader of philosophy
  must be prepared for the possibility of this experience. While reading
  philosophy can help one clarify one’s values, and even make one self-conscious
  for the first time of the fact that there are good reasons for believing what
  one believes, it can also generate unremediated doubt that is difficult to
  live with.\footcite{concepcion_reading_2019}
\end{quotation}
While this hazard may be alright for those of us who volunteered, we should be
cognizant that the vast majority of our students were drafted.
We have a duty of care to help them back to safe harbor.
It is a problem, then, that aporia is purely destructive.
Wonder, however, as conative and even nonrational is an aspect of creativity
because of its subjectivity.
It's conative nature---the desire for meaning-making---is a crucial
action-guiding motivation to help them pursue new beliefs, values, and ways of
living instead of angst.

Moreover, it would be disingenuous to suggest that the subjective and
speculative nature of wonder as an epistemic emotion is not critical to
philosophy as an activity.
Philosophy is creative.
It is creative in the particular way that relies on what makes wonder
essentially special: by instantiating the kind of open, loose, and spontaneous
play between abstract concepts that generates intriguing connections and
delightful moves that no one else finds amusing.
Creativity abounds in philosophy even if it is forged and tempered by cold hard
rationality.
I'll begrudge that this is a bit like playing jazz.

Secondly, it might be objected still that holding wonder as our primary purpose
and relegating knowledge and skills relevant to doing philosophy to secondary
and tertiary status is inappropriate because it moves us too far afield from
what we are actually doing.
That is to say, even if we are not training philosophers, what we are supposed
to be doing is uniquely qualified to serve the purpose of inculcating wonder
towards the end of a liberal arts general education.
This, however, is contentious.
Science has a knack for inspiring wonder in the general public.
Or, more specifically, popular science has.
Moreover, the role of wonder in popular science is not unknown.

Adam Savage, one of the \textit{Mythbusters} and responsible figures for the
recent surge in popular science, is not quiet about wonder.
He recognizes the foundational role wonder serves in scientific inquiry:
\begin{quotation}
  'Failure is always an option' came up as a joke in season two, when we were
  screwing something up over and over again, but it’s an awesome way to think
  about the scientific method. We tend to think about science as a series of
  facts and absolutes that we need to study in order to understand stuff; a
  scientist saying, “I want to prove this thing,” and then coming up with an
  experiment to prove it. Nothing could be further from the truth on both
  counts. The scientist simply says, “I wonder if?” and then builds a
  methodology to test whether his theory is correct, or even to figure out what
  his theory might be. So to think that an experiment could “fail” is ludicrous.
  Every experiment tells you something, even if it’s just don’t do that
  experiment the same way again.\footcite{lahey_mythbusters_2014}
\end{quotation}
At a popular science demonstration, he describes the conative aspect of wonder
and how the desire to keep looking can serve an epistemic function:
\begin{quotation}
  I used to practice juggling for hours in my upstairs bedroom, and the sound of
  me dropping the balls over and over again---\textit{thump-thump-thump,
  thump-thump-thump}--—as they hit the ground was the sound of my teenage years.
  I spent entire afternoons practicing tricks that just would not work. But as I
  slept, and my brain fermented on them overnight, the next morning they would
  suddenly work. I thought I was just learning how to juggle, but I wasn't. I
  was learning how to learn.\footcite{lahey_mythbusters_2014}
\end{quotation}
Lastly, he recognizes the role the creative aspect of wonder can serve: ``Play
is simply a process of running experiments. We do things because they are fun.
And remember the difference between screwing around and science is writing it
down.''\footcite{lahey_mythbusters_2014}

It is interesting that the contemporary philosophical literature on wonder is
indebted to popular science.
Rachel Carson's writing on wonder is influential.
She, too, recognizes its intrinsic value:
\begin{quotation}
If I  had  influence  with  the  good  fairy  who  is  supposed  to  preside
over  the christening of all children, I should ask that her gift to each child
in the world be a sense of wonder so indestructible that it would last
throughout life, as an unfailing antidote against the boredom and disenchantment
of later years, the sterile preoccupation with things that are artificial, the
alienation from the sources of our strength.\footcite[p.
46]{rachel_carson_help_nodate}
\end{quotation}
The kind of wonder she is interested in, however, is that towards the natural
world.
In this advocacy of wonder towards the natural world, Carson inspired an
understanding of its moral significance:
\begin{quotation}
  One ghost crab, one haunted woman in the dark by the edge of the sea: in
  this image, Carson shows us that a sense of wonder is not just a way of
  feeling or a way of seeing, it is a way of being in the world. To contemplate,
  and thereby acknowledge the meaningfulness and significance of the other,
  opens the door to a moral relationship.\footcite[p. 271]{moore_truth_2005}
\end{quotation}
Wonder, then, is not just an epistemic emotion but also a moral one.
If this is the case, then the unique qualification of philosophy to inculcate
wonder is brought back into focus.
The wonder popular science inculcates is directed towards the natural world.
Insofar as the moral relationship generated by wonder is between ourselves and
the object of our wonder.

In which case, as science studies the natural world, philosophy, at least in
part,  studies the human world.
The role philosophy plays in wonder as a moral emotion should be clear.
As Ahmed asserts, ``it is through wonder that pain and anger come to life, as
wonder allows us to realize what hurts, and what causes pain, and what we feel
is wrong, is not necessary, and can be unmade as well as made. Wonder energises
the hope of transformation, and the will for politics.''\footcite[p.
181]{ahmed_cultural_2004}
This explains the unconvincingness in which Wallace insists that wonder is not a
virtue and not moral as well as the discomfort in the classist imagery.
As he advises,
\begin{quotation}
But most days, if you're aware enough to give yourself a choice, you can choose
to look differently at this fat, dead-eyed, over-made-up lady who just screamed
at her kid in the checkout line. Maybe she's not usually like this. Maybe she's
been up three straight nights holding the hand of a husband who is dying of bone
cancer. Or maybe this very lady is the low wage clerk at the motor vehicle
department, who just yesterday helped your spouse resolve a horrific,
infuriating, red-tape problem through some small act of bureaucratic kindness.
Of course, none of this is likely, but it's also not impossible. It just depends
what you what to consider. If you're automatically sure that you know what
reality is, and you are operating on your default setting, then you, like me,
probably won't consider possibilities that aren't annoying and miserable. But if
you really learn how to pay attention, then you will know there are other
options. It will actually be within your power to experience a crowded, hot,
slow, consumer-hell type situation as not only meaningful, but sacred, on fire
with the same force that made the stars: love, fellowship, the mystical oneness
of all things deep down.\footcite{david_foster_wallace_this_2005}
\end{quotation}
If wonder as a moral emotion builds a moral relationship with its object by
contemplating and acknowledging its meaningfulness and significance, then it
functions as an instrument by which we extend moral regard to its object.
How or what we chose to wonder about is morally charged, then.
If wonder can generate respect and care for nature when it contemplates nature,
wonder should generate respect and care for people when it contemplates people.

In considering the relationship between wonder and popular science, it is
obvious that it is in nearly all respects reflected to that of wonder and
philosophy.
Still, philosophy is uniquely qualified to inculcate wonder, alongside popular
science, in the liberal arts education because they have parallel purposes.
As science wondering about the natural world builds a moral relationship to the
natural world, philosophy---particularly ethics and political
philosophy---wonders about the human world and builds moral relationships
between individuals.

One last observation can be drawn here.
The strong analogy to popular science reveals why it is wrong to object that
what we are supposed to be doing is providing an education in academic
philosophy and not inculcating wonder.
It is interesting to note that Adam Savage, whose background is in art and
special effects, is very clear about whatever hat he is currently wearing:
``Look, we never set out to make something that was education. That was not on
the list of things we were thinking about.''\footcite{lahey_mythbusters_2014}
He is very careful to identify himself as a science communicator rather than a
science educator.
The mistake, then, is to think we are doing academic philosophy education in the
first place.

We are philosophy communicators doing popular philosophy.
Or, rather, public philosophers doing public philosophy.
This is only made salient by the analogy to science.
In 2016, the American Philosophical Association's Committee on Public Philosophy
was compiling a list of philosophers who had careers in public philosophy either
exclusively or concurrently with careers in academic philosophy.
They posted on the \textit{Daly Nous} to call for names of those who, given the
variety ways we might do public philosophy, met at least one of four criteria.
Of relevance is the fourth: 
\begin{quotation}
  are regularly involved in activities that bring
philosophy to the public, including broad promotional activities (e.g.,
festivals), websites and shows that aim to bring philosophy to a
non-philosophical audience, and outreach programs for traditionally
philosophically-underserved populations (such as prisoners, the elderly,
pre-college students, and the lay community more
generally).\footcite{weinberg_who_2016}
\end{quotation}
If this is a reasonably acceptable definition, then the kind of activity
addressed here fits the bill.
As already noted, given the liberal arts general education context in which our
teaching is situated in, we cannot conceive of our classroom as that of
philosophers.
It is comprised of students from the general student body as a whole.
While it may be physically located in the same building, it is, by these
standards, outside the philosophical academy.
Our students are a non-philosophical audience, arguably representative of the
kind of lay people that would show up to a public philosophy talk.
If this feature pertains, then there is no feature I can see that would rule out
such an activity that wouldn't also rule out some more conventionally accepted
case of public philosophy.
It can only be excluded by singling it out explicitly.
This may be good enough for definitions in legislature, but not for
philosophical work.
As an instance of public philosophy, we can appropriately understand ourselves
as communicators of philosophical wonder.

There are two graduate sole instructors heading towards their respective
classrooms, and they happen to meet a tenured professor walking the other way,
who nods at them and says, ``Morning, boys, how's the water? And the two grad
instructors walk on for a bit and then eventually one of them looks over at the
other and goes, ``What the hell is wonder?''

\nocite{plato_theaetetus_nodate}

\printbibliography

end{document}
