\documentclass[letterpaper,notitlepage,12pt]{article}
\usepackage[
  letterpaper,tmargin=1in,bmargin=1in,lmargin=1in,rmargin=1in]
{geometry}
\usepackage[backend=biber]{biblatex-chicago}
\addbibresource{wonder.bib}
\usepackage{setspace}
\usepackage[super]{nth}


\title{This is Wonder}
\author{Alexander Zhang}
\date{}

\begin{document}

\maketitle

\section{Introduction}

There is, I suspect, a point for every philosophy professor when they think
about teaching their class and ask themselves, ``What am a I even doing
here?''
It might be during new course preparation to comply with the latest
administrative-bureaucratic demands.
It might be in the middle of grading undergraduate papers or reading their
student--teacher evaluations.
It might just creep up on you whenever what you do for a living is mentioned in
public.
My father, an engineer, once told me a story.
He was in a meeting.
During his introduction, he proudly mentioned that his son was a philosopher.
A woman then audibly commented, ``How fucking useless.''
I still haven't figured out why he felt he had to mention that last bit.

Some might simply say that teaching is a requirement to get paid to research.
Others, hopefully, will believe that teaching philosophy is valuable in of
itself.
My initial response to to my father's story was to recall David Foster Wallace's
``This is Water'' commencement speech where he defends the value of a liberal
arts education and then I imagined myself giving an impassioned appeal to the
speech to this stranger presumably to uproarious applause and a standing 
ovation.
I suspect I am not alone in seeking benediction in Wallace's declaration that
``this is what the real, no bullshit value of your liberal arts education is
supposed to be about: how to keep from going through your comfortable,
prosperous, respectable adult life dead, unconscious, a slave to your head and
to your natural default setting of being uniquely, completely, imperially alone
day in and day out.''\footcite{david_foster_wallace_this_2005}
Something might be said here about a cave or a life not worth living.
However, there is a point worth considering in Wallace that isn't salient in
Plato.

After a didactic story about two young fish baffled by an older fish asking them
how's the water (``What the hell is water?''), Wallace proceeds to address the
other trope of a commencement speech: ``Of course the main requirement of
speeches like this is that I’m supposed to talk about your liberal arts
education’s meaning, to try to explain why the degree you are about to receive
has actual human value instead of just a material
payoff.''\footcite{david_foster_wallace_this_2005} 
I suggest that rather than consider the value of philosophy to a life as a
whole, in reflecting on Wallace, we consider the value of philosophy in the
context of a liberal arts general education some of us teach in.
In finding our purpose---both function and intention---in teaching a
philosophy class, it would be fruitful to consider the activity and purpose in
the context the activity is situated in.
That is to say, the question I want to answer is ``What are we doing when we
teach philosophy as part of a liberal arts general education?''

To do this, we have to also make sense about the point of a liberal arts
education is.
A common response educators like to give is that, rather than merely dispensing
knowledge, we are teaching students how to think.
Wallace proposes that what this actually means, in his words, is ``that the 
liberal arts clich\'e about teaching you how to think is actually shorthand for a
much deeper, more serious idea: learning how to think really means learning how
to exercise some control over how and what you think. 
It means being conscious
and aware enough to choose what you pay attention to and to choose how you
construct meaning from experience.  
Because if you cannot exercise this kind of choice in adult life, you will be
totally hosed.''
What I want to suggest is that Wallace's amendment is actually about having a
sense of wonder.
Moreover, I want to suggest that in contextualizing how some of teaching is not
merely in a liberal arts institution but a foundational component of a
liberal arts general education, the purpose of that kind of teaching is also to
inculcate a sense of wonder in a way philosophy is uniquely qualified to do.

Wonder, as an affective and conative attitude, is in contrast to learning
objectives in the cognitive domain like skills and knowledge.
While these can and should be outcomes in a philosophy class, they are secondary
and tertiary purposes.
We cannot appropriately understand primary purpose, as situated in the context
of providing in a liberal arts general education, in terms of learning
objectives in the cognitive domain alone.
This includes the misapprehension that our purpose is train philosophers or
teach how to do philosophy.
It may be asked what we kind of education are we providing if not an
education in academic philosophy given that we are philosophy professors.
I contend that is also a mistake.
Rather, in that specific context, we are public philosophers doing public
philosophy and not philosopher educators in the same sense as when we are
teaching in other contexts like high-level courses offered primarily to
philosophy program students.
In the context of providing classes that fulfill part of a liberal arts general
education, our primary purpose is most appropriately understood as
communicating, and hopefully inculcating, a sense of philosophical wonder.

To this end, I argue that given the liberal arts general education context in
which our introductory philosophy courses are situated in, we should
conceptualize these courses within that rather than in the context of academic
philosophy solely.
Given the likely facts of our students, we cannot appropriately hold that our
primary, but not exclusive, purpose is to train philosophers.
Moreover, it is similarly inappropriate from the broader liberal arts
perspective to conceive of such courses as a subject matter course readily
reducible to learning objectives in the cognitive domain only.
Rather, we should conceptually orient our general education classes towards the
purpose they are uniquely qualified to do: inculcate philosophical wonder.
This is rooted, in part, from the observation that such courses, and the
subjective experience of students, are not obviously aimed at the acquisition of
learning outcomes in the cognitive domain like philosophical skills or knowledge
of philosophy.
By making this purpose explicit to at least ourselves, we can hopefully better
facilitate a necessary and important epistemic and moral function by inculcating
the epistemic and moral emotion of wonder.

Finally, some brief notes on the following account.
I will be concerned primarily on the American liberal arts general education
system because that is what I have experience teaching in.
How the following applies to other national higher education contexts depends
upon the degree to which they relevantly resemble the American one.
That being said, it is my understanding that in places where higher education
consists primarily of courses within the subject of study, like the United
Kingdom, a something that resembles a liberal arts general education takes
place in secondary education rather than post-secondary.
To that end, what follows may be applicable, and even reason for, earlier
philosophy education.
I won't however, comment on this as it is outside my experience.
In light of that, the following should not be taken as making
assertions to the activity of teaching philosophy in of itself.
While it may be pedagogically or rhetorical to consider how we can encourage
wonder, I do not argue that this is the primary purpose in all philosophy
courses.
Graduate seminars, for examples, are appropriately understood in terms of
learning objectives in the cognitive domain because it is obvious that we are,
in that context, training philosophers.
Given the contingency on the context of the activity of teaching, I will rely in
part on autoethnographic reflections on teaching in this analysis.
In light of the contingent nature of the project, hopefully resonance on these
points rather than analytic or empirical truth suffices.
Finally, throughout I will refer to philosophy for the sake of brevity  but am 
open to it being understood in the sense of philosophy and adjacent subjects 
like bioethics.

\section{}

I will admit that it may sound peculiar to suggest our purpose in teaching
philosophy is not, first and foremost, student acquisition of learning outcomes
in the cognitive domain like knowledge or skills.
Some teachers may already be sympathetic to where I'm heading with this but I
still want to note that, for many, this may be counter-intuitive or sound like
schmaltzy sentimentalism.
I'd have probably felt this way when I started out.
Here's my concession to them: There is nothing intrinsic to philosophy itself
that entails it cannot be taught in terms of cognitive learning outcomes.
However, we do not teach philosophy in a vacuum.
Consequently, the nature of what we are doing, the kind of activity we are
engaged in, is molded by the context it is situated in.
Thus, the context should inform our activity.
It is only continent on the context of teaching philosophy as part of a liberal
arts general education.
In doing so, I contend, it becomes clear that while cognitive learning outcomes
are salient, they are not the primary purpose.

The claim here is not that cognitive learning outcomes do not exist in
philosophy or that they are intrinsically undesirable.
I do not advocate for striking learning outcomes and course objectives from your
syllabi.
There are, if nothing else, pragmatic reasons of a compliance with
administrative-bureaucratic demands.
The claim is simply that, when we consider our pedagogical goals within the
context of a liberal arts general education, it is inappropriate to put it
purely in terms of acquiring cognitive outcomes.
When I say teaching in a liberal arts general education, I mean it in a narrow
sense.
I am referring to teaching only those classes that are offered with the
intuition of serving that liberal arts general education.
This obviously includes courses that satisfy a compulsory coursework requirement
for a student's general education: introductory philosophy courses, introductory
ethics courses, and professional ethics courses.
It may also include ``fluff'' electives insofar as they share the relevant
characteristics of the prior.
It does not include upper-level courses intended for students committed to a
philosophy or philosophy-adjacent program of study: undergraduate majors or
minors and graduate students.
The liberal arts general education context referred to here is not that any and
all teaching at a school with a liberal arts program but only that teaching that
is intended as part of the general education of the general student body.

It was not immediately apparent to me that contextualizing teaching in this way
is significant.
I had long struggled with what my purpose was in this activity and how to square
that with assessments and other administrative-bureaucratic miscellany.
The puzzle was helpfully clarified for me in a conversation with a colleague.
We were discussing new prep to meet the school's new core requirements.
He had said, ``I don't understand why we have to teach these reading
requirements.
I hate teaching all this modern and contemporary stuff.
If had my way, I would just teach only Plato for the entire semester.''
My jaw hit the floor.
After prompting, he followed up with the following reasoning: 
``Everything in philosophy is a response to Plato. 
It doesn't make sense to teach later stuff before teaching Plato. 
It's like jazz. 
If you want to play jazz, you don't start with improvisation and stuff.
In order to learn to play jazz, you have to learn the fundamentals first.''
This seemed to be palpably not right.
It took me a while to realize why.

The difficulty this posed for me stemmed from the fact that I was not inclined
to think that he was wrong.
I narrowly avoided becoming a musician growing up when I realized how many hours
I would have to dedicate to practicing scales If I wanted to become more
competent.
I had gone to a music camp in the mountains.
I walked around seeing the good students and the professionals there to teach
stood around doing scales over and over again constantly.
This, teenaged me decided, was not how I was going to spend my time.
Nevertheless, I agree with the general principle because of this background.
Moreover, I'm inclined to think it's an apt analogy.
I did my undergraduate degree in the United Kingdom.
In my program, it was compulsory to take six specific introductory courses on
different aspects of philosophy.
One requirement for the first semester was \textit{Introduction to History of
Philosophy I} which consisted of only ancient Greek readings.
In other words, I was an example of what my colleague had prescribed.
That education was the foundation to whatever debatable competencies I have as a
philosopher, also debatable, today.
Consequently, I could not hold that his reasoning was wrong without having to
cast doubts on my past that would open the door to ever more impostor syndrome.
Yet, I had a conviction that what he had said was problematic in some way.

I contend that while his reasoning was not wrong per se but that it was
inappropriate.
It was inappropriate given the fact that it was uttered within the context of an
American liberal arts general education system.
In such a context, it is a mistake to think that what we are doing is like
teaching people how to play jazz.
If we believe that our primary purpose in a general education classroom is to
train philosophers, then we have simply misapprehended what we are doing.

The brute realities of our activity do not support this belief.
There are two particularly problematic issues here.
First, given the fact that such courses are offered as part of a liberal arts
general education, it is not the case that our students are philosophers.
These courses are typically compulsory---or, at least, an option within a
compulsory choice---for the general undergraduate student body.
Consequently, the students we teach in such classes are diverse ranging from the
humanities to medicine to STEM.
Only a small minority of them are likely to be philosophy students.
Anecdotally, I estimate the most I have had is somewhere in the region of
1--2\%.
At worst, in a professional ethics course, like business ethics or nursing
ethics, there are likely to be no philosophy students given the course is being
offered primarily to students of that particular professional program.
Given the demographics of these classes are dramatically different than other
philosophy courses offered primarily to philosophy students, it must surely be
the case that this should inform how we understand the activity we're engaged in
and, thereafter, the primary purpose we have in mind.

As such, it is patently the case that it is inappropriate to think that what we
are doing is training philosophers and, therefore, primarily interested in
cognitive learning outcomes relating to competence in philosophy. For one, it is
inappropriate because it is premised on a problematic relationship between the
students and the professor.
It is wrong to neglect the vast majority---possibly all---of the students in our
classroom.
It is neglect to conceive ourselves as primarily training philosophers and
putting cognitive learning outcomes relevant primarily or even only to academic
philosophers.
If fails to properly recognize, consider, and respond to the alternative
interests of those in the classroom who are not interested in pursuing academic
philosophy.
To put it another way, it is inappropriate because it is a failure of our
responsibility to care for our students.
It would be inconsiderate and callous not to make the class a worthwhile
experience for the students.
Coercion is perhaps too strong of a word.
The fact of the matter is that the vast majority of them were compelled to
enroll in our class contrary to what they would have chosen in the absence of
mandatory graduation requirements.
It is wrong to begrudge them of their compliance to a system they cannot change.
Moreover, it is a compelled decision for which they---or at least someone---is
paying a significant sum of money for.
This is not to say that our class should be understood as a commodity they have
bought and paid for.
Rather, that these are normative reasons to try to do good by them.
Call it fairness, justice, care, or whatever.
Obviously, I do not take this to mean what we provide should be understood in
financial terms---what is the financial value of wonder?
Rather, we should aim at outcomes that are worthwhile and valuable to everyone
generally generally the possible diversity of backgrounds and interests.

It might be argued that the acquisition of skills relevant to being a
philosopher has generalizable value.
After all, to be successful in philosophy, one must demonstrate competence in
logic, analysis, argument, rhetoric, reading complicated texts, writing,
communicating complicated content, problem solving, et cetera, et cetera.
I held this view for a while.
I still think these are valuable secondary purposes for us to pursue.
However, I am dubious to the extent we can be understood as sincere in this
regard.
To put it another way, I suspect that in most philosophy classes this is,
unintentionally, mere lip service.
What I suggest is that if we were sincerely committed to the notion that we are
primarily aiming to track the skills relevant to doing philosophy as
generalizably useful and not mere to train what few philosophers are in the
room, then we would be committed to explicit instruction of these skills as
generalizable.
I can note some philosophers that have done this.
But in them being notable, it suggests that they are exceptions to the rule
rather than representatives of it.

Certainly, on reflection of my own experiences, I realize now that this was not
the case.
If reading, comprehending, and analyzing philosophy texts is a transferable
skill, it is not obvious to me that I was given instruction in it besides
``sometimes it will take an hour to read and understand a single paragraph.''
As David W. Concepci\'on observes:
\begin{quotation}
The relative absence of appropriate ``How to Read'' material is peculiar. 
As years of listening to plaintive students teaches, intelligent and literate 
general studies and early major students lack the skills needed to read 
philosophy well.
Students are not familiar with the folkways of academic philosophy and are too
often left to learn them through trail [sic] and error. 
But we do our students a disservice if we let them flounder through with 
nothing but trial and error.
Philosophy professors should not ask students in introductory or early major
classes to spend three hours or more per week doing something they have never
done before (i.e., read like a philosopher) without telling them how to do it.
This is particularly true since we know they are likely to think reading
philosophy is just like reading anything
else.\footcite[p.351-2]{concepcion_reading_2004} 
\end{quotation}
That is to say, that explicit instruction on those skills ought to do  if we
expect students to use them.
Yet this is something that is notably neglected.
In light of this, it seems hard to sincerely say that we teach generally 
valuable skills in the course of doing philosophy if there is a dearth of
explicit instruction on that skill at all.

Or, consider writing.
The majority of my education in philosophy consisted in final papers as the sole
assessment.
Consequently, the only writing feedback I received was well after the course had
ended.
What competence in writing I acquired, if any, was a function of trial and error
over an entire degree program rather than in any course itself.
Further, the majority of my feedback was chiefly concerned with the
philosophical substance of the argument.
This is troubling in light of Jonathan Bennett and Samuel Gorovitz's observation
that ``If we only look to the substance, we encourage our students to undervalue
the importance of the quality of writing.''\footcite[p.
10]{bennett_improving_1997}
That is to say, if what feedback we do give is primarily in regards to the
quality of their philosophy then we are not likely to be helping students
improve the quality of their skills as writers.
They charge that ``too much academic writing, even in prestigious
journals and books, is `academic' in the worst sense.''\footcite[p.
10]{bennett_improving_1997}
In this light, it looks less than hopeful that we can sincerely say that we are
trying to teach transferable writing skills as part of our primary purpose in
these classes.
There is simply too little explicit instruction to this effect in most cases to
support that claim.

For the sake of brevity, I will only make a few comments in regards to the
suggestion that the skill we teach is how to think or how to be rational.
I hope that we can at least make a better case that we do this.
I also hope that we do a better job in this regard.
What I will say is that I find it difficult to gauge my confidence if it is the
case that there is doubt regarding the competencies of how well students provide
input into their critical thinking process, e.g. reading, and how well students
provide the outputs from that process, e.g. writing.
If they struggle at either end, I'm not sure how to assess the black box in the
middle.
I'm less than hopeful, however, that we give much of any explicit instruction on
how to whatever philosophical thinking skills they exercise can be transferred
outside the field with it's idiosyncrasies.
When I have mentioned to other teachers that I incorporate some superficial
details from computer science when I teach logical connectives, they generally
respond in either surprise or confusion.
This, to me, is the bare minimum I could do to show that the lesson has broader
connection and utility than in just analyzing philosophical arguments.
If even this is novel, I am gloomy that there is much explicit instruction on
how whatever way we teach students how to think is useful outside of our
classroom.

This focus on explicit instruction as the relevant evidence on the degree to
which we can sincerely say that teaching students these skills is part of our
primary purpose might be contentious.
For one, it might be said that this is a philosophy class, after all, we both do
not have the time nor the responsibility to provide explicit instruction on all
these things and teach about philosophy in any depth.
I agree.
That is why I suggest that the acquisition of these cognitive learning outcomes
is not our first and foremost purpose even though it is hopefully secondary or
tertiary.
It might otherwise be objected that these outcomes can be acquired without
explicit instruction: they are learned implicitly through philosophical
training.
I suspect this is how most of us think this is supposed to work.
We teach students to use these skills in philosophy and then they will just work
out how to apply them outside of philosophy on their own.
If they do well in philosophy, then we can assume they have become better
thinkers in general.
This seems problematic to me because I have a hard time distinguishing between
the degree to which this actually happens and surviorship bias of the trial and
error training that tends to exist in academic training.
If our classroom is sink or swim given the absence of explicit instruction, then
it seems odd to just throw them in the deep and point to the ones that haven't
sunk and say this is evidence we are teaching people to swim.

\section{}

% reframe as making distinct philosophy classes as something special

Alternatively, it might be argued that there is something intrinsically valuable
about acquring knowledge about philosophy that is our primary purpose.
This, I think, is unlikely to be the kind of tack philosophers will take.
Rather, I suspect it is the kind of reasoning used by non-philosophers
justifying the requirment of philosophy in a liberal arts general education.
That is to say, that philosophy is simply a subject matter course students are
expected to have some familiarity with like english literature or history.
No offense intended.
This approach escapes the problem of the prior by situating the activity in its
context.
However, I do not take it to be a successful one because it attempts still to
conceptualize our purpose i just cognitive learning outcomes.
This is inappropriate because it is not at all obvious that a well-designed and
succesful philosophy course has to do beyond a minmimum.
To put it another way, philosophy is not appropriately measured by and then
asked to design around cognitive learning outcomes because we are doing
something outside the cognitive domain.

I take as salient example to this kind of position the
bureaucratic adminsitrators who insist on faculty using the language of Bloom's
taxonomy.
Bloom's taxonomy is meant to be a standardized measure by which different
concerns from different institutions can be compared.
One problem with the measure, though, is that it's use focuses nearly
exclusively on knowledge acquisition.
Note, that in it's original proposal it was never suggest that it was an
exhaustive model.
Bloom originally proposed three domains captured in the model: the
cognitive, affective, and psychomotor.
Yet in actual practice, ``Bloom's taxonomy'' is near universally used to refer
to the taxonomy of the cognitive domain.
In the 2001 revised edition, only the cognitive domain is discussed in depth.
It offers the following hierarchal model of educational learning objectives:
Remember, Understand, Apply, Analyze, Evaluate, and Create.
Learning objectives are phrased in terms of a verb associated with one of these
levels and an object of knowledge or information.
This ranges evaluations of whether the student can recall the information,
explain it, use it in a new way, and make distinctions to things like whether
they can defend some view about the information or create some knew knowledge
from the evidence they have.
Notably, these are about either acquisition or deployment of either inforation
or knowledge.\footcite{armstrong_blooms_nodate}

The use of these measures is pervasive in education.
The difficulty is that they are treated in practice as exhuasive and imposed as
stipulations around which a course should be designed around.  As one university
resource notes, ``Section III of \textit{A Taxonomy for Learning, Teaching, and
Assessing: A Revision of Bloom’s Taxonomy of Educational Objectives}, entitled
“The Taxonomy in Use,” provides over 150 pages of examples of applications of
the taxonomy. Although these examples are from the K-12 setting, they are easily
adaptable to the university setting.''\footcite{armstrong_blooms_nodate}
It is presumed in use that the taxonomy can readily applicable to any course and
captures all possible, or all the important, pedagogical objectives.
Further, it is imposed as the language by which we establish our objectives and
deliver appropriate instruction and assessment.
I suspect quite a few philosophers, myself included, experience either
frustration or unease when volunteered to do tihs.
This, I contend, is a reasonable response because it responds to the
inappropriateness of trying to reduce our purpose into the cognitive domain
alone.
These do not exhuastively or aptly describe the kind of pedagogical goods a
succesful philosophy course may be designed around.

Ironically, to see this we need only to turn to Plato.
We have in the Apology the case of Socratic ignorance---the awareness of our own
ignorance---as the explanation provided for the attribution of being the wisest
of all to Socrates.
More relevantly, this outcome is used as a pedagogical device in Plato's early
works as \textit{aporia}.
This is the state of puzzlement that occurs when faced with a philosophical
dilemma.
Significantly, aporia is used as a desirable outcome, not the starting point of
learning, in many of the early Socratic dialogues.
Socrates engages with an interlocutor who claims to have knowledge and offers
this to Socrates.
Socrates then considers this and then points out the flaws in the interlocutor's
account.
The interlocutor will then offer an alternative in response and this process is
repeated until, in the aporetic dialogues, Socrate's interlocutor admits that
they do not know and walks away.
Notably, in these texts, Socrates does not provide an answer he claims to be the
truth.
Significantly, what is demonstrated in these works is the acclaimed Socratic
method.
This well established pedagogical method is used from the starting point of
knowledge to arrive at a state of ignorance as an outcome.
I once explained this to one of those education people leading a compulsory
course for faculty when they asked me to describe my learning objectives purely
in terms of Bloom's taxonomy verbs. 
I stated I was unsure how to do this so I asked him how I would translate aporia
as a learning objective.
He wasn't happy.

Of course, what is really going on is the destruction of the appearance of
knowledge, rather than knowledge per se.
Still, it is not obvious that this is neatly categorized in Bloom's taxonomy.
This is especially not the way they want the taxonomy to be used: to be ale to
point at some stuff and say that the students walked out of the classroom with
more stuff in their heads than before.
However, what we do when we teach introductory philosophy courses is not
education in the conventional, or at least this, sense.
We can frame aporia in Bloom's verb-object formula if we're a little bit
creative.
Dispell bad beliefs.
Analyze and critique things they mistake to be knowledge.
But these are either impermissible verbs (use the ones lister here!) or
undesirable objects.
I suggested these in that mandatory workshop.
He, even more unhappily, asked me how I was going to objectively assess these
objectives and then told me to just give them some basic comprehension test.

It is, I think, obvious to many of us and our students that a philosophy class
can be a unique experience distinct from the other courses they may take.
The more successful classes I've seen have done this.
Classes based primarily on didactic lectures have tended to have difficulties.
Students tend to get lost and quickly check out.
They express their frustration and bafflement.
We can be quick to put the blame on them, but we should consider that what we're
asking them to do may be quite novel to them as students.
It is unclear to them what knowledge they are supposed to be memorizing and
regurgitating and, as a consequence, become disoriented.
Of course, this is not what we want them to do but this can be more or less
apparent to tem.
The more successful have largely integrated dialogues.
While this is not always possible, I think it's worth noting that one thing that
this methodology can do implicitly is make the experience of the class
distinctive and, by association, help the students recognized that the purpose
is not the same as in other classes they may have taken.

This is important for the students to recognize because the way we tend to
incorporate aporia in our own pedagogy.
It seems to be the case that students in the primarily didactic classes
struggled with philosophy as an activity rather than as a subject.
They are perfectly capable of the more conventional educational tasks.
They can recall, digest, and explain what whoever said about whatever.
However, when aske to analyze and evaluate, they clam up.
I will admit in my first class I taught, I tended to be didactic.
However, there was one point when I had asked a question and a student had
spoken up frustratedly, ``But you're just asking for opinions, not facts. When
are we going to get to the asnwers?''
I realized then that I had screwed up.
I realized that the students had been misapprehending philosophy as a mere
subject matter course and not as an activity I was asking them to engage in.
They had grown up developing an aptitude as students their entire life in
courses where they learned about a thing rather than really doing it.
While the latter depends in part on the prior, I suggest that this is not really
the kind of activity we're engaged in.

In part, it is inappropriate to conceive of a philosopy course as a subject
matter course because an element of aporia is inevitable.












































\nocite{plato_theaetetus_nodate}

\printbibliography

\end{document}
