\documentclass[12pt, letterpaper]{article}

\title{When we wonder about technology, should we hope or fear?}
\author{Alexander Zhang}

\begin{document}

\maketitle

Imagine a piece of beige cardstock about seven inches long and three inches
wide.
It is punched with an array of seemingly random rectangular holes.
These holes are on a grid of  12-rows ans 80-columns.
Each column contains either a pair of holes or none at all.
This is data.
Each pair of holes represents a character of an alphabet to be read or written
by a computer.
The punchcard is a physical and analog manifestation of a computer's input, output, or
storage.
There is, in principle, no reason why a modern computer could not still be used
via a punchcard interface besides the impracticability.
That little piece of paper is digital.
That fact unerringly sparks a sense of wonder in me.
I have a reasonably competent understanding of the mechanics of how the
punchcard works.
Yet, the notion I can hold in my hands and see digital data in much the same way
as a computer still smacks of the bizarre.

There is a quote by Arthur C. Clarke that is well-known to science fiction
geeks: ``any sufficiently
advanced technology is indistinguishable from magic.''
While this claim is typically applied to the speculative, it seems to quite
aptly describe our phenomenal experience of wonder at technology.
We have taught tiny bits of sand and spinning plates of rust to think.
It might be objected that this is merely poetic license.
But then, I try to fully comprehend that I am writing this essay on transistors
measurable in double-digits of atoms.
Or, I imagine the five terabyte hard drive that it will be saved on made with
the vacuum tube technology of the punchcard era.
It would be the size of North America but now it fits neatly in my pocket.
It's not Harry Potter magic with silly incantations and smelly cauldrons.
Yet, fully grasping these notions feels slippery.
The significance of the punchcard seems just outside what should be possible.
It is, at least in part, fundamentally mysterious.

This resemblance of magic is arguably essential to wonder.
It is easy to see how this relates to childlike wonder.
It is the strange and the bizarre that grips a child's curiosity and
imagination.
It causes them to ask how such a thing can be possible.
However, this element of mystery is at the root of all wonder.
This is a significant point for philosophers.
When Theatus declares, ``I perpetually wonder---by the gods I do!---how to make
sense of it; sometimes just looking it makes me literally quite dizzy,''
Socrates asserts he has the nature of a philosophy.
Wonder is the mark of philosopher.

But this wonder is not the same as rosy childlike wonder.
As Theatus observes, it can be deeply disorienting---even unpleasant.
This phenomenology is no accident.
As Descartes posits, it is surprise---when something is very different from what
we know or from what we suppose ought to be---that causes us to wonder.
This comes before any understanding of the positive or the negative associated
with the object in question.
Consequently, Descartes declares wonder as the first of all passions.
But surprise is not benign.
It confronts us.
As such, this resemblance of magic presents us with something alien, outside of
our prior understanding.
It forces us to recognize the strangeness of the world.

Thus, we should recognize that wonder is not merely innocent and naive.
Wonder can be quite emotionally charged.
The confrontation of the strange, the other, the alien, can trigger other
emotions in us like hope or fear.
There is no better example of this than our reactions to technology.

Consider the reactions of technological optimists.
Some adopt the ``fail fast'' mentality of Silicon Valley.
Technology will solve the problems of technology.
We put a great faith in technology that it will always be bountiful.
Something new that will solve just that problem is coming down the line.

This is, perhaps, epitomized in Moore's law.
This ``law''---really a marketing strategy---was an observation in 1965 
by Gordon Moore,
co-founder of Intel, that the number of components in an integrated circuit, and
therefore performance, would double roughly every year.
This was revised in 1976 to roughly every two years, which is the most
frequently cited formulation of the law.
This was again revised, implicitly, by Intel when they publicly moved from a two
year tick-tock release cycle to a three year process-architecture-optimization
model after troubles shrinking process nodes from 14nm to 10nm.

Despite the observable slowing and looming physical limitations as transistors
reach atomic scale, hopeful technologists like Ray Kurzweil still hold faith
that technological growth is accelerating.
He argues that even if Moore's law ends, some replacement technology will
inevitably be discovered or perfected to carry on the exponential growth.
This bold claim emerges out of features Kurzweil takes to be intrinsic to
technology itself.
It creates positive feedback loops that will inevitably lead to the
singularity---a point when technological growth reaches a point it becomes
a runaway reaction of self-improving cycles that outstrips the human capacity to
follow it.

Surprisingly, technological pessimists hold similar views.
Rather than the hope felt by futurists like Kurzweil, the belief that technology
has such a teleology is something to fear.
Bernard Stiegler believes such a future holds an age of disruption.
As he puts it, accelerating technological development will break the ``time
barrier'' and---like a supersonic jet creates a sonic boom---will break the
``wall of time''.
Put in another way, with fewer rhetorical flourishes, Stiegler holds that
technological innovation is disruptive.
It disorients the cultural formations around the prior technical system.
As such, culture must reform and reorient around the new innovation.
This, however, takes a certain amount of time.
As such, if technological development continues to accelerate, it will at
some point be innovating faster than culture can subsume it.

For Stiegler, this would mean the end of culture itself.
As he holds it, the process of cultural evolution is necessarily a long circuit.
However, permanent technological innovation short circuits those long circuits
ans prevents them from occurring.
Without the stability of cultural structures integrating technological
developments, individuals will not be able to be able to connect with each other
or even make sense of themselves as individuals.
Consequently, accelerating technological innovation will result in a complete
and utter alienation for everyone.

The natural question, then, is whether technology should inspire hope or fear---do the
optimists or the pessimists get it right?
The answer seems to be neither.
It is noteworthy that both see technology as an outside force exerting change
onto humanity.
Both positions, the optimist and the pessimist, see technology as an abstract
thing in of itself.
Technology with a capital T.
We should question that basic premise to begin with.

This approach is intuitive insofar as technology, or some part of our experience
of technology, bears this resemblance to magic.
The unceasing sense of mystery that such things can be possible inclines us to
think of technology as something outside of us as surprise confronts us.
That surprise creates the impression of an alien force and causes the reactions
of hope and fear.

However, we should remember that, of course, there is no real magic here.
The punchcard is just a piece of paper with some holes in it.
Removed from human intention, it is nothing more and nothing less.
Without human interpretation, it is meaningless.
Without human manipulation, it is inert.
It may represent information for us to digest in sea of information presented to
us to consume, but we should consider whether we need to digest it at all.
It holds as much power over me, if I choose it to be so, as the number of grains
of sand on a beach waiting to be counted.

We ought to remember that technology is not only a product of human artifice,
but meaningful through human activity.
Whether wonder should lead us to hope and fear rests on how we should view each
other rather than an abstract idealization of Technology with a capital T.
The wonder I feel towards the punchcard ceases when I put it between two pages
as a quirky, nerdy bookmark.
The wonder I feel towards my computer or to my smartphone ceases when I stop
pondering them abstractly and use them instrumentally.
They are, ultimately, tools.

The appropriate object of our emotions towards technology, then, isn't the
phenomenology of its magic but the reality of its use as tools.
Consequently, wonder ought to be directed towards this human activity.
Rather than speculate over the metaphysical nature of Technology, philosophy of
technology should concern itself with technologies first.
That is to say, we should make the methodological choice to appreciate that such
an inquiry spans the course of human history from the plow to cryptocurrency and
beyond.
It is bound up with the richness and diversity of all those human experiences.
It is fundamentally complex.
We can only make sense of project by examining technologies and humans as they
actually are embodied in the world rather than trying to pierce the noumenal to
the essence of Technology itself.

We can understand this methodological approach as an application of Charles W.
Mills's nonideal theory to the philosophy of technology.
His distinction between ideal and nonideal theory is a question of how we try to
construct theories about the world: in what direction do we proceed from,
descending from heaven to earth or ascending from earth to heaven?
While Mills is particularly concerned with the implications of ideological bias
in ethics as a result of ideal theory, the fundamental point that ideal theory
has blindspots remains.
Indeed, insofar as our theories of technology informs our ethics of technology
which serve to justify and permit real action, it matters that we get it right
for the right reasons.

It may seem to be the case that any ethical theory entails some kind of ideal.
Mills target, however, is a particular sense.
Out of the gate, the intended referent is not the trivial sense of
ideal-as-normative in which any normative ethics that asserts an ought is
committed to.
Rather, Mills is interested in a particular way in how we create ideals of, or,
rather, idealize, the world as models.
Again, it seems that any attempt to render the world more intelligible would
entail idealization in this sense.
However, he distinguishes between two different ways in which we---in ethics,
philosophy in general, or natural and social sciences---create idealized models:
ideal-as-descriptive-model and ideal-as-idealized-model.
The ideal-as-descriptive-model attempts to merely describe where as the
ideal-as-idealized-model attempts to define and construct an exemplar.

Of course, the answer to how useful an ideal-as-idealized-model is as a starting
point depends on the fact of how closely it resembles the actual.
Since we cannot know a priori the degree to which this is the case, it is
methodologically unsound to proceed on an ideal-as-idealized-model based on a
suppressed premise that it is sufficiently approximate to the
ideal-as-descriptive-model.
In the case that it is not, then our conclusions will be invalid and fail to
capture the actual phenomenon we are attempting to model.

In looking at the history of philosophy, ideal theory has failed to obviate the
need for a nonideal theory for all these millennia.
We can grant that philosophers are well-intentioned and that the recognize that
more will have to be said in the application of their idealized principles.
But the question remains: when will the work of accounting for the nonideal be
done?
Historically, it seems rarely if ever.
If concessions and promises are never followed up by action, they hardly rise
above the level of lip service.
Moreover, this implies the concession that ideal theories are evidentially
impoverished.
This presses the need for ideal theorists to satisfactorily account for the
history all the more.

The kinds of problems that we want an ethics of technology to address are not
purely academic.
They are instantiated in the real world affecting and affected by real people.
Insofar as we are interested in applying philosophy towards understanding
problems and finding solutions, then we are ultimately interested in applied
philosophy.
As such, an inquiry that attempts to make intelligible actual
dialogue across disciplines and the thresholds of ivory towers is necessary.

To be clear, this should not be construed as stipulating philosophy's role as an
apologist for technology.
It might be objected that this way lies pragmatism, relativism, nihilism,
or---even more disrespectfully---mere sophistry.
It might be claimed that in being useful, we never actually question the project
of technology but only tweak it, thereby allowing the producers of technology to
continue their work without questioning their fundamental assumptions.
Such claims are fundamentally misguided.
The mistake is in thinking that we can philosophize about
Technology with a capital T without examining the real world
instantiations and experiences of actual technologies.

This is all to say that while philosophy of technology begins in wonder, we
should careful in getting carried away in our fancy.
There is a temptation in wonder to become fascinated with the mysteriousness.
However, chasing after this leads us astray from the appropriate object of our
inquiry.
Wonder at technology should lead us to ponder humans and their tool use.

\end{document}
