\documentclass{article}

\begin{document}

Imagine a piece of beige cardstock about seven inches long and three inches
wide.
It is punctuated with an array of seemingly random rectangular holes.
These holes, arranged in 12-rows ans 80-columns with either two holes or none in
each column, are recorded data.
Each pair of holes represents a character of an alphabet to be read or written
by a computer.
It is a physical and analog manifestation of a computer's input, output, or
storage.
There is, in principle, no reason why a modern computer could not still be used
via a punchcard interface besides the impracticability.
That little piece of paper is digital.
Considering such an object, I find my sense of wonder in technology rekindled.
How can it not at something so strange?

It is common to quote Arthur C. Clarke at this point: ``any sufficiently
advanced technology is indistinguishable from magic.''
While this phrase is slightly worn, there is, I think, something to it.
We might say we've already achieved this by teaching tiny bits of sand and
spinning plates of rust to think.
It might be objected that this is merely poetic characterization.
But then again, I am writing this essay on transistors measurable in
double-digits of atoms, on miles of tiny copper and gold wires, and a five
terabyte hard drive that, if made with the vacuum tubes of the punchcard era,
would be the size of North America but now fits in my pocket.
It might not be Harry Potter magic with silly incantations and smelly cauldrons.
Yet the brute facts seem to resist full comprehension as if they are just
outside what should be possible.

It is easy to see how the resemblence of magic can inspire childlike wonder.
It is this strageness, this mystreriousness, that grips a child's curiosity and
imagination.
It causes them to ask how can such a thing be possible.
But  this incomprehension is at the root of all wonder.
As Descartes posits, it is surprise---when something is very different from what
we know or from what we suppose ought to be---that causes us to wonder.
This element is what puts wonder as the first of all passions for Decartes.
To put a finer point on it, it is Theatus's declaration, ``I perpetually
wonder---by the gods I do!---how to make sense of it; sometimes just looking at
it makes me literally quite dizzy,'' that Socrates ascribes him as having the
nature of a philosopher.

But this element of magic is also something alien.
It confronts us.
It forces us to recognize the strangeness of the world.
It represents an existence outside of ourselves, outside the immediate gasp of
our comprehension.
We should recognize, then, that wonder is not merely innocent or naive.
Despite our nostalgia for our childhoods.
The kind of wonder we feel as we mature is emotionally charged.
It can be, as Theatus notes, deeply disorienting.

The confrontation of the strange, the other, the alien, can trigger other
emotions in us like hope or fear.
There is no better example of this then our reactions as we try to come to grips
with technology, particularly when we approach it as something in of itself or
something other, an alien entity.
Regarding hoper, we can loot at the deep faith we put in technology like that of
the singularists or fail-fast Silicon Valley.
There is a conviction that Moore's Law will be ressurected.
This ``Law'', really a marketing strategy, was an observed trend in 1965 by
Gordon Moore, co-founder of Intel, that the number of components in an
integrated circuit, and therefore performance, would double every roughly every
year.
This was revised in 1975 to every two years.
In 2016, this was implicitly revised again when Intel publically moved from a two 
year tick-tock release cycle to a three year process-architecture-optimization
model.
Despite the observable slowing and looming physical limitations as transistors
reach the atomic scale, hopeful technologits like Ray Kurzweil still hold faith
that technological growth is accelerating premised on ideas like Moore's Law.
Such hope is founded on a speculation that some future innovation will overcome
and revitalize technological growth.
It is, allegedly, the nature of technology itself.

Conversely, technological pessimists hold similar similar attitudes.
Bernard Stiegler fears that an accelerating technological development means it
wo go so fast it will break the ``time barrier'' much like a supersonic jet
breaks the sound barrier.
It holds that there is a certain pace of cultural evolution.
As new technological innovations come about, they disrupt the prevailing social
construction.
As such, society must go through a period of disorientation and the
ireorientation as new cultures are built around the technological innovation.
Such a process, however, takes time.
Stiegler fears that if technological innovation continues to accelerate, then it
will outstrip the pace at which cultures can reorient to new innovation.
It would be, as he coins it, an age of disruption.

Both positions, the optimist and the pessimist, see technology as an abstract
thing in of itself.
Technology with a capital T.
This approach is intuitive insofar as technology, or some part of our
experience of technology, bears this resemblence to magic.
The strangeness of the punchcard puts it outside the human realm despite being
an artifice of humanity.
Technology appears alien ans so becomes a kind of alien force of its own.
As so we not only wonder at it but feel hope and fear as well.

The question, then, is where should wonder lead us.
It seems to me that attending too much to the phenomenal experience of wonder
leads us astray.
There is no real magic, of course.
The punchcard is just a piece of paper with some holes in it.
Without a human being to read it, to manipulate it, to interpret it, it is
meaningless.
It remains information.
But, so then, does the number of grains of sand on a beach or the e-wase buried
in a landfill.
More dere data is a drop of water in the ocean of the universe.

Ultimately, what is significant about technlology is technlogy as human
activity.
As so our wonder towards it should rest on the wonder we hold towards ourselves.
The hope and faer we have are the hopes and fears we have of each other.
There is no magic.
Only us, for better or for worse.

\end{document}
