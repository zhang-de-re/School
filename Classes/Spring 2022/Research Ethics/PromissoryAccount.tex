\documentclass[letterpaper,notitlepage,12pt]{article}
\usepackage[
  letterpaper,tmargin=1in,bmargin=1in,lmargin=1in,rmargin=1in]
{geometry}
\usepackage[backend=biber]{biblatex-chicago}
\addbibresource{workscited.bib}
\usepackage{setspace}
\usepackage[super]{nth}

\title{Promissory Account of Research Participation}
\author{Alexander Zhang}
\date{}

\begin{document}

\maketitle

\section{Introduction}

Since the dawn of time, the great philosophers of mankind have wondered why
Webster's dictionary says the things it says.
Aliens ask much the same but with Bilbdoolpoolp's \textit{Book of Words for
Words}.

It is sometimes helpful to draw analogy between apparently disparate discourse
to see what emerges in the juxtaposition.
In this paper, I suggest that it may be fruitful for bioethics to juxtapose
research ethics to the ethics of war.
Immediately an interesting parallel be observed.
In the ethics of war there are two or three conceptual categories: \textit{
jus ad bellum} which considers the conditions necessary to instantiate a just
war; \textit{jus in bello} which is concerned with the ethical conduct of
combatants during a war; and, recently, \textit{jus post bellum} which is
concerned with justice in terminating a war, agreeing to peace, and subsequent
actions.
Similarly, we can categorize issues in research ethics along these lines.

This appears superficial.
However, it raises an interesting observation.
While debates regarding \textit{jus ad bellum} are preoccupied with when armed
conflict can ever be just if at all, heretofore, debates in an analogous
\textit{jus ad research} have been concerned with when medical research is
merely permissible.
While adjacent, the concepts of justice and moral permissibility are distinct.
Indeed, some take it to be the case that we can talk about one without the
other.
In \textit{Rethinking the Ethics of Clinical Research}, Alan Wertheimer
stipulates the following: \begin{quotation}
I believe that the allocation of resources for research may be the
big---macro---issue of research ethics. This book focuses on a smaller (not
small)---micro---discussion of research ethics, namely, the recruitment and use
of research participants. For whether the research in question serves the needs
of developed nations, or the citizens of undeveloped nations, and whether the
participants come from developed or undeveloped nations, we must still ask: When
is it ethically justifiable to recruit and use people as research
subjects?\footcite[p. 8]{wertheimer_rethinking_2011}
\end{quotation}
In light of this juxtaposition, it is not clear to me why research ethics should
begin by attempting to hold separate justice and moral permissibility.

I offer an alternative perspective that attempts to integrate both the question
of justice and moral permissibility of clinical research.
By drawing analogy to the ethics of military privatization, I propose a
promissory account of clinical research participation.
In doing so, the promissory account provides a conceptual framework grounded in
principles of justice, personal autonomy, and paternalism.
As such, in the second section, I examine Hans Jonas's account of permissibility
and the dilemma it sets.
In the third section, I motivate the methodological move to draw analogy to
soldiers and mercenaries.
In the fourth section, I pivot to Chiara Cordelli's promissory account of
soldiers and mercenaries.
Finally, in the following section, I sketch out a promissory account of clinical
participation and note how such an account dissolve's Jonas's dilemma.

\section{The Permissibility of Clinical Research}

If there is a cohesive element within the literature on the permissibility of
clinical research, it is the quandary of how to balance the apparently
contradictory ethical demands to the individual research participant and to the
general population that might benefit from possible discover.
Otherwise, the inquiry is---at the pace philosophy tends to take---quite young
emerging around the turn of the \nth{20} century and then seriously after World
War II.
Commentators on these issues have noted this approach.
As Wertheimer stresses: \begin{quotation}
This book does not aim to provide the philosophical \textit{foundation} of
research ethics. Research ethics is a practical enterprise that has developed in
response to specific historical events. It is not built on any general or
overarching theory. The reigning principles---there are plural principles and
they are reigning principles---respond to the desire to square the genuine need
for biomedical research with the protection of human subjects in the context of
a history that contains several episodes of serious abuse and exploitation of
subjects. The question is which principles can be given sound philosophical
support. \footcite[p. 3]{wertheimer_rethinking_2011}
\end{quotation}
However, this approach has also led commentators to note, ``To judge from a
survey of recent literature, the foundations of research ethics are in
disarray.''\footcite[p. 99]{london_two_2007}
As such, I will throw another hat in the ring by attempting to conceptualize the
problem from an already established philosophical foundation first.
I hope it may be fruitful.

This is not to say, however, that this project starts \textit{ex nihilo}.
It will be helpful to start with Jonas's position.
While not presently popular, his comments are influential.
At the very least, I take the dilemma he sets out to get at the root of the
quandary.
Any viable theory, then, should be able to at least resolve it.
However, the dilemma also sets out a red herring---that the issue can be
isolated as when we are morally permitted to ask individuals to participate in
clinical research.

Jonas's dilemma for clinical research, or also calls it ``human
experimentation'', is the apparent contradiction between what is owed to the
participant and what is owed to society writ large.
It presents two horns.
First, the moral appeal to the common good does not license generally research
participant recruitment because the appeal to the common good is morally
gratuitous and unnecessary.
Second, the social appeal to the common good can only license generally
recruitment if informed consent is superfluous and participants can be simply
conscripted.
Neither horn is appealing.

This dilemma is rooted in Jonas's conceptualization of the issue as the
individual versus society: ``the possible tension between the individual good
and the common good, between private and public
welfare.''\footcite[p. 221--2]{jonas_philosophical_1969}
While biomedical research promises to advance scientific knowledge that can be
used to help the sick and ailing, it has historically been paid for with unjust
and unethical treatment of research subjects.
Jonas, then, is concerned to what extent potential benefit to society, or the
common good, justifies making individuals research participants at potential
risk to their rights, dignity, and sacrosanctity.

The appeal to the common good to license enrolling research participants is seen
by Jonas as grounded within the conceptual framework of the social contract
despite his misgivings.
Indeed, he seems to take this to be the conceptual framework generally used
whenever the common good has some pragmatic precedence over the individual
good.\footcite[p. 221]{jonas_philosophical_1969}
Consequently, by referring to this political abstraction, Jonas comes to phrase
the issue of clinical research in terms of rights and obligations: ``In terms
of rights, we let some of the basic rights of the individual be overruled by
the acknowledged rights of society---as a matter of right and moral justness
and not of mere force or dire necessity (much as such necessity may be adduced
in defense of that right).''\footcite[p. 221]{jonas_philosophical_1969}
As a result, Jonas is reluctant to defend research participation as he is
critical of the notion that we have a right to an individual's participation.

Moreover, the conceptual framework of the social contract is contrasted against that of informed consent.
Jonas asserts, ``Note that in putting the matter in this way---that is, in
asking about the right of society to individual sacrifice---the consent of the
sacrificed subject is no necessary part of the \textit{basic}
question.''\footcite[p. 221]{jonas_philosophical_1969}
While consent, particularly informed consent, is conventionally at the heart of
medical ethics, it is in tension with the social contract framing.
The result of this tension drives the wedge between the two horns of Jonas's
dilemma.
As he puts it, ``If society has a right, its exercise is not contingent on
volunteering. On the other hand, if volunteering is fully genuine, no public
right to the volunteered not need be
construed.''\footcite[p. 221]{jonas_philosophical_1969}
It distinguishes what Jonas sees as the moral claim and the social claim.

This characterization, however, minimizes the severity of the moral claim arm
of the dilemma.
It is not merely the case for Jonas that, if we make a moral appeal for
research participation, we need not phrase it in terms of the common good.
Rather, we ought not ask at all.
The force of Jonas's argument rests on his characterization of the common good.

Jonas argues that health as a public good is morally unnecessary and gratuitous.
He does not dispute the loss that may arise without biomedical research.
He disputes that this loss is a moral evil.
Jonas asserts, ``What is it that \textit{society} can or cannot
afford---leaving aside for the moment the question of what it has a
\textit{right} to? It surely can afford to lose members through death;
more than that, it is built on the balance of death and birth decreed by
the order of life.''\footcite[p. 228]{jonas_philosophical_1969}
Even if some countless number would be healed with discoveries from
biomedical research, Jonas does not believe this is a hinderence to
society flourishing
``in every way''.\footcite[p. 228]{jonas_philosophical_1969}
The only exception in this regard are cases of ``clear and present danger''
which threaten the natural order of life---namely, when the numerical
ratios of birth and death are out of
balance.\footcite[p. 228]{jonas_philosophical_1969}

Consequently, the promise of biomedical research to improve society---the notion of progress---is unnecessary and gratuitous.
Society can afford it.
``A permanent death rate from heart failure or cancer does not threaten
society. SO long as certain statistical ratios are maintained, the
incidence of disease and disease-induced mortality is not (in the strict
sense) a `social' misfortune.''\footcite[p. 229]{jonas_philosophical_1969}
Progress beyond what society cannot afford is, therefore, not a matter of necessity.
Moreover, Jonas argues, ``Unless the present state is intolerable, the melioristic goal is in a sense gratuitous, and not only from the vantage point of the present. Our descendants have a right to be left an unplundered planet; they do not have a right to new miracle cures.''\footcite[p. 230]{jonas_philosophical_1969}
Medicine, to Jonas, is indirectly melioristic as a social good.
Consequently, biomedical research is merely a societal interest.
Jonas does not dispute the possible goods.
He admits that the commitment to progress can be noble and the results can
be regarded as something akin to grace.
It is not, however, a moral interest outside disaster conditions, because
it is unnecessary and gratuitous even if it is noble.

Weighed against this is the individual good.
Here, Jonas asserts there is clear necessity.
As he contrasts, ``Society, in a subtler sense, cannot `afford' a simple
mercenary of justice, a single inequity in the dispensation of its laws,
the violations of the rights of even the tiniest minority, because these
undermine the moral basis on which society's existence
rests''\footcite[p. 228]{jonas_philosophical_1969}
Given that participation in research can, and has, come at great cost to
the individual subject, the permissibility of clinical research depends on
the degree to which it makes possible unjust abuse of those involved.
This, contrary to progress, is a moral necessity for Jonas: ``And in short,
society cannot afford the absence among its members of \textit{virtue}
with its readiness to sacrifice beyond defined
duty.''\footcite[p. 228]{jonas_philosophical_1969}

As such, the moral appeal asks us to risk what must necessarily be
protected from abuse for the sake of something unnecessary and gratuitous.
The common good does not license individual sacrifice in of itself.
This is not to say that Jonas prohibits any research whatsoever.
It is only to say that only some minimum amount of clinical research is
permissible.
Moreover, Jones holds that ``the mere issuing of the appeal, the calling
for volunteers, with the moral and social pressures it inevitably generates,
amounts even under the most meticulous rules of consent to a sort of
\textit{conscripting}.''\footcite[p. 233]{jonas_philosophical_1969}
Given this attitude, the conclusion of this horn of the dilemma is a general prohibition against recruiting research participants outside of specific circumstances.
Researchers should not issue general appeals for participants.

Alternatively, we might attempt the social appeal to the common good.
Jonas, as discussed above, takes this as an appeal to the social contract.
This, however, brings us to the other horn of the dilemma.
Appeal to the social claim of the common good implies that informed consent is strictly superfluous and we may simply conscript subjects.
The social contract, as a political abstraction, is argued to be the legitimating source of the political authority of the state generally.
This comes, either explicitly in the form of a formal constitution or implicitly in some other form, prior to any research and, likely existence of involved individuals.
As Jonas astutely observes, such an appeal implies ``the idea of a public right conceived independently of (and valid prior to) consent.''\footcite[p. 222]{jonas_philosophical_1969}

A social appeal to the common good via a social contract, then, is an appeal to the political right to an individual's participation in clinical research. The social contract, after all, is an agreement among its signatories to transfer some portion of their individual rights into a governing body to wield over all for the benefit of all.
Every right implies an obligation.
Thus, individuals, under such an appeal, are obligated to participate in clinical research if called upon to do so.
In short, it is owed to society.
Jonas does not expand beyond this to conclude that an appeal to the social contract abrogates the need for informed consent.
To me, the picture must be something like this.
If we have consented to the social contract, then we have consented to the obligations entailed by that contract.
Thus, if the social contract claims a right to biomedical research and we have consented to that social contract, then we have already implicitly consented to, sight unseen, participating in clinical research when called upon.
In virtue of citizenship or political membership, additional, informed consent at the outset of participation is strictly superfluous even if it is desirable.

Despite this apparent dilemma, Jonas accepts that at least some individuals can permissibly participate in clinical research.
For one thing, Jonas thinks it natural that researchers themselves be their own subjects.
This, he claims, dissolves all logical, ethical, and metaphysical problems.\footcite[p. 234]{jonas_philosophical_1969}
Beyond this, however, he offers the rule of ``descending order''.
His is a descending order of permissibility of which populations may be appealed to be participants.
Namely, ``those patients who most identify with and are cognizant of the cause of research---members oft the medical profession (who are after all sometimes patients themselves)---come first; the highly motivated and educated, also least dependent, among the lay patients come next\ and so on down the line.''\footcite[p. 239--40]{jonas_philosophical_1969}
An additional stipulation is ``that emphatic rule that patients should be experimental upon, if at all, \textit{only} with reference to \textit{their disease}. Never should they be added to the gratuitousness of the experiment as such the gratuitousness of service to an unrelated cause.''\footcite[p. 241]{jonas_philosophical_1969}
Let this brief summary of the rule of descending order suffice.
More relevant to the later discussion is Jonas's reasoning behind it.

Jonas takes these candidate populations to permissible participate on what he terms a principle of identification.
Jonas infers from the permissibility of researchers to participate that ``one should look for additional subjects where a maximum of identification, understanding, and spontaneity can be expected---that is, among the most highly motivated, the most highly educated, and the least ``captive'' members of the community.''\footcite[p. 235]{jonas_philosophical_1969}
The moral reasoning behind this stipulation is Jonas's characterization of research participation as a kind of objectification.
``What is wrong with making a person an experimental subject is not so much that we make him thereby a means (which happens in social contexts of all kinds), as that we make him a thing---a passive thing merely to be acted on, and passive not even for real action, but for token action whose token object he is. His being is reduced to that of mere token or `sample'.''\footcite[p. 235]{jonas_philosophical_1969}
Mere consent as no more that permission does not address this objectification.
The only redress, according to Jonas, is ``by such authentic identification with the cause that it is the subject's as well as the researcher's cause---whereby his role in its service is not just permitted by him, but \textit{willed}.''\footcite[p. 236]{jonas_philosophical_1969}
Then, the participant is not a mere object of the researcher.

Ultimately, Jonas's principle of identity is an appeal to the participant's good will.
Of this, Jonas speaks religiously.
He appeals to the virtue of the participant when their will is motivated by ``compassion with human suffering, zeal for humanity, reverence for the Golden Rule, enthusiasm for progress, homage to the cause of knowledge, even longing for sacrificial justification (do not call that that ``masochism'' please)''\footcite[p. 236]{jonas_philosophical_1969}
Consequently, we must put faith ``in the transcendent potential of man'' to express their good will.
Through this, presumably, the participant is acting virtuously.
This motivational condition will have some bearing later in the promissory account of research participation.

We can conclude this exegesis by observing that Jonas's discussion of research ethics set the tone.
Here, he is concerned with the question of the permissibility of enrolling individuals as research participants.
Questions of justice---or allocation of resources for research as Wertheimer puts it---are only addressed insofar as they are dismissed out of hand as frivolous and gratuitous
The dilemma focuses on the permissibility of enrollment; we can either not issue general appeals for participants outside a handful of exception or we can simply conscript subjects as informed consent is superfluous
Granted, it is simple enough to deny the dilemma by dismissing the basic premises that progress is unnecessary and that appeal to the common good is an appeal to the social contract.
However, insofar as subsequents accounts do not abandon Jonas's individual versus society framework, they tend to focus on the question of the permissibility of enrollment in isolation to justice.
I suggest that by stepping outside of the individual versus society account, we can see that the two are necessarily intertwined.



\printbibliography

\end{document}
