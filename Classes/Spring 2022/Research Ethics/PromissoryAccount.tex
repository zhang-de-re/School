\documentclass[letterpaper,notitlepage,12pt]{article}
\usepackage[
  letterpaper,tmargin=1in,bmargin=1in,lmargin=1in,rmargin=1in]
{geometry}
\usepackage[backend=biber]{biblatex-chicago}
\addbibresource{workscited.bib}
\usepackage{setspace}
\usepackage[super]{nth}

\title{Promissory Account of Research Participation}
\author{Alexander Zhang}
\date{}

\begin{document}

\maketitle

\section{Introduction}

Since the dawn of time, the great philosophers of mankind have wondered why
Webster's dictionary says the things it says.
Aliens ask much the same but with Bilbdoolpoolp's \textit{Book of Words for
Words}.

It is sometimes helpful to draw analogy between apparently disparate discourse
to see what emerges in the juxtaposition.
In this paper, I suggest that it may be fruitful for bioethics to juxtapose
research ethics to the ethics of war.
Immediately an interesting parallel be observed.
In the ethics of war there are two or three conceptual categories: \textit{
jus ad bellum} which considers the conditions necessary to instantiate a just
war; \textit{jus in bello} which is concerned with the ethical conduct of
combatants during a war; and, recently, \textit{jus post bellum} which is
concerned with justice in terminating a war, agreeing to peace, and subsequent
actions.
Similarly, we can categorize issues in research ethics along these lines.

This appears superficial.
However, it raises an interesting observation.
While debates regarding \textit{jus ad bellum} are preoccupied with when armed
conflict can ever be just if at all, heretofore, debates in an analogous
\textit{jus ad research} have been concerned with when medical research is
merely permissible.
While adjacent, the concepts of justice and moral permissibility are distinct.
Indeed, some take it to be the case that we can talk about one without the
other.
In \textit{Rethinking the Ethics of Clinical Research}, Alan Wertheimer
stipulates the following: \begin{quotation}
I believe that the allocation of resources for research may be the
big---macro---issue of research ethics. This book focuses on a smaller (not
small)---micro---discussion of research ethics, namely, the recruitment and use
of research participants. For whether the research in question serves the needs
of developed nations, or the citizens of undeveloped nations, and whether the
participants come from developed or undeveloped nations, we must still ask: When
is it ethically justifiable to recruit and use people as research
subjects?\footcite[p. 8]{wertheimer}
\end{quotation}
In light of this juxtaposition, it is not clear to me why research ethics should
begin by attempting to hold separate justice and moral permissibility.

I offer an alternative perspective that attempts to integrate both the question
of justice and moral permissibility of clinical research.
By drawing analogy to the ethics of military privitzation, I propose a
promissory account of clinical research participation.
In doing so, the promissory account provieds a conceptual framework grounded in
principles of justice, personal autonomy, and paternalism.
As such, in the second section, I examine Hans Jonas's account of permissibility
and the dilemma it sets.
In the third section, I motivate the methodological move to draw analogy to
soldiers and mercenaries.
In the fourth section, I pivot to Chiara Cordelli's promissory account of
soldiers and mercenaries.
Finally, in the following section, I sketch out a promssory account of clinical
participation and note how such an account dissolve's Jonas's dillema.

\section{The Permissibility of Clinical Research}









\printbibliography

\end{document}
